\documentclass{article}
\usepackage[a4paper,margin=1.5cm]{geometry}
%\usepackage{lua-visual-debug}
\usepackage{luaotfload,luacode}
\usepackage[hidelinks]{hyperref}
\usepackage{pdflscape}
\usepackage{xcolor}
\usepackage{longtable}
\usepackage{booktabs}
\usepackage{array}
\usepackage{graphicx}
\usepackage{fontspec}
\usepackage[utf8]{inputenc}
\usepackage[russian]{babel}
\usepackage{tipa}
\usepackage[style=russian]{csquotes}

%\usepackage[datamodel=archive,backend=biber,style=gost-numeric]{biblatex}
\usepackage[datamodel=archive,backend=biber]{biblatex}

\setmainfont{Noto Serif}

\newfontfamily\tenevilfont[Renderer = HarfBuzz]{tenevil-font.otf}

\begin{luacode}
  documentdata       = documentdata or { }

  local stringformat = string.format
  local texsprint    = tex.sprint
  local slot_of_name = luaotfload.aux.slot_of_name

  documentdata.fontchar = function (chr)
    local chr = slot_of_name(font.current(), chr, false)
    if chr and type(chr) == "number" then
      texsprint
        (stringformat ([[\char"%X]], chr))
    end
  end
\end{luacode}

\def\fontchar#1{\directlua{documentdata.fontchar "#1"}}

\newcounter{glyph}
\setcounter{glyph}{1}

\DeclareDocumentCommand{\tenevilglyph}{o o m}{%
\def\tmpyes{yes}
\def\tmpone{#1}
\theglyph~{\IfNoValueTF{#2}{}{(#2)}}\hfill~\linebreak{%
	\IfNoValueTF{#1}{}{%
		\ifx\tmpone\tmpyes%
			%
		\else%
			\color{gray}%
		\fi%
		}
		\tenevilfont\fontsize{40pt}{40pt}\selectfont\fontchar{#3}
		}%
\stepcounter{glyph}%
}

%
%	первый опциональный аргумент — найден ли этот знак, написанный рукой Теневиля, yes/no
%
%	второй — уверенность в интерпретации:
%		0:	нет вообще ничего
%		1: 	есть только фонетическая какая-то наводка, значение которой неясно
%		2:	есть интерпретация, но сомнительная, либо есть противоречия между интерпретациями
%		3:	есть однозначная интерпретация, но источник всего один (или источники зависимые) и это не Теневиль,
%			либо интерпретация написанного Теневилем вызывает сомнения
%		4:	есть однозначная интерпретация, совпадающая или близкая в нескольких источниках, 
%			либо есть однозначный перевод приведенный Теневилем, а противоречия, если есть, объясняются ошибкой
%		5:	полная уверенность, много независимых источников сходятся на одном значении, работает в контексте,
%			не вызывает сомнений
%

\addbibresource{tenevil.bib}

\input bibliography-macros.tex

\begin{document}
\begin{landscape}
\begin{longtable}{p{1.25cm}>{\raggedright}p{9.5cm}p{3cm}>{\raggedright}p{3cm}>{\raggedright}p{3cm}>{\raggedright}p{4.75cm}}
\toprule
 & СПбФ АРАН \cite{spbfaran79} & Богораз \cite{bogoraz1934} & Миндалевич \cite{mindalevich1934} & Лавров \cite{lavrov1969} & Теневиль \cite{davydova2015a,lavrov1969,bogoraz1934} \tabularnewline \midrule
\tenevilglyph[yes][4]{i_2cU_2cD}
	&	qlaul [ӄԓявыԓ = мужчина] \cite[л. 64 об.]{spbfaran79} % ӄԓявыԓ
	& 
	& 
	& 	мужчина 
	&	[38.1]
		\tabularnewline \midrule
\tenevilglyph[yes][2]{i_2cU_2cD_'}
	&	отец \cite[л. 40, 55]{spbfaran79}\linebreak
		әtlьgьn [ытԓыгын = отец] \cite[л. 52]{spbfaran79}\linebreak % ытԓыгын
		әtlьgә \cite[л. 52]{spbfaran79}\linebreak
		etlьgьn [әtlьgьn] \cite[л. 52 об.]{spbfaran79}\linebreak
		ьnpenacgen [ынпыначгын = старик] \cite[л. 64]{spbfaran79} % ынпыначгын
	& 	отец
	& 
	& 
	&	\cite[360, 364]{davydova2015a} \linebreak
		старек [старик] [34.12об]
		\tabularnewline \midrule
\tenevilglyph[yes][3]{i_2cU_j_2cD}
	&	uwaqug [uwæquc, ы'вэӄуч = муж] \cite[л. 65 об.]{spbfaran79} % ы'вэӄуч
	& 
	&
	& 
	&	\cite[364]{davydova2015a} \tabularnewline \midrule
\tenevilglyph[yes][4]{i_2cU_2C}
	&	ŋәucqan [ŋausqan, ӈэвысӄэт = женщина] \cite[л. 65 об.]{spbfaran79} % ӈэвысӄэт
	& 
	&	a woman
	& 
	&	\cite[364]{davydova2015a} \linebreak
		женжны [женщины] [34.16]
		\tabularnewline \midrule
\tenevilglyph[yes][3]{i_2cU_j_2C}
	&	жена \cite[л. 65 об.]{spbfaran79}
	& 
	&	
	& 
	&	\cite[364]{davydova2015a}
		\tabularnewline \midrule
\tenevilglyph[yes][3]{i_2cU_l_2C}
	&	мать \cite[л. 64]{spbfaran79}\linebreak
		әtla [ытԓя = мать] \cite[л. 52]{spbfaran79}\linebreak % ытԓя
		etla [әtla] \cite[л. 52 об., 56]{spbfaran79}
	& 
	&	
	& 
	&	\cite[360, 364]{davydova2015a}
		\tabularnewline \midrule
\tenevilglyph[no][3]{i_2cU_t_2C}
	&	родившая мать \cite[л. 64]{spbfaran79}
	& 
	&	a woman awaiting the birth of her child
	& 
	&	\tabularnewline \midrule
\tenevilglyph[yes][3]{i_2cU_2C_h}
	&	ьnpьŋәu [ьnpь-ŋæw, ынпыӈэв = старуха] \cite[л. 65 об]{spbfaran79} % ынпыӈэв
	& 
	&	
	& 
	 &	[25.6]
	 	\tabularnewline \midrule
\tenevilglyph[yes][4]{i_2CF}
	&	сын \cite[л. 52]{spbfaran79}\linebreak
		сыновья \cite[л. 52]{spbfaran79} \linebreak
		әkkot [ækkæt, эккэт = сыновья] \cite[л. 39]{spbfaran79} \linebreak % экык, эккэт
		син [сын] \cite[л. 67]{spbfaran79}
	& 	сын
	&	
	& 	сын
	&	\cite[364]{davydova2015a} \linebreak
		\cite{bogoraz1934}
		\tabularnewline \midrule
\tenevilglyph[yes][3]{i_2cU_CF}
	&	доса [дочь] \cite[л. 67]{spbfaran79}
	&	
	&	
	& 
	 &	[25.8об]
	 	\tabularnewline \midrule
\tenevilglyph[no][3]{i_2cU_3CF}
	&	ситра [сестра] \cite[л. 67]{spbfaran79} 
	& 
	&	
	& 
	& 	\tabularnewline \midrule
\tenevilglyph[no][3]{i_2CF_v_q_'}
	&	прат [брат] \cite[л. 67]{spbfaran79}
	& 	
	&	
	& 
	& 	\tabularnewline \midrule
\tenevilglyph[yes][4]{i_vd_q_i} 
	&	
	& 	
	&	
	& 	друг
	& 	\cite[364]{davydova2015a} \linebreak
		таварес [товарищ] [34.8об]
		\tabularnewline \midrule
\tenevilglyph[yes][3]{i_2CF_j}
	&	qlaul nenene [qlaul nænænæ, ӄԓявыԓ нэнэны = мужчина младенец] \cite[л. 65 об]{spbfaran79} % ӄԓявыԓ нэнэны
	& 
	&	
	& 
	& 	\cite[364]{davydova2015a} 
		\tabularnewline \midrule
\tenevilglyph[yes][3]{i_2cU_CF_h}
	&	ŋәusqan neneneŋ [ŋausqan nænænæŋ, ӈэвысӄэт нэнэны = женщина мланедец] \cite[л. 65 об]{spbfaran79} % ӈэвысӄэт нэнэны
	& 
	&	
	& 
	& 	[34.9]
		\tabularnewline \midrule
\tenevilglyph[yes][4]{o-_p_j}
	&	он \cite[л. 40]{spbfaran79} \linebreak 
		әtlon [ытԓён = он] \cite[л. 39 об, 52, 65 об]{spbfaran79} % ытԓён
	& 	он
	&	
	& 
	& 	\cite[360]{davydova2015a} \linebreak
		ылтон [ытԓён] [32.16об]
		\tabularnewline \midrule
\tenevilglyph[yes][4]{o_2j}
	&	наш \cite[л. 40]{spbfaran79} \linebreak
		murgin [мургин = наш] \cite[л. 52]{spbfaran79} \linebreak % мургин
		muri [мури = мы] \cite[л. 39 об, 65 об]{spbfaran79} \linebreak % мури
		мы \cite[л. 68]{spbfaran79} \linebreak
		наса [наша] \cite[л. 68]{spbfaran79}
	& 	наш
	&	we
	& 
	& 	\cite[364]{davydova2015a} \linebreak
		\cite[28]{lavrov1969} 
		\tabularnewline \midrule
\tenevilglyph[yes][4]{o_j}
	&	мой \cite[л. 40, 55]{spbfaran79} \linebreak
		gьmnin [гымнин = мой] \cite[л. 56]{spbfaran79} \linebreak % гымнин
		gumnin [gьmnin] \cite[л. 52 об, 65]{spbfaran79}
	& 	мой
	&	
	& 
	&	[2.1?] 
		мена [меня] [37.2]
		\tabularnewline \midrule
\tenevilglyph[yes][4]{o}
	&	я \cite[л. 40, 53, 65 об]{spbfaran79} \linebreak
		gьm [гым = я]\cite[л. 52,56]{spbfaran79} \linebreak % гым
		gum [gьm] \cite[л. 52 об, 65 об]{spbfaran79}
	& 	я
	&	
	& 
	& 	\cite[364]{davydova2015a} \linebreak
		хым [гым] [34.8] \linebreak
		я [34.11]
		\tabularnewline \midrule
\tenevilglyph[yes][4]{o_j_q}
	&	мне \cite[л. 66]{spbfaran79} \linebreak
		в \textit{«мне»}, \textit{«я} восму» \cite[л. 66]{spbfaran79} \linebreak
		в \textit{«я} ниснаю», \textit{«я} упрала» \cite[л. 79]{spbfaran79}
	& 	
	&	
	& 
	& 	\cite{bogoraz1934}
		\tabularnewline \midrule
\tenevilglyph[no][3]{o-_s}
	&	gьt [гыт = ты] \cite[л. 65 об]{spbfaran79} % гыт
	& 	
	&	
	& 
	& 	\tabularnewline \midrule
\tenevilglyph[no][3]{o-_jY}
	&	turi [тури = вы] \cite[л. 65 об]{spbfaran79} % тури
	& 	
	&	
	& 
	& 	\tabularnewline \midrule
\tenevilglyph[yes][1]{o_j_j}
	&	или такои [?] \cite[л. 67]{spbfaran79} \linebreak
		эвлм [?] \cite[л. 68]{spbfaran79}
	& 	
	&	
	& 
	& 	[38.1]
		\tabularnewline \midrule
\tenevilglyph[yes][4]{R_2bN}
	&	сеть \cite[л. 40]{spbfaran79} \linebreak
		giŋingi [giŋingь, гиӈынгиӈ = сеть] \cite[л. 39]{spbfaran79} \linebreak % гиӈынгиӈ
		сетка \cite[л. 68]{spbfaran79}
	& 	сеть
	&	a net
	& 
	& 	\cite[361]{davydova2015a} \linebreak
		\cite{bogoraz1934} 
		\tabularnewline \midrule
\tenevilglyph[yes][2]{sME_2b}
	&	Teŋiwil — автор записей \cite[л. 40, 52, 54]{spbfaran79}
	& 	
	&	
	& 
	& 	\cite[360–364]{davydova2015a} \linebreak
		атехае [?] [33.5об] \linebreak
		Тынеувел [Теневиль] [35.3]
		\tabularnewline \midrule
\tenevilglyph[yes][4]{sME}
	&
	& 	
	&	
	& 	Теневиль
	& 	\cite[361]{davydova2015a} \linebreak
		\cite[28]{lavrov1969} \linebreak
		Тынеувел [Теневиль] [33.5об]
		\tabularnewline \midrule
\tenevilglyph[yes][3]{i_2lY}
	&
	& 	
	&	
	& 	Раглине
	& 	\cite[364]{davydova2015a} \linebreak
		\cite[28]{lavrov1969} 
		\tabularnewline \midrule
\tenevilglyph[yes][4]{i_2cY}
	&
	& 	
	&	
	& 	Этувьи
	& 	\cite[361, 363]{davydova2015a} \linebreak
		\cite[28]{lavrov1969} \linebreak
		Еекор [Егор] [33.5об]
		\tabularnewline \midrule
\tenevilglyph[yes][2]{i_c_C_i_j}
	&	мать \cite[л. 40]{spbfaran79} \linebreak
		Upenkew — враг автора \cite[л. 40]{spbfaran79} % Ближайшее по звучанию — шаман Упетеу \cite{mindalevich1934a}
	& 	мать
	&	
	& 	мать
	& 	[1.1] 
		\tabularnewline \midrule
\tenevilglyph[no][1]{i_c_C}
	&	Utenkew \cite[л. 52 об]{spbfaran79} \linebreak
		Utenkew (?) \cite[л. 56]{spbfaran79}
	& 	
	&	
	& 	
	& 	\tabularnewline \midrule
\tenevilglyph[yes][4]{iY_j}
	&	сам \cite[л. 40, 53]{spbfaran79} \linebreak
		cinit [cinit, чинит = сам] \cite[л. 52]{spbfaran79} \linebreak % чинит
		ꞓinit [cinit] \cite[л. 52 об]{spbfaran79}
	& 	сам
	&	
	& 	
	& 	\cite[364]{davydova2015a} \linebreak
		\cite{bogoraz1934} \linebreak
		чам [сам] [32.6об] \linebreak
		сам [34.8об]
		\tabularnewline \midrule
\tenevilglyph[yes][4]{iY}
	&	тот \cite[л. 40]{spbfaran79} \linebreak
		әnqon [әnqan, ынӄэн = тот, этот] \cite[л. 52, 54]{spbfaran79} % ынӄэн
	& 	тот
	&	
	& 	
	& 	\cite[360, 361, 364]{davydova2015a} \linebreak
		\cite[28]{lavrov1969} \linebreak
		nqen [әnqan] [4.10об] \linebreak
		ето [это] [32.16]
		\tabularnewline \midrule
\tenevilglyph[yes][4]{d_C}
	&	нет \cite[л. 40]{spbfaran79} \linebreak
		ujŋә [уйӈэ = нет чего-нибудь] \cite[л. 39]{spbfaran79} \linebreak % уйӈэ
		нету \cite[л. 66 об]{spbfaran79} \linebreak
		в \textit{«не}било» \cite[л. 66]{spbfaran79}
	& 	нет
	&	no
	& 	
	& 	\cite[360, 361, 364]{davydova2015a} \linebreak
		\cite[28]{lavrov1969} \linebreak
		нету [32.18]
		\tabularnewline \midrule
\tenevilglyph[yes][4]{G}
	&	теперь \cite[л. 40]{spbfaran79} \linebreak
		igьt [игыр = сегодня, теперь] \cite[л. 39, 52 об]{spbfaran79} \linebreak % игыр
		чиперче [теперь] \cite[л. 67 об]{spbfaran79} \linebreak
		вонтенеи (?) \cite[л. 67 об]{spbfaran79} 
	& 	теперь
	&	
	& 	
	& 	\cite[361, 364]{davydova2015a} \linebreak
		\cite[28]{lavrov1969} \linebreak
		сесеяс [сейчас] [37.2] \linebreak
		севотни [сегодня] [37.2] \linebreak
		\tabularnewline \midrule
\tenevilglyph[yes][4]{o_q}
	&	там \cite[л. 50]{spbfaran79} \linebreak
		әnkь [ынкы = там] \cite[л. 39 об]{spbfaran79} \linebreak % ынкы
		тут \cite[л. 66]{spbfaran79} \linebreak
		дут [тут] \cite[л. 68]{spbfaran79}
	& 	там
	&	
	& 	
	& 	\cite[360, 361, 364]{davydova2015a}\linebreak 
		\cite[28]{lavrov1969}\linebreak 
		тот [тут] [32.13] \linebreak
		в «вотут» [вот тут] [32.6] \linebreak
		тут [32.15об] \linebreak
		ынкы [= там] [32.16об]
		\tabularnewline \midrule
\tenevilglyph[yes][3]{o_q_'}
	&	с тех пор \cite[л. 40]{spbfaran79} \linebreak
		әnkәtegnek \cite[л. 39]{spbfaran79} \linebreak % НУЖЕН ПЕРЕВОД ынкатагнэпы ? ынкатагнык ?
		әnketegnek \cite[л. 39 об]{spbfaran79} \linebreak
		әnkәtegnьik \cite[л. 54]{spbfaran79} 
	& 	с тех пор
	&	
	& 	
	& 	\cite[360, 364]{davydova2015a} 
		\tabularnewline \midrule
\tenevilglyph[yes][4]{l-l}
	&	
	& 	
	&	
	& 	
	& 	нутку [ӈутку? = здесь, тут] [34.8]
		\tabularnewline \midrule
\tenevilglyph[no][3]{l-l_i}
	&	с этих пор \cite[л. 40]{spbfaran79} \linebreak
		wutkelegnek (?) \cite[л. 54]{spbfaran79} % НУЖЕН ПЕРЕВОД выткутэгнык ?
	& 	с этих пор
	&	
	& 	
	& 	\tabularnewline \midrule
\tenevilglyph[yes][3]{2i_P}
	&	на реке \cite[л. 41]{spbfaran79} \linebreak
		veemьk [vææmьk, вээмык = на реке] \cite[л. 39]{spbfaran79} % вээмык
	& 	на реке
	&	
	& 	
	& 	\cite[361]{davydova2015a} \linebreak
		\tabularnewline \midrule
\tenevilglyph[yes][4]{2i_2q}
	&	vaamete [vææmьte, вээмэты = к реке] \cite[л. 56]{spbfaran79} \linebreak % вээмэты
		около рещки [около речки] \cite[л. 68 об]{spbfaran79}
	& 	
	&	
	& 	
	& 	\cite[361]{davydova2015a} \linebreak
		\cite[28]{lavrov1969} 
		\tabularnewline \midrule
\tenevilglyph[yes][4]{i_g_b_jX}
	&	хариус \cite[л. 41, 54 об]{spbfaran79} \linebreak
		qeꞓaw [kьcaw, кычав = хариус] \cite[л. 39]{spbfaran79} % кычав
	& 	хариус
	&	
	& 	
	& 	\cite[361]{davydova2015a} \linebreak
		хартон [?] [33.4]
		\tabularnewline \midrule
\tenevilglyph[yes][4]{i_g_b}
	&	кета \cite[л. 44, 45, 54 об]{spbfaran79} \linebreak
		рипа [рыба] \cite[л. 68 об]{spbfaran79}
	& 	кета
	&	
	& 	рыба кета
	& 	\cite[361]{davydova2015a} \linebreak 
		\cite[26]{lavrov1969} \linebreak
		кетат [ӄэтаӄэт, кета] [33.4] \linebreak
		эрепо [рыба?] [33.4]
		\tabularnewline \midrule
\tenevilglyph[no][3]{i_g_2b}
	&	налим \cite[л. 45, 54 об]{spbfaran79} 
	& 	налим
	&	
	& 	налим
	& 	\tabularnewline \midrule
\tenevilglyph[yes][3]{i_g_b_z}
	&	сиг \cite[л. 45]{spbfaran79} 
	& 	
	&	
	& 	
	& 	[2.101]* 
		\tabularnewline \midrule
\tenevilglyph[yes][4]{i_g_b_hL}
	&	щука \cite[л. 45]{spbfaran79} 
	& 	щука
	&	
	& 	щука
	& 	[2.101] 
		\tabularnewline \midrule %
\tenevilglyph[no][3]{i_g_2b_q_k}
	&	бычок \cite[л. 45]{spbfaran79} 
	& 	
	&	
	& 	
	& 	\tabularnewline \midrule
\tenevilglyph[yes][3]{i_g_b_2cD}
	&	 [без перевода] \cite[л. 54 об]{spbfaran79} 
	& 	
	&	
	& 	нярка
	& 	\cite[361]{davydova2015a} 
		\tabularnewline \midrule
\tenevilglyph[yes][3]{u_j_jX_j}
	&	еда, есть \cite[л. 41]{spbfaran79} \linebreak
		roьlqal [roolqәl, рооԓӄыԓ = пища, продукты] \cite[л. 39]{spbfaran79} % рооԓӄыԓ
	& 	еда, есть
	&	food
	& 	
	& 	\cite[364]{davydova2015a} 
		\tabularnewline \midrule
\tenevilglyph[yes][2]{u_j_jX} 
	&	
	& 	
	&	
	& 	
	& 	в «чаеопат» [«чай \textit{пить}» или чайпат = вскипевший чай] [32.16об] % чайпат?
		\tabularnewline \midrule
\tenevilglyph[yes][4]{i_I_2qY}
	&	бояться \cite[л. 41]{spbfaran79} \linebreak
		в «мы \textit{боимся»} \cite[л. 52]{spbfaran79} \linebreak
		в «я оцин \textit{боюс»} [«я очень боюсь»] \cite[л. 67 об]{spbfaran79}
	& 	бояться
	&	
	& 	
	& 	поеысея [боялся] [37.2] 
		\tabularnewline \midrule
\tenevilglyph[yes][4]{o_q_jF}
	&	сделал \cite[л. 41]{spbfaran79} \linebreak
		елат [делать] \cite[л. 68]{spbfaran79}
	& 	сделал
	&	made
	& 	
	& 	\cite[361, 364]{davydova2015a} 
		\tabularnewline \midrule
\tenevilglyph[yes][4]{o_q_jF_b}
	&	
	& 	
	&	
	& 	делать, работать
	& 	\cite[364]{davydova2015a} \linebreak
		ырыпотали [работали] [34.8об]
		\tabularnewline \midrule
\tenevilglyph[yes][4]{U_v}
	&	сказал* \cite[л. 41]{spbfaran79} \linebreak % тут везде знак перевернут, похоже
		giulin* \cite[л. 52]{spbfaran79} % НУЖЕН ПЕРЕВОД гэвъилин ?
	& 	сказал*
	&	
	& 	
	& 	каварйт [говорит] [32.16]
		\tabularnewline \midrule
\tenevilglyph[yes][4]{U_b}
	&	
	& 	
	&	
	& 	
	& 	еесик [язык] [32.6об] \linebreak
		еегиг [язык] [34.16]
		\tabularnewline \midrule
\tenevilglyph[yes][4]{c_CE}
	&	пребывать, быть \cite[л. 41]{spbfaran79} \linebreak
		в \textit{«било} он», «не\textit{било»}, «какои» \cite[л. 66]{spbfaran79}
	& 	пребывать, быть
	&	
	& 	
	& 	\cite[360, 361, 364]{davydova2015a} \linebreak
		\cite[28]{lavrov1969} \linebreak
		ееч [есть] [32.16] \linebreak
		еес [есть] [33.4] \linebreak
		\tabularnewline \midrule
\tenevilglyph[yes][4]{UD_2B}
	&	жить \cite[л. 41]{spbfaran79} \linebreak
		nyegtel \cite[л. 39]{spbfaran79} \linebreak % НУЖЕН ПЕРЕВОД ны-егтэԓ ?
		nьegtel \cite[л. 39 об]{spbfaran79} \linebreak
		сивоы [живой] \cite[л. 68]{spbfaran79}
	& 	жить
	&	lived
	& 	жить
	& 	\cite[360, 364]{davydova2015a} 
		\tabularnewline \midrule
\tenevilglyph[yes][3]{UE}
	&	становиться \cite[л. 41]{spbfaran79} \linebreak
		mьnneel \cite[л. 39]{spbfaran79} \linebreak % НУЖЕН ПЕРЕВОД мыннъэԓ ? мытнъэԓ?
		mьtьnnel \cite[л. 39 об]{spbfaran79} \linebreak
		mьn-neel \cite[л. 52]{spbfaran79}
	& 	становиться
	&	
	& 	
	& 	\cite[360, 364]{davydova2015a} 
		\tabularnewline \midrule
\tenevilglyph[yes][3]{2OX_j}
	&	расти и большой \cite[л. 41]{spbfaran79} \linebreak
		nьmejŋqin* [nьmæjьŋqin, нымэйыӈӄин = большой] \cite[л. 54]{spbfaran79} \linebreak % нымэйыӈӄин
		mejn* [mæjŋ, мэйыӈ = большой (основа)] \cite[л. 39 об]{spbfaran79} % мэйыӈ
	& 	расти, большой
	&	
	& 	
	& 	\cite[360, 364]{davydova2015a} 
		\tabularnewline \midrule
\tenevilglyph[yes][2]{2OX} 
	&	mejŋь [mæjŋ, мэйыӈ = большой (основа)] \cite[л. 64 об]{spbfaran79} \linebreak % мэйыӈ
		оыё [?] \cite[л. 66]{spbfaran79} \linebreak
		коломеи [коԓё мэй = очень большой] \cite[л. 68 об]{spbfaran79} % коԓё мэй
	& 	
	&	
	& 	
	& 	\cite[361, 364]{davydova2015a} \linebreak
		\cite[28]{lavrov1969} 
		\tabularnewline \midrule
\tenevilglyph[yes][3]{o_4i}
	&	nenanmuqen \cite[л. 54]{spbfaran79} \linebreak % НУЖЕН ПЕРЕВОД
		в \textit{«убил} волка» \cite[л. 68 об]{spbfaran79} 
	& 	
	&	
	& 	
	& 	\cite[360, 361]{davydova2015a} \linebreak
		\cite{bogoraz1934} 
		\tabularnewline \midrule
\tenevilglyph[yes][4]{o_4i_k}
	&	умереть \cite[л. 41]{spbfaran79} \linebreak
		умирают \cite[л. 52]{spbfaran79} \linebreak
		wu [vi, въи = умереть (основа)] \cite[л. 52]{spbfaran79} \linebreak % въи
		vu [vi] \cite[л. 52]{spbfaran79} 
	& 	
	&	
	& 	
	& 	\cite[360]{davydova2015a} \linebreak
		омейр [умер] [32.16]
		\tabularnewline \midrule
\tenevilglyph[yes][4]{c_JY}
	&	покинул \cite[л. 41]{spbfaran79} \linebreak
		enapelae [энапэԓя = оставлять (основа)] \cite[л. 52]{spbfaran79} \linebreak % энапэԓя 
		enapela \cite[л. 56]{spbfaran79} \linebreak
		оставил \cite[л. 68 об]{spbfaran79}
	& 	
	&	
	& 	
	& 	[25.3] 
		\tabularnewline \midrule
\tenevilglyph[yes][2]{b_2q_L}
	&	покинул \cite[л. 41]{spbfaran79} \linebreak
		pela [пэԓя = покидать, оставлять (основа)] \cite[л. 52]{spbfaran79} % пэԓя
	& 	
	&	
	& 	
	& 	\cite[364]{davydova2015a} \linebreak
		чкорйе [?] [32.16]
		\tabularnewline \midrule
\tenevilglyph[yes][3]{4L}
	&	плакать \cite[л. 41]{spbfaran79} \linebreak
		nьtergьtьm \cite[л. 52]{spbfaran79} % НУЖЕН ПЕРЕВОД нытэргат-?
	& 	
	&	
	& 	
	& 	\cite[360]{davydova2015a} 
		\tabularnewline \midrule
\tenevilglyph[yes][4]{a}
	&	стадо, олень \cite[л. 42]{spbfaran79} \linebreak
		ŋәlvil [ŋælvьl, ӈэԓвыԓ = стадо (преимущественно оленей)] \cite[л. 56]{spbfaran79} % ӈэԓвыԓ
	& 	стадо, олень
	&	reindeer
	& 	олень
	& 	\cite[364]{davydova2015a} \linebreak
		\cite{bogoraz1934} \linebreak
		олене [олень] [33.4]
		\tabularnewline \midrule
\tenevilglyph[yes][3]{a_k}
	&	силонок [теленок] \cite[л. 68 об]{spbfaran79} 
	& 	
	&	
	& 	
	& 	\cite[362]{davydova2015a} 
		\tabularnewline \midrule
\tenevilglyph[no][3]{a_k_j}
	&
	& 	
	&	
	& 	теленок
	& 	\tabularnewline \midrule
\tenevilglyph[yes][4]{a_q}
	&	васонка [важенка] \cite[л. 68 об]{spbfaran79} 
	& 	
	&	
	& 	важенка
	& 	[25.6об] 
		\tabularnewline \midrule
\tenevilglyph[yes][4]{a_q_l}
	&	 
	& 	
	&	
	& 	
	& 	ванкаскор [ваӈӄасӄор = важенка в возрасте двух лет, яловая важенка] [25.6об] % ваӈӄасӄор
		\tabularnewline \midrule
\tenevilglyph[yes][3]{a_t}
	&	силилас [телилась] \cite[л. 68 об]{spbfaran79} 
	& 	
	&	
	& 	
	& 	\cite[362]{davydova2015a} \linebreak
		\cite[26]{lavrov1969} 
		\tabularnewline \midrule
\tenevilglyph[yes][4]{aB}
	&	в «стойбище, олени (богатый оленевод)» \cite[л. 47]{spbfaran79} \linebreak
		табун \cite[л. 55]{spbfaran79} 
	& 	
	&	reindeer herd
	& 	стадо оленей
	& 	\cite[361]{davydova2015a} \linebreak
		\cite[26, 28]{lavrov1969} \linebreak
		тапон [табун] [33.4]
		\tabularnewline \midrule
\tenevilglyph[yes][3]{a_o}
	&	
	& 	
	&	
	& 	дикий олень
	& 	[18.1об?] 
		\tabularnewline \midrule
\tenevilglyph[yes][3]{a_jT}
	&	бик [бык] \cite[л. 68 об]{spbfaran79} 
	& 	
	&	
	& 	
	& 	[1.2] 
		\tabularnewline \midrule
\tenevilglyph[yes][3]{a_2jX}
	&	хромой олень \cite[л. 43]{spbfaran79} 
	& 	
	&	
	& 	
	& 	[25.9] \tabularnewline \midrule
\tenevilglyph[yes][4]{s_b}
	&	много \cite[л. 42]{spbfaran79} \linebreak
		много \cite[л. 37]{spbfaran79} \linebreak
		numkәqin [nьmkәqin, нымкыӄин = многочисленный] \cite[л. 54]{spbfaran79} \linebreak % нымкыӄин
		nьemkәqin [nьmkәqin] \cite[л. 54]{spbfaran79} \linebreak
		nьmkәqin \cite[л. 52 об]{spbfaran79} \linebreak
		мноко [много] \cite[л. 66 об, 67]{spbfaran79}
	& 	
	&	
	& 	
	& 	\cite[360–364]{davydova2015a} \linebreak
		\cite[28]{lavrov1969} \linebreak
		\cite{bogoraz1934} \linebreak
		мноко [много] [34.11]
		\tabularnewline \midrule
\tenevilglyph[yes][4]{f}
	&	человек, народ \cite[л. 42]{spbfaran79} \linebreak
		человек \cite[л. 53]{spbfaran79} \linebreak
		соловк [человек] \cite[л. 68 об]{spbfaran79} 
	& 	человек
	&	
	& 	человек
	& 	\cite[360, 361, 364]{davydova2015a} \linebreak
		\cite{bogoraz1934} \linebreak
		соловек [человек] [37.7об]
		\tabularnewline \midrule
\tenevilglyph[yes][4]{i_l}
	&	другой \cite[л. 42]{spbfaran79} \linebreak
		другой \cite[л. 53]{spbfaran79} 
	& 	другой
	&	
	& 	
	& 	\cite[361–364]{davydova2015a} \linebreak
		\cite{bogoraz1934} \linebreak
		трхой [другой] [32.16]
		\tabularnewline \midrule
\tenevilglyph[yes][3]{v_l}
	&	раньше, ꞓit [cit, чит = раньше, прежде] \cite[л. 42]{spbfaran79} \linebreak % чит
		ꞓit [cit] \cite[л. 52 об, 56]{spbfaran79} 
	& 	
	&	
	& 	
	& 	\cite[364]{davydova2015a} \linebreak
		\cite[28]{lavrov1969} 
		\tabularnewline \midrule
\tenevilglyph[yes][3]{i_LX}
	&	действительно, qәjve [ӄэйвэ = право, и действительно] \cite[л. 42]{spbfaran79} \linebreak % ӄэйвэ
		qojve [qәjve] \cite[л. 56]{spbfaran79} \linebreak
		qoive [qәjve] \cite[л. 54, 52 об]{spbfaran79}
	& 	действительно
	&	
	& 	
	& 	\cite[360–362, 364]{davydova2015a} 
		\tabularnewline \midrule
\tenevilglyph[yes][3]{w}
	&	еще не, jep [еп = еще; пока еще; еще в то время, как] \cite[л. 42]{spbfaran79} \linebreak % еп
		jep \cite[л. 52, 52 об, 56]{spbfaran79}
	& 	еще не
	&	
	& 	
	& 	\cite[360, 364]{davydova2015a} 
		\tabularnewline \midrule
\tenevilglyph[yes][3]{2c}
	&	кое как, насилу \cite[л. 42]{spbfaran79} \linebreak
		metkiit [mætkiit, мэткиит = насилу, еле-еле] \cite[л. 39, 52]{spbfaran79} % мэткиит
	& 	
	&	somehow
	& 	
	& 	 \cite{bogoraz1934} 
		\tabularnewline \midrule
\tenevilglyph[yes][4]{I_2l}
	&	совсем, qonpь [ӄонпыӈ = совсем, совершенно] \cite[л. 42]{spbfaran79} \linebreak % ӄонпыӈ
		qonpь* \cite[л. 39]{spbfaran79} \linebreak
		совсем \cite[л. 67]{spbfaran79}
	& 	совсем
	&	
	& 	
	& 	\cite[360, 361, 364]{davydova2015a} \linebreak
		\cite[28]{lavrov1969} 
		\tabularnewline \midrule
\tenevilglyph[yes][3]{wD}
	&	довлно [довольно] \cite[л. 68 об]{spbfaran79} 		
	& 	
	&	
	& 	
	& 	[33.14]
		\tabularnewline \midrule
\tenevilglyph[yes][3]{wD2E}
	&	притом, qәnver [qanver, ӄоныры = также] \cite[л. 42]{spbfaran79} \linebreak % ӄоныры
		qanver \cite[л. 39, 56]{spbfaran79} \linebreak
		qәnver \cite[л. 52, 56]{spbfaran79} 		
	& 	притом
	&	
	& 	
	& 	\cite[360, 361]{davydova2015a} 
		\tabularnewline \midrule
\tenevilglyph[yes][4]{2o_2iY}
	&	давно \cite[л. 42]{spbfaran79} \linebreak	
		telenjep [tælænjæp, тэԓенъеп = давно] \cite[л. 39 об, 52, 56]{spbfaran79} \linebreak % тэԓенъеп
		давно \cite[л. 66 об]{spbfaran79}
	& 	
	&	
	& 	
	& 	\cite[360]{davydova2015a} 
		\tabularnewline \midrule
\tenevilglyph[yes][3]{b_q}
	&	также, ꞓama [cama, чама = тоже] \cite[л. 42]{spbfaran79} \linebreak % чама
		тоже \cite[л. 37]{spbfaran79} \linebreak
		ꞓama [cama] \cite[л. 39 об, 54]{spbfaran79}
	& 	так же, тоже
	&	
	& 	
	& 	\cite[360, 361, 364]{davydova2015a} \linebreak
		\cite[28]{lavrov1969} \linebreak	
		\cite{bogoraz1934} 
		\tabularnewline \midrule
\tenevilglyph[yes][3]{2i_2cD_2l}
	&	все \cite[л. 42]{spbfaran79} \linebreak	
		ьmьlo [ымыԓьо = весь] \cite[л. 52 об]{spbfaran79} % ымыԓьо
	& 	все
	&	all
	& 	
	& 	\cite[360, 361, 364]{davydova2015a} 
		\tabularnewline \midrule
\tenevilglyph[yes][4]{U_q}
	&	но \cite[л. 42]{spbfaran79} \linebreak	
		naqam [наӄам = но, однако] \cite[л. 39, 52 об, 54, 56]{spbfaran79} % наӄам
	& 	
	&	
	& 	
	& 	\cite[360, 361, 364]{davydova2015a} \linebreak
		нахам [наӄам] [34.8]
		\tabularnewline \midrule
\tenevilglyph[yes][4]{o_2CY}
	&	хорошо \cite[л. 43]{spbfaran79} \linebreak	
		meꞓәnkь [mæсәnkь, мэчынкы = довольно, достаточно, хорошо, в состоянии] \cite[л. 39, 52]{spbfaran79} \linebreak % мэчынкы
		латно [ладно] \cite[л. 67]{spbfaran79} \linebreak
		хвачит [хватит] \cite[л. 68 об]{spbfaran79}
	& 	
	&	well
	& 	
	& 	\cite[360, 361, 364]{davydova2015a} 
		\tabularnewline \midrule
\tenevilglyph[no][3]{SMY_iX}
	&	слабый \cite[л. 43]{spbfaran79} \linebreak	
		nьrulqin [ныруԓӄин = слабый] \cite[л. 52, 52 об]{spbfaran79} \linebreak % ныруԓӄин
		nьrulium \cite[л. 52 об, 56]{spbfaran79} \linebreak
		nьrulqinet \cite[л. 39 об]{spbfaran79}
	& 	
	&	
	& 	
	& 	\tabularnewline \midrule
\tenevilglyph[yes][3]{S_iX}
	&	не могли* \cite[л. 43]{spbfaran79} \linebreak % вверх ногами опять
		nanilrawcьm* (?) \cite[л. 39]{spbfaran79} % НУЖЕН ПЕРЕВОД
	& 	
	&	
	& 	
	& 	[33.7]
		\tabularnewline \midrule
\tenevilglyph[yes][3]{i_4l_2l}
	&	тосковать* \cite[л. 43]{spbfaran79} 
	& 	
	&	
	& 	
	& 	[9.3] 
		\tabularnewline \midrule % тут вверх ногами
\tenevilglyph[yes][1]{i_4l}
	&	
	& 	
	&	
	& 	
	& 	ом [?] [37.2] 
		\tabularnewline \midrule
\tenevilglyph[yes][4]{U2E_JX}
	&	лето наступило \cite[л. 43]{spbfaran79} \linebreak	
		elerurkьn = лето начинается [ælærurkьn = лето начинается] \cite[л. 52 об]{spbfaran79} \linebreak % НУЖНА ТРАНСКРИПЦИЯ 
		лет [лето] \cite[л. 66]{spbfaran79}
	& 	
	&	
	& 	
	& 	\cite[362]{davydova2015a} \linebreak
		\cite[28]{lavrov1969} 
		\tabularnewline \midrule
\tenevilglyph[yes][4]{U_JX_3'}
	&	симои [зимой] \cite[л. 66]{spbfaran79}
	& 	
	&	
	& 	
	& 	семои [зимой] [34.11]
		\tabularnewline \midrule
\tenevilglyph[yes][4]{2O}
	&	пусть, чтобы, opopo [опопы = пусть] \cite[л. 43]{spbfaran79} \linebreak % опопы
		пусть \cite[л. 53]{spbfaran79} \linebreak
		opopo \cite[л. 52 об]{spbfaran79} 
	& 	
	&	
	& 	
	& 	\cite[364]{davydova2015a} \linebreak
		опопы [= пусть] [34.8]
		\tabularnewline \midrule
\tenevilglyph[yes][4]{o_3iS}
	&	пускай, maꞓьnan [macьnan, мачынан = пусть] \cite[л. 43]{spbfaran79} \linebreak % мачынан
		maꞓьnan [macьnan] \cite[л. 52 об, 56]{spbfaran79} \linebreak
		мачнан [macьnan] \cite[л. 68]{spbfaran79} 
	& 	
	&	
	& 	
	& 	\cite[364]{davydova2015a} \linebreak
		\cite{bogoraz1934} 
		\tabularnewline \midrule
\tenevilglyph[yes][4]{u-o_b}
	&	как же, miŋkri [миӈкри = куда, как] \cite[л. 43]{spbfaran79} \linebreak % миӈкри
		miŋkri-lu \cite[л. 56]{spbfaran79} \linebreak % НУЖЕН ПЕРЕВОД ԓымиӈкыри? миӈкри-?
		кута [куда] \cite[л. 66]{spbfaran79} \linebreak
		как \cite[л. 66 об]{spbfaran79} \linebreak
		в «какои» \cite[л. 66]{spbfaran79} 
	& 	
	&	
	& 	
	& 	\cite[364]{davydova2015a} 
		\tabularnewline \midrule
\tenevilglyph[yes][1]{u-o}
	&	аке [?] \cite[л. 68]{spbfaran79}
	& 	
	&	
	& 	
	& 	[25.9] 
		\tabularnewline \midrule
\tenevilglyph[no][2]{U_iX_b}
	&	завидовать \cite[л. 43]{spbfaran79}
	& 	
	&	divide the reindeer herd
	& 	
	& 	\tabularnewline \midrule
\tenevilglyph[yes][4]{i_2kU_2kD}
	&	шкура \cite[л. 44]{spbfaran79} \linebreak
		nelgen [nælgьn, нэԓгын = шкура]* \cite[л. 49 об]{spbfaran79} \linebreak % нэԓгын
		писик [пыжик] \cite[л. 68]{spbfaran79}
	& 	
	&	
	& 	
	& 	\cite[364]{davydova2015a} 
		\tabularnewline \midrule
\tenevilglyph[yes][3]{i_2kU_kD_2Q}
	&	неторос [недоросль] \cite[л. 68]{spbfaran79} 
	& 	
	&	
	& 	
	& 	\cite[364]{davydova2015a} 
		\tabularnewline \midrule
\tenevilglyph[yes][3]{i_2kU_kD_2Q_iX}
	&	посела [постель (шкура взрослого оленя)] \cite[л. 68]{spbfaran79} 
	& 	
	&	
	& 	
	& 	[32.17об]
		\tabularnewline \midrule
\tenevilglyph[yes][1]{i_kU_b_3Q_c}
	&	саски [?] \cite[л. 68]{spbfaran79} 
	& 	
	&	
	& 	
	& 	\cite[364]{davydova2015a} 
		\tabularnewline \midrule
\tenevilglyph[yes][3]{k_o_oN}
	&	випороток [выпороток] \cite[л. 68]{spbfaran79} 
	& 	
	&	
	& 	
	& 	[1.1] \tabularnewline \midrule
\tenevilglyph[yes][4]{2k}
	&	мука \cite[л. 44]{spbfaran79} \linebreak
		мука \cite[л. 66 об]{spbfaran79}
	& 	мука
	&	
	& 	
	& 	[25.6]
		\tabularnewline \midrule
\tenevilglyph[yes][4]{vY_z}
	&	русский \cite[л. 44]{spbfaran79} 
	& 	
	&	Russian holding fire-arms
	& 	русский*
	& 	\cite[364]{davydova2015a} 
		\tabularnewline \midrule
\tenevilglyph[yes][4]{a_vY_z}
	&	
	& 	
	&	
	& 	
	& 	совхос [совхоз] [32.13] \linebreak % \cite[148]{sergeev1956} «мельгитаньги чаучуа» — русские оленные люди
		оленсохоси [оленсовхоз] [34.11об]
		\tabularnewline \midrule
\tenevilglyph[yes][4]{bD_b_vY_z}
	&	
	& 	
	&	
	& 	Ленин
	& 	ленн [Ленин] [34.9] % часть «ԓыгэн» тут, видимо, в качестве фонетической компоненты 
		\tabularnewline \midrule
\tenevilglyph[no][4]{zR_v}
	&	следы \cite[л. 45]{spbfaran79} 
	& 	следы
	&	
	& 	
	& 	\tabularnewline \midrule
\tenevilglyph[yes][4]{c_2cD_q}
	&	волк \cite[л. 45, 53]{spbfaran79} \linebreak
		волк \cite[л. 68 об]{spbfaran79} \linebreak
		в «убил \textit{волка»} \cite[л. 68 об]{spbfaran79}
	& 	волк
	&	
	& 	волк
	& 	\cite[360]{davydova2015a} \linebreak
		воки [волки] [34.12]
		\tabularnewline \midrule
\tenevilglyph[yes][4]{cD_b}
	&	мтвет [медведь] \cite[л. 68 об]{spbfaran79}
	& 	
	&	
	& 	
	& 	мысвеси [медведи] [34.12]
		\tabularnewline \midrule
\tenevilglyph[yes][4]{I-IE} 
	&	
	& 	
	&	
	& 	
	& 	рысомака [росомаха] [34.12]
		\tabularnewline \midrule
\tenevilglyph[yes][4]{2CY} % С лисами сплошная путаница
	&	reqokalgьn [рэӄокаԓгын = лисица, песец] \cite[л. 54]{spbfaran79} % рэӄокаԓгын
	& 	
	&	
	& 	песец
	& 	[11.3] \linebreak
		лесесея [лисица] [34.12]
		\tabularnewline \midrule
\tenevilglyph[no][3]{2CY_c} 
	&	голубой песец \cite[л. 46]{spbfaran79} 
	& 	голубой песец
	&	
	& 	
	& 	\tabularnewline \midrule
\tenevilglyph[no][2]{2CY_2c} 
	&	песец \cite[л. 45]{spbfaran79} \linebreak
		сорнпур [чернобурая] \cite[л. 69 об]{spbfaran79} 
	& 	
	&	
	& 	
	& 	\tabularnewline \midrule
\tenevilglyph[yes][3]{2CY_cFD} 
	&	красная лисица* \cite[л. 45]{spbfaran79} \linebreak
		лисита [лисица] \cite[л. 69 об]{spbfaran79}
	& 	
	&	
	& 	
	& 	[11.3] 
		\tabularnewline \midrule
\tenevilglyph[yes][2]{2CY_o_I_3q} 
	&	огневка \cite[л. 45]{spbfaran79} \linebreak
		песеч [песец] \cite[л. 69 об]{spbfaran79}
	& 	
	&	
	& 	
	& 	[4.4об]*
		\tabularnewline \midrule
\tenevilglyph[no][2]{2CY_o_I_3q_c} 
	&	чернобурая \cite[л. 45]{spbfaran79} \linebreak
		кулубои [голубой] \cite[л. 69 об]{spbfaran79}
	& 	
	&	
	& 	
	& 	\tabularnewline \midrule
\tenevilglyph[no][3]{2CY_o_I_3q_2jF} 
	&	сиводушка \cite[л. 45]{spbfaran79}
	& 	
	&	
	& 	
	& 	\tabularnewline \midrule
\tenevilglyph[yes][4]{2cF_k_2qY} 
	&	заяц \cite[л. 46]{spbfaran79} \linebreak
		melitolgьn [melotalgьn, мэԓётаԓгын = заяц] \cite[л. 54]{spbfaran79} % мэԓётаԓгын
	& 	заяц
	&	a hare
	& 	
	& 	[рядом с изображением кролика] [11.1]
		\tabularnewline \midrule
\tenevilglyph[yes][4]{v-_jF}
	&	кострула [кастрюля] \cite[л. 68]{spbfaran79}
	& 	
	&	
	& 	
	& 	\cite[364]{davydova2015a} \linebreak
		в «7 ветра» [7 ведер] [34.19об]
		\tabularnewline \midrule
\tenevilglyph[no][3]{O_v}
	&	тас [таз] \cite[л. 66]{spbfaran79}
	& 	
	&	
	& 	
	& 	\tabularnewline \midrule
\tenevilglyph[no][3]{O_v_vD}
	&	в «кастрюлька» \cite[л. 46]{spbfaran79}
	& 	кастрюля
	&	
	& 	
	& 	\tabularnewline \midrule
\tenevilglyph[no][3]{O_v_2jF}
	&	тарелка \cite[л. 46]{spbfaran79}
	& 	тарелка
	&	
	& 	
	& 	\tabularnewline \midrule
\tenevilglyph[yes][3]{i_c_c_2j}
	&	
	& 	свечка
	&	
	& 	
	& 	\cite[364]{davydova2015a}
		\tabularnewline \midrule
\tenevilglyph[yes][3]{R_o-o}
	&	молоко \cite[л. 49]{spbfaran79} 
	& 	молоко
	&	
	& 	
	& 	[4.4об]
		\tabularnewline \midrule
\tenevilglyph[yes][3]{R_o-o_2j}
	&	молоко \cite[л. 49]{spbfaran79} 
	& 	молоко
	&	
	& 	
	& 	[2.101]
		\tabularnewline \midrule
\tenevilglyph[no][3]{R_o-o_2b}
	&	банка с керосином \cite[л. 46]{spbfaran79} 
	& 	банка с керосином
	&	
	& 	
	& 	\tabularnewline \midrule
\tenevilglyph[no][3]{R-o-o_3iS_'}
	&	банка с жиром \cite[л. 46]{spbfaran79} 
	& 	банка с жиром
	&	
	& 	
	& 	\tabularnewline \midrule
\tenevilglyph[yes][3]{R_o-o_c_zR}
	&	банка с салом (с маслом) \cite[л. 46]{spbfaran79} 
	& 	банка с салом
	&	
	& 	
	& 	[4.1]
		\tabularnewline \midrule
\tenevilglyph[yes][3]{R_o-o_2CE}
	&	банка с сахаром \cite[л. 49]{spbfaran79} 
	& 	
	&	
	& 	
	& 	[4.7]
		\tabularnewline \midrule
\tenevilglyph[yes][3]{C_c_zR} 
	&	в «олений \textit{жир} это» \cite[л. 46]{spbfaran79}
	& 	
	&	
	& 	
	& 	[4.5об]
		\tabularnewline \midrule
\tenevilglyph[yes][4]{c_2b}
	&	белый \cite[л. 46]{spbfaran79} \linebreak
		пелои [белый] \cite[л. 68]{spbfaran79}
	& 	белый
	&	
	& 	
	& 	\cite[362–364]{davydova2015a} \linebreak
		\cite[28]{lavrov1969}
		\tabularnewline \midrule
\tenevilglyph[yes][4]{o-o_J}
	&	маленький \cite[л. 46]{spbfaran79} \linebreak
		nьpluqin [ныппыԓюӄин = маленький] \cite[л. 46]{spbfaran79} % ныппыԓюӄин
	& 	маленький
	&	
	& 	
	& 	\cite[360]{davydova2015a} \linebreak
		малйнкй [маленький] [37.7об]
		\tabularnewline \midrule
\tenevilglyph[yes][3]{O_bN}
	&	
	& 	крадет
	&	
	& 	
	& 	\cite{bogoraz1934}
		\tabularnewline \midrule
\tenevilglyph[yes][3]{U_bN}
	&	вороватый \cite[л. 47]{spbfaran79} 
	& 	вороватый
	&	
	& 	
	& 	\cite{bogoraz1934}
		\tabularnewline \midrule
\tenevilglyph[yes][4]{i_G}
	&	хороший \cite[л. 47]{spbfaran79} \linebreak
		хоросой [хороший]* \cite[л. 66, 68 об]{spbfaran79} 
	& 	хороший
	&	
	& 	
	& 	\cite[360, 364]{davydova2015a} \linebreak
		\cite{bogoraz1934} \linebreak
		хоросо [хорошо] [33.4] 
		\tabularnewline \midrule
\tenevilglyph[yes][3]{i_o_G}
	&	
	& 	мастероватый
	&	
	& 	
	& 	\cite{bogoraz1934} \linebreak
		[25.13об]
		\tabularnewline \midrule
\tenevilglyph[yes][3]{i_G_b}
	&	поправилас [поправилась] \cite[л. 66 об]{spbfaran79}
	& 	
	&	
	& 	
	& 	[25.13]
		\tabularnewline \midrule
\tenevilglyph[yes][1]{i_G_bX}
	&	блно [?] \cite[л. 66]{spbfaran79}
	& 	
	&	
	& 	
	& 	[4.8] 
		\tabularnewline \midrule
\tenevilglyph[yes][4]{BD}
	&	худой \cite[л. 47]{spbfaran79} \linebreak
		хутои [худой] \cite[л. 68 об]{spbfaran79} 
	& 	худой
	&	
	& 	
	& 	\cite[364]{davydova2015a} \linebreak
		\cite{bogoraz1934} \linebreak
		iэтке [э'тки = плохо, скверно] [34.8] % э'тки
		плохо [34.11]
		\tabularnewline \midrule
\tenevilglyph[yes][4]{BD_cD}
	&	
	& 	
	&	
	& 	
	& 	плохои [плохой] [37.2]
		\tabularnewline \midrule
\tenevilglyph[yes][3]{O_jN}
	&	ночью \cite[л. 47]{spbfaran79} 
	& 	
	&	
	& 	
	& 	\cite[360, 362]{davydova2015a} 
		\tabularnewline \midrule
\tenevilglyph[yes][2]{2o_2j}
	&	в «стойбище, олени (богатый оленевод)» \cite[л. 47]{spbfaran79} \linebreak
		в «на стойбище» \cite[л. 53]{spbfaran79} \linebreak
		отлакир [әtlьq-, эԓԓыӄ- = тундра] \cite[л. 68]{spbfaran79} % эԓԓыӄ-
	& 	
	&	
	& 	
	& 	\cite[364]{davydova2015a} 
		\tabularnewline \midrule
\tenevilglyph[no][3]{i_j_jF}
	&	ложка \cite[л. 48]{spbfaran79}
	& 	ложка
	&	
	& 	
	& 	\tabularnewline \midrule
\tenevilglyph[yes][4]{u_p}
	&	чайник \cite[л. 48]{spbfaran79} \linebreak
		саиник [чайник] \cite[л. 53]{spbfaran79}
	& 	чайник
	&	
	& 	
	& 	\cite[364]{davydova2015a}
		\tabularnewline \midrule
\tenevilglyph[yes][3]{u_p_b}
	&	белый чайник \cite[л. 48]{spbfaran79} 
	& 	белый чайник
	&	
	& 	
	& 	\cite[364]{davydova2015a}
		\tabularnewline \midrule
\tenevilglyph[no][3]{u_pD_bD}
	&	медный чайник \cite[л. 48]{spbfaran79} 
	& 	медный чайник
	&	
	& 	
	& 	\tabularnewline \midrule
\tenevilglyph[yes][3]{u_p_2b}
	&	широкий чайник \cite[л. 48]{spbfaran79} 
	& 	
	&	
	& 	
	& 	\cite[364]{davydova2015a}
		\tabularnewline \midrule
\tenevilglyph[yes][4]{jFY_jF}
	&	ремень* \cite[л. 48]{spbfaran79} \linebreak
		ремен [ремень] \cite[л. 66 об]{spbfaran79}
	& 	
	&	
	& 	
	& 	[32.2об]
		\tabularnewline \midrule
\tenevilglyph[no][4]{O_jXX}
	&	тюленья шкура \cite[л. 48]{spbfaran79} \linebreak
		нерпа \cite[л. 66 об]{spbfaran79}
	& 	тюленья шкура
	&	
	& 	
	& 	\tabularnewline \midrule
\tenevilglyph[no][4]{O_2b}
	&	шкура лахтака \cite[л. 48]{spbfaran79} \linebreak
		лахтак \cite[л. 66 об]{spbfaran79}
	& 	
	&	
	& 	
	& 	\tabularnewline \midrule
\tenevilglyph[no][3]{O_2b_c_zR}
	&	морча [морж] \cite[л. 66 об]{spbfaran79}
	& 	
	&	
	& 	
	& 	\tabularnewline \midrule
\tenevilglyph[yes][3]{2CE}
	&	сахар \cite[л. 44, 49]{spbfaran79}
	& 	
	&	
	& 	
	& 	[25.6] 
		\tabularnewline \midrule
\tenevilglyph[no][3]{I_q} 
	&	трубка \cite[л. 49]{spbfaran79} 
	& 	трубка
	&	a pipe
	& 	
	& 	\tabularnewline \midrule
\tenevilglyph[no][3]{I_q_UE_JX}
	&	папироска \cite[л. 49]{spbfaran79} 
	& 	папироска
	&	
	& 	
	& 	\tabularnewline \midrule
\tenevilglyph[no][3]{I_q_UE_JX_b_q}
	&	трубка с мундштуком \cite[л. 49]{spbfaran79} 
	& 	
	&	
	& 	
	& 	\tabularnewline \midrule
\tenevilglyph[yes][4]{UE_JX} 
	&	
	& 	
	&	
	& 	
	& 	kelekel [kælikæl, кэԓикэԓ = рисунок, картина] [4.10об] \linebreak % кэԓикэԓ
		в «писат» [писать] [32.6] \linebreak
		в «песмо» [письмо] [34.8об] \linebreak
		в «касет» [газета] [34.8об]
		\tabularnewline \midrule
\tenevilglyph[yes][4]{UE_JX_j_q} 
	&	
	& 	
	&	
	& 	
	& 	еенки [деньги] [34.1]
		\tabularnewline \midrule
\tenevilglyph[yes][2]{l_JXE} % знаки на л. 50 и л. 66 зеркальны
	&	не мог* \cite[л. 50]{spbfaran79} \linebreak
		нимнок [?] \cite[л. 66 об]{spbfaran79}
	& 	
	&	
	& 	
	& 	[9.1]
		\tabularnewline \midrule
\tenevilglyph[yes][4]{cF_CF}
	&	так \cite[л. 50]{spbfaran79} \linebreak
		әnmen [энмэн = также, итак] \cite[л. 39 об]{spbfaran79} \linebreak % энмэн
		дак [так] \cite[л. 66 об]{spbfaran79}
	& 	
	&	
	& 	
	& 	\cite[360, 361, 364]{davydova2015a} \linebreak
		\cite[26, 28]{lavrov1969} \linebreak
		етак [так] [36.1]
		\tabularnewline \midrule
\tenevilglyph[yes][4]{o_jX}
	&	так себе, attaw [а'тав = напрасно, зря] \cite[л. 50]{spbfaran79} \linebreak % а'тав
		attaw \cite[л. 52 об]{spbfaran79} \linebreak
		всетакии (нука) (?)  \cite[л. 53]{spbfaran79} 
	& 	так себе
	&	
	& 	
	& 	\cite[361]{davydova2015a} \linebreak
		так [32.6]
		\tabularnewline \midrule % [25.12]
\tenevilglyph[yes][1]{o_qX_f}
	&	
	& 	
	&	
	& 	
	& 	iатаро [?] [34.10]
		\tabularnewline \midrule % [25.12]
\tenevilglyph[yes][2]{i_2l_iSY}
	&	несмотря на то, что* \cite[л. 50]{spbfaran79} % TODO: проверить, не похоже ли это на «толстый»
	& 	
	&	
	& 	
	& 	\cite[360]{davydova2015a} 
		\tabularnewline \midrule
\tenevilglyph[yes][3]{B_2BD}
	&	быть \cite[л. 50]{spbfaran79} 
	& 	
	&	
	& 	
	& 	\cite[364]{davydova2015a} 
		\tabularnewline \midrule
\tenevilglyph[yes][4]{o_l}
	&	мало \cite[л. 50]{spbfaran79} \linebreak
		kitkit [киткит = мало, немного] \cite[л. 39 об]{spbfaran79} % киткит
	& 	
	&	
	& 	
	& 	\cite[360, 361, 364]{davydova2015a} \linebreak
		немноско [немножко] [34.11]
		\tabularnewline \midrule
\tenevilglyph[yes][2]{oI_vD}
	&	вероятно \cite[л. 50]{spbfaran79} \linebreak
		наверно \cite[л. 67]{spbfaran79}
	& 	
	&	
	& 	
	& 	\cite[364]{davydova2015a} \linebreak
		анако [?] [33.4]
		\tabularnewline \midrule
\tenevilglyph[yes][3]{bD_b}
	&	только \cite[л. 50]{spbfaran79} \linebreak
		lien [liæn, ԓыгэн = как только] \cite[л. 52 об, 56]{spbfaran79} % ԓыгэн
	& 	только
	&	
	& 	
	& 	\cite[361, 364]{davydova2015a} \linebreak
		\cite[28]{lavrov1969} 
		\tabularnewline \midrule
\tenevilglyph[yes][2]{u_2k_uN_2k}
	&	спали (ушли?) \cite[л. 50]{spbfaran79}
	& 	
	&	
	& 	
	& 	[25.9] \linebreak
		спас [спать?] [33.4]
		\tabularnewline \midrule
\tenevilglyph[yes][3]{cU_2q_cD_2q}
	&	словно, как бы \cite[л. 50]{spbfaran79} \linebreak
		qajaagьtkь \cite[л. 52 об]{spbfaran79} % НУЖЕН ПЕРЕВОД
	& 	
	&	
	& 	
	& 	\cite[360–362, 364]{davydova2015a} 
		\tabularnewline \midrule
\tenevilglyph[yes][3]{i_oB}
	&	тайно \cite[л. 50]{spbfaran79} \linebreak
		wьnvь [vinvә, винвэ = тайно, крадучись] \cite[л. 56]{spbfaran79} % винвэ
	& 	тайный
	&	
	& 	
	& 	\cite[364]{davydova2015a} \linebreak
		\cite{bogoraz1934}
		\tabularnewline \midrule
\tenevilglyph[yes][4]{c_J}
	&	в поле \cite[л. 50]{spbfaran79} \linebreak
		nutek [нутэк = на земле] \cite[л. 56]{spbfaran79} % нутэк
	& 	в поле
	&	
	& 	земля
	& 	\cite[360]{davydova2015a} \linebreak
		\cite[28]{lavrov1969}
		\tabularnewline \midrule
\tenevilglyph[yes][3]{c_J_2j}
	&	nutesqan [nutæsqәn, нутэсӄын = земля, почва] \cite[л. 39]{spbfaran79} % нутэсӄын
	& 	
	&	
	& 	
	& 	\cite[362, 364]{davydova2015a} \linebreak
		\cite[28]{lavrov1969} 
		\tabularnewline \midrule
\tenevilglyph[yes][3]{i_2bX}
	&	сразу \cite[л. 51]{spbfaran79} \linebreak
		awetuwaq [авэтываӄ = быстро, проворно, сразу] \cite[л. 56]{spbfaran79} % авэтываӄ
	& 	сразу
	&	
	& 	
	& 	[2.1?]* 
		\tabularnewline \midrule
\tenevilglyph[yes][4]{o_m_j}
	&	мама \cite[л. 51, 37]{spbfaran79} \linebreak
		мама \cite[л. 67]{spbfaran79} 
	& 	
	&	
	& 	
	& 	\cite[362]{davydova2015a} \linebreak
		\cite[28]{lavrov1969} \linebreak
		мама [33.5об]
		\tabularnewline \midrule
\tenevilglyph[yes][4]{B_b_oX}
	&	остров Кулючин \cite[л. 51]{spbfaran79} \linebreak
		на Колючено \cite[л. 37]{spbfaran79} 
	& 	
	&	
	& 	
	& 	\cite[360]{davydova2015a} 
		\tabularnewline \midrule
\tenevilglyph[yes][4]{UD_i_2l}
	&	покочевали* \cite[л. 51]{spbfaran79}  % вверх ногами
	& 	
	&	
	& 	кочевать*
	& 	[25.8об] \linebreak
		ялхытык [яԓгытык = кочевать] [34.8] % яԓгытык
		\tabularnewline \midrule
\tenevilglyph[yes][3]{i_2iY}
	&	вверх \cite[л. 51]{spbfaran79} 
	& 	вверх
	&	
	& 	
	& 	\cite[361]{davydova2015a} 
		\tabularnewline \midrule
\tenevilglyph[yes][3]{u_v_cD}
	&	вполне \cite[л. 51]{spbfaran79} \linebreak
		arala [аръаԓя = совсем, вовсе] \cite[л. 52]{spbfaran79} % аръаԓя
	& 	
	&	
	& 	
	& 	\cite[361, 364]{davydova2015a} \linebreak
		\cite[28]{lavrov1969} 
		\tabularnewline \midrule
\tenevilglyph[yes][4]{cF-cF}
	&	тоже, опять \cite[л. 51]{spbfaran79} \linebreak
		опять \cite[л. 53]{spbfaran79} 
	& 	тоже, опять
	&	
	& 	
	& 	\cite[361, 362]{davydova2015a} \linebreak
		апес [опять] [33.4]
		\tabularnewline \midrule
\tenevilglyph[yes][4]{oF_2l_lG}
	&	близко \cite[л. 51, 53]{spbfaran79} \linebreak
		ꞓьmꞓә [cьmcь, чымче = близко] \cite[л. 54]{spbfaran79} \linebreak % чымче
		плиско [близко] \cite[л. 68 об]{spbfaran79}
	& 	
	&	
	& 	
	& 	\cite[364]{davydova2015a} \linebreak 
		\cite{bogoraz1934} 
		\tabularnewline \midrule
\tenevilglyph[yes][3]{cU_2cD}
	&	после того, әŋqorә \cite[л. 51, 53]{spbfaran79} \linebreak
		әnqre \cite[л. 39]{spbfaran79} 
	& 	после того
	&	
	& 	
	& 	\cite[361, 362, 364]{davydova2015a} \linebreak
		\cite[28]{lavrov1969} 
		\tabularnewline \midrule
\tenevilglyph[yes][4]{o_2CE}
	&	milgьrә [milgьr, миԓгэр = ружье] \cite[л. 54]{spbfaran79} \linebreak % миԓгэр
		русёь [ружье] \cite[л. 68 об]{spbfaran79}
	& 	
	&	
	& 	
	& 	\cite[360, 364]{davydova2015a} \linebreak
		\cite[28]{lavrov1969} 
		\tabularnewline \midrule
\tenevilglyph[yes][4]{o_2q}
	&	1 \cite[л. 64]{spbfaran79} \linebreak
		amunen [әnnæn?, ыннэн? = один] \cite[л. 39 об]{spbfaran79} % ыннэн
	& 	
	&	
	& 	1
	& 	1 \cite[360, 362]{davydova2015a} \linebreak
		\cite[361, 364]{davydova2015a} \linebreak
		\cite[26]{lavrov1969} 
		\tabularnewline \midrule
\tenevilglyph[yes][4]{o_2q_j}
	&	20 \cite[л. 64]{spbfaran79} 
	& 	
	&	
	& 	20
	& 	20 \cite[360, 362]{davydova2015a} \linebreak
		\cite[361, 363]{davydova2015a} \linebreak
		\cite[26]{lavrov1969}
		\tabularnewline \midrule
\tenevilglyph[yes][4]{i_b_s_j_o_2q}
	&	
	& 	
	&	
	& 	1000*
	& 	1000 [25.1об] 
		\tabularnewline \midrule
\tenevilglyph[yes][4]{i_b_s_j_o_q_j}
	&	
	& 	
	&	
	& 	
	& 	20000 [36.2] \tabularnewline \midrule
\tenevilglyph[yes][4]{B-}
	&	2 \cite[л. 64]{spbfaran79} \linebreak
		двоих \cite[л. 68]{spbfaran79}
	& 	
	&	
	& 	2
	& 	2 \cite[360, 362]{davydova2015a} \linebreak
		\cite[361, 363, 364]{davydova2015a} \linebreak
		\cite[28]{lavrov1969} 
		\tabularnewline \midrule
\tenevilglyph[yes][4]{B-_j}
	&	
	& 	
	&	
	& 	40
	& 	40 \cite[360]{davydova2015a} 
		\tabularnewline \midrule
\tenevilglyph[yes][4]{B-_2oI_jF_j}
	&	
	& 	
	&	
	& 	400
	& 	[25.2] 
		\tabularnewline \midrule
\tenevilglyph[yes][4]{i_b_s_j_B-}
	&	
	& 	
	&	
	& 	
	& 	2000 [36.2] 
		\tabularnewline \midrule
\tenevilglyph[yes][4]{o_2q_q_l}
	&	три \cite[л. 41]{spbfaran79} \linebreak
		ŋьroq [ӈыроӄ = три] \cite[л. 39]{spbfaran79} \linebreak % ӈыроӄ
		3 \cite[л. 64]{spbfaran79}
	& 	
	&	
	& 	3
	& 	3 \cite[360, 362]{davydova2015a} \linebreak
		\cite[361, 363, 364]{davydova2015a} 
		\tabularnewline \midrule
\tenevilglyph[yes][4]{o_2q_q_l_j}
	&	
	& 	
	&	
	& 	60
	& 	60 \cite[360]{davydova2015a} \linebreak
		\cite[26]{lavrov1969} 
		\tabularnewline \midrule
\tenevilglyph[yes][3]{o_q_q_l_2oI_jF_j}
	&	
	& 	
	&	
	& 	600
	& 	[11.4об]
		\tabularnewline \midrule
\tenevilglyph[yes][3]{i_b_s_j_o_q_q_l}
	&	
	& 	
	&	
	& 	
	& 	[3000] [32.13об] 
		\tabularnewline \midrule
\tenevilglyph[yes][4]{o_q_c_T}
	&	4 \cite[л. 64]{spbfaran79}
	& 	
	&	
	& 	4
	& 	4 \cite[360]{davydova2015a} \linebreak
		\cite[361]{davydova2015a} \linebreak
		\cite[26]{lavrov1969} 
		\tabularnewline \midrule
\tenevilglyph[yes][3]{o_q_c_T_j}
	&	
	& 	
	&	
	& 	80
	& 	[25.4]
		\tabularnewline \midrule
\tenevilglyph[yes][3]{o_c_T_2oI_jF_j}
	&	
	& 	
	&	
	& 	800
	& 	[25.4] 
		\tabularnewline \midrule
\tenevilglyph[yes][4]{oI_2j}
	&	5 \cite[л. 64]{spbfaran79}
	& 	
	&	
	& 	5
	& 	5 \cite[360]{davydova2015a} \linebreak
		\cite[361, 364]{davydova2015a} 
		\tabularnewline \midrule
\tenevilglyph[no][3]{i_b_s_j_oI_2j}
	&	
	& 	
	&	
	& 	5000
	& 	\tabularnewline \midrule
\tenevilglyph[yes][4]{oI_3j}
	&	
	& 	
	&	
	& 	100
	& 	\cite[361]{davydova2015a} \linebreak
		100 [34.19]
		\tabularnewline \midrule
\tenevilglyph[yes][4]{o-_q_jF_o}
	&	6 \cite[л. 64]{spbfaran79}
	& 	
	&	
	& 	6
	& 	6 \cite[360]{davydova2015a}
		\tabularnewline \midrule
\tenevilglyph[yes][4]{o-_q_jF_o_j}
	&	
	& 	
	&	
	& 	
	& 	120 [34.20об]
		\tabularnewline \midrule
\tenevilglyph[yes][4]{o_j_2q}
	&	7 \cite[л. 64]{spbfaran79}
	& 	
	&	
	& 	7
	& 	7 \cite[360]{davydova2015a} \linebreak
		\cite[361]{davydova2015a}
		\tabularnewline \midrule
\tenevilglyph[yes][4]{o_j_2q_j}
	&	
	& 	
	&	
	& 	
	& 	140 [2.101] 
		\tabularnewline \midrule
\tenevilglyph[yes][4]{o-_2q_j}
	&	8 \cite[л. 64]{spbfaran79}
	& 	
	&	
	& 	8
	& 	8 \cite[360]{davydova2015a} 
		\tabularnewline \midrule
\tenevilglyph[yes][4]{o-_2q_j_j}
	&	
	& 	
	&	
	& 	
	& 	в «161» [160] [32.15] 
		\tabularnewline \midrule
\tenevilglyph[yes][4]{o_2q_jN_jF_o}
	&	9 \cite[л. 64]{spbfaran79}
	& 	
	&	
	& 	9
	& 	9 \cite[360]{davydova2015a} 
		\tabularnewline \midrule
\tenevilglyph[yes][3]{o_2q_jN_jF_o_j}
	&	
	& 	
	&	
	& 	
	& 	[180] [25.3об] 
		\tabularnewline \midrule
\tenevilglyph[yes][4]{2oI_2jF}
	&	10 \cite[л. 64]{spbfaran79}
	& 	
	&	
	& 	10
	& 	10 \cite[360]{davydova2015a} \linebreak
		\cite[361, 363]{davydova2015a} \linebreak
		\cite[26]{lavrov1969} 
		\tabularnewline \midrule
\tenevilglyph[yes][3]{2oI_2jF_j}
	&	
	& 	
	&	
	& 	200
	& 	[25.3об] 
		\tabularnewline \midrule
\tenevilglyph[yes][4]{o_T_2q_2o_l}
	&	
	& 	
	&	
	& 	15
	& 	15 \cite[360]{davydova2015a} \linebreak 
		\cite[361]{davydova2015a} 
		\tabularnewline \midrule
\tenevilglyph[yes][4]{o_T_2q_2o_l_j} 
	&	
	& 	
	&	
	& 	
	& 	[300] \cite[26]{lavrov1969} \linebreak % Как 15, но с крючком, обычно обозначающим 20. Кроме того, на «лунном календаре», в контексте, видимо, числа, где есть этот знак, 60, 4 и 1, то есть если это в самом деле 15*20=300, то сумма будет 365
		в 301, 302, 303 и т. д. [300] [4.9об]
		\tabularnewline \midrule
\tenevilglyph[yes][2]{CD_CDY}
	&	тем не менее, wenlьgь [vænligi, вэнԓыги = тем не менее] \cite[л. 42]{spbfaran79} \linebreak % вэнԓыги
		wenlьgь [vænligi] \cite[л. 52 об]{spbfaran79} \linebreak
		силно [сильно?] \cite[л. 66 об]{spbfaran79} 
	& 	
	&	
	& 	
	& 	\cite{bogoraz1934} 
		\tabularnewline \midrule
\tenevilglyph[yes][4]{UD_2c}
	&	мэсяч [месяц] \cite[л. 66]{spbfaran79} 
	& 	
	&	
	& 	
	& 	\cite[362]{davydova2015a} \linebreak
		\cite[26, 28]{lavrov1969} \linebreak
		мэсыс [месяц] [34.19]
		\tabularnewline \midrule
\tenevilglyph[yes][3]{o_8q}
	&	
	& 	
	&	the sun
	& 	
	& 	[25.8об]
		\tabularnewline \midrule
\tenevilglyph[yes][4]{o_7q_Q}
	&	сонсо [солнце] \cite[л. 66]{spbfaran79} 
	& 	
	&	
	& 	солнце
	& 	\cite[361, 364]{davydova2015a}
		еен [день] [34.11об, 34.16об]
		\tabularnewline \midrule
\tenevilglyph[yes][4]{o_7q_L}
	&	
	& 	
	&	
	& 	
	& 	утырым [утром] [34.19об]
		\tabularnewline \midrule
\tenevilglyph[yes][3]{rI_l_b}
	&	топор \cite[л. 68 об]{spbfaran79} 
	& 	
	&	
	& 	
	& 	\cite[364]{davydova2015a} 
		\tabularnewline \midrule
\tenevilglyph[yes][3]{c_c_2k}
	&	лодка \cite[л. 68 об]{spbfaran79} 
	& 	
	&	
	& 	
	& 	\cite[361]{davydova2015a} 
		\tabularnewline \midrule
\tenevilglyph[yes][3]{i_2l}
	&	долстои [толстый] \cite[л. 69 об]{spbfaran79} 
	& 	
	&	
	& 	
	& 	\cite[364]{davydova2015a} \linebreak
		\cite[28]{lavrov1969} 
		\tabularnewline \midrule
\tenevilglyph[yes][3]{i_2j_l}
	&	донкои [тонкий] \cite[л. 69 об]{spbfaran79} 
	& 	
	&	
	& 	
	& 	[25.4] 
		\tabularnewline \midrule
\tenevilglyph[yes][1]{i_2c}
	&	курба [?] \cite[л. 68 об]{spbfaran79} 
	& 	
	&	
	& 	
	& 	\cite[361, 364]{davydova2015a} 
		\tabularnewline \midrule
\tenevilglyph[yes][3]{u_2l}
	&	ŋagcьnьn \cite[л. 64 об]{spbfaran79} \linebreak 
		в собки [в сопки] \cite[л. 68 об]{spbfaran79}
	& 	
	&	
	& 	
	& 	\cite[361]{davydova2015a} 
		\tabularnewline \midrule
\tenevilglyph[yes][1]{i_jX}
	&	 ?...gite...* \cite[л. 39 об]{spbfaran79}  % НУЖНА РАСШИФРОВКА, НУЖЕН ПЕРЕВОД
	& 	
	&	
	& 	
	& 	\cite[360, 362, 364]{davydova2015a} 
		\tabularnewline \midrule
\tenevilglyph[no][1]{i_jX_o}
	&	 mьgitegәn [?mьgitægәn = пусть я на него посмотрю] \cite[л. 64 об]{spbfaran79} 
	& 	
	&	
	& 	
	& 	\tabularnewline \midrule
\tenevilglyph[yes][3]{i_jX_z}
	&	ime renut [imь-rәnut, имыръэнут = что угодно] \cite[л. 51]{spbfaran79} % имыръэнут
	& 	
	&	
	& 	
	& 	\cite[364]{davydova2015a} 
		\tabularnewline \midrule
\tenevilglyph[yes][3]{U_qD}
	&	камусы \cite[л. 37]{spbfaran79} 
	& 	
	&	
	& 	
	& 	\cite[362, 364]{davydova2015a} 
		\tabularnewline \midrule
\tenevilglyph[yes][3]{U_qD_b}
	&	рукавицы \cite[л. 37]{spbfaran79} 
	& 	
	&	
	& 	
	& 	\cite[362]{davydova2015a} 
		\tabularnewline \midrule
\tenevilglyph[yes][4]{sE}
	&	юукула [юкола]* \cite[л. 68 об]{spbfaran79} 
	& 	
	&	
	& 	юкола сушеная рыба
	& 	\cite[361]{davydova2015a} 
		\tabularnewline \midrule
\tenevilglyph[yes][3]{sE_jFE}
	&	всала [взяла] \cite[л. 68 об]{spbfaran79} \linebreak
		в «я \textit{возьму»} \cite[л. 66]{spbfaran79}
	& 	
	&	
	& 	
	& 	\cite[360]{davydova2015a} 
		\tabularnewline \midrule
\tenevilglyph[yes][4]{sE_jFE_qY}
	&	
	& 	
	&	
	& 	
	& 	в «ырыпота» [работа] [35.1] \linebreak
		в «рыпосе» [работе] [34.18об]
		\tabularnewline \midrule
\tenevilglyph[yes][3]{w_j}
	&	оцин [очень] \cite[л. 66]{spbfaran79} \linebreak
		в «я \textit{оцин} боюс», «я \textit{оцин} писпокоюс» \cite[л.66]{spbfaran79}
	& 	
	&	
	& 	
	& 	\cite[364]{davydova2015a} 
		\tabularnewline \midrule
\tenevilglyph[yes][3]{BR}
	&	вместе \cite[л. 55]{spbfaran79} 
	& 	
	&	
	& 	
	& 	\cite[360, 364]{davydova2015a}
		\tabularnewline \midrule
\tenevilglyph[yes][1]{SFE_jF}
	&	нарта [?] \cite[л. 68]{spbfaran79} 
	& 	
	&	
	& 	
	& 	\cite[360, 361, 364]{davydova2015a}
		\tabularnewline \midrule
\tenevilglyph[yes][3]{O_L_qE}
	&	доз [дождь] \cite[л. 68]{spbfaran79} 
	& 	
	&	
	& 	
	& 	\cite[361, 364]{davydova2015a}
		\tabularnewline \midrule
\tenevilglyph[yes][3]{O_L_2q}
	&	холот [холод] \cite[л. 66]{spbfaran79} 
	& 	
	&	
	& 	холодный ветер (в~тексте)
	& 	 \cite[26]{lavrov1969} 
		\tabularnewline \midrule
\tenevilglyph[no][3]{O_L}
	&	бурка [пурга] \cite[л. 68 об]{spbfaran79} 
	& 	
	&	
	& 	
	& 	 \tabularnewline \midrule
\tenevilglyph[yes][3]{i_SX}
	&	alьmь [аԓымы? = положим, что] \cite[л. 52 об]{spbfaran79} % аԓымы?
	& 	
	&	
	& 	
	& 	\cite[361, 364]{davydova2015a}
		\tabularnewline \midrule
\tenevilglyph[yes][4]{2C_2c} 
	&	
	& 	
	&	
	& 	вода
	& 	\cite[364]{davydova2015a} \linebreak 
		\cite[26, 28]{lavrov1969} \linebreak
		вотой [водой] [32.15об]
		\tabularnewline \midrule
\tenevilglyph[yes][3]{2kU_2QY} 
	&	
	& 	
	&	
	& 	снег
	& 	\cite[361, 364]{davydova2015a} 
		\tabularnewline \midrule
\tenevilglyph[yes][3]{U_ux} 
	&	витал [видал] \cite[л. 67 об, 68 об]{spbfaran79}
	& 	
	&	
	& 	
	& 	\cite[360, 364]{davydova2015a} 
		\tabularnewline \midrule
\tenevilglyph[no][3]{U_ux_j} 
	&	нивидал [не видал] \cite[л. 66 об]{spbfaran79}
	& 	
	&	
	& 	
	& 	\tabularnewline \midrule
\tenevilglyph[yes][3]{V_2l_i_2q} 
	&	крепкои [крепкий] \cite[л. 69 об]{spbfaran79}
	& 	
	&	
	& 	
	& 	\cite[28]{lavrov1969} 
		\tabularnewline \midrule
\tenevilglyph[no][3]{V_l_lU_i_q_qU} 
	&	нирепкои [некрепкий] \cite[л. 69 об]{spbfaran79}
	& 	
	&	
	& 	
	& 	\tabularnewline \midrule
\tenevilglyph[yes][4]{v_i_2CX} 
	&	
	& 	
	&	
	& 	приходить, приезжать
	& 	\cite[360]{davydova2015a}\linebreak
		\cite[26]{lavrov1969}\linebreak
		прлехалй [приехали] [32.13об]
		\tabularnewline \midrule
\tenevilglyph[yes][4]{i_i_bX} 
	&	ŋocьm [ŋocьn, ӈъочьын = бедняк] \cite[л. 39 об]{spbfaran79} % ӈъочьын
	& 	богатый
	&	
	& 	
	& 	петнаска [бедняжка] [34.8об]
		\tabularnewline \midrule
\tenevilglyph[no][2]{oEN_q} 
	&	goymьcьl(?) [gajmьcьjьn, гаймычьыԓьын = богач] \cite[л. 39 об]{spbfaran79} % гаймычьыԓьын
	& 	бедный
	&	
	& 	
	& 	\tabularnewline \midrule
\tenevilglyph[yes][3]{2i_2iX_4q} 
	&	прсол [пришел] \cite[л. 68 об]{spbfaran79}
	& 	
	&	
	& 	
	& 	\cite[361]{davydova2015a} 
		\tabularnewline \midrule
\tenevilglyph[yes][3]{2i_iX_2q_cF_jF} 
	&	прныси [принеси] \cite[л. 68 об]{spbfaran79}
	& 	
	&	
	& 	
	& 	[4.3об] 
		\tabularnewline \midrule
\tenevilglyph[yes][1]{i_CD_2jF} 
	&	длко [?] \cite[л. 68]{spbfaran79}
	& 	
	&	
	& 	
	& 	\cite[364]{davydova2015a} \linebreak
		толко [?] [34.11об]
		\tabularnewline \midrule
\tenevilglyph[yes][1]{uD_jN} 
	&	кус [?] \cite[л. 66]{spbfaran79}
	& 	
	&	
	& 	
	& 	\cite[28]{lavrov1969} 
		\tabularnewline \midrule
\tenevilglyph[yes][4]{i_u_uD_b} 
	&	grep [græp, грэп = песня] \cite[л. 64 об]{spbfaran79} % грэп
	& 	
	&	
	& 	
	& 	поеот* [поёт] [36.1]
		\tabularnewline \midrule
\tenevilglyph[yes][4]{i_u_uD_k_r} 
	&	
	& 	
	&	
	& 	
	& 	кармоска [гармошка] [36.1]
		\tabularnewline \midrule
\tenevilglyph[yes][3]{oF_oN_z} 
	&	колова [голова] \cite[л. 68]{spbfaran79}
	& 	
	&	
	& 	
	& 	\cite[364]{davydova2015a} 
		\tabularnewline \midrule
\tenevilglyph[yes][3]{o_jN_m_z} 
	&	домои [домой] \cite[л. 66 об]{spbfaran79}
	& 	
	&	
	& 	
	& 	\cite[363]{davydova2015a} 
		\tabularnewline \midrule
\tenevilglyph[yes][3]{iE_b_i} 
	&	мало \cite[л. 67]{spbfaran79}
	& 	
	&	
	& 	
	& 	\cite[361]{davydova2015a} 
		\tabularnewline \midrule
\tenevilglyph[yes][3]{j_b_q} 
	&	посла [пошла] \cite[л. 66]{spbfaran79}
	& 	
	&	
	& 	
	& 	\cite[360]{davydova2015a} 
		\tabularnewline \midrule
\tenevilglyph[yes][3]{j_b_q_2q} 
	&	слы [шли] \cite[л. 68]{spbfaran79} \linebreak
		пёс [вёз] \cite[л. 66 об]{spbfaran79}
	&
	&	
	& 	
	& 	\cite[360]{davydova2015a} 
		\tabularnewline \midrule
\tenevilglyph[yes][3]{i_2j_2cY} 
	&	
	& 	
	&	
	& 	орел*
	& 	\cite[28]{lavrov1969} 
		\tabularnewline \midrule
\tenevilglyph[yes][3]{C-C_q_j} 
	&	
	& 	
	&	
	& 	ворон
	& 	[25.13] 
		\tabularnewline \midrule
\tenevilglyph[yes][2]{CD-CDX} 
	&	недавно* \cite[л. 50]{spbfaran79} \linebreak % значок только умеренно похож
		цирас [?] \cite[л. 67 об]{spbfaran79} \linebreak
		в «провсерас» [?] \cite[л. 67 об]{spbfaran79}
	& 	
	&	
	& 	
	& 	[25.4об] 
		\tabularnewline \midrule
\tenevilglyph[yes][1]{CD-CDX_l} 
	&	ониметнис [?] \cite[л. 66 об]{spbfaran79}
	& 	
	&	
	& 	
	& 	\cite[364]{davydova2015a} 
		\tabularnewline \midrule
\tenevilglyph[yes][3]{CD-CDX_2q} 
	&	прослокот [прошлый год] \cite[л. 66 об]{spbfaran79}
	& 	
	&	
	& 	
	& 	[25.4] 
		\tabularnewline \midrule
\tenevilglyph[yes][4]{CD-CDX_q_2b_c} 
	&	
	& 	
	&	
	& 	
	& 	кота [когда] [37.2] 
		\tabularnewline \midrule
\tenevilglyph[yes][2]{i_b_qY} 
	&	нусно [нужно] \cite[л. 66]{spbfaran79} \linebreak
		в «понравилас» [?] \cite[л. 66]{spbfaran79}
	& 	
	&	
	& 	
	& 	[25.7] 
		\tabularnewline \midrule
\tenevilglyph[yes][1]{3k} 
	&	в «понравилас» [?] \cite[л. 66]{spbfaran79}
	& 	
	&	
	& 	
	& 	\cite[364]{davydova2015a} 
		\tabularnewline \midrule
\tenevilglyph[yes][1]{i_j_3b} 
	&	паска патро [?] \cite[л. 68 об]{spbfaran79}
	& 	
	&	
	& 	
	& 	\cite[364]{davydova2015a} 
		\tabularnewline \midrule
\tenevilglyph[yes][2]{u_q_l} 
	&	талок [далеко] \cite[л. 68 об]{spbfaran79}
	& 	
	&	
	& 	
	& 	\cite[360, 364]{davydova2015a} \linebreak
		\cite[28]{lavrov1969} 
		\tabularnewline \midrule
\tenevilglyph[yes][2]{2cD_jY} 
	&	илплыл* [?] \cite[л. 68]{spbfaran79} % зеркально
	& 	
	&	
	& 	вьюга (в~тексте)
	& 	\cite[361]{davydova2015a} \linebreak
		\cite[26]{lavrov1969} 
		\tabularnewline \midrule
\tenevilglyph[yes][3]{u_2j} 
	&	прошол [прошел] \cite[л. 66 об]{spbfaran79} % зеркально
	& 	
	&	
	& 	
	& 	[25.4] 
		\tabularnewline \midrule
\tenevilglyph[yes][3]{c_C_2j} 
	&	собака \cite[л. 68 об]{spbfaran79}
	& 	
	&	
	& 	
	& 	[25.3] 
		\tabularnewline \midrule
\tenevilglyph[yes][4]{k_2j} 
	&	щаи [чай] \cite[л. 68 об]{spbfaran79}
	& 	
	&	
	& 	
	& 	[25.9] \linebreak
		в «чаеопат» [\textit{«чай} пить» или чайпат = вскипевший чай] [32.16об] % чайпат?
		\tabularnewline \midrule
\tenevilglyph[yes][1]{2LE} 
	&	полноь [?] \cite[л. 66 об]{spbfaran79}
	& 	
	&	
	& 	
	& 	[25.4об] \linebreak
		полнои [?] [34.11]
		\tabularnewline \midrule
\tenevilglyph[yes][3]{uD_z} 
	&	рука \cite[л. 68]{spbfaran79}
	& 	
	&	
	& 	
	& 	[25.13] 
		\tabularnewline \midrule
\tenevilglyph[yes][3]{o-o_z} 
	&	клас [глаз] \cite[л. 68]{spbfaran79}
	& 	
	&	
	& 	
	& 	[6.1] 
		\tabularnewline \midrule
\tenevilglyph[yes][3]{l_i} 
	&	нос \cite[л. 68]{spbfaran79}
	& 	
	&	
	& 	
	& 	[25.15об] 
		\tabularnewline \midrule
\tenevilglyph[yes][1]{2c_2bX} 
	&	в «провсерас» \cite[л. 67 об]{spbfaran79}
	& 	
	&	
	& 	
	& 	[25.7] 
		\tabularnewline \midrule
\tenevilglyph[yes][4]{o_2q_2j} 
	&	послы [пошли (от «слать»)] \cite[л. 68 об]{spbfaran79}
	& 	
	&	
	& 	
	& 	[25.10] \linebreak
		посевыке [посылке] [34.8об] \linebreak
		в «песмо» [письмо] [34.8об] 
		\tabularnewline \midrule
\tenevilglyph[yes][1]{o-o-o} 
	&	
	& 	
	&	
	& 	
	& 	осенако [?] [34.12] 
		\tabularnewline \midrule
\tenevilglyph[yes][4]{vD_2qY} 
	&	aagek [aacek, а'ачек = молодой человек] \cite[л. 65 об]{spbfaran79} % а'ачек
	& 	
	&	
	& 	
	& 	[4.2?] \linebreak
		в «молотои» [молодой] [34.11об]
		\tabularnewline \midrule
\tenevilglyph[yes][3]{2o_2jY} 
	&	в «я упрала» [«я убрала»] \cite[л. 67]{spbfaran79}
	& 	
	&	
	& 	
	& 	[4.2?] 
		\tabularnewline \midrule
\tenevilglyph[yes][4]{CD_jFN} 
	&	кова [?] \cite[л. 66]{spbfaran79} \linebreak
		ковта [?] \cite[л. 66]{spbfaran79}
	& 	
	&	
	& 	
	& 	[4.2?] \linebreak
		какта [как-то?] [34.11]
		\tabularnewline \midrule
\tenevilglyph[yes][4]{i_b_jX} 
	&	
	& 	
	&	
	& 	
	& 	\cite[363]{davydova2015a} \linebreak
		вот [32.6] \linebreak
		ŋotьqen [ŋotьnqәn = вон тут] [4.10об] % НУЖЕН ПЕРЕВОД ӈотыӈкы? ӈотыӈӄэн?
		\tabularnewline \midrule
\tenevilglyph[yes][4]{2b_2l} 
	&	
	& 	
	&	
	& 	
	& 	стотакои [что такое] [35.6]
		\tabularnewline \midrule
\tenevilglyph[yes][4]{G_t} 
	&	
	& 	
	&	
	& 	
	& 	теыеене [тэгйиӈ = кашель, грипп] [34.8] \linebreak % тэгйиӈ
		касли [кашель] [34.11]
		\tabularnewline \midrule
\tenevilglyph[yes][4]{r_t} 
	&	
	& 	
	&	
	& 	
	& 	пыраснек [праздник] [34.10об] \linebreak
		в «захтре перхоймая» [завтра первое мая] [39.7об]
		\tabularnewline \midrule
\tenevilglyph[yes][4]{i_b_JX} 
	&	
	& 	
	&	
	& 	
	& 	\cite[360]{davydova2015a} \linebreak
		сахтре [завтра] [34.18об] \linebreak
		в «захтре перхоймая» [завтра первое мая] [39.7об]
		\tabularnewline \midrule
\tenevilglyph[yes][4]{U2E} 
	&	
	& 	
	&	
	& 	учиться
	& 	усеся [учиться] [34.12] \linebreak
		в «усеся» [учиться] [34.11об] 
		\tabularnewline \midrule
\tenevilglyph[yes][4]{cD_2k} 
	&	
	& 	
	&	
	& 	
	& 	там [33.4, 34.1] 
		\tabularnewline \midrule
\tenevilglyph[yes][3]{i_qY_vD} 
	&	тапак [табак] \cite[л. 68 об.]{spbfaran79}
	& 	
	&	
	& 	
	& 	[4.1об]
		\tabularnewline \midrule
\tenevilglyph[yes][4]{c_q_cD_q} 
	&	
	& 	
	&	
	& 	
	& 	осын [очень] [37.2]
		\tabularnewline \midrule
\tenevilglyph[yes][4]{UD_uDE} 
	&	
	& 	
	&	
	& 	
	& 	натваре [на дворе] [34.11об]
		\tabularnewline \midrule
\tenevilglyph[yes][4]{q_c_cD_q} 
	&	
	& 	
	&	
	& 	
	& 	сороно [всё равно] [34.11об]
		\tabularnewline \midrule
\tenevilglyph[yes][1]{O_JX_b} 
	&	
	& 	
	&	
	& 	
	& 	насоеас [?] [33.4]
		\tabularnewline \midrule
\tenevilglyph[yes][1]{3iX} 
	&	
	& 	
	&	
	& 	
	& 	месе [?] [34.21об]
		\tabularnewline \midrule
\tenevilglyph[yes][4]{k_j_jF} 
	&	
	& 	
	&	
	& 	
	& 	нехосыт [не хочет] [37.2, 37.2об]
		\tabularnewline \midrule
\tenevilglyph[yes][4]{i_2q_l_q_i_L} 
	&	
	& 	
	&	
	& 	
	& 	ролтырхын [руԓтыркын = сторониться] [34.10] % руԓтыркын
		\tabularnewline \midrule
\tenevilglyph[yes][4]{o_2q_l} 
	&	
	& 	
	&	
	& 	
	& 	iесо [еще] [32.1] \linebreak
		iечо [еще] [32.15об] \linebreak
		iезо [еще] [39.1об]
		\tabularnewline \midrule
\bottomrule
\end{longtable}
\end{landscape}

\section{Описания источников} 

\subsection{СПбФ АРАН. Ф. 250. Оп. 1. Д. 79}

\subsubsection{Листы 40–51}

Сводная таблица с символами и переводами на русский и иногда чукотский (латиницей).

\subsubsection{Листы 39 и об., 52 и об., 54, 56}

Листы с символами и переводами на чукотский (латиницей).

\subsubsection{Листы 37, 52, 53, 54 об, 55}

Листы со словами на русском и отдельными символами.

\subsubsection{Листы 64, 65 с оборотами}

Листы с символами и переводами на чукотский (латиницей) и русский. На л. 64 приведены числительные от 1 до 10, плюс 19 и 20.

\subsubsection{Листы 66–69 с оборотами}

Листы из тетради с символами и переводами на русский и, местами, чукотский (кириллицей). Почерк неуверенный, плохая орфография.

\subsubsection{Остальные листы}

Символов не содержат, кроме 38 об., где шесть символов без переводов, но с комментариями.

\subsection{Статья В. Г. Богораз-Тана «Луораветланский (чукотский) язык»}

Таблица с переводом ряда символов на русский. Изображение (перерисовка) одной из табличек Теневиля с переводом одной из трех строк на ней на русский язык.

\subsection{Статья А. М. Миндалевича Hieroglyphic Characters of the Chuckchees}

Переводы на английский ряда отдельных символов. Перевод на английский двух фрагментов записей Теневиля.

\subsection{Статья И. П. Лаврова «Чукотский феномен»}

Три таблицы с переводами ряда символов на русский. Изображения символов сильно стилизованы, поэтому их затруднительно использовать в отсутствие других источников. Фотографии «лунного календаря» и «родового дерева» , нарисованных Теневилем.

\subsection{Диссертация Е. А. Давыдовой «Властные отношения в семейно-родственных коллективах оленных чукчей»}

Пять фотографий «дневниковых записей» Теневиля из архива отдела этнографии Сибири МАЭ РАН.

\printbibliography

\end{document}