\documentclass{article}
\usepackage[a4paper,margin=1.5cm]{geometry}
%\usepackage{lua-visual-debug}
\usepackage{luaotfload,luacode}
\usepackage[hidelinks]{hyperref}
\usepackage{pdflscape}
\usepackage{xcolor}
\usepackage{longtable}
\usepackage{booktabs}
\usepackage{array}
\usepackage{graphicx}
\usepackage{fontspec}
\usepackage[utf8]{inputenc}
\usepackage[russian]{babel}
\usepackage{tipa}
\usepackage[style=russian]{csquotes}

%\usepackage[datamodel=archive,backend=biber,style=gost-numeric]{biblatex}
\usepackage[datamodel=archive,backend=biber]{biblatex}

\setmainfont{Noto Serif}

\newfontfamily\tenevilfont[Renderer = HarfBuzz]{tenevil-font.otf}

\begin{luacode}
 documentdata = documentdata or { }

 local stringformat = string.format
 local texsprint = tex.sprint
 local slot_of_name = luaotfload.aux.slot_of_name

 documentdata.fontchar = function (chr)
 local chr = slot_of_name(font.current(), chr, false)
 if chr and type(chr) == "number" then
 texsprint
 (stringformat ([[\char"%X]], chr))
 end
 end
\end{luacode}

\def\fontchar#1{\directlua{documentdata.fontchar "#1"}}

\newcounter{glyph}
\setcounter{glyph}{1}

\DeclareDocumentCommand{\tenevilglyph}{o o m}{%
\def\tmpyes{yes}
\def\tmpone{#1}
\theglyph~{\IfNoValueTF{#2}{}{(#2)}}\hfill~\linebreak{%
	\IfNoValueTF{#1}{}{%
		\ifx\tmpone\tmpyes%
			%
		\else%
			\color{gray}%
		\fi%
		}
		\tenevilfont\fontsize{40pt}{40pt}\selectfont\fontchar{#3}
		}%
\stepcounter{glyph}%
}

%
%	первый опциональный аргумент — найден ли этот знак, написанный рукой Теневиля, yes/no
%
%	второй — уверенность в интерпретации: см. пояснение под таблицей
%		
%

\addbibresource{tenevil.bib}

\input bibliography-macros.tex

\begin{document}
\begin{landscape}
\begin{longtable}{p{1.25cm}>{\raggedright}p{8cm}>{\raggedright}p{4cm}>{\raggedright}p{4cm}>{\raggedright}p{8cm}}
\toprule
 	& 	СПбФ АРАН \cite{spbfaran79} 
 	& 	Опубликованные интерпретации \cite{bogoraz1934,mindalevich1934,lavrov1969} 
 	&	Картотека И. П. Лаврова
	& 	В записях самого Теневиля \cite{davydova2015a,lavrov1969,bogoraz1934} 
		\tabularnewline \midrule
\tenevilglyph[yes][4]{i_2cU_2cD}
	&	qlaul [ӄԓявыԓ = мужчина] \cite[л. 64 об.]{spbfaran79} % ӄԓявыԓ
	&	мужчина \cite{lavrov1969}
	&	клявыл [ӄԓявыԓ], мужчина [126]
	&	[38.1]
		\tabularnewline \midrule
\tenevilglyph[yes][2]{i_2cU_2cD_'}
	&	отец \cite[л. 40, 55]{spbfaran79} \linebreak
		әtlьgьn [ытԓыгын = отец] \cite[л. 52]{spbfaran79}\linebreak % ытԓыгын
		әtlьgә \cite[л. 52]{spbfaran79}\linebreak
		etlьgьn [әtlьgьn] \cite[л. 52 об.]{spbfaran79}\linebreak
		ьnpenacgen [ынпыначгын = старик] \cite[л. 64]{spbfaran79} % ынпыначгын
	& 	отец \cite{bogoraz1934}
	&	ынпыначгын, старик [126]
	&	\cite[360, 364]{davydova2015a} \linebreak
		әtlьgьn [ытԓыгын; напечатано в книге, знак рядом] [12.17] \linebreak
		старек [старик]* [34.12об] \linebreak % черточка тут справа, но не очень понятно, значит ли это что-то
		pelqэrkьt [ИЛИ:1.17] % TODO: нужен перевод
		\tabularnewline \midrule
\tenevilglyph[yes][3]{i_2cU_j_2cD}
	&	uwaqug [uwæquc, ы'вэӄуч = муж] \cite[л. 65 об.]{spbfaran79} % ы'вэӄуч
	&	
	&
	&	\cite[364]{davydova2015a} \tabularnewline \midrule
\tenevilglyph[yes][4]{i_cUY_2cD}
	&	
	&	
	&
	&	irьn [иръын = кухлянка, мужская меховая рубаха, пальто, верхняя одежда; напечатано в книге, знак рядом] [7.13, 12.23]
		\tabularnewline \midrule
\tenevilglyph[yes][4]{i_cUY_2cD_2q}
	&	
	&	
	&
	&	tur-irьn [туриръын = новая кухлянка; напечатано в книге, знак рядом] [7.13] % TODO: уточнить перевод
		\tabularnewline \midrule
\tenevilglyph[yes][4]{i_2cU_2C}
	&	ŋәucqan [ŋausqan, ӈэвысӄэт = женщина] \cite[л. 65 об.]{spbfaran79} % ӈэвысӄэт
	&	a woman \cite{mindalevich1934}
	&	н'эвыскэт [ӈэвысӄэт], женщина [26]
	&	\cite[364]{davydova2015a} \linebreak
		женжны [женщины] [34.16]
		\tabularnewline \midrule
\tenevilglyph[yes][3]{i_2cU_j_2C}
	&	жена \cite[л. 65 об.]{spbfaran79}
	&	
	&
	&	\cite[364]{davydova2015a}
		\tabularnewline \midrule
\tenevilglyph[yes][4]{i_2cU_l_2C}
	&	мать \cite[л. 64]{spbfaran79}\linebreak
		әtla [ытԓя = мать] \cite[л. 52]{spbfaran79}\linebreak % ытԓя
		etla [әtla] \cite[л. 52 об., 56]{spbfaran79}
	&	
	&
	&	\cite[360, 364]{davydova2015a} \linebreak
		ьla [әtla, ытԓя] [ИЛИ:1.21]
		\tabularnewline \midrule
\tenevilglyph[no][3]{i_2cU_t_2C}
	&	родившая мать \cite[л. 64]{spbfaran79}
	&	a woman awaiting the birth of her child \cite{mindalevich1934}
	&
	&	\tabularnewline \midrule
\tenevilglyph[yes][3]{i_2cU_2C_h}
	&	ьnpьŋәu [ьnpь-ŋæw, ынпыӈэв = старуха] \cite[л. 65 об]{spbfaran79} % ынпыӈэв
	&	
	&
	&	[25.6]
	 	\tabularnewline \midrule
\tenevilglyph[yes][4]{i_2CF}
	&	сын \cite[л. 52]{spbfaran79}\linebreak
		сыновья \cite[л. 52]{spbfaran79} \linebreak
		әkkot [ækkæt, эккэт = сыновья] \cite[л. 39]{spbfaran79} \linebreak % экык, эккэт
		син [сын] \cite[л. 67]{spbfaran79}
	& 	сын \cite{bogoraz1934}\linebreak
		сын \cite{lavrov1969}
	&	экык [экык = сын], сын [18]
	&	\cite[364]{davydova2015a} \linebreak 
		\cite{bogoraz1934}
		\tabularnewline \midrule
\tenevilglyph[yes][4]{i_2cU_CF}
	&	доса [дочь] \cite[л. 67]{spbfaran79}
	&	
	&
	&	ееуске [девушка] [29.2об] \linebreak
	 	~[25.8об] \linebreak
	 	ŋэkьk [ӈээкык = дочь] [ИЛИ:1.24]
	 	\tabularnewline \midrule
\tenevilglyph[no][3]{i_2cU_3CF}
	&	ситра [сестра] \cite[л. 67]{spbfaran79} 
	&	
	&
	& 	\tabularnewline \midrule
\tenevilglyph[no][3]{i_2CF_v_q_'}
	&	прат [брат] \cite[л. 67]{spbfaran79}
	&	
	&
	& 	\tabularnewline \midrule
\tenevilglyph[yes][4]{i_vd_q_i} 
	&	
	&	друг \cite{lavrov1969}
	&	тумгытум [= товарищ], товарищ [50]
	& 	\cite[364]{davydova2015a} \linebreak
		tumgьt [тумгыт = друзья; напечатано в книге, знак рядом] [12.20] \linebreak % TODO: уточнить транскрипцию и перевод
		таварес [товарищ] [34.8об]
		\tabularnewline \midrule
\tenevilglyph[yes][4]{i_2CF_j}
	&	qlaul nenene [qlaul nænænæ, ӄԓявыԓ нэнэны = мужчина младенец] \cite[л. 65 об]{spbfaran79} % ӄԓявыԓ нэнэны
	&	
	&
	& 	\cite[364]{davydova2015a} \linebreak
		nenenь [нэнэны = дитя, ребенок; напечатано в книге, знак рядом] [7.13] \linebreak
		ŋenqej [ŋenqәj, ӈинӄэй = мальчик] [ИЛИ:1.7]
		\tabularnewline \midrule
\tenevilglyph[yes][3]{i_2cU_CF_h}
	&	ŋәusqan neneneŋ [ŋausqan nænænæŋ, ӈэвысӄэт нэнэны = женщина мланедец] \cite[л. 65 об]{spbfaran79} % ӈэвысӄэт нэнэны
	&	
	&
	& 	[34.9]
		\tabularnewline \midrule
\tenevilglyph[yes][4]{o-_p_j}
	&	он \cite[л. 40]{spbfaran79} \linebreak 
		әtlon [ытԓён = он] \cite[л. 39 об, 52, 65 об]{spbfaran79} % ытԓён
	& 	он \cite{bogoraz1934}
	&	ытлён [ытԓён], он [92]
	& 	\cite[360]{davydova2015a} \linebreak
		ылтон [ытԓён] [32.16об] \linebreak
		ьlon [ытԓён] [ИЛИ:2.3]
		\tabularnewline \midrule
\tenevilglyph[yes][4]{o_2j}
	&	наш \cite[л. 40]{spbfaran79} \linebreak
		murgin [мургин = наш] \cite[л. 52]{spbfaran79} \linebreak % мургин
		muri [мури = мы] \cite[л. 39 об, 65 об]{spbfaran79} \linebreak % мури
		мы \cite[л. 68]{spbfaran79} \linebreak
		наса [наша] \cite[л. 68]{spbfaran79}
	& 	наш \cite{bogoraz1934}\linebreak
		we \cite{mindalevich1934}
	&	мури, мы [92]
	& 	\cite[364]{davydova2015a} \linebreak
		\cite[28]{lavrov1969} 
		\tabularnewline \midrule
\tenevilglyph[yes][4]{o_2j_l}
	&	
	& 	
	&	
	& 	\cite[364]{davydova2015a} \linebreak
		murgьnan [morgьnan, моргынан = мы (форма подлежащего при переходном глаголе)] [ИЛИ:2.1]
		\tabularnewline \midrule
\tenevilglyph[yes][4]{o_j}
	&	мой \cite[л. 40, 55]{spbfaran79} \linebreak
		gьmnin [гымнин = мой] \cite[л. 56]{spbfaran79} \linebreak % гымнин
		gumnin [gьmnin] \cite[л. 52 об, 65]{spbfaran79}
	& 	мой \cite{bogoraz1934}
	&	гымнин, мой [93]
	&	[2.1?] 
		мена [меня] [37.2]
		\tabularnewline \midrule
\tenevilglyph[yes][4]{o}
	&	я \cite[л. 40, 53, 65 об]{spbfaran79} \linebreak
		gьm [гым = я]\cite[л. 52,56]{spbfaran79} \linebreak % гым
		gum [gьm] \cite[л. 52 об, 65 об]{spbfaran79}
	& 	я \cite{bogoraz1934}
	&	гым, я [92]
	& 	\cite[364]{davydova2015a} \linebreak
		хым [гым] [34.8] \linebreak
		я [34.11] \linebreak
		gьm [гым] [ИЛИ:2.6]
		\tabularnewline \midrule
\tenevilglyph[yes][4]{o_j_q}
	&	мне \cite[л. 66]{spbfaran79} \linebreak
		в \textit{«мне»}, \textit{«я} восму» \cite[л. 66]{spbfaran79} \linebreak
		в \textit{«я} ниснаю», \textit{«я} упрала» \cite[л. 79]{spbfaran79}
	&	
	&	
	&	\cite{bogoraz1934} \linebreak
		gьmnan [гымнан = я, форма подлежащего при переходных глаголах; напечатано в книге, знак рядом] [12.28] \linebreak
		gьmnan [гымнан] [ИЛИ:1.10] 
		\tabularnewline \midrule
\tenevilglyph[yes][4]{o-_s}
	&	gьt [гыт = ты] \cite[л. 65 об]{spbfaran79} % гыт
	&	
	&
	& 	в «ььt» [гыт] [5.1об] % закорючка справа, вероятно, ошибка
		\tabularnewline \midrule
\tenevilglyph[no][4]{o-_jY}
	&	turi [тури = вы] \cite[л. 65 об]{spbfaran79} % тури
	&	
	&	тури, вы [92]
	& 	\tabularnewline \midrule
\tenevilglyph[yes][1]{o_j_j}
	&	или такои [?] \cite[л. 67]{spbfaran79} \linebreak
		эвлм [?] \cite[л. 68]{spbfaran79}
	&	
	&
	& 	[38.1] \linebreak
		eulьm [ИЛИ:1.10] % TODO: нужен перевод
		\tabularnewline \midrule
\tenevilglyph[yes][4]{o-_j}
	&	
	&	
	&	ынин [= его, её], его [93]
	& 	\cite[360, 361, 362, 364]{davydova2015a} \linebreak
		ьnen [ынин] [ИЛИ:1.4] \linebreak % TODO: нужна транскрипция латиницей
		ьnnen [ынин] [ИЛИ:1.4]
		\tabularnewline \midrule
\tenevilglyph[yes][4]{o-_j_2cD}
	&	хозяин* \cite[л. 51]{spbfaran79}
	&	
	&	
	& 	eetьn [etьn, этын = хозяин] [ИЛИ:1.9, ИЛИ:2.7] \linebreak
		eetьnwьt [?] [ИЛИ:1.17] % TODO: нужен перевод
		\tabularnewline \midrule
\tenevilglyph[yes][1]{o-_j_jY}
	&	
	&	
	&	
	& 	qьmelьrgьnan [?] [ИЛИ:1.19] % TODO: нужен перевод
		\tabularnewline \midrule
\tenevilglyph[yes][4]{o_l}
	&	
	&	
	&	
	& 	ьnan [ынан = он (употребляется с переходным глаголом), он сам] [ИЛИ:1.9]
		\tabularnewline \midrule
\tenevilglyph[yes][4]{o_l_jY}
	&	
	&	
	&	ыргынан [ыргынан = они (употребляется с переходным глаголом)], ими [94]
	& 	\cite[364]{davydova2015a} \linebreak
		ьrgьnan [ыргынан] [ИЛИ:1.3, ИЛИ:1.11]
		\tabularnewline \midrule
\tenevilglyph[yes][4]{o_l_jY_j}
	&	
	&	
	&	
	& 	ьrgen [ыргин = их, принадлежащий им] [ИЛИ:1.3, ИЛИ:2.1]
		\tabularnewline \midrule
\tenevilglyph[yes][4]{R_2bN}
	&	сеть \cite[л. 40]{spbfaran79} \linebreak
		giŋingi [giŋingь, гиӈынгиӈ = сеть] \cite[л. 39]{spbfaran79} \linebreak % гиӈынгиӈ
		сетка \cite[л. 68]{spbfaran79}
	& 	сеть \cite{bogoraz1934}\linebreak
		a net \cite{mindalevich1934}
	&	купрэн [= сеть, невод], гин'инги [гиӈынгиӈ], сеть [113]
	& 	\cite[361]{davydova2015a} \linebreak
		\cite{bogoraz1934} \linebreak
		kepret [купрэт = сети; напечатано в книге, знак рядом] [12.25]
		\tabularnewline \midrule
\tenevilglyph[yes][2]{sME_2b}
	&	Teŋiwil — автор записей \cite[л. 40, 52, 54]{spbfaran79}
	&	как звать? \cite{lavrov1969}
	&	как тебя звать [16]
	& 	\cite[360–364]{davydova2015a} \linebreak
		%Tьŋweil [7.29об] \linebreak
		атехае [?] [33.5об] \linebreak
		атегаи [?] [30.3] \linebreak
		Тынеувел [Теневиль] [35.3] \linebreak
		в «папа» [30.1об]
		\tabularnewline \midrule
\tenevilglyph[yes][4]{sME}
	&
	&	Теневиль \cite{lavrov1969}
	&	имя Теневиля
	& 	\cite[361]{davydova2015a} \linebreak
		\cite[28]{lavrov1969} \linebreak
		Тынеувел [Теневиль] [33.5об]
		\tabularnewline \midrule
\tenevilglyph[yes][2]{i_2lY}
	&
	&	Раглине [жена Теневиля] \cite{lavrov1969}
	&	Сергей (имя) [117]
	& 	\cite[364]{davydova2015a} \linebreak
		\cite[28]{lavrov1969} 
		\tabularnewline \midrule
\tenevilglyph[yes][4]{i_l_q_lY}
	&
	&	
	&	Раглынэ (Рая), имя жены Теневиля* [118]
	& 	Rая [Рая\cite{druri1989} = Раулина, жена Теневиля] [32.10]
		\tabularnewline \midrule
\tenevilglyph[yes][4]{i_2cY}
	&
	&	Этувьи \cite{lavrov1969}
	&	Этувьи (Егор), имя сына Теневиля [7]
	& 	\cite[361, 363]{davydova2015a} \linebreak
		\cite[28]{lavrov1969} \linebreak
		Еекор [Егор, Этувьи\cite{lavrov1969} — сын Теневиля] [33.5об]
		\tabularnewline \midrule
\tenevilglyph[yes][4]{UD_2b}
	&
	&	
	&	Эйчинкеу, имя сына Теневиля [115]
	& 	\cite[362, 363]{davydova2015a} \linebreak
		\cite[28]{lavrov1969} \linebreak
		Эехынкеу [Игеннеу\cite{mindalevich1934a}, Эгенкау\cite{sergeev1956}, Эйчинкеу [46] — второй сын Теневиля] [29.13, 33.5об, 35.3]
		\tabularnewline \midrule
\tenevilglyph[yes][4]{b-B}
	&
	&	
	&	Маусек, имя старшего сына Теневиля
	& 	\cite[361, 362, 363]{davydova2015a} \linebreak
		Mаусек [[46] старший сын Теневиля] [33.5об]
		\tabularnewline \midrule
\tenevilglyph[yes][1]{i_2cU_CF_i_2l} % TODO: Вероятно имя собственное, но надо проверить
	&
	&	
	&
	& 	Еетхеут [?] [29.2, 35.3]
		\tabularnewline \midrule
\tenevilglyph[yes][1]{f_i_2l} % TODO: Вероятно имя собственное, но надо проверить
	&
	&	
	&
	& 	Еатхеынын [?] [35.6об]
		\tabularnewline \midrule
\tenevilglyph[yes][4]{i_2cU_CF_i_2j}
	&
	&	
	&	Эвнэут, имя дочери Теневиля [125]
	& 	iеунеут [Эвнэут] [29.13, 32.10, 33.5об, 35.3] 
		\tabularnewline \midrule
\tenevilglyph[yes][1]{iY_2cDX_jF} % TODO: Вероятно имя собственное, но надо проверить
	&
	&	
	&
	& 	нэiэн [?] [29.10] \linebreak
		неiенхе [?] [30.1об]
		\tabularnewline \midrule
\tenevilglyph[yes][2]{i_c_C_i_j}
	&	мать \cite[л. 40]{spbfaran79} \linebreak
		Upenkew — враг автора \cite[л. 40]{spbfaran79} % Ближайшее по звучанию — шаман Упетеу \cite{mindalevich1934a}
	& 	мать \cite{bogoraz1934}\linebreak
		мать \cite{lavrov1969}
	&	ытля [= мать], мать [69]
	& 	[1.1] 
		\tabularnewline \midrule
\tenevilglyph[no][1]{i_c_C}
	&	Utenkew \cite[л. 52 об]{spbfaran79} \linebreak
		Utenkew (?) \cite[л. 56]{spbfaran79}
	&	
	&
	& 	\tabularnewline \midrule
\tenevilglyph[yes][4]{IY_j}
	&	сам \cite[л. 40, 53]{spbfaran79} \linebreak
		cinit [cinit, чинит = сам] \cite[л. 52]{spbfaran79} \linebreak % чинит
		ꞓinit [cinit] \cite[л. 52 об]{spbfaran79}
	& 	сам \cite{bogoraz1934}
	&	чиниткин [= свой], свой, своя, свое, чинит [= сам], сам, сама, само [6]
	& 	\cite[364]{davydova2015a} \linebreak
		\cite{bogoraz1934} \linebreak
		чам [сам] [32.6об] \linebreak
		сам [34.8об]
		\tabularnewline \midrule
\tenevilglyph[yes][4]{iY}
	&	тот \cite[л. 40]{spbfaran79} \linebreak
		әnqon [әnqan, ынӄэн = тот, этот] \cite[л. 52, 54]{spbfaran79} % ынӄэн
	& 	тот \cite{bogoraz1934}
	&	ынкэн [ынӄэн], тот, та, то [52]
	& 	\cite[360, 361, 364]{davydova2015a} \linebreak
		\cite[28]{lavrov1969} \linebreak
		nqen [ынӄэн] [4.10об] \linebreak
		ето [это] [32.16] \linebreak
		ьnqen [ынӄэн] [ИЛИ:1.4]
		\tabularnewline \midrule
\tenevilglyph[yes][4]{iY_q}
	&	
	& 	
	&	наанкэн [ӈанӄэн = вон тот (видимый), далекий], вон (там) [52]
	& 	\cite[364]{davydova2015a} \linebreak
		\cite[28]{lavrov1969} \linebreak
		ŋanqen [ŋanqәn, ӈанӄэн] [ИЛИ:1.2]
		\tabularnewline \midrule
\tenevilglyph[yes][4]{d_C}
	&	нет \cite[л. 40]{spbfaran79} \linebreak
		ujŋә [уйӈэ = нет чего-нибудь] \cite[л. 39]{spbfaran79} \linebreak % уйӈэ
		нету \cite[л. 66 об]{spbfaran79} \linebreak
		в \textit{«не}било» \cite[л. 66]{spbfaran79}
	& 	нет \cite{bogoraz1934}\linebreak
		no \cite{mindalevich1934}
	&	уйн'э [уйӈэ], нет [51]
	& 	\cite[360, 361, 364]{davydova2015a} \linebreak
		\cite[28]{lavrov1969} \linebreak
		нету [32.18]
		\tabularnewline \midrule
\tenevilglyph[yes][4]{G}
	&	теперь \cite[л. 40]{spbfaran79} \linebreak
		igьt [igьr, игыр = сегодня, теперь] \cite[л. 39, 52 об]{spbfaran79} \linebreak % игыр
		чиперче [теперь] \cite[л. 67 об]{spbfaran79} \linebreak
		вонтенеи (?) \cite[л. 67 об]{spbfaran79} 
	& 	теперь \cite{bogoraz1934}
	&	игыр, игыт, сегодня, сейчас, теперь [64] % TODO: нужна транскрипция
	& 	\cite[361, 364]{davydova2015a} \linebreak
		\cite[28]{lavrov1969} \linebreak
		сесеяс [сейчас] [37.2] \linebreak
		севотни [сегодня] [37.2] \linebreak
		ieьr [igьr, игыр] [ИЛИ:1.7]
		\tabularnewline \midrule
\tenevilglyph[yes][4]{G_'}
	&	
	& 	
	&	
	& 	ieьteьnk [игыттэгнык = до сего дня, до сих пор] [ИЛИ:1.4] % TODO: нужна транскрипция латиницей; обрати внимание ` тут и еще в паре мест обозначает -тэгнык
		\tabularnewline \midrule
\tenevilglyph[yes][4]{o_q}
	&	там \cite[л. 50]{spbfaran79} \linebreak
		әnkь [ынкы = там] \cite[л. 39 об]{spbfaran79} \linebreak % ынкы
		тут \cite[л. 66]{spbfaran79} \linebreak
		дут [тут] \cite[л. 68]{spbfaran79}
	& 	там \cite{bogoraz1934}
	&	ынкы, там
	& 	\cite[360, 361, 364]{davydova2015a}\linebreak 
		\cite[28]{lavrov1969}\linebreak 
		әnkь [напечатано в книге, знак рядом] [12.17] \linebreak
		тот [тут] [32.13] \linebreak
		в «вотут» [вот тут] [32.6] \linebreak
		тут [32.15об] \linebreak
		ынкы [= там] [32.16об] \linebreak
		ьnkт [ынкы] [ИЛИ:1.3]
		\tabularnewline \midrule
\tenevilglyph[yes][4]{o_q_'}
	&	с тех пор \cite[л. 40]{spbfaran79} \linebreak
		әnkәtegnek \cite[л. 39]{spbfaran79} \linebreak % TODO: нужен перевод ынкатагнэпы ? ынкатагнык ?
		әnketegnek \cite[л. 39 об]{spbfaran79} \linebreak
		әnkәtegnьik \cite[л. 54]{spbfaran79} 
	& 	с тех пор \cite{bogoraz1934}
	&	ынкэтэгнык, с тех пор [46] % TODO: нужна транскрипция
	& 	\cite[360, 364]{davydova2015a} \linebreak
		ьnketeьnek [ИЛИ:1.13] % TODO: нужна транскрипция
		\tabularnewline \midrule
\tenevilglyph[yes][4]{o_q_b}
	&	
	& 	
	&	
	& 	lьmьnkь [ԓымынкы = повсюду, всюду, везде] [ИЛИ:2.2] % TODO: нужна транскрипция латиницей
		\tabularnewline \midrule
\tenevilglyph[yes][4]{l-l}
	&	
	&	
	&	нутку [ӈутку = здесь, тут, здесь [49]
	& 	нутку [ӈутку] [34.8] \linebreak
		ŋutku [ӈутку] [ИЛИ:2.4]
		\tabularnewline \midrule
\tenevilglyph[yes][4]{l-l_'}
	&	с этих пор \cite[л. 40]{spbfaran79} \linebreak
		wutkelegnek (?) \cite[л. 54]{spbfaran79} % TODO: нужен перевод выткутэгнык ?
	& 	с этих пор \cite{bogoraz1934}
	&	с этих пор (Б. Т. 39)
	& 	[1.4об, 4.2об] \linebreak
		ŋutketeьnk [ИЛИ:1.4] % TODO: нужна транскрипция
		\tabularnewline \midrule
\tenevilglyph[yes][4]{l-l_'_2cD}
	&	
	& 	
	&	
	&	ŋanqo [ӈанӄо = оттуда сюда (к говорящему); напечатано в книге, знак рядом]  [12.20об] \linebreak
		ŋanqorь [ӈанӄоры = сюда, оттуда сюда (к говорящему)] [ИЛИ:1.20] % TODO: нужна транскрипция латиницей
		\tabularnewline \midrule
\tenevilglyph[yes][3]{2i_P}
	&	на реке \cite[л. 41]{spbfaran79} \linebreak
		veemьk [vææmьk, вээмык = на реке] \cite[л. 39]{spbfaran79} % вээмык
	& 	на реке \cite{bogoraz1934}
	&
	& 	\cite[361]{davydova2015a} 
		\tabularnewline \midrule
\tenevilglyph[yes][3]{2i_2q}
	&	vaamete [vææmьte, вээмэты = к реке] \cite[л. 56]{spbfaran79} \linebreak % вээмэты
		около рещки [около речки] \cite[л. 68 об]{spbfaran79}
	&	
	&	веем [вээм = река], река [41]
	& 	\cite[361]{davydova2015a} \linebreak
		\cite[28]{lavrov1969} \linebreak
		weemьk [вээмык; напечатано в книге, знак рядом] [12.19об]
		\tabularnewline \midrule
\tenevilglyph[yes][4]{i_g_b_jX}
	&	хариус \cite[л. 41, 54 об]{spbfaran79} \linebreak
		qeꞓaw [kьcaw, кычав = хариус] \cite[л. 39]{spbfaran79} % кычав
	& 	хариус \cite{bogoraz1934}
	&	кычав, хариус  [22] % 
	& 	\cite[361]{davydova2015a} \linebreak
		хартон [?] [33.4]
		\tabularnewline \midrule
\tenevilglyph[yes][4]{i_g_b}
	&	кета \cite[л. 44, 45, 54 об]{spbfaran79} \linebreak
		рипа [рыба] \cite[л. 68 об]{spbfaran79}
	& 	кета \cite{bogoraz1934}\linebreak
		рыба кета \cite{lavrov1969}
	&	ыннээн [= рыба], рыба [22]
	& 	\cite[361]{davydova2015a} \linebreak 
		\cite[26]{lavrov1969} \linebreak
		ьnnьt [ынныт = рыбы; напечатано в книге, знак рядом] [12.24об] \linebreak % TODO: нужна транскрипция латиницей
		qetaqet [ӄэтаӄэт; напечатано в книге, знак рядом] [12.24об] \linebreak % TODO: нужна транскрипция латиницей
		кетат [ӄэтаӄэт, кета] [33.4] \linebreak
		эрепо [рыба?] [33.4]
		\tabularnewline \midrule
\tenevilglyph[yes][3]{i_g_2b}
	&	налим \cite[л. 45, 54 об]{spbfaran79} 
	& 	налим \cite{bogoraz1934}\linebreak
		налим \cite{lavrov1969}
	&	гачгагыргын [= налим], налим [22]
	& 	[1.30]
		\tabularnewline \midrule
\tenevilglyph[yes][3]{i_g_b_z}
	&	сиг \cite[л. 45]{spbfaran79} 
	&	
	&
	& 	[15.11] 
		\tabularnewline \midrule
\tenevilglyph[yes][4]{i_g_b_hL}
	&	щука \cite[л. 45]{spbfaran79} 
	& 	щука \cite{bogoraz1934}\linebreak
		щука \cite{lavrov1969}
	&	юуткуннээн [юукуннээн], ээун, щука [22]  % TODO: нужна транскрипция
	& 	[2.101] 
		\tabularnewline \midrule %
\tenevilglyph[no][3]{i_g_2b_q_k}
	&	бычок \cite[л. 45]{spbfaran79} 
	&	
	&
	& 	\tabularnewline \midrule
\tenevilglyph[yes][3]{i_g_b_2cD}
	&	 [без перевода] \cite[л. 54 об]{spbfaran79} 
	&	нярка \cite{lavrov1969}
	&	?, нярка (род рыбы) [22] % TODO: нужна транскрипция, первое слово неразборчиво
	& 	\cite[361]{davydova2015a} \linebreak
		viruvir [вирувир = нерка; напечатано в книге, знак рядом] [12.24об]
		\tabularnewline \midrule
\tenevilglyph[yes][4]{u_j_jX_j}
	&	еда, есть \cite[л. 41]{spbfaran79} \linebreak
		roьlqal [roolqәl, рооԓӄыԓ = пища, продукты] \cite[л. 39]{spbfaran79} % рооԓӄыԓ
	& 	еда, есть \cite{bogoraz1934} \linebreak
		food \cite{mindalevich1934}
	&
	& 	\cite[364]{davydova2015a} \linebreak
		roolqьl [roolqәl, рооԓӄыԓ] [ИЛИ:2.24]
		\tabularnewline \midrule
\tenevilglyph[yes][3]{u_j_jX} 
	&	
	&	
	&	нэнунэт, съели, роолкыл [рооԓӄыԓ], еда [111] % TODO: нужна транскрипция
	& 	\cite[364]{davydova2015a} \linebreak
		в «чаеопат» [«чай \textit{пить}» или чайпат = вскипевший чай] [32.16об] % чайпат?
		\tabularnewline \midrule
\tenevilglyph[yes][4]{I_iX_2qY}
	&	бояться \cite[л. 41]{spbfaran79} \linebreak
		в «мы \textit{боимся»} \cite[л. 52]{spbfaran79} \linebreak
		в «я оцин \textit{боюс»} [«я очень боюсь»] \cite[л. 67 об]{spbfaran79}
	& 	бояться \cite{bogoraz1934}
	&	бояться (Б. Т. 26), апылгавык, экэлин'этык, пъэчикэтык [8] % TODO: нужна транскрипция
	& 	поеысея [боялся] [37.2]  \linebreak
		ajlgawk [айыԓгавык = бояться, испугаться] [ИЛИ:1.22]
		\tabularnewline \midrule
\tenevilglyph[yes][1]{I_iX_u_2qY}
	&	
	& 	
	&	
	& 	ajьla [ИЛИ:1.3] \linebreak % TODO: нужен перевод
		ajьlga [ИЛИ:1.3] \linebreak
		ajlga [ИЛИ:1.4] \linebreak
		\tabularnewline \midrule
\tenevilglyph[yes][4]{o_q_jF}
	&	сделал \cite[л. 41]{spbfaran79} \linebreak
		елат [делать] \cite[л. 68]{spbfaran79}
	& 	сделал \cite{bogoraz1934}\linebreak
		made \cite{mindalevich1934}
	&	сделал, тэйкык, рытчик, делать* [67] % TODO: нужна транскрипция
	& 	\cite[361, 364]{davydova2015a} 
		\tabularnewline \midrule
\tenevilglyph[yes][4]{o_q_jF-c}
	&	
	& 	
	&	тайкыё, делаемое [80] % TODO: нужна транскрипция, уточнить перевод
	& 	tajkjo [тайкыё] [ИЛИ:1.4]
		\tabularnewline \midrule
\tenevilglyph[yes][4]{o_l_jF}
	&	конщил* [кончил] \cite[л. 66 об]{spbfaran79} \linebreak % вверх ногами
	& 	
	&	пылиткунин, закончил [79] % TODO: здесь и ниже нужна транскрипция, нужен перевод
	& 	зиокочайс [закончилось] [32.15] \linebreak
		эemlьkuke [ИЛИ:1.5] \linebreak
		gemlьtkulen [ИЛИ:1.11]
		\tabularnewline \midrule
\tenevilglyph[yes][4]{o_q_jF_b}
	&	
	&	делать, работать \cite{lavrov1969}
	&	нуван'экэн, работает [79] % TODO: нужна транскрипция
	& 	\cite[364]{davydova2015a} \linebreak
		ырыпотали [работали] [34.8об]
		\tabularnewline \midrule
\tenevilglyph[yes][4]{U_2Q}
	&	сказал* \cite[л. 41]{spbfaran79} \linebreak % тут везде знак перевернут, похоже
		giulin* \cite[л. 52]{spbfaran79} % НУЖЕН ПЕРЕВОД гэвъилин ?
	& 	сказал* \cite{bogoraz1934}
	&	нивкин, говорит [69] % TODO: нужна транскрипция
	& 	каварйт [говорит] [32.16] \linebreak
		кавареу [говорю] [30.7] \linebreak
		кавареят [говорят] [30.7] \linebreak
		neuqen [ИЛИ:1.4] % TODO: нужна транскрипция, нужен перевод
		\tabularnewline \midrule
\tenevilglyph[yes][4]{U_Q_b}
	&	
	&	
	&
	& 	wetgau [wetgaw, вэтгав = слово, речь; напечатано в книге, знак рядом] [12.24]
		\tabularnewline \midrule
\tenevilglyph[yes][4]{U_b}
	&	
	&	
	&
	& 	еесик [язык] [32.6об] \linebreak
		еегиг [язык] [34.16]
		\tabularnewline \midrule
\tenevilglyph[yes][4]{c_CE}
	&	пребывать, быть \cite[л. 41]{spbfaran79} \linebreak
		в \textit{«било} он», «не\textit{било»}, «какои» \cite[л. 66]{spbfaran79}
	& 	пребывать, быть \cite{bogoraz1934}
	&	варкын, иметься, есть (что-нибудь), (пребывать, быть, Б. Т. 32) [71] % TODO: нужна транскрипция
	& 	\cite[360, 361, 364]{davydova2015a} \linebreak
		\cite[28]{lavrov1969} \linebreak
		ееч [есть] [32.16] \linebreak
		еес [есть] [33.4] \linebreak
		warkьn [ѵarkьn = пребывает, живет] [ИЛИ:2.21] % TODO: нужна транскрипция
		\tabularnewline \midrule
\tenevilglyph[yes][4]{UD_2B}
	&	жить \cite[л. 41]{spbfaran79} \linebreak
		nyegtel \cite[л. 39]{spbfaran79} \linebreak % НУЖЕН ПЕРЕВОД ны-егтэԓ ?
		nьegtel \cite[л. 39 об]{spbfaran79} \linebreak
		сивоы [живой] \cite[л. 68]{spbfaran79}
	& 	жить \cite{bogoraz1934}\linebreak
		lived \cite{mindalevich1934}\linebreak
		жить \cite{lavrov1969}
	&	нымытвак, жить (жизнь) [27] % TODO: нужна транскрипция
	& 	\cite[360, 364]{davydova2015a} \linebreak
		jeьtel [егтэԓ = жизнь, средства к существованию] [ИЛИ:2.27]
		\tabularnewline \midrule
\tenevilglyph[yes][3]{UE}
	&	становиться \cite[л. 41]{spbfaran79} \linebreak
		mьnneel \cite[л. 39]{spbfaran79} \linebreak % НУЖЕН ПЕРЕВОД мыннъэԓ ? мытнъэԓ?
		mьtьnnel \cite[л. 39 об]{spbfaran79} \linebreak
		mьn-neel \cite[л. 52]{spbfaran79}
	& 	становиться \cite{bogoraz1934}
	&
	& 	\cite[360, 364]{davydova2015a} \linebreak
		nelgi [= стал; напечатано в книге, знак рядом] [11.22] \linebreak % TODO: уточнить перевод, нужна транскрипция
		nilgei [ИЛИ:1.7] % TODO: нужен перевод
		\tabularnewline \midrule
\tenevilglyph[yes][4]{2OX} 
	&	mejŋь [mæjŋ, мэйыӈ = большой (основа)] \cite[л. 64 об]{spbfaran79} \linebreak % мэйыӈ
		оыё [?] \cite[л. 66]{spbfaran79} \linebreak
		коломеи [коԓё мэй = очень большой] \cite[л. 68 об]{spbfaran79} % коԓё мэй
	&	
	&	нымэйынкин [нымэйыӈӄин], большой [123]
	& 	\cite[361, 364]{davydova2015a} \linebreak
		\cite[28]{lavrov1969} \linebreak
		kolomj [kolo mej, коԓё мэй] [ИЛИ:2.27]
		\tabularnewline \midrule
\tenevilglyph[yes][4]{2OX_j}
	&	расти и большой \cite[л. 41]{spbfaran79} \linebreak
		nьmejŋqin* [nьmæjьŋqin, нымэйыӈӄин = большой] \cite[л. 54]{spbfaran79} \linebreak % нымэйыӈӄин
		mejn* [mæjŋ, мэйыӈ = большой (основа)] \cite[л. 39 об]{spbfaran79} % мэйыӈ
	& 	расти, большой \cite{bogoraz1934}
	&	
	& 	\cite[360, 364]{davydova2015a} \linebreak
		в «полчои» [большой]* [29.12]
		\tabularnewline \midrule
\tenevilglyph[yes][4]{2OX_l} 
	&	
	&	
	&	ныкетгукин [ныкэтгуӄин = сильный, мощный могучий], сильный, мэйн'этык [мэйӈэтык = расти, развиваться, повышаться], увеличиваться* [123] % зеркально
	& 	\cite[364]{davydova2015a} \linebreak
		nьketguqenet [ныкэтгуӄинэт] [ИЛИ:1.3] % TODO: нужна транскрипция латиницей
		\tabularnewline \midrule
\tenevilglyph[yes][4]{o_4i}
	&	nenanmuqen \cite[л. 54]{spbfaran79} \linebreak % TODO: нужен перевод
		в \textit{«убил} волка» \cite[л. 68 об]{spbfaran79} 
	&	
	&	убить, тымык [= убить зверя, добыть зверя] (жив.), тэгынн'этык [тэгинӈэтык = убить (человека)] (чел.) [131]
	& 	\cite[360, 361]{davydova2015a} \linebreak
		\cite{bogoraz1934} 
		\tabularnewline \midrule
\tenevilglyph[yes][4]{o_4i_k}
	&	умереть \cite[л. 41]{spbfaran79} \linebreak
		умирают \cite[л. 52]{spbfaran79} \linebreak
		wu [vi, въи = умереть (основа)] \cite[л. 52]{spbfaran79} \linebreak % въи
		vu [vi] \cite[л. 52]{spbfaran79} 
	&	
	&	въик [въик = умереть, скончаться], вайн'ык [вайӈык = угасать, умирать], умирать [132]
	& 	\cite[360]{davydova2015a} \linebreak
		омейр [умер] [32.16]
		\tabularnewline \midrule
\tenevilglyph[yes][4]{c_JY}
	&	покинул \cite[л. 41]{spbfaran79} \linebreak
		enapelae [энапэԓя = оставлять (основа)] \cite[л. 52]{spbfaran79} \linebreak % энапэԓя 
		enapela \cite[л. 56]{spbfaran79} \linebreak
		оставил \cite[л. 68 об]{spbfaran79}
	&	
	&	ныппелакенат, (их) покидают [8] % TODO: уточнить перевод, нужна транскрипция
	& 	nьpelaqenat [= покидали, напечатано в книге, знак рядом] [12.20об] \linebreak
		[25.3] 
		\tabularnewline \midrule
\tenevilglyph[yes][4]{c_sY} 
	&	
	&	
	&	
	& 	qutti [ӄутти =другие; напечатано в книге, знак рядом] [12.20] \linebreak % ср. ӄоԓ, совсем другой знак
		qute [ӄутти] [ИЛИ:1.10]
		\tabularnewline \midrule
\tenevilglyph[yes][4]{b_2q_L}
	&	покинул \cite[л. 41]{spbfaran79} \linebreak % Надо полагать, ошибка из-за схожести слов
		pela [пэԓя = покидать, оставлять (основа)] \cite[л. 52]{spbfaran79} % пэԓя, вероятно, неправильно 
	&	
	&
	& 	\cite[364]{davydova2015a} \linebreak
		чкорйе [вскоре] [32.16] \linebreak
		ычкоре [вскоре] [30.6] \linebreak
		pele [petlә, пэтԓе = скоро, быстро, вскоре] [ИЛИ:1.20]
		\tabularnewline \midrule
\tenevilglyph[yes][3]{4L}
	&	плакать \cite[л. 41]{spbfaran79} \linebreak
		nьtergьtьm \cite[л. 52]{spbfaran79} % НУЖЕН ПЕРЕВОД нытэргат-?
	&	
	&
	& 	\cite[360]{davydova2015a} 
		\tabularnewline \midrule
\tenevilglyph[yes][4]{a}
	&	стадо, олень \cite[л. 42]{spbfaran79} \linebreak
		ŋәlvil [ŋælvьl, ӈэԓвыԓ = стадо (преимущественно оленей)] \cite[л. 56]{spbfaran79} % ӈэԓвыԓ
	& 	стадо, олень \cite{bogoraz1934}\linebreak
		reindeer \cite{mindalevich1934}\linebreak
		олень \cite{lavrov1969}
	&	коран'ы [ӄораӈы = олень], олень [82]
	& 	\cite[364]{davydova2015a} \linebreak
		\cite{bogoraz1934} \linebreak
		олене [олень] [33.4] \linebreak
		qoraŋь [ӄораӈы] [ИЛИ:2.18]
		\tabularnewline \midrule
\tenevilglyph[yes][3]{a_k}
	&	силонок [теленок] \cite[л. 68 об]{spbfaran79} 
	&	
	&
	& 	\cite[362]{davydova2015a} \linebreak
		[1.61]
		\tabularnewline \midrule
\tenevilglyph[yes][4]{a_k_j}
	&
	&	теленок \cite{lavrov1969}
	&
	& 	[1.61] \linebreak
		qьleken [ӄԓикин = теленок-бычок возроастом до года] [ИЛИ:2.10]
		\tabularnewline \midrule
\tenevilglyph[yes][4]{a_q}
	&	васонка [важенка] \cite[л. 68 об]{spbfaran79} 
	&	важенка \cite{lavrov1969}
	&
	& 	[25.6об] \linebreak
		rekwьt [rækwut, рэквыт = важенка трех лет и старше] [ИЛИ:2.12]
		\tabularnewline \midrule
\tenevilglyph[yes][4]{a_q_l}
	&	 
	&	
	&
	& 	ванкаскор [ваӈӄасӄор = важенка в возрасте двух лет, яловая важенка] [25.6об] % ваӈӄасӄор
		\tabularnewline \midrule
\tenevilglyph[yes][4]{a_t}
	&	силилас [телилась] \cite[л. 68 об]{spbfaran79} 
	&	
	&	гыргольгын [гыръоԓьын = отелившийся, ощенившийся], отелившаяся самка* [84]
	& 	\cite[362]{davydova2015a} \linebreak
		\cite[26]{lavrov1969} 
		\tabularnewline \midrule
\tenevilglyph[yes][4]{aB}
	&	в «стойбище, олени (богатый оленевод)» \cite[л. 47]{spbfaran79} \linebreak
		табун \cite[л. 55]{spbfaran79} 
	&	reindeer herd \cite{mindalevich1934}\linebreak
		стадо оленей \cite{lavrov1969}
	&	н'элвыл [ӈэԓвыԓ = стадо (преимущественно оленей)], стадо оленей [82]
	& 	\cite[361]{davydova2015a} \linebreak
		\cite[26, 28]{lavrov1969} \linebreak
		тапон [табун] [33.4] \linebreak
		ŋelwьl [ŋælvьl, ӈэԓвыԓ] [ИЛИ:2.7]
		\tabularnewline \midrule
\tenevilglyph[yes][3]{a_o}
	&	
	&	дикий олень \cite{lavrov1969}
	&
	& 	[18.1об?] 
		\tabularnewline \midrule
\tenevilglyph[yes][4]{a_jT}
	&	бик [бык] \cite[л. 68 об]{spbfaran79} 
	&	
	&	чим'ны [чымӈы = старый бык], бык* [82]
	& 	[1.2] \linebreak
		cьmŋь [чымӈы] [ИЛИ:2.5]
		\tabularnewline \midrule
\tenevilglyph[yes][3]{a_2jX}
	&	хромой олень \cite[л. 43]{spbfaran79} 
	&	
	&
	& 	[25.9] \tabularnewline \midrule
\tenevilglyph[yes][1]{b_a}
	&	
	&	
	&
	& 	капетка (каретка?) [?] [29.12] % TODO: проверить, может это тот же знак, что и предыдущий, что-то вроде atka-qor
		\tabularnewline \midrule
\tenevilglyph[yes][4]{a_b}
	&	
	&	
	&	
	& 	пеcвак [пээчвак = молодой олень-самец] [29.11] \linebreak
		печвак [пээчвак] [30.5об] 		
		\tabularnewline \midrule
\tenevilglyph[yes][4]{a_bD}
	&	
	&	
	&	пеечвак [пээчвак = молодой олень-самец], теленок оленя до одного года* [82] % вероятно, ошибка
	& 	tacьmьtь [taacьmentь, таачымынты = олень в возрасте четырех лет] [ИЛИ:2.10] 		
		\tabularnewline \midrule
\tenevilglyph[yes][4]{aE}
	&	
	&	
	&	
	& 	moqor [mooqor, мооӄор = упряжной олень (грузовой)] [ИЛИ:1.3]
		\tabularnewline \midrule
\tenevilglyph[yes][4]{aE_'}
	&	
	&	
	&	
	& 	эkweu [ækwæw, эквэв = левый олень (грузовой)] [ИЛИ:2.10]
		\tabularnewline \midrule
\tenevilglyph[yes][4]{a_jX}
	&	
	&	
	&	каанми [ӄаанмэ = неразделанный убитый олень], каанмыйо [ӄаанмыё = тот, для которого забили, убили оленя], убитый олень, оленья туша [82]
	& 	qaanmj [qaanmi, ӄаанмэ] [ИЛИ:2.13]
		\tabularnewline \midrule
\tenevilglyph[yes][4]{a_lD}
	&	
	&	
	&	самец 2-х лет [84]
	& 	penwel [пэнвэԓ = олень-самец в возрасте двух лет, морж в возрасте двух лет] [ИЛИ:2.10]
		\tabularnewline \midrule
\tenevilglyph[yes][4]{aY}
	&	
	&	
	&	гекенылын [гэкэӈыԓьын = ездок на оленьей упряжке], на оленях [86]
	& 	gakaqor [гакаӈӄор = ездовой олень, правый упряжной олень] [ИЛИ:2.10] % TODO: нужна транскрипция латиницей
		\tabularnewline \midrule
\tenevilglyph[yes][4]{s_b}
	&	много \cite[л. 42]{spbfaran79} \linebreak
		много \cite[л. 37]{spbfaran79} \linebreak
		numkәqin [nьmkәqin, нымкыӄин = многочисленный] \cite[л. 54]{spbfaran79} \linebreak % нымкыӄин
		nьemkәqin [nьmkәqin] \cite[л. 54]{spbfaran79} \linebreak
		nьmkәqin \cite[л. 52 об]{spbfaran79} \linebreak
		мноко [много] \cite[л. 66 об, 67]{spbfaran79}
	&	
	&
	& 	\cite[360–364]{davydova2015a} \linebreak
		\cite[28]{lavrov1969} \linebreak
		\cite{bogoraz1934} \linebreak
		nьmkәqin [нымкыӄин; напечатано в книге, знак рядом] [12.17об] \linebreak
		мноко [много] [34.11] \linebreak
		nьmkьqen [нымкыӄин] [ИЛИ:1.5]
		\tabularnewline \midrule
\tenevilglyph[yes][4]{s_j_b}
	&	
	&	
	&
	& 	nьmkeu [нымкъэв = много; напечатано в книге, знак рядом] [12.16] \linebreak
		nьmkieu [нымкъэв] [ИЛИ:1.7,ИЛИ:2.21] % TODO: нужна транскрипция латиницей
		\tabularnewline \midrule
\tenevilglyph[yes][4]{s_b_jFY}
	&	
	&	
	&
	& 	mьkьciьn [мыкычьын = большинство] [ИЛИ:1.11]
		\tabularnewline \midrule
\tenevilglyph[yes][4]{s_b_jFE}
	&	
	&	
	&	мыкычин [мыкычьын = большинство], многочисленный [56]
	& 	mьkeliьn [мыкыԓьын = многочисленный] [ИЛИ:1.4] % TODO: нужна транскрипция латиницей
		\tabularnewline \midrule
\tenevilglyph[yes][4]{f}
	&	человек, народ \cite[л. 42]{spbfaran79} \linebreak
		человек \cite[л. 53]{spbfaran79} \linebreak
		соловк [человек] \cite[л. 68 об]{spbfaran79} 
	& 	человек \cite{bogoraz1934}\linebreak
		человек \cite{lavrov1969}
	&
	& 	\cite[360, 361, 364]{davydova2015a}\linebreak
		\cite{bogoraz1934} \linebreak
		соловек [человек] [37.7об] \linebreak
		iorawelian [orawetlan, о'равэтԓьан = человек] [ИЛИ:1.3]
		\tabularnewline \midrule
\tenevilglyph[yes][3]{f_4q}
	&	
	&	
	&
	& 	сапране [собрание] [36.1]
		\tabularnewline \midrule
\tenevilglyph[yes][4]{f_c}
	&	
	&	
	&	варат [варат = народ], рэмкын, народ [39] % TODO: нужна транскрипция
	& 	\cite[364]{davydova2015a} \linebreak
		човет [совет] [30.3об] \linebreak
		в «советски» [советский] [34.8об] \linebreak
		warat [варат] [ИЛИ:2.17]
		\tabularnewline \midrule
\tenevilglyph[yes][4]{i_l}
	&	другой \cite[л. 42]{spbfaran79} \linebreak
		другой \cite[л. 53]{spbfaran79} 
	& 	другой \cite{bogoraz1934}
	&	кол [ӄоԓ = другой, один из], другой [49]
	& 	\cite[361–364]{davydova2015a} \linebreak
		\cite{bogoraz1934} \linebreak
		трхой [другой] [32.16] \linebreak
		qol [ӄоԓ] [ИЛИ:1.20]
		\tabularnewline \midrule
\tenevilglyph[yes][4]{i_jF_q}
	&	раньше, ꞓit [cit, чит = раньше, прежде] \cite[л. 42]{spbfaran79} \linebreak % чит
		ꞓit [cit] \cite[л. 52 об, 56]{spbfaran79} 
	&	
	&	чит, раньше [54]
	& 	\cite[364]{davydova2015a} \linebreak
		ceet [cit, чит] [ИЛИ:1.7]
		\tabularnewline \midrule
\tenevilglyph[yes][4]{i_LX}
	&	действительно, qәjve [ӄэйвэ = право, и действительно] \cite[л. 42]{spbfaran79} \linebreak % ӄэйвэ
		qojve [qәjve] \cite[л. 56]{spbfaran79} \linebreak
		qoive [qәjve] \cite[л. 54, 52 об]{spbfaran79}
	& 	действительно \cite{bogoraz1934}
	&	кейве [ӄэйвэ], правда [50]
	& 	\cite[360–362, 364]{davydova2015a} \linebreak
		qejwe [qәjve, ӄэйвэ] [ИЛИ:1.11]
		\tabularnewline \midrule
\tenevilglyph[yes][4]{b_2j}
	&	еще не, jep [еп = еще; пока еще; еще в то время, как] \cite[л. 42]{spbfaran79} \linebreak % еп
		jep \cite[л. 52, 52 об, 56]{spbfaran79}
	& 	еще не \cite{bogoraz1934}
	&	jep, еще, пока еще, еще в то время как, еще не [11]
	& 	\cite[360]{davydova2015a} \linebreak
		jep [напечатано в книге, знак рядом] [12.20об] \linebreak
		eep [jep, еп] [ИЛИ:1.10]
		\tabularnewline \midrule
\tenevilglyph[yes][3]{b_2jF} 
	&	
	&	
	&	валы [ваԓы = нож], рэскын [rәsqәn = боевой нож], нож [114] % TODO: нужна транскрипция
	&	\cite[361]{davydova2015a} \linebreak
		valь [выԓы] [12.15] 
		\tabularnewline \midrule
\tenevilglyph[yes][3]{b_2jF_2q} 
	&	
	&	
	&	торвалы [торвыԓы = новый нож], новый нож [114] % TODO: уточнить транскрипцию
	&	tor-valь [торвыԓы; напечатано в книге, знак рядом] [12.15] % TODO: уточнить транскрипцию
		\tabularnewline \midrule 
\tenevilglyph[yes][4]{2c}
	&	кое как, насилу \cite[л. 42]{spbfaran79} \linebreak
		metkiit [mætkiit, мэткиит = насилу, еле-еле] \cite[л. 39, 52]{spbfaran79} % мэткиит
	&	somehow \cite{mindalevich1934}
	&
	&	\cite{bogoraz1934} \linebreak
		metkeet [mætkiit, мэткиит] [ИЛИ:1.4]
		\tabularnewline \midrule
\tenevilglyph[yes][4]{I_2l}
	&	совсем, qonpь [ӄонпыӈ = совсем, совершенно] \cite[л. 42]{spbfaran79} \linebreak % ӄонпыӈ
		qonpь* \cite[л. 39]{spbfaran79} \linebreak
		совсем \cite[л. 67]{spbfaran79}
	& 	совсем \cite{bogoraz1934}
	&	конпын' [ӄонпыӈ], совсем, вовсе
	& 	\cite[360, 361, 364]{davydova2015a} \linebreak
		\cite[28]{lavrov1969} \linebreak
		qonpь [ӄонпыӈ] [ИЛИ:1.18]
		\tabularnewline \midrule
\tenevilglyph[yes][4]{wD}
	&	довлно [довольно] \cite[л. 68 об]{spbfaran79} 		
	&	
	&	раттанн'авн'эн [rattanŋawŋьn, ратанӈавӈэн = довольно, хватит, достаточно], довольно, хватит, худо, перестаньте [26]
	& 	[33.14] \linebreak
		rataŋauŋen [rattanŋawŋьn, ратанӈавӈэн] [ИЛИ:1.13]
		\tabularnewline \midrule
\tenevilglyph[yes][4]{wD2E}
	&	притом, qәnver [qanver, ӄынвэр = также] \cite[л. 42]{spbfaran79} \linebreak % TODO: нужна транскрипция
		qanver \cite[л. 39, 56]{spbfaran79} \linebreak
		qәnver \cite[л. 52, 56]{spbfaran79} 		
	& 	притом \cite{bogoraz1934}
	&	кынвэр [ӄынвэр], поэтому [26]
	& 	\cite[360, 361]{davydova2015a} \linebreak
		qьnwer [qanver, ӄынвэр] [ИЛИ:1.4]
		\tabularnewline \midrule
\tenevilglyph[yes][4]{2o_2iY}
	&	давно \cite[л. 42]{spbfaran79} \linebreak
		telenjep [tælænjæp, тэԓенъеп = давно] \cite[л. 39 об, 52, 56]{spbfaran79} \linebreak % тэԓенъеп
		давно \cite[л. 66 об]{spbfaran79}
	&	
	&	айгоон [= давно], давно [6]
	& 	\cite[360]{davydova2015a} \linebreak
		telen-jep [тэԓенъеп; напечатано в книге, знак рядом] [12.20о] \linebreak
		тавыно [давно] [30.7об] \linebreak
		teleeip [тэԓенъеп] [ИЛИ:2.2]
		% teleŋken [тэԓеӈкин = старинный, древний] [ИЛИ:1.16] % другой знак
		\tabularnewline \midrule
\tenevilglyph[yes][4]{2o_2iY_j}
	&	
	&	
	&	
	& 	teleŋken [тэԓеӈкин = старинный, древний] [ИЛИ:1.16]
		\tabularnewline \midrule
\tenevilglyph[yes][4]{b_q}
	&	также, ꞓama [cama, чама = тоже] \cite[л. 42]{spbfaran79} \linebreak % чама
		тоже \cite[л. 37]{spbfaran79} \linebreak
		ꞓama [cama] \cite[л. 39 об, 54]{spbfaran79}
	& 	так же, тоже \cite{bogoraz1934}
	&	чама, тоже [113]
	& 	\cite[360, 361, 362, 364]{davydova2015a} \linebreak
		\cite[28]{lavrov1969} \linebreak 
		\cite{bogoraz1934} \linebreak
		cama [чама] [ИЛИ:1.19]
		\tabularnewline \midrule
\tenevilglyph[yes][4]{2i_2cD_2l}
	&	все \cite[л. 42]{spbfaran79} \linebreak	
		ьmьlo [ымыԓьо = весь] \cite[л. 52 об]{spbfaran79} % ымыԓьо
	& 	все \cite{bogoraz1934}\linebreak
		all \cite{mindalevich1934}
	&
	& 	\cite[360, 361, 364]{davydova2015a} \linebreak
		сех [всех] [29.11об] \linebreak
		чех [всех] [29.11об] \linebreak
		ьmьlio [ымыԓьо] [ИЛИ:1.3]
		\tabularnewline \midrule
\tenevilglyph[yes][4]{U_q}
	&	но \cite[л. 42]{spbfaran79} \linebreak	
		naqam [наӄам = но, однако] \cite[л. 39, 52 об, 54, 56]{spbfaran79} % наӄам
	&	
	&	накам [наӄам] (союз), но, к [36]
	& 	\cite[360, 361, 364]{davydova2015a} \linebreak
		нахам [наӄам] [34.8] \linebreak
		naqam [наӄам] [ИЛИ:1.5]
		\tabularnewline \midrule
\tenevilglyph[yes][4]{o_2JY}
	&	хорошо \cite[л. 43]{spbfaran79} \linebreak	
		meꞓәnkь [mæсәnkь, мэчынкы = довольно, достаточно, хорошо, в состоянии] \cite[л. 39, 52]{spbfaran79} \linebreak % мэчынкы
		латно [ладно] \cite[л. 67]{spbfaran79} \linebreak
		хвачит [хватит] \cite[л. 68 об]{spbfaran79}
	&	well \cite{mindalevich1934}
	&	мечынкы [мэчынкы], достаточно [7]
	& 	\cite[360, 361, 364]{davydova2015a} \linebreak
		mecьnk [мэчынкы] [ИЛИ:2.9]
		\tabularnewline \midrule
\tenevilglyph[yes][3]{o_JY_JE}
	&	
	&	
	&	
	& 	әtopel [этъопэԓ = лучше, лучше сделать так; напечатано в книге, знак рядом] [12.25]
		\tabularnewline \midrule
\tenevilglyph[yes][4]{o_2JE}
	&	milgьrә [milgьr, миԓгэр = ружье]* \cite[л. 54]{spbfaran79} \linebreak % миԓгэр
		русёь [ружье] \cite[л. 68 об]{spbfaran79}
	&	
	&	милгэр [миԓгэр], ружье, милгэрыткук [миԓгэрыткук = стрелять], стрелять [133]
	& 	\cite[360, 364]{davydova2015a} \linebreak
		\cite[28]{lavrov1969} \linebreak
		milgьrә [миԓгэр; напечатано в книге, знак рядом] [12.25]
		\tabularnewline \midrule
\tenevilglyph[yes][4]{SMY_iX}
	&	слабый* \cite[л. 43]{spbfaran79} \linebreak
		nьrulqin* [ныруԓӄин = слабый] \cite[л. 52, 52 об]{spbfaran79} \linebreak % ныруԓӄин
		nьrulium* \cite[л. 52 об, 56]{spbfaran79} \linebreak
		nьrulqinet* \cite[л. 39 об]{spbfaran79}
	&	
	&
	& 	\cite[360]{davydova2015a} \linebreak
		* [11.5] \linebreak % TODO: проверить, это вариант написания или другой знак
		nьrulqen [ныруԓӄин] [ИЛИ:1.3]
		\tabularnewline \midrule
\tenevilglyph[yes][3]{S_iX}
	&	не могли* \cite[л. 43]{spbfaran79} \linebreak % вверх ногами опять
		nanilrawcьm* (?) \cite[л. 39]{spbfaran79} % TODO: нужен перевод
	&	
	&
	& 	[33.7]
		\tabularnewline \midrule
\tenevilglyph[yes][3]{i_4l_2l}
	&	тосковать* \cite[л. 43]{spbfaran79} 
	&	
	&
	& 	[9.3] 
		\tabularnewline \midrule % тут вверх ногами
\tenevilglyph[yes][4]{i_4l}
	&	
	&	
	&	чимгун [чимгъун = ум, разум, дума], гемгогыргын, разум [125] % TODO: нужна транскрипция
	& 	ом [ум] [37.2] \linebreak
		cemion [cimŋun, чимгъун] [ИЛИ:1.9]
		\tabularnewline \midrule
\tenevilglyph[yes][4]{U2EN_JX}
	&	лето наступило \cite[л. 43]{spbfaran79} \linebreak	
		elerurkьn = лето начинается [ælærurkьn = лето начинается] \cite[л. 52 об]{spbfaran79} \linebreak % TODO: нужна транскрипция 
		лет [лето] \cite[л. 66]{spbfaran79}
	&	
	&	элен'ит, элеэл, лето [60] % TODO: нужна транскрипция 
	& 	\cite[362]{davydova2015a} \linebreak
		\cite[28]{lavrov1969} \linebreak
		летым [летом] [30.6]
		\tabularnewline \midrule
\tenevilglyph[yes][4]{U_JX_3'}
	&	симои [зимой] \cite[л. 66]{spbfaran79}
	&	
	&	леэлен' [ԓьэԓеӈ = зима], зима
	& 	семои [зимой] [34.11] \linebreak
		lieleŋkь [lәlæŋqь, ԓьэԓеӈкы = зимой] [ИЛИ:1.6]
		\tabularnewline \midrule
\tenevilglyph[yes][4]{U_JX_j}
	&	
	&	
	&	кыткыт [= поздняя весна], анон, весна [60] % TODO: нужен перевод
	& 	kьtkьtьk [кыткытык = весной; напечатано в книге, знак рядом] [12.11, 12.20об] % TODO: нужна транскрипция латиницей
		\tabularnewline \midrule
\tenevilglyph[yes][4]{2O}
	&	пусть, чтобы, opopo [опопы = пусть] \cite[л. 43]{spbfaran79} \linebreak % опопы
		пусть \cite[л. 53]{spbfaran79} \linebreak
		opopo \cite[л. 52 об]{spbfaran79} 
	&	
	&	опопы, пусть [123]
	& 	\cite[364]{davydova2015a} \linebreak
		опопы [34.8] \linebreak
		opopь [опопы] [ИЛИ:2.3]
		\tabularnewline \midrule
\tenevilglyph[yes][4]{o_3iS}
	&	пускай, maꞓьnan [macьnan, мачынан = пусть] \cite[л. 43]{spbfaran79} \linebreak % мачынан
		maꞓьnan [мачынан] \cite[л. 52 об, 56]{spbfaran79} \linebreak
		мачнан [мачынан] \cite[л. 68]{spbfaran79} 
	&	
	&	мачынан, пускай, пусть [46]
	& 	\cite[364]{davydova2015a} \linebreak
		\cite{bogoraz1934} \linebreak
		почкаи [пускай] [30.7] \linebreak
		macьnan [мачынан] [ИЛИ:1.4]
		\tabularnewline \midrule
\tenevilglyph[yes][4]{u-o_b}
	&	как же, miŋkri [миӈкыри = куда, как] \cite[л. 43]{spbfaran79} \linebreak % миӈкыри TODO: перепроврить
		miŋkri-lu \cite[л. 56]{spbfaran79} \linebreak % НУЖЕН ПЕРЕВОД ԓымиӈкыри? миӈкыри-?
		кута [куда] \cite[л. 66]{spbfaran79} \linebreak
		как \cite[л. 66 об]{spbfaran79} \linebreak
		в «какои» \cite[л. 66]{spbfaran79} 
	&	
	&	мин'кри [миӈкыри], куда [114]
	& 	\cite[364]{davydova2015a} \linebreak
		miŋkri [миӈкыри; напечатано в книге, знак рядом] [12.25] \linebreak
		mekьrelь [ИЛИ:2.25] % TODO: нужна транскрипция
		\tabularnewline \midrule
\tenevilglyph[yes][1]{u-o}
	&	аке [?] \cite[л. 68]{spbfaran79}
	&	
	&
	& 	[25.9] \linebreak
		mekь [miŋkь = где] [ИЛИ:2.2]
		\tabularnewline \midrule
\tenevilglyph[yes][2]{U_iX_b}
	&	завидовать \cite[л. 43]{spbfaran79}
	&	divide the reindeer herd \cite{mindalevich1934}
	&
	& 	[12.1об]
		\tabularnewline \midrule
\tenevilglyph[yes][4]{i_2kU_2kD}
	&	шкура \cite[л. 44]{spbfaran79} \linebreak
		nelgen [nælgьn, нэԓгын = шкура]* \cite[л. 49 об]{spbfaran79} \linebreak % нэԓгын
		писик [пыжик] \cite[л. 68]{spbfaran79}
	&	
	&
	& 	\cite[364]{davydova2015a} \linebreak
		wlqьnalgьn [выԓгынаԓгын = тонкошерстная летняя шкура оленя, неблюй] [ИЛИ:1.19]
		\tabularnewline \midrule
\tenevilglyph[yes][3]{i_2kU_kD_2Q}
	&	неторос [недоросль] \cite[л. 68]{spbfaran79} 
	&	
	&
	& 	nlgьn [nælgьn, нэԓгын = шкура] [ИЛИ:1.6]
		\cite[364]{davydova2015a} 
		\tabularnewline \midrule
\tenevilglyph[yes][3]{i_2kU_kD_2Q_iX}
	&	посела [постель (шкура взрослого оленя)] \cite[л. 68]{spbfaran79} 
	&	
	&
	& 	[32.17об]
		\tabularnewline \midrule
\tenevilglyph[yes][1]{i_kU_b_3Q_c}
	&	саски [?] \cite[л. 68]{spbfaran79} 
	&	
	&
	& 	\cite[364]{davydova2015a} 
		\tabularnewline \midrule
\tenevilglyph[yes][3]{k_o_oN}
	&	випороток [выпороток] \cite[л. 68]{spbfaran79} 
	&	
	&
	& 	[1.1] \tabularnewline \midrule
\tenevilglyph[yes][4]{2k}
	&	мука \cite[л. 44]{spbfaran79} \linebreak
		мука \cite[л. 66 об]{spbfaran79}
	& 	мука \cite{bogoraz1934}
	&	пин'вытрын [пиӈвытрын = мука], пинмэкычгын, мука [14] % TODO: нужна транскрипция 
	& 	[25.6]
		\tabularnewline \midrule
\tenevilglyph[yes][4]{vY_z}
	&	русский \cite[л. 44]{spbfaran79} 
	&	Russian holding fire-arms \cite{mindalevich1934} \linebreak 
		русский* \cite{lavrov1969}
	&	тангыт, руссилын [русиԓьын = русский (народ)], русский [119] % TODO: нужна транскрипция 
	& 	\cite[364]{davydova2015a} \linebreak
		melgьtaŋь [melgьt-tanŋьtan, мэԓгытанӈын = русский] [ИЛИ:1.2]
		\tabularnewline \midrule
\tenevilglyph[yes][4]{a_vY_z}
	&	
	&	
	&	каатангыт, совхоз (русские олени) [83] \linebreak % TODO: нужна транскрипция 
		совхоз (русский олень) [119]
	& 	совхос [совхоз] [32.13] \linebreak % \cite[148]{sergeev1956} «мельгитаньги чаучуа» — русские оленные люди
		оленсохоси [оленсовхоз] [34.11об]
		\tabularnewline \midrule
\tenevilglyph[yes][4]{bD_b_vY_z}
	&	
	&	Ленин \cite{lavrov1969}
	&	Ленин [118]
	& 	ленн [Ленин] [34.9] % часть «ԓыгэн» тут, видимо, в качестве фонетической компоненты 
		\tabularnewline \midrule
\tenevilglyph[no][3]{zR_v}
	&	следы \cite[л. 45]{spbfaran79} 
	& 	следы \cite{bogoraz1934}
	&	вины [= след], след [117]
	& 	\tabularnewline \midrule
\tenevilglyph[yes][4]{c_2cD_q}
	&	волк \cite[л. 45, 53]{spbfaran79} \linebreak
		волк \cite[л. 68 об]{spbfaran79} \linebreak
		в «убил \textit{волка»} \cite[л. 68 об]{spbfaran79}
	& 	волк \cite{bogoraz1934}\linebreak
		волк \cite{lavrov1969}
	&	пинны, эгичгын [э'гычгын = волк (волчище)], волк [50] % TODO: нужна транскрипция
	& 	\cite[360]{davydova2015a} \linebreak
		воки [волки] [34.12]
		\tabularnewline \midrule
\tenevilglyph[yes][4]{J_b_i}
	&	мтвет [медведь] \cite[л. 68 об]{spbfaran79}
	&	
	&	кейнын [кэйӈын = медведь (бурый)], медведь бурый [86]
	& 	мысвеси [медведи] [34.12]
		\tabularnewline \midrule
\tenevilglyph[yes][4]{J_b_2b_c}
	&	
	&	
	&	
	& 	umqь [umqæ, умӄы = белый медведь; напечатано в книге, знак рядом] [12.16]
		\tabularnewline \midrule
\tenevilglyph[yes][4]{I-IE_'} 
	&	
	&	
	&
	& 	рысомака [росомаха] [34.12]
		\tabularnewline \midrule
\tenevilglyph[yes][4]{2CY} % С лисами сплошная путаница
	&	reqokalgьn [рэӄокаԓгын = лисица, песец] \cite[л. 54]{spbfaran79} % рэӄокаԓгын
	&	песец \cite{lavrov1969}
	&	рекокалгын [рэӄокаԓгын], элгар [эԓгар = песец], песец [97]
	& 	[11.3] \linebreak
		лесесея [лисица] [34.12]
		\tabularnewline \midrule
\tenevilglyph[yes][3]{2CY_c} 
	&	голубой песец \cite[л. 46]{spbfaran79} 
	& 	голубой песец \cite{bogoraz1934}
	&	кытгым [= соболь], голубой песец [97]
	& 	[1.14об] %\linebreak
		%riqukete [?; напечатано в книге, знак рядом] [12.20] % TODO: нужен перевод, уточнить транскрипцию; вероятно, форма рэӄокаԓгын
		\tabularnewline \midrule
\tenevilglyph[no][2]{2CY_2c} 
	&	песец \cite[л. 45]{spbfaran79} \linebreak
		сорнпур [чернобурая] \cite[л. 69 об]{spbfaran79} 
	&	
	&
	& 	\tabularnewline \midrule
\tenevilglyph[yes][3]{2CY_cFD} 
	&	красная лисица* \cite[л. 45]{spbfaran79} \linebreak
		лисита [лисица] \cite[л. 69 об]{spbfaran79}
	&	
	&
	& 	[11.3] 
		\tabularnewline \midrule
\tenevilglyph[yes][2]{2CY_o_I_3q} 
	&	огневка \cite[л. 45]{spbfaran79} \linebreak
		песеч [песец] \cite[л. 69 об]{spbfaran79}
	&	
	&
	& 	[1.14об]
		\tabularnewline \midrule
\tenevilglyph[no][2]{2CY_o_I_3q_c} 
	&	чернобурая \cite[л. 45]{spbfaran79} \linebreak
		кулубои [голубой] \cite[л. 69 об]{spbfaran79}
	&	
	&
	& 	\tabularnewline \midrule
\tenevilglyph[no][3]{2CY_o_I_3q_2jF} 
	&	сиводушка \cite[л. 45]{spbfaran79}
	&	
	&
	& 	\tabularnewline \midrule
\tenevilglyph[yes][4]{2cF_k_2qY} 
	&	заяц \cite[л. 46]{spbfaran79} \linebreak
		melitolgьn [melotalgьn, мэԓётаԓгын = заяц] \cite[л. 54]{spbfaran79} % мэԓётаԓгын
	& 	заяц \cite{bogoraz1934}\linebreak
		a hare \cite{mindalevich1934}
	&	мелёталгын [мэԓётаԓгын], заяц [23]
	& 	milutet [миԓютэт = зайцы; напечатано в книге, знак рядом] [7.3] \linebreak
		~[рядом с изображением кроликов] [11.1]
		\tabularnewline \midrule
\tenevilglyph[yes][4]{v-_jF}
	&	кострула [кастрюля] \cite[л. 68]{spbfaran79}
	&	
	&	кукэн'ы [кукэӈы = котел, кастрюля, ведро], котел [52]
	& 	\cite[364]{davydova2015a} \linebreak
		в «7 ветра» [7 ведер] [34.19об]
		\tabularnewline \midrule
\tenevilglyph[no][3]{O_v}
	&	тас [таз] \cite[л. 66]{spbfaran79}
	&	
	&
	& 	\tabularnewline \midrule
\tenevilglyph[no][3]{O_v_vD}
	&	в «кастрюлька» \cite[л. 46]{spbfaran79}
	& 	кастрюля \cite{bogoraz1934}
	&
	& 	\tabularnewline \midrule
\tenevilglyph[no][3]{O_v_2jF}
	&	тарелка \cite[л. 46]{spbfaran79}
	& 	тарелка \cite{bogoraz1934}
	&	вычгоккам, илгуккэм, тарелка [133] % TODO: нужна транскрипция 
	& 	\tabularnewline \midrule
\tenevilglyph[yes][3]{i_c_c_2j}
	&	
	& 	свечка \cite{bogoraz1934}
	&	пин'инэн [пиӈинэӈ = лучина, свеча], свеча [97]
	& 	\cite[364]{davydova2015a}
		\tabularnewline \midrule
\tenevilglyph[yes][3]{R_o-o}
	&	молоко \cite[л. 49]{spbfaran79} 
	& 	молоко \cite{bogoraz1934}
	&	молоко (Б. Т. 69, 70) [109]
	& 	[4.4об]
		\tabularnewline \midrule
\tenevilglyph[yes][3]{R_o-o_2j}
	&	молоко \cite[л. 49]{spbfaran79} 
	& 	молоко \cite{bogoraz1934}
	&	молоко (Б. Т. 69, 70) [109]
	& 	[2.101]
		\tabularnewline \midrule
\tenevilglyph[no][3]{R_o-o_2b}
	&	банка с керосином \cite[л. 46]{spbfaran79} 
	& 	банка с керосином \cite{bogoraz1934}
	&	банка с керосином (Б. Т. 68) [109]
	& 	\tabularnewline \midrule
\tenevilglyph[no][3]{R-o-o_3iS_'}
	&	банка с жиром \cite[л. 46]{spbfaran79} 
	& 	банка с жиром \cite{bogoraz1934}
	&	банка с жиром (Б. Т. 66) [109]
	& 	\tabularnewline \midrule
\tenevilglyph[yes][3]{R_o-o_c_zR}
	&	банка с салом (с маслом) \cite[л. 46]{spbfaran79} 
	& 	банка с салом \cite{bogoraz1934}
	&	банка с салом (Б. Т. 68) [109]
	& 	[4.1]
		\tabularnewline \midrule
\tenevilglyph[yes][3]{R_o-o_2CE}
	&	банка с сахаром \cite[л. 49]{spbfaran79} 
	&	
	&	кэнтикэй, конфеты [98] % TODO: нужна транскрипция 
	& 	[4.7]
		\tabularnewline \midrule
\tenevilglyph[yes][4]{C_c_zR} 
	&	в «олений \textit{жир} это» \cite[л. 46]{spbfaran79}
	&	
	&
	& 	[4.5об] \linebreak
		ьpalgьn [ыпаԓгын = топленый жир] [ИЛИ:1.15]
		\tabularnewline \midrule
\tenevilglyph[yes][4]{c_2b}
	&	белый \cite[л. 46]{spbfaran79} \linebreak
		пелои [белый] \cite[л. 68]{spbfaran79}
	& 	белый \cite{bogoraz1934}
	&	нильгыкин [ниԓгыӄин = белый], белый [113]
	& 	\cite[360, 364]{davydova2015a} \linebreak
		\cite[28]{lavrov1969}
		\tabularnewline \midrule
\tenevilglyph[yes][4]{o-o_J}
	&	маленький \cite[л. 46]{spbfaran79} \linebreak
		nьpluqin [ныппыԓюӄин = маленький] \cite[л. 46]{spbfaran79} % ныппыԓюӄин
	& 	маленький \cite{bogoraz1934}
	&	ныппылюкин [ныппыԓюӄин], маленький-ая-ое [34]
	& 	\cite[360]{davydova2015a} \linebreak
		малйнкй [маленький] [37.7об] \linebreak
		nьpьluqen [ныппыԓюӄин] [ИЛИ:1.3]
		\tabularnewline \midrule
\tenevilglyph[no][3]{o-o_J_2q}
	&	ысовая [дешевая] \cite[л. 69 об]{spbfaran79} \linebreak
	& 	
	&	
	& 	
		\tabularnewline \midrule
\tenevilglyph[yes][3]{O_bN}
	&	
	& 	крадет \cite{bogoraz1934}
	&
	&	\cite{bogoraz1934}
		\tabularnewline \midrule
\tenevilglyph[yes][4]{U_bN}
	&	вороватый \cite[л. 47]{spbfaran79} 
	& 	вороватый \cite{bogoraz1934}
	&
	&	\cite{bogoraz1934} \linebreak
		tulerkьnin [= украл; напечатано в книге, знак рядом] [12.13об] % TODO: нужен перевод, нужна транскрипция
		\tabularnewline \midrule
\tenevilglyph[yes][4]{i_G}
	&	хороший \cite[л. 47]{spbfaran79} \linebreak
		хоросой [хороший]* \cite[л. 66, 68 об]{spbfaran79} 
	& 	хороший \cite{bogoraz1934}
	&	нытэн'кин [нытэӈӄин = славный, добрый, хороший], хороший [64]
	& 	\cite[360, 364]{davydova2015a} \linebreak
		\cite{bogoraz1934} \linebreak
		хоросо [хорошо] [33.4] \linebreak
		nьteqen [нытэӈӄин] [ИЛИ:2.23] % TODO: нужна транскрипция латиницей
		\tabularnewline \midrule
\tenevilglyph[yes][4]{i_J}
	&	лучей [лучший] \cite[л. 66 об]{spbfaran79} \linebreak
		со споконо [?] \cite[л. 67 об]{spbfaran79} \linebreak
	&	
	&	
	& 	nьmeleu [нымԓьав = ловко, юрко, проворно] [12.17] \linebreak % TODO: нужна транскрипция латиницей
		nьmelieu [нымԓьав] [ИЛИ:2.1] \linebreak 
		nьmliu [нымԓьав] [ИЛИ:2.12]
		\tabularnewline \midrule
\tenevilglyph[yes][3]{i_o_G}
	&	
	& 	мастероватый \cite{bogoraz1934}
	&	нытэминн'ыкин [нытэминӈыӄин = умелый, умелец, мастер], мастер [64]
	&	\cite{bogoraz1934} \linebreak
		[25.13об]
		\tabularnewline \midrule
\tenevilglyph[yes][3]{i_G_b}
	&	поправилас [поправилась] \cite[л. 66 об]{spbfaran79}
	&	
	&
	& 	[25.13]
		\tabularnewline \midrule
\tenevilglyph[yes][1]{i_G_bX}
	&	блно [?] \cite[л. 66]{spbfaran79}
	&	
	&
	& 	[4.8] \linebreak
		jьqajiьm [ИЛИ:1.18] % TODO: нужен перевод
		\tabularnewline \midrule
\tenevilglyph[yes][4]{i_G_cD}
	&	
	&	
	&	кавэты [= удобно, уютно], удобно [64]
	& 	kawetь [кавэты] [ИЛИ:1.5]
		\tabularnewline \midrule
\tenevilglyph[yes][4]{BD}
	&	худой \cite[л. 47]{spbfaran79} \linebreak
		хутои [худой] \cite[л. 68 об]{spbfaran79} 
	& 	худой \cite{bogoraz1934}
	&	этки [э'тки = плохо, скверно], плохой [10]
	& 	\cite[364]{davydova2015a} \linebreak 
		\cite{bogoraz1934} \linebreak
		etqi [э'тки;  напечатано в книге, знак рядом] [12.18] \linebreak
		iэтке [э'тки] [34.8] \linebreak % э'тки
		плохо [34.11]
		\tabularnewline \midrule
\tenevilglyph[yes][4]{BD_cD}
	&	
	&	
	&
	& 	плохои [плохой] [37.2, 37.2об]
		\tabularnewline \midrule
\tenevilglyph[yes][3]{O_jN}
	&	ночью \cite[л. 47]{spbfaran79} 
	&	
	&	никитэ [ныкитэ = ночью], ночью [65]
	& 	\cite[360, 362]{davydova2015a} 
		\tabularnewline \midrule
\tenevilglyph[yes][4]{2o_2j}
	&	в «стойбище, олени (богатый оленевод)» \cite[л. 47]{spbfaran79} \linebreak
		в «на стойбище» \cite[л. 53]{spbfaran79} \linebreak
		отлакир [әtlьq-, эԓԓыӄ- = тундра] \cite[л. 68]{spbfaran79} % эԓԓыӄ- , вероятно, ошибка
	&	
	&	нымным [= поселок, селение], селение [121]
	& 	\cite[364]{davydova2015a} \linebreak
		nьmnьm [нымным] [ИЛИ:2.10]
		\tabularnewline \midrule
\tenevilglyph[yes][3]{2o_2j_JFE}
	&	
	&	
	&	рэмкын* [народ, толпа] [121] % *«гребенка» отдельно
	& 	remkьn [ræmkьn, рэмкын; напечатано в книге, знак рядом] [12.12об]
		\tabularnewline \midrule
\tenevilglyph[yes][4]{2o_2j_a}
	&	
	&	
	&	
	& 	çawçuwaçgьn [= оленеводы; напечатано в книге, знак рядом] [12.12об] \linebreak % TODO: уточнить перевод, нужна транскрипция
		cawcu [чавчыв = богатый оленями, оленевод, оленный] [ИЛИ:2.14]
		\tabularnewline \midrule
\tenevilglyph[no][3]{i_j_jF}
	&	ложка \cite[л. 48]{spbfaran79}
	& 	ложка \cite{bogoraz1934}
	&
	& 	\tabularnewline \midrule
\tenevilglyph[yes][4]{u_p}
	&	чайник \cite[л. 48]{spbfaran79} \linebreak
		саиник [чайник] \cite[л. 53]{spbfaran79}
	& 	чайник \cite{bogoraz1934}
	&	чайкок [= чайник], чайник [79]
	& 	\cite[364]{davydova2015a}
		\tabularnewline \midrule
\tenevilglyph[yes][3]{u_p_b}
	&	белый чайник \cite[л. 48]{spbfaran79} 
	& 	белый чайник \cite{bogoraz1934}
	&
	& 	\cite[364]{davydova2015a}
		\tabularnewline \midrule
\tenevilglyph[no][3]{u_pD_bD}
	&	медный чайник \cite[л. 48]{spbfaran79} 
	& 	медный чайник \cite{bogoraz1934}
	&
	& 	\tabularnewline \midrule
\tenevilglyph[yes][3]{u_p_2b}
	&	широкий чайник \cite[л. 48]{spbfaran79} 
	&	
	&
	& 	\cite[364]{davydova2015a}
		\tabularnewline \midrule
\tenevilglyph[yes][4]{jFY_jF}
	&	ремень* \cite[л. 48]{spbfaran79} \linebreak
		ремен [ремень] \cite[л. 66 об]{spbfaran79}
	&	
	&
	& 	[32.2об] \linebreak
		ырымен [ремень] [29.11об] \linebreak
		ŋelgьn [ИЛИ:2.27] % TODO: нужна транскрипция 
		\tabularnewline \midrule
\tenevilglyph[yes][4]{O_jXX} % TODO: проверить размер
	&	тюленья шкура \cite[л. 48]{spbfaran79} \linebreak
		нерпа \cite[л. 66 об]{spbfaran79}
	& 	тюленья шкура \cite{bogoraz1934}
	&	тюленья шкура (Б. Т. 6) [25]
	& 	[1.13] \linebreak
		tenugьn [тэнуйгын = тюленья шкура]* [ИЛИ:2.27] % зеркально
		\tabularnewline \midrule
\tenevilglyph[yes][2]{O_2b}
	&	шкура лахтака* \cite[л. 48]{spbfaran79} \linebreak
		лахтак \cite[л. 66 об]{spbfaran79}
	&	
	&
	& 	потоска [?] [29.11об]
		\tabularnewline \midrule
\tenevilglyph[no][3]{O_2b_c_zR}
	&	морча [морж] \cite[л. 66 об]{spbfaran79}
	&	
	&
	& 	\tabularnewline \midrule
\tenevilglyph[yes][4]{O_jXX_2b}
	&	
	&	
	&
	& 	kultalgьn [qoltalgьn, коԓтаԓгын = шкура лахтака] [ИЛИ:2.27]
		\tabularnewline \midrule
\tenevilglyph[yes][4]{O_jXXE}
	&	
	&	морж \cite{lavrov1969}
	&	рыркы [= морж], морж [24]
	& 	rьrkь [рыркы; напечатано в книге, знак рядом] [12.12] \linebreak
		rьrken [рыркэн = моржовый; напечатано в книге, знак рядом] [12.12об]
		\tabularnewline \midrule
\tenevilglyph[yes][3]{O_jXX_C_c}
	&	
	&	нерпа \cite{lavrov1969}
	&	мемыл [мэмыԓ = тюлень, нерпа], нерпа (тюлень) [23]
	& 	memьl [мэмыԓ; напечатано в книге, знак рядом] [14.10]
		\tabularnewline \midrule
\tenevilglyph[yes][3]{O_jXX_2zRX}
	&	
	&	
	&	унел [унъэԓ = лахтак], лахтак (морской заяц), Erignathus barbatus [23]
	& 	% unele [унъэԓ; напечатано в книге, знак рядом] [12.11об] \linebreak
		unel [unæl, унъэԓ; напечатано в книге, знак рядом] [12.12]
		\tabularnewline \midrule
\tenevilglyph[yes][3]{O_jXX-2b}
	&	
	&	
	&	келилин [кеԓиԓьын = нерпа], пестрая нерпа [24]
	& 	kelilьt [кеԓиԓьыт = нерпы; напечатано в книге, знак рядом] [12.22] % TODO: нужна транскрипция латиницей
		\tabularnewline \midrule
\tenevilglyph[yes][4]{2CE}
	&	сахар \cite[л. 44, 49]{spbfaran79}
	&	
	&	чакар [= сахар], сахар [97]
	& 	[25.6] \linebreak
		caqar [чакар] [ИЛИ:1.15]
		\tabularnewline \midrule
\tenevilglyph[no][3]{I_q} 
	&	трубка \cite[л. 49]{spbfaran79} 
	& 	трубка \cite{bogoraz1934} \linebreak
		a pipe \cite{mindalevich1934}
	&	таакойнын [таа'койӈын = курительная трубка], трубка (курительная) [49]
	& 	\tabularnewline \midrule
\tenevilglyph[no][3]{I_q_UE_JX}
	&	папироска \cite[л. 49]{spbfaran79} 
	& 	папироска \cite{bogoraz1934}
	&	сигрис [чиӄырич = папироса, сигарета], папироса [61]
	& 	\tabularnewline \midrule
\tenevilglyph[no][3]{I_q_UE_JX_b_q}
	&	трубка с мундштуком \cite[л. 49]{spbfaran79} 
	&	
	&
	& 	\tabularnewline \midrule
\tenevilglyph[yes][4]{UE_JX} 
	&	
	&	
	&	кэликэл [кэԓикэԓ = рисунок, картина], книга, письмо, рисунок [60]
	& 	\cite[364]{davydova2015a} \linebreak
		kelekel [kælikæl, кэԓикэԓ] [4.10об] \linebreak % кэԓикэԓ
		в «писат» [писать] [32.6] \linebreak
		в «песмо» [письмо] [34.8об] \linebreak
		в «касет» [газета] [34.8об]
		\tabularnewline \midrule
\tenevilglyph[yes][4]{UE_JX_j_q} 
	&	
	&	
	&
	& 	еенки [деньги] [34.1] \linebreak
		keletiul [kælitul, кэԓитъуԓ = бумага, деньги] [ИЛИ:2.25]
		\tabularnewline \midrule
\tenevilglyph[yes][2]{l_JXE} % знаки на л. 50 и л. 66 зеркальны
	&	не мог* \cite[л. 50]{spbfaran79} \linebreak
		нимнок [?] \cite[л. 66 об]{spbfaran79}
	&	
	&	тъытлен [тъытԓьэн = больной], больной [71]
	& 	[9.1]
		\tabularnewline \midrule
\tenevilglyph[yes][4]{cF_CF}
	&	так \cite[л. 50]{spbfaran79} \linebreak
		әnmen [энмэн = также, итак] \cite[л. 39 об]{spbfaran79} \linebreak % энмэн, возможно, ошибка
		дак [так] \cite[л. 66 об]{spbfaran79}
	&	
	&	ынн'ын, ынн'от [ынӈин, ынӈот = так, такой], так, эдак [54]
	& 	\cite[360, 361, 364]{davydova2015a} \linebreak
		\cite[26, 28]{lavrov1969} \linebreak
		етак [так] [36.1] \linebreak
		ьŋen [ынӈин] [ИЛИ:1.3] \linebreak
		ьnŋen [ынӈин] [ИЛИ:1.6]
		\tabularnewline \midrule
\tenevilglyph[yes][4]{o_jX}
	&	так себе, attaw [а'тав = ну, что же, напрасно, зря] \cite[л. 50]{spbfaran79} \linebreak % а'тав
		attaw \cite[л. 52 об]{spbfaran79} \linebreak
		всетакии (нука) (?) \cite[л. 53]{spbfaran79} 
	& 	так себе \cite{bogoraz1934}
	&	атав [а'тав], аттав,  давай, ну (подбадривающее междометие) % TODO: нужна транскрипция 
	& 	\cite[361]{davydova2015a} \linebreak
		так [32.6] \linebreak
		iatau [а'тав] [ИЛИ:1.2]
		\tabularnewline \midrule % [25.12]
\tenevilglyph[yes][1]{o_qX_f}
	&	
	&	
	&
	& 	iатаро [?] [34.10] \linebreak
		iataru [ИЛИ:2.9] % TODO: нужен перевод
		\tabularnewline \midrule % [25.12]
\tenevilglyph[yes][3]{i_2j}
	&	долстои [толстый]* \cite[л. 69 об]{spbfaran79} % TODO: убедиться, что l и j тут одно и то же
	&	
	&
	& 	\cite[364]{davydova2015a} \linebreak
		\cite[28]{lavrov1969} 
		\tabularnewline \midrule
\tenevilglyph[yes][4]{i_2j_iSY}
	&	несмотря на то, что \cite[л. 50]{spbfaran79}
	&	
	&	эквырга* [э'квырга = однако же, тем не менее], однако [90]
	& 	\cite[360]{davydova2015a} \linebreak
		iekwьrga* [ekwurga, э'квырга] [ИЛИ:1.9] 
		\tabularnewline \midrule
\tenevilglyph[yes][3]{B_2BD}
	&	быть \cite[л. 50]{spbfaran79} 
	&	
	&
	& 	\cite[364]{davydova2015a} 
		\tabularnewline \midrule
\tenevilglyph[yes][4]{o_L}
	&	мало \cite[л. 50]{spbfaran79} \linebreak
		kitkit [киткит = мало, немного] \cite[л. 39 об]{spbfaran79} % киткит
	&	
	&	кит-кит [киткит], чуть чуть [46]
	& 	\cite[360, 361, 364]{davydova2015a} \linebreak
		немноско [немножко] [34.11] \linebreak
		ketket [киткит] [ИЛИ:1.17]
		\tabularnewline \midrule
\tenevilglyph[yes][2]{oI_vD}
	&	вероятно \cite[л. 50]{spbfaran79} \linebreak
		наверно \cite[л. 67]{spbfaran79}
	&	
	&	этым [эты-ым = значит, наверное], наверно [95]
	& 	\cite[364]{davydova2015a} \linebreak
		анако [?] [33.4] \linebreak
		эtiьm [эты-ым] [ИЛИ:1.5]
		\tabularnewline \midrule
\tenevilglyph[yes][4]{bD_b}
	&	только \cite[л. 50]{spbfaran79} \linebreak
		lien [liæn, ԓыгэн = как только] \cite[л. 52 об, 56]{spbfaran79} % ԓыгэн
	& 	только \cite{bogoraz1934}
	&	лыгэн [ԓыгэн], инстинно, несомненно, только [113]
	& 	\cite[361, 364]{davydova2015a} \linebreak
		\cite[28]{lavrov1969} \linebreak
		lien [liæn; напечатано в книге, знак рядом] [12.20об]
		lьen [liæn, ԓыгэн] [ИЛИ:2.5]
		\tabularnewline \midrule
\tenevilglyph[yes][4]{u_2k_uN_2k}
	&	спали (ушли?) \cite[л. 50]{spbfaran79} % судя по всему "спали"
	&	
	&
	& 	[25.9] \linebreak
		ычпау [спал?] [30.6] \linebreak
		спас [спать?] [33.4] \linebreak
		nejlqetqen [ИЛИ:1.8] % TODO: нужна транскрипция
		\tabularnewline \midrule
\tenevilglyph[yes][3]{cU_2q_cD_2q}
	&	словно, как бы \cite[л. 50]{spbfaran79} \linebreak
		qajaagьtkь \cite[л. 52 об]{spbfaran79} % TODO: нужен перевод
	&	
	&
	& 	\cite[360–362, 364]{davydova2015a} \linebreak
		rajarkьnen [ИЛИ:1.5] % TODO: нужен перевод 
		\tabularnewline \midrule
\tenevilglyph[yes][3]{i_oB}
	&	тайно \cite[л. 50]{spbfaran79} \linebreak
		wьnvь [vinvә, винвэ = тайно, крадучись] \cite[л. 56]{spbfaran79} % винвэ
	& 	тайный \cite{bogoraz1934}
	&	винвыкин [винвэткин = тайный], тайный* [68]
	& 	\cite[364]{davydova2015a} \linebreak
		\cite{bogoraz1934}
		\tabularnewline \midrule
\tenevilglyph[yes][4]{i_u} 
	&	посиму [почему] \cite[л. 66 об]{spbfaran79}
	&	
	&	
	& 	ejam [iam, ьjam, ийъам = почему] [5.1] \linebreak
		посиму [почему] [37.2об] \linebreak
		ijam [ийъам] [ИЛИ:1.4] \linebreak
		iejam [ийъам] [ИЛИ:1.6]
		\tabularnewline \midrule
\tenevilglyph[yes][4]{c_J}
	&	в поле \cite[л. 50]{spbfaran79} \linebreak
		nutek [нутэк = на земле] \cite[л. 56]{spbfaran79} % нутэк
	& 	в поле \cite{bogoraz1934}\linebreak
		земля \cite{lavrov1969}
	&	нутэнут [= нутен = тундра, земля, страна], земля [67]
	& 	\cite[360]{davydova2015a} \linebreak
		\cite[28]{lavrov1969} \linebreak
		nutenut [нутэнут] [ИЛИ:2.1]
		\tabularnewline \midrule
\tenevilglyph[yes][4]{c_J_2j}
	&	nutesqan [nutæsqәn, нутэсӄын = земля, почва] \cite[л. 39]{spbfaran79} % нутэсӄын
	&	
	&	нутэкин [= земляной, тундровый], земной, полевой [67]
	& 	\cite[362, 364]{davydova2015a} \linebreak
		\cite[28]{lavrov1969} \linebreak
		nutecqьn [nutæsqәn, нутэсӄын] [ИЛИ:1.12]
		\tabularnewline \midrule
\tenevilglyph[yes][4]{O_cN_JN}
	&	
	&	
	&	камлелы нутэнут, вокруг земли [67] % TODO: нужна транскрипция, нужен перевод
	& 	\cite[364]{davydova2015a} \linebreak
		kamleleь [камԓеԓыӈ = вокруг] [ИЛИ:1.12]
		\tabularnewline \midrule
\tenevilglyph[yes][3]{i_2bX}
	&	сразу \cite[л. 51]{spbfaran79} \linebreak
		awetuwaq [авэтываӄ = быстро, проворно, сразу] \cite[л. 56]{spbfaran79} % авэтываӄ
	& 	сразу \cite{bogoraz1934}
	&	авэтывак [авэтываӄ], сразу [126]
	& 	[7.7] % [2.1?]* 
		\tabularnewline \midrule
\tenevilglyph[yes][4]{o_m_j}
	&	мама \cite[л. 51, 37]{spbfaran79} \linebreak
		мама \cite[л. 67]{spbfaran79} 
	&	
	&	ыммэмы [= мама (в речи близких людей)], ыммэй [=  мама (при обращении)], мама [121]
	& 	\cite[362]{davydova2015a} \linebreak
		\cite[28]{lavrov1969} \linebreak
		мама [33.5об] \linebreak
		ьmeqj [әmmәqәj = мать, кормилица] [ИЛИ:1.9] % TODO: нужна транскрипция
		\tabularnewline \midrule
\tenevilglyph[yes][4]{B_b_oX}
	&	остров Кулючин \cite[л. 51]{spbfaran79} \linebreak
		на Колючено \cite[л. 37]{spbfaran79} 
	&	
	&
	& 	\cite[360]{davydova2015a} 
		\tabularnewline \midrule
\tenevilglyph[yes][4]{UD_i_2l}
	&	покочевали* \cite[л. 51]{spbfaran79} % вверх ногами
	&	кочевать* \cite{lavrov1969}
	&	ялгытык [яԓгытык = кочевать], кочевать [103]
	& 	[25.8об] \linebreak
		ялхытык [яԓгытык] [34.8] \linebreak % яԓгытык
		покочоеоч [покочуешь] [30.5об]
		\tabularnewline \midrule
\tenevilglyph[yes][4]{i_2iY}
	&	вверх \cite[л. 51]{spbfaran79} 
	& 	вверх \cite{bogoraz1934}
	&	гыргоягты, вверх [51] % TODO: нужна транскрипция
	& 	\cite[361]{davydova2015a} \linebreak
		grguca [гыргоча = выше, над, поверх чего-л., вверх (по реке)] [ИЛИ:2.6]
		\tabularnewline \midrule
\tenevilglyph[yes][4]{i_o_iY}
	&	
	& 	
	&	
	& 	эeuca [euca, эвыча = внизу] [ИЛИ:2.6]
		\tabularnewline \midrule
\tenevilglyph[yes][4]{u_v_CD}
	&	вполне \cite[л. 51]{spbfaran79} \linebreak
		arala [аръаԓя = совсем, вовсе] \cite[л. 52]{spbfaran79} % аръаԓя
	&	
	&
	& 	\cite[361, 364]{davydova2015a} \linebreak
		\cite[28]{lavrov1969} \linebreak
		ariala [аръаԓя] [ИЛИ:1.9]
		\tabularnewline \midrule
\tenevilglyph[yes][4]{cF-cF}
	&	тоже, опять \cite[л. 51]{spbfaran79} \linebreak
		опять \cite[л. 53]{spbfaran79} 
	& 	тоже, опять \cite{bogoraz1934}
	&	неме [нэмэ = опять, снова], опять [19]
	& 	\cite[361, 362]{davydova2015a} \linebreak
		апес [опять] [33.4]
		\tabularnewline \midrule
\tenevilglyph[yes][2]{c_cD_'} 
	&	кута [?] \cite[л. 66 об]{spbfaran79}
	&	
	&	
	& 	camiam [camam, чамъам = не мочь] [ИЛИ:2.1] 
		\tabularnewline \midrule
\tenevilglyph[yes][4]{oF_2l_lG}
	&	близко \cite[л. 51, 53]{spbfaran79} \linebreak
		ꞓьmꞓә [cьmcь, чымче = близко] \cite[л. 54]{spbfaran79} \linebreak % чымче
		плиско [близко] \cite[л. 68 об]{spbfaran79}
	&	
	&
	& 	\cite[364]{davydova2015a} \linebreak 
		\cite{bogoraz1934} \linebreak
		cьmce [cьmcь, чымче] [ИЛИ:1.20]
		\tabularnewline \midrule
\tenevilglyph[yes][4]{cU_2cD}
	&	после того, әŋqorә [ынӄоры = потом, затем, оттуда] \cite[л. 51, 53]{spbfaran79} \linebreak
		әnqre \cite[л. 39]{spbfaran79} 
	& 	после того \cite{bogoraz1934}
	&	ынкоры [ынӄоры], после того, оттуда, потом [127]
	& 	\cite[361, 362, 364]{davydova2015a} \linebreak
		\cite[28]{lavrov1969} \linebreak
		ьnqurь [ынӄоры] [ИЛИ:1.5]
		\tabularnewline \midrule
\tenevilglyph[yes][3]{2cU_cD_jFY}
	&	
	& 	
	&	
	& 	\cite[364]{davydova2015a} \linebreak
		qonьrьm [ӄоныръым = между прочим, притом, впрочем, да и] [12.20об]
		\tabularnewline \midrule
\tenevilglyph[yes][4]{o_2q}
	&	1 \cite[л. 64]{spbfaran79} \linebreak
		amunen [әnnæn?, ыннэн? = один] \cite[л. 39 об]{spbfaran79} % ыннэн
	&	1 \cite{lavrov1969}
	&	ыннэн, 1, один, одна, одно [43] % TODO: нужна транскрипция
	& 	1 \cite[360, 362]{davydova2015a} \linebreak
		\cite[361, 364]{davydova2015a} \linebreak
		\cite[26]{lavrov1969} 
		\tabularnewline \midrule
\tenevilglyph[yes][4]{o_2q_j}
	&	20 \cite[л. 64]{spbfaran79} 
	&	20 \cite{lavrov1969}
	&	кликкин, 20, двадцать [44] % TODO: нужна транскрипция
	& 	20 \cite[360, 362]{davydova2015a} \linebreak
		\cite[361, 363]{davydova2015a} \linebreak
		\cite[26]{lavrov1969}
		\tabularnewline \midrule
\tenevilglyph[yes][4]{i_b_s_j_o_2q}
	&	
	&	1000* \cite{lavrov1969}
	&	тытлынча, мынгытклеккан, тысяча [57] % TODO: нужна транскрипция
	& 	1000 [25.1об] 
		\tabularnewline \midrule
\tenevilglyph[yes][4]{i_b_s_j_o_q_j}
	&	
	&	
	&
	& 	20000 [36.2] \tabularnewline \midrule
\tenevilglyph[yes][4]{B-}
	&	2 \cite[л. 64]{spbfaran79} \linebreak
		двоих \cite[л. 68]{spbfaran79}
	&	2 \cite{lavrov1969}
	&	н'ирэк [ӈирэк = два], 2, два, две, двое [115]
	& 	2 \cite[360, 362]{davydova2015a} \linebreak
		\cite[361, 363, 364]{davydova2015a} \linebreak
		\cite[28]{lavrov1969} 
		\tabularnewline \midrule
\tenevilglyph[yes][4]{B-_j}
	&	
	&	40 \cite{lavrov1969}
	&	н'ирэккликкин [ӈирэӄӄԓиккин = сорок], 40, сорок [114]
	& 	40 \cite[360]{davydova2015a} 
		\tabularnewline \midrule
\tenevilglyph[yes][4]{B-_2oI_jF_j}
	&	
	&	400 \cite{lavrov1969}
	&	кликликкин, 400, четыреста [122] % TODO: нужна транскрипция
	& 	[25.2] 
		\tabularnewline \midrule
\tenevilglyph[yes][4]{i_b_s_j_B-}
	&	
	&	
	&
	& 	2000 [36.2] 
		\tabularnewline \midrule
\tenevilglyph[yes][4]{o_2q_q_l}
	&	три \cite[л. 41]{spbfaran79} \linebreak
		ŋьroq [ӈыроӄ = три] \cite[л. 39]{spbfaran79} \linebreak % ӈыроӄ
		3 \cite[л. 64]{spbfaran79}
	&	3 \cite{lavrov1969}
	&	н'ырок, 3, три [43] % TODO: нужна транскрипция
	& 	3 \cite[360, 362]{davydova2015a} \linebreak
		\cite[361, 363, 364]{davydova2015a} 
		\tabularnewline \midrule
\tenevilglyph[yes][4]{o_2q_q_l_j}
	&	
	&	60 \cite{lavrov1969}
	&	н'ырокклеккэн, 60, шестдесят [44] % TODO: нужна транскрипция
	& 	60 \cite[360]{davydova2015a} \linebreak
		\cite[26]{lavrov1969} 
		\tabularnewline \midrule
\tenevilglyph[yes][3]{o_q_q_l_2oI_jF_j}
	&	
	&	600 \cite{lavrov1969}
	&	кликкликан мынгытклеккэн, пароль [?], 600, шестьсот [122] % TODO: нужна транскрипция
	& 	[11.4об]
		\tabularnewline \midrule
\tenevilglyph[yes][3]{i_b_s_j_o_q_q_l}
	&	
	&	
	&
	& 	[3000] [32.13об] 
		\tabularnewline \midrule
\tenevilglyph[yes][4]{o_q_c_T}
	&	4 \cite[л. 64]{spbfaran79}
	&	4 \cite{lavrov1969}
	&	н'ырак, 4, четыре [43] % TODO: нужна транскрипция
	& 	4 \cite[360]{davydova2015a} \linebreak
		\cite[361]{davydova2015a} \linebreak
		\cite[26]{lavrov1969} 
		\tabularnewline \midrule
\tenevilglyph[yes][3]{o_q_c_T_j}
	&	
	&	80 \cite{lavrov1969}
	&	н'ыракклеккен, 80, восемдесят [44] % TODO: нужна транскрипция
	& 	[25.4]
		\tabularnewline \midrule
\tenevilglyph[yes][3]{o_c_T_2oI_jF_j}
	&	
	&	800 \cite{lavrov1969}
	&	н'ыроча мынгытклеккэн, 800, восемьсот [122] % TODO: нужна транскрипция
	& 	[25.4] 
		\tabularnewline \midrule
\tenevilglyph[yes][4]{oI_2j}
	&	5 \cite[л. 64]{spbfaran79}
	&	5 \cite{lavrov1969}
	&	мытлын'эн, 5, пять [43] % TODO: нужна транскрипция
	& 	5 \cite[360]{davydova2015a} \linebreak
		\cite[361, 364]{davydova2015a} 
		\tabularnewline \midrule
\tenevilglyph[yes][3]{i_b_s_j_oI_2j}
	&	
	&	5000 \cite{lavrov1969}
	&
	& 	[15.9]
		\tabularnewline \midrule
\tenevilglyph[yes][4]{oI_3j}
	&	
	&	100 \cite{lavrov1969}
	&	мытлын'клеккэн, 100, сто [44] % TODO: нужна транскрипция
	& 	\cite[361]{davydova2015a} \linebreak
		100 [34.19]
		\tabularnewline \midrule
\tenevilglyph[yes][4]{o-_q_jF_o}
	&	6 \cite[л. 64]{spbfaran79}
	&	6 \cite{lavrov1969}
	&	ыннанмытлын'эн, 6, шесть [43] % TODO: нужна транскрипция
	& 	6 \cite[360]{davydova2015a}
		\tabularnewline \midrule
\tenevilglyph[yes][4]{o-_q_jF_o_j}
	&	
	&	
	&	120, мытлын'клеккэн кликкин, сто двадцать [45] % TODO: нужна транскрипция
	& 	120 [34.20об]
		\tabularnewline \midrule
\tenevilglyph[yes][4]{o_j_2q}
	&	7 \cite[л. 64]{spbfaran79}
	&	7 \cite{lavrov1969}
	&	н'эрэкмытлын'эн, 7, семь [43] % TODO: нужна транскрипция
	& 	7 \cite[360]{davydova2015a} \linebreak
		\cite[361]{davydova2015a}
		\tabularnewline \midrule
\tenevilglyph[yes][4]{o_j_2q_j}
	&	
	&	
	&	130, мытлын'клеккэн н'ирэккликкин, сто сорок [45] % TODO: нужна транскрипция
	& 	140 [2.101] 
		\tabularnewline \midrule
\tenevilglyph[yes][4]{o-_2q_j}
	&	8 \cite[л. 64]{spbfaran79}
	&	8 \cite{lavrov1969}
	&	амн'ырооткэн, 8, восемь [44] % TODO: нужна транскрипция
	& 	8 \cite[360]{davydova2015a} 
		\tabularnewline \midrule
\tenevilglyph[yes][4]{o-_2q_j_j}
	&	
	&	
	&	160, мытлын'клеккэн н'ырокклеккэн [45] % TODO: нужна транскрипция
	& 	в «161» [160] [32.15] 
		\tabularnewline \midrule
\tenevilglyph[yes][4]{o_2q_jN_jF_o}
	&	9 \cite[л. 64]{spbfaran79}
	&	9 \cite{lavrov1969}
	&	коначгын'кэн, 9, девять [44] % TODO: нужна транскрипция
	& 	9 \cite[360]{davydova2015a} 
		\tabularnewline \midrule
\tenevilglyph[yes][3]{o_2q_jN_jF_o_j}
	&	
	&	
	&	180, мытлын'клеккэн н'ыракклеккэн, сто восемдесят [45] % TODO: нужна транскрипция
	& 	[180] [25.3об] 
		\tabularnewline \midrule
\tenevilglyph[yes][4]{2oI_2jF}
	&	10 \cite[л. 64]{spbfaran79}
	&	10 \cite{lavrov1969}
	&	мынгыткэн, 10, десять [122] % TODO: нужна транскрипция
	& 	10 \cite[360]{davydova2015a} \linebreak
		\cite[361, 363]{davydova2015a} \linebreak
		\cite[26]{lavrov1969} 
		\tabularnewline \midrule
\tenevilglyph[yes][3]{2oI_2jF_j}
	&	
	&	200 \cite{lavrov1969}
	&	мынгытклеккен, 200, двести [122] % TODO: нужна транскрипция
	& 	[25.3об] 
		\tabularnewline \midrule
\tenevilglyph[yes][4]{o_T_2q_2o_l}
	&	
	&	15 \cite{lavrov1969}
	&	15, кылгынкэн [45] % TODO: нужна транскрипция
	& 	15 \cite[360]{davydova2015a} \linebreak 
		\cite[361]{davydova2015a} 
		\tabularnewline \midrule
\tenevilglyph[yes][4]{o_T_2q_2o_l_j} 
	&	
	&	
	&
	& 	[300] \cite[26]{lavrov1969} \linebreak % Как 15, но с крючком, обычно обозначающим 20. Кроме того, на «лунном календаре», в контексте, видимо, числа, где есть этот знак, 60, 4 и 1, то есть если это в самом деле 15*20=300, то сумма будет 365
		в 301, 302, 303 и т. д. [300] [4.9об]
		\tabularnewline \midrule
\tenevilglyph[yes][4]{CD_CDY}
	&	тем не менее, wenlьgь [vænligi, вэнԓыги = тем не менее] \cite[л. 42]{spbfaran79} \linebreak % вэнԓыги
		wenlьgь [vænligi] \cite[л. 52 об]{spbfaran79} \linebreak
		силно [сильно?] \cite[л. 66 об]{spbfaran79} 
	&	
	&
	&	\cite{bogoraz1934} \linebreak
		wlьe [vænligi, вэнԓыги] [ИЛИ:1.4]
		\tabularnewline \midrule
\tenevilglyph[yes][4]{UD_2c}
	&	мэсяч [месяц] \cite[л. 66]{spbfaran79} 
	&	
	&	йилгын [йъиԓгын = луна, месяц], чатам, луна, месяц [62] % TODO: нужна транскрипция
	& 	\cite[362]{davydova2015a} \linebreak
		\cite[26, 28]{lavrov1969} \linebreak
		мэсыс [месяц] [34.19] \linebreak
		jelgьn [jilgьn, йъиԓгын = луна, месяц] [ИЛИ:1.13]
		\tabularnewline \midrule
\tenevilglyph[yes][3]{o_8q}
	&	
	&	the sun \cite{mindalevich1934}
	&
	& 	[25.8об] \linebreak
		jьnjьn [йынйын = огонь, пламя; напечатано в книге, знак рядом] [12.21]
		\tabularnewline \midrule
\tenevilglyph[yes][4]{o_7q_Q}
	&	сонсо [солнце] \cite[л. 66]{spbfaran79} 
	&	солнце \cite{lavrov1969}
	&	элон'эт [ы'ԓёӈэт = день], течение дня, ыле [ы'ԓён = день] — день, тиркытир [= солнце] — солнце [131] 
	& 	\cite[361, 364]{davydova2015a}
		еен [день] [34.11об, 34.16об]
		\tabularnewline \midrule
\tenevilglyph[yes][4]{o_7q_L}
	&	
	&	
	&
	& 	утырым [утром] [34.19об]
		\tabularnewline \midrule
\tenevilglyph[yes][4]{o_7q_LE}
	&	
	&	
	&	омом [= тепло, жара], чиитэв, тепло [130] % TODO: уточнить перевод
	& 	в «сопло» [тепло] [34.8]
		\tabularnewline \midrule
\tenevilglyph[yes][1]{o_O_8qX}
	&	
	&	
	&
	& 	секраиче [?] [30.2об]
		\tabularnewline \midrule
\tenevilglyph[yes][3]{rI_l_b}
	&	топор \cite[л. 68 об]{spbfaran79} 
	&	
	&
	& 	\cite[364]{davydova2015a} 
		\tabularnewline \midrule
\tenevilglyph[yes][2]{c_2k}
	&	
	&	
	&	валын [ваԓьын = живущий], сущий [103] % TODO: нужна транскрипция
	& 	eun [ИЛИ:2.7]
		\tabularnewline \midrule
\tenevilglyph[yes][4]{c_c_2k}
	&	лодка \cite[л. 68 об]{spbfaran79} 
	&	
	&	этвет [ы'твъэт = байдара, лодка], судно, лодка [103]
	& 	\cite[361]{davydova2015a} \linebreak
		әttwet [әtwьt, ы'твъэт; напечатано в книге, знак рядом] [12.16]
		\tabularnewline \midrule
\tenevilglyph[yes][3]{C_pF_c_2k}
	&	
	&	
	&	тевылын, гребец [103] % TODO: уточнить перевод, нужна транскрипция
	& 	teula [= гребцы; напечатано в книге, знак рядом] [12.16] % TODO: уточнить перевод, нужна транскрипция
		\tabularnewline \midrule
\tenevilglyph[yes][3]{i_2j_l}
	&	донкои [тонкий] \cite[л. 69 об]{spbfaran79} 
	&	
	&
	& 	[25.4] 
		\tabularnewline \midrule
\tenevilglyph[yes][4]{i_2c}
	&	курба [крупа] \cite[л. 68 об]{spbfaran79} 
	&	
	&
	& 	\cite[361, 364]{davydova2015a} \linebreak
		корпаи [крупой] [29.11об] \linebreak
		в «макаро» [макарон] [29.11об]
		\tabularnewline \midrule
\tenevilglyph[yes][4]{u_2l}
	&	ŋagcьnьn \cite[л. 64 об]{spbfaran79} \linebreak  % TODO: нужен перевод, нужна транскрипция
		в собки [в сопки] \cite[л. 68 об]{spbfaran79}
	&	
	&
	& 	\cite[361]{davydova2015a} \linebreak
		~[30.8об] \linebreak
		ŋagcьŋьn [= скала; напечатано в книге, знак рядом] [12.23] % TODO: нужен перевод, нужна транскрипция
		\tabularnewline \midrule
\tenevilglyph[yes][4]{uF_2l} % TODO: убедиться, что это не то же самое, что предыдущий, пока похоже, что нет, в 30.8об встречается, но лучше найти на одном листе
	&	
	&	
	&	кымэк [ӄымэк = чуть-чуть, почти, едва не, чуть не], почти [110]
	& 	чочо [чуть-чуть] [30.7об] \linebreak
		сосо	[чуть-чуть] [34.11] \linebreak
		qьmek [ӄымэк] [ИЛИ:1.5] % TODO: нужна транскрипция латиницей
		\tabularnewline \midrule
\tenevilglyph[yes][4]{i_jX}
	&	 ?...gite...* \cite[л. 39 об]{spbfaran79} % TODO:  нужна расшифровка, нужен перевод
	&	
	&
	& 	\cite[360, 362, 364]{davydova2015a} \linebreak
		rinut [ръэнут = что, что-нибудь] [ИЛИ:1.15] \linebreak % TODO: нужна транскрипция латиницей
		riэnut [ръэнут] [ИЛИ:1.2]
		\tabularnewline \midrule
\tenevilglyph[no][1]{i_jX_o}
	&	 mьgitegәn [?] \cite[л. 64 об]{spbfaran79} % TODO: нужен перевод
	&	
	&
	& 	\tabularnewline \midrule
\tenevilglyph[yes][4]{i_jX_z}
	&	ime renut [imь-rәnut, имыръэнут = что угодно] \cite[л. 51]{spbfaran79} % имыръэнут
	&	
	&	имырэнут [имыръэнут], всякая всячина [117] 
	& 	\cite[364]{davydova2015a} \linebreak
		ime-r\={ә}nut [имыръэнут; напечатано в книге, знак рядом] [12.24] \linebreak
		iэmьrienut [имыръэнут] [ИЛИ:1.8]
		\tabularnewline \midrule
\tenevilglyph[yes][4]{i_jX_2z}
	&	
	&	
	&	имыкееречин [имы- = всякий, каждый; кээрэчьын = муха, всякое живое существо (кроме человека), шероховатость],  все живое [117] % TODO: уточнить перевод 
	& 	\cite[28]{lavrov1969} \linebreak
		iэmьkььrecien [имыкээрэчьын] [ИЛИ:1.8] % TODO: уточнить транскрипцию, нужна транскрипция латиницей
		\tabularnewline \midrule
\tenevilglyph[yes][4]{i_jX_z_c-l}
	&	
	&	
	&	кимитын [кимитъын = товар, продукция], товар, груз, кладь [118]
	& 	в «канпенат» [34.10] \linebreak
		в «канпинати» [37.5] \linebreak
		kemetiьk [ИЛИ:1.7] \linebreak % TODO: нужна транскрипция
		kemetiьn [кимитъын] [ИЛИ:1.7]
		\tabularnewline \midrule
\tenevilglyph[yes][4]{U_qD}
	&	камусы \cite[л. 37]{spbfaran79} 
	&	
	&	панралгын [панраԓгын = камус], камус, шкура, содранная с ног оленя [78]
	& 	\cite[362, 364]{davydova2015a} 
		\tabularnewline \midrule
\tenevilglyph[yes][4]{U_qD_b}
	&	рукавицы \cite[л. 37]{spbfaran79} 
	&	
	&	лилит [ԓиԓит = варежки], рукавицы [79]
	& 	\cite[362]{davydova2015a} 
		\tabularnewline \midrule
\tenevilglyph[yes][4]{sE}
	&	юукула [юкола]* \cite[л. 68 об]{spbfaran79} 
	&	юкола сушеная рыба \cite{lavrov1969}
	&	тевел [тэвъэԓ = юкола], юкола (сушеная рыба) [88]
	& 	\cite[361]{davydova2015a} \linebreak
		tewel [tawal, тэвъэԓ; напечатано в книге, знак рядом] [12.13об]
		\tabularnewline \midrule
\tenevilglyph[yes][4]{sE_jFE}
	&	всала [взяла] \cite[л. 68 об]{spbfaran79} \linebreak
		в «я \textit{восму»} \cite[л. 66]{spbfaran79}
	&	
	&	нийгулеткин, учиться [74] % похоже на ошибку, TODO: нужна транскрипция
	& 	\cite[360]{davydova2015a} \linebreak
		pirirkьn [= берет; напечатано в книге, знак рядом] [12.23об] \linebreak % TODO: уточнить перевод, нужна транскрипция
		в «опере» [берет] [30.7]
		\tabularnewline \midrule
\tenevilglyph[yes][4]{sE_jFE_qY}
	&	
	&	
	&
	& 	в «ырыпота» [работа] [35.1] \linebreak
		в «рыпосе» [работе] [34.18об]
		\tabularnewline \midrule
\tenevilglyph[yes][3]{sE_jFE_qYE}
	&	
	&	
	&
	& 	meьcergьrgьn [мэгчергыргын = опыт работы?] [ИЛИ:1.6] % TODO: уточнить перевод
		\tabularnewline \midrule
\tenevilglyph[yes][3]{w_j}
	&	оцин [очень] \cite[л. 66]{spbfaran79} \linebreak
		в «я \textit{оцин} боюс», «я \textit{оцин} писпокоюс» \cite[л.66]{spbfaran79}
	&	
	&
	& 	\cite[364]{davydova2015a} 
		\tabularnewline \midrule
\tenevilglyph[yes][4]{w_j_'}
	&	
	&	
	&
	& 	jarat [йъарат = очень, весьма, слишком, особенно; напечатано в книге, знак рядом] [12.19об] \linebreak
		jarat [йъарат] [ИЛИ:1.6. ИЛИ:1.18]
		\tabularnewline \midrule
\tenevilglyph[yes][4]{B}
	&	вместе \cite[л. 55]{spbfaran79} 
	&	
	&	кынмал [кынмаԓ = быть вместе], вместе [36] % кынмаԓ
	& 	\cite[360, 364]{davydova2015a} \linebreak
		kьnmal [кынмаԓ] [ИЛИ:2.12]
		\tabularnewline \midrule
\tenevilglyph[yes][4]{B_2qX}
	&	
	&	
	&	винрэтык [= помогать], помогать [36]
	& 	помокаи [помогай] [30.8]
		\tabularnewline \midrule
\tenevilglyph[yes][2]{SFE_jF}
	&	нарта [?] \cite[л. 68]{spbfaran79} 
	&	
	&	эвыр [= если], если, и [88]
	& 	\cite[360, 361, 364]{davydova2015a} \linebreak
		eur [эвыр; напечатано в книге, знак рядом] [12.18об] \linebreak
		eur [эвыр] [ИЛИ:1.5]
		\tabularnewline \midrule
\tenevilglyph[yes][4]{O_L_q}
	&	
	&	
	&	ёо [= пурга], чеченёо , непогода, холодный ветер [47] % TODO: нужна транскрипция
	& 	cacajokь [чьэчеӈэюк = начало зимы] [ИЛИ:1.19] \linebreak
		joo [ёо] [ИЛИ:1.5]
		\tabularnewline \midrule
\tenevilglyph[yes][4]{O_L_l}
	&	
	&	
	&	
	& 	cieceŋkь [чьэчеӈкы = во время мороза] [ИЛИ:1.19] % TODO: нужна транскрипция латиницей
		\tabularnewline \midrule
\tenevilglyph[yes][4]{O_L_qE}
	&	доз [дождь] \cite[л. 68]{spbfaran79} 
	&	
	&	элыёо, ветер со снегом [47] % TODO: нужна транскрипция
	& 	\cite[361, 364]{davydova2015a} \linebreak
		ilel [иԓииԓ = дождь] [ИЛИ:2.8] \linebreak
		ŋlgqn [?] [5.1об] % TODO: найти перевод
		\tabularnewline \midrule
\tenevilglyph[yes][3]{O_L_2q}
	&	холот [холод] \cite[л. 66]{spbfaran79} 
	&	холодный ветер (в~тексте) \cite{lavrov1969}
	&
	& 	 \cite[26]{lavrov1969} 
		\tabularnewline \midrule
\tenevilglyph[no][3]{O_L}
	&	бурка [пурга] \cite[л. 68 об]{spbfaran79} 
	&	
	&
	& 	 \tabularnewline \midrule
\tenevilglyph[yes][4]{i_SX}
	&	alьmь* [аԓымы = положим, что] \cite[л. 52 об]{spbfaran79} % аԓымы?
	&	
	&	alьmь(ŋ) (союз), положим, что… [35] 
	& 	\cite[361, 364]{davydova2015a} \linebreak
		alьmь [аԓымы] [ИЛИ:1.4]
		\tabularnewline \midrule
\tenevilglyph[yes][4]{2C_2c} 
	&	
	&	вода \cite{lavrov1969}
	&	мимыл, вода [100] % TODO: нужна транскрипция
	& 	\cite[364]{davydova2015a} \linebreak 
		\cite[26, 28]{lavrov1969} \linebreak
		вотой [водой] [32.15об] \linebreak
		вота [вода] [30.6] \linebreak
		memьl [mimьl, мимыԓ] [ИЛИ:2.8]
		\tabularnewline \midrule
\tenevilglyph[yes][4]{2C_2c_q} 
	&	
	&	
	&	ан'кы [аӈӄы = море], море [101] 
	& 	aŋqajpu [аӈӄэпы = из моря;  напечатано в книге, знак рядом] [12.24 об] % TODO: уточнить перевод, уточнить транскрипцию
		\tabularnewline \midrule
\tenevilglyph[yes][3]{2C_2c_q_z} 
	&	
	&	
	&	
	& 	әweneьrgьn [эвэнэгыргын = охота на морского зверя;  напечатано в книге, знак рядом] [12.25] % TODO: уточнить перевод, уточнить транскрипцию
		\tabularnewline \midrule
\tenevilglyph[yes][4]{2kU_2QY} 
	&	
	&	снег \cite{lavrov1969}
	&	ыльыль [ы'ԓьыԓ], лежащий снег [7] 
	& 	\cite[361, 364]{davydova2015a} \linebreak
		ычнек [снег] [30.6] \linebreak
		iliьl [ы'ԓьыԓ] [ИЛИ:2.7] % TODO: добавить транскрипцию латиницей
		\tabularnewline \midrule
\tenevilglyph[yes][3]{U_ux} 
	&	витал [видал] \cite[л. 67 об, 68 об]{spbfaran79}
	&	
	&
	& 	\cite[360, 364]{davydova2015a} \linebreak
		lūәn [= видел; напечатано в книге, знак рядом][12.25] \linebreak % TODO: уточнить перевод, нужна транскрипция
		вечым [?] [30.6об]
		\tabularnewline \midrule
\tenevilglyph[no][3]{U_ux_j} 
	&	нивидал [не видал] \cite[л. 66 об]{spbfaran79}
	&	
	&
	& 	\tabularnewline \midrule
\tenevilglyph[yes][3]{V_2l_i_2q} 
	&	крепкои [крепкий] \cite[л. 69 об]{spbfaran79}
	&	
	&	нъомрыкэн, крепкий, прочный (о мехе) [30] % TODO: нужна транскрипция
	& 	\cite[28]{lavrov1969} 
		\tabularnewline \midrule
\tenevilglyph[no][3]{V_l_lU_i_q_qU} 
	&	нирепкои [некрепкий] \cite[л. 69 об]{spbfaran79}
	&	
	&
	& 	\tabularnewline \midrule
\tenevilglyph[yes][4]{v_i_2CX} 
	&	
	&	приходить, приезжать \cite{lavrov1969}
	&	пикирык, прихожу, приходит [30] % TODO: нужна транскрипция
	& 	\cite[360]{davydova2015a} \linebreak
		\cite[26]{lavrov1969} \linebreak
		прлехалй [приехали] [32.13об]
		\tabularnewline \midrule
\tenevilglyph[yes][4]{i_i_bX} 
	&	ŋocьm [ŋocьn, ӈъочьын = бедняк] \cite[л. 39 об]{spbfaran79} % ӈъочьын
	& 	богатый \cite{bogoraz1934} % явная ошибка
	&	н'ъочын [ӈъочьын = бедняк], н'ъочыян, бедный [10]
	& 	петнаска [бедняжка] [34.8об] \linebreak
		петнак [бедняк] [30.3об] \linebreak
		ŋiociьn [ӈъочьын] [ИЛИ:2.14]
		\tabularnewline \midrule
\tenevilglyph[yes][4]{oEN_q} 
	&	goymьcьl(?) [gajmьcьjьn, гаймычьыԓьын = богач] \cite[л. 39 об]{spbfaran79} % гаймычьыԓьын
	& 	бедный \cite{bogoraz1934} % явная ошибка
	&	гаймычыллын [гаймычьыԓьын], богатый [95]
	& 	ьgamьciьliьn [гаймычьыԓьын] [ИЛИ:2.6]
		\tabularnewline \midrule
\tenevilglyph[yes][3]{2i_2iX_4q} 
	&	прсол [пришел] \cite[л. 68 об]{spbfaran79}
	&	
	&	qeetlin % TODO: нужен перевод, ужна транскрипция
	& 	\cite[361]{davydova2015a} \linebreak
		geetlin [?; напечатано в книге, знак рядом] [12.19об] % TODO: нужен перевод, нужна транскрипция
		еееот [?] [30.6] \linebreak
		пырыеехали [приехали] [30.6об]
		\tabularnewline \midrule
\tenevilglyph[yes][3]{2i_iX_2q_cF_jF} 
	&	прныси [принеси] \cite[л. 68 об]{spbfaran79}
	&	
	&
	& 	[4.3об] 
		\tabularnewline \midrule
\tenevilglyph[yes][4]{i_CD} 
	&	
	&	
	&	тэльпыйе, кончился, тэльпык [тэԓпык = кончаться (завершаться), пройти (о месяце)], кончаться [54] % TODO: нужна транскрипция
	& 	telpьie [ИЛИ:1.24] % TODO: нужна транскрипция
		\tabularnewline \midrule
\tenevilglyph[yes][4]{i_CD_2jF} 
	&	длко [только] \cite[л. 68]{spbfaran79}
	&	
	&
	& 	\cite[364]{davydova2015a} \linebreak
		толко [только] [34.11об] \linebreak
		ьriэc [әrræc, ытръэч = только, всё (конец)] [ИЛИ:1.6]
		\tabularnewline \midrule
\tenevilglyph[yes][4]{uD_jN} 
	&	кус [гусь] \cite[л. 66]{spbfaran79}
	&	
	&
	& 	\cite[28]{lavrov1969} \linebreak
		gatlь [гатԓе = птица, утка, промысловая водяная птица] [12.17об]
		\tabularnewline \midrule
\tenevilglyph[yes][3]{i_u_uD} 
	&	
	&	
	&	
	& 	ejŋei [= крикнул;  напечатано в книге, знак рядом] [12.22об] % TODO: уточнить перевод, явно от ejŋæ = хрипеть, нужна транскрипция
		\tabularnewline \midrule
\tenevilglyph[yes][4]{i_u_uD_b} 
	&	grep [græp, грэп = песня] \cite[л. 64 об]{spbfaran79} % грэп
	&	
	&	греп [грэп], песня [125]
	& 	grep [грэп] [12.23; напечатано в книге, знак рядом] \linebreak
		поеот* [поёт] [36.1]
		\tabularnewline \midrule
\tenevilglyph[yes][4]{i_u_uD_k_r} 
	&	
	&	
	&
	& 	кармоска [гармошка] [36.1]
		\tabularnewline \midrule
\tenevilglyph[yes][4]{oF_oN_z} 
	&	колова [голова] \cite[л. 68]{spbfaran79}
	&	
	&
	& 	\cite[364]{davydova2015a} \linebreak
		leut [læwt, ԓевыт = голова; напечатано в книге, знак рядом] [12.12об] \linebreak
		leut [ԓевыт] [ИЛИ:1.10]
		\tabularnewline \midrule
\tenevilglyph[yes][4]{o_jN_m} 
	&	
	&	
	&	яранга, чукотское жилище [106]
	& 	\cite[363,364]{davydova2015a} \linebreak
		яранка [яранга] [29.2об] \linebreak
		ярнка [яранга] [29.2об] \linebreak
		еярнка [яранга] [29.2]
		\tabularnewline \midrule
\tenevilglyph[yes][4]{o_lN_l} % значок почти наверняка другой, но в чем разница — непонятно
	&	
	&	
	&	
	& 	jaraŋь [яраӈы = яранга] [ИЛИ:1.12]
		\tabularnewline \midrule
\tenevilglyph[yes][3]{o_jN_m_z} 
	&	домои [домой] \cite[л. 66 об]{spbfaran79}
	&	
	&
	& 	\cite[363]{davydova2015a} 
		\tabularnewline \midrule
\tenevilglyph[yes][3]{o_lN_l_2jF}
	&	
	&	
	&	% яракача [?], около дома [107] % TODO: нужна транскрипция; больше похоже еа другой знак со 106, но там крючок сверху другой
	& 	tьttьl [тытыԓ = дверь, вход; напечатано в книге, знак рядом] [14.10] \linebreak
		товерес [?] [37.7об] % TODO: нужна интерпретация
		\tabularnewline \midrule
\tenevilglyph[yes][4]{iE_b_i} 
	&	мало \cite[л. 67]{spbfaran79}
	&	
	&	мъэркин, киткит [= немного, чуть-чуть], мало [56] % TODO: нужна транскрипция
	& 	\cite[361]{davydova2015a} \linebreak
		мало [29.11об, 30.6] \linebreak
		tierken [тъэркин = мало, немного, недостаточно] [ИЛИ:2.16] % TODO: нужна транскрипция латиницей
		\tabularnewline \midrule
\tenevilglyph[yes][4]{iE_b_i_jL} 
	&	
	&	
	&
	& 	немночко [немножко] [30.7об]
		\tabularnewline \midrule
\tenevilglyph[yes][3]{iE_b_i_jR} 
	&	
	&	
	&	тъэр [сколько], сколько [56]
	& 	ter [tær, тъэр; напечатано в книге, знак рядом] [12.16]
		\tabularnewline \midrule
\tenevilglyph[yes][4]{iE-q_b_i} 
	&	
	&	
	&	
	& 	tereu [tæræw, тъэръэв = недостаточно, мало; напечатано в книге, знак рядом] [12.16] \linebreak
		мало [34.12]
		\tabularnewline \midrule
\tenevilglyph[yes][4]{j_b_q} 
	&	посла [пошла] \cite[л. 66]{spbfaran79}
	&	
	&	кет'и, пошел, кетыркын — уходит [85] % TODO: нужна транскрипция
	& 	\cite[360]{davydova2015a} \linebreak
		qәtji [= пошел; напечатано в книге, знак рядом] [12.22] \linebreak % TODO: нужна транскрипция, уточнить перевод
		оеехау [уехал] [30.6об]
		\tabularnewline \midrule
\tenevilglyph[yes][3]{j_b_q_2q} 
	&	слы [шли] \cite[л. 68]{spbfaran79} \linebreak
		пёс [вёз] \cite[л. 66 об]{spbfaran79}
	&	
	&
	& 	\cite[360]{davydova2015a} 
		\tabularnewline \midrule
\tenevilglyph[yes][3]{i_2j_2cY} 
	&	
	&	орел* \cite{lavrov1969}
	&	тилмытил [тиԓмытиԓ = орел, орлан], орел [98]
	& 	\cite[28]{lavrov1969} \linebreak
		tilme [напечатано в книге, знак рядом] [12.13] % TODO: нужен перевод, нужна транскрипция
		\tabularnewline \midrule
\tenevilglyph[yes][3]{i_j_cY_s} 
	&	
	&	
	&	
	& 	raulьŋ [rauleŋ, равыԓьэӈ = белка; напечатано в книге, знак рядом] [12.19об] 
		\tabularnewline \midrule
\tenevilglyph[yes][4]{C-C_q_j} 
	&	
	&	ворон \cite{lavrov1969}
	&	вэтлы [вэтԓы = ваԓвыйӈын = ворон], ворон [23]
	& 	[25.13] \linebreak
		wetlь [vætlь, вэтԓы; напечатано в книге, знак рядом] [12.13об]
		\tabularnewline \midrule
\tenevilglyph[yes][3]{CD-CDX} 
	&	недавно* \cite[л. 50]{spbfaran79} \linebreak % значок только умеренно похож
		цирас [?] \cite[л. 67 об]{spbfaran79} \linebreak
		в «провсерас» [?] \cite[л. 67 об]{spbfaran79}
	&	
	&
	& 	[25.4об] \linebreak
		ajwә [ajvә, айвэ = вчера; напечатано в книге, знак рядом] [12.19] \linebreak
		черач [?] [30.6об]
		\tabularnewline \midrule
\tenevilglyph[yes][1]{CD-CDX_l} 
	&	ониметнис [?] \cite[л. 66 об]{spbfaran79}
	&	
	&
	& 	\cite[364]{davydova2015a} 
		\tabularnewline \midrule
\tenevilglyph[yes][3]{CD-CDX_2q} 
	&	прослокот [прошлый год] \cite[л. 66 об]{spbfaran79}
	&	
	&
	& 	[25.4] 
		\tabularnewline \midrule
\tenevilglyph[yes][4]{CD-CDX_q_2b_c} 
	&	
	&	
	&
	& 	кота [когда] [37.2] 
		\tabularnewline \midrule
\tenevilglyph[yes][2]{i_b_qY} 
	&	нусно [нужно] \cite[л. 66]{spbfaran79} \linebreak
		в «понравилас» [?] \cite[л. 66]{spbfaran79}
	&	
	&
	& 	[25.7] \linebreak
		teggeŋu [напечатано в книге, знак рядом] [12.19об] % TODO: нужен перевод, нужна транскрипция
		\tabularnewline \midrule
\tenevilglyph[yes][1]{3k} 
	&	в «понравилас» [?] \cite[л. 66]{spbfaran79}
	&	
	&
	& 	\cite[364]{davydova2015a} 
		lьŋьtkь [напечатано в книге, знак рядом] [12.19об] % TODO: нужен перевод, нужна транскрипция
		\tabularnewline \midrule
\tenevilglyph[yes][3]{i_j_3b} 
	&	паска патро [?] \cite[л. 68 об]{spbfaran79}
	&	
	&	мэкым [мэӄым = стрела], рынныкон [= стрела с костяным наконечником], стрела [114]
	& 	\cite[364]{davydova2015a} \linebreak
		mәqьm [mæqьm, мэӄым; напечатано в книге, знак рядом] [12.19об]
		\tabularnewline \midrule
\tenevilglyph[yes][1]{jY_3b} 
	&	
	&	
	&	
	& 	\cite[364]{davydova2015a} \linebreak
		lьe [ИЛИ:2.11] \linebreak % TODO: нужен перевод
		lьeэ [ИЛИ:1.9]
		\tabularnewline \midrule
\tenevilglyph[yes][4]{u_q_l} 
	&	талок [далеко] \cite[л. 68 об]{spbfaran79}
	&	
	&
	& 	\cite[360, 364]{davydova2015a} \linebreak
		\cite[28]{lavrov1969}  \linebreak
		ejaa [jaa, ыяа = далеко, вдали] [ИЛИ:2.21]
		\tabularnewline \midrule
\tenevilglyph[yes][4]{2cD_jY} 
	&	илплыл* [ы'ԓьыԓ = снег] \cite[л. 68]{spbfaran79} % зеркально
	&	вьюга (в~тексте) \cite{lavrov1969}
	&
	& 	\cite[361]{davydova2015a} \linebreak
		\cite[26]{lavrov1969} 
		\tabularnewline \midrule
\tenevilglyph[yes][4]{u_2j} 
	&	прошол [прошел] \cite[л. 66 об]{spbfaran79} % зеркально
	&	
	&	галяк [гаԓяк = миновать, проходить], пройти, проехать [77]
	& 	[25.4] \linebreak
		ьalj [5.1] % TODO: нужен перевод, нужна транскрипция
		\tabularnewline \midrule
\tenevilglyph[yes][4]{c_C_2j} 
	&	собака \cite[л. 68 об]{spbfaran79}
	&	
	&
	& 	[25.3] \linebreak
		attьn [әttьn, ы'ттъын = собака; напечатано в книге, знак рядом] [7.13, 12.10об] \linebreak
		\tabularnewline \midrule
\tenevilglyph[yes][4]{c_C_2j_f} 
	&	
	&	
	&
	& 	[в книге, рядом с изображением собачьей упряжки] [7.30] \linebreak
		maьlaliьt [магԓяԓьыт = ездоки на собаках] [ИЛИ:1.19]
		\tabularnewline \midrule
\tenevilglyph[yes][4]{k_2j} 
	&	щаи [чай] \cite[л. 68 об]{spbfaran79}
	&	
	&	чай [6]
	& 	[25.9] \linebreak
		в «чаеопат» [\textit{«чай} пить» или чайпат = вскипевший чай] [32.16об] % чайпат?
		\tabularnewline \midrule
\tenevilglyph[yes][4]{c_cD_b} 
	&	
	&	
	&	экымыл [э'ӄимыԓ = водка], водка, спирт [10]
	& 	в «пееано» [пьяный] [30.3] \linebreak
		eqemьl [э'ӄимыԓ] [ИЛИ:2.24]
		\tabularnewline \midrule
\tenevilglyph[yes][4]{c-c_cD_b} 
	&	
	&	
	&	
	& 	celgьmemьl [celgь-memьl = красное вино] [ИЛИ:2.24] % TODO: нужна транскрипция, чеԓгы мимыԓ?
		\tabularnewline \midrule
\tenevilglyph[yes][3]{2LE} 
	&	полноь [полный] \cite[л. 66 об]{spbfaran79}
	&	
	&
	& 	[25.4об] \linebreak
		полнои [полный] [34.11]
		\tabularnewline \midrule
\tenevilglyph[yes][4]{uD_z} 
	&	рука \cite[л. 68]{spbfaran79}
	&	
	&
	& 	[25.13] \linebreak
		mьngьtlьŋьn [= рука] [12.17об]
		\tabularnewline \midrule
\tenevilglyph[yes][3]{I_q_iSY} 
	&	нока [нога] \cite[л. 68]{spbfaran79} 
	&	
	&
	& 	[19.6]
		\tabularnewline \midrule
\tenevilglyph[yes][3]{I_q_iSY_p} 
	&	сутав [сустав] \cite[л. 68]{spbfaran79} 
	&	
	&
	& 	[32.7]
		\tabularnewline \midrule
\tenevilglyph[yes][3]{o-o_z} 
	&	клас [глаз] \cite[л. 68]{spbfaran79}
	&	
	&	лилет, глаза [32] % TODO: нужна транскрипция
	& 	[6.1] 
		\tabularnewline \midrule
\tenevilglyph[yes][4]{l_i} 
	&	нос \cite[л. 68]{spbfaran79}
	&	
	&	екаак [еӄааӄ = человеческий нос, перед обуви], ин'ин' [и'ӈъиӈ = нос животного, человека, клюв, носовая часть людки], нос [50]
	& 	[25.15об] 
		\tabularnewline \midrule
\tenevilglyph[yes][1]{2c_2bX} 
	&	в «провсерас» \cite[л. 67 об]{spbfaran79}
	&	
	&
	& 	талче [?] [29.13] \linebreak
		в «талсе» [?] [29.12] \linebreak
		[25.7]
		\tabularnewline \midrule
\tenevilglyph[yes][4]{o_2q_2j} 
	&	послы [пошли (от «слать»)] \cite[л. 68 об]{spbfaran79}
	&	
	&
	& 	[25.10] \linebreak
		genŋiulinet [напечатано в книге, знак рядом] [12.20об] \linebreak % TODO: нужен перевод, нужна транскрипция
		посевыке [посылке] [34.8об] \linebreak
		в «песмо» [письмо] [34.8об] 
		\tabularnewline \midrule
\tenevilglyph[yes][4]{o-o-o} 
	&	
	&	
	&	иннаны [ыннаны = одинаково, одинаковые], одинаково [122]
	& 	осенако [одинаково] [34.12] \linebreak
		ьnanь [әnnanьŋ, ыннаны] [ИЛИ:1.5]
		\tabularnewline \midrule
\tenevilglyph[yes][4]{vD_2qY} 
	&	aagek [aacek, а'ачек = молодой человек] \cite[л. 65 об]{spbfaran79} % а'ачек
	&	
	&
	& 	[4.2?] \linebreak
		oraçekqaj [= юноша; напечатано в книге, знак рядом] [12.22] \linebreak % TODO: уточнить перевод, нужна транскрипция
		в «молотои» [молодой] [34.11об] \linebreak
		молотои [молодой] [29.12] \linebreak
		iaacek [aacek, а'ачек] [ИЛИ:2.22]
		\tabularnewline \midrule
\tenevilglyph[yes][3]{2o_2jY} 
	&	в «я упрала» [«я убрала»] \cite[л. 67]{spbfaran79}
	&	
	&
	& 	[4.2?] 
		\tabularnewline \midrule
\tenevilglyph[yes][4]{CD_jFN} 
	&	кова [?] \cite[л. 66]{spbfaran79} \linebreak
		ковта [?] \cite[л. 66]{spbfaran79}
	&	
	&
	& 	[4.2?] \linebreak
		какта [когда] [34.11] \linebreak
		tete [titә, титэ = когда] [ИЛИ:1.13]
		\tabularnewline \midrule
\tenevilglyph[yes][4]{i_b_jX} 
	&	
	&	
	&	н'откэн [ӈотӄэн = этот, эта, это], вот, этот [34]
	& 	\cite[363]{davydova2015a} \linebreak
		вот [32.6] \linebreak
		ŋotьqen [ŋotьnqәn, ӈотӄэн = вон тут] [4.10об] \linebreak
		ŋotqen [ӈотӄэн] [ИЛИ:1.5]
		\tabularnewline \midrule
\tenevilglyph[yes][4]{i_b_jX_2cD} 
	&	
	&	
	&	н'откоры [ӈотӄоры], н'отко [ӈотӄо = с этого места, отсюда], отсюда [34]
	& 	ŋotьqorь [ӈотӄоры] [ИЛИ:1.19] % TODO: нужен перевод
		\tabularnewline \midrule
\tenevilglyph[yes][4]{2b_2l} 
	&	
	&	
	&
	& 	reqә [напечатано в книге, знак рядом] [12.19] \linebreak % TODO: нужен перевод, нужна транскрипция
		стотакои [что такое] [35.6, 37.2, 37.2об]
		\tabularnewline \midrule
\tenevilglyph[yes][4]{G_t} 
	&	
	&	
	&
	& 	теыеене [тэгйиӈ = кашель, грипп] [34.8] \linebreak % тэгйиӈ
		касли [кашель] [34.11]
		\tabularnewline \midrule
\tenevilglyph[yes][4]{r_t} 
	&	
	&	
	&
	& 	пыраснек [праздник] [34.10об] \linebreak
		прачнек [праздник] [30.3] \linebreak
		в «перхомаи» [первомай] [30.3] \linebreak
		в «захтре перхоймая» [завтра первое мая] [39.7об] \linebreak
		эeŋeŋkь [ИЛИ:1.14] % TODO: нужна транскрипция
		\tabularnewline \midrule
\tenevilglyph[yes][4]{i_b_JX} 
	&	
	&	
	&	эргатык [= завтра], завтра [34]
	& 	\cite[360]{davydova2015a} \linebreak
		сахтре [завтра] [34.18об] \linebreak
		в «захтре перхоймая» [завтра первое мая] [39.7об] \linebreak
		эrgatьk [әrgatьk, эргатык] [ИЛИ:1.15]
		\tabularnewline \midrule
\tenevilglyph[yes][4]{U2EN} 
	&	
	&	учиться \cite{lavrov1969}
	&	рыгъюлевык, нинэйвык, учить, учиться [74] % TODO: нужна транскрипция
	& 	усеся [учиться] [34.12] \linebreak
		в «усеся» [учиться] [34.11об] 
		\tabularnewline \midrule
\tenevilglyph[yes][3]{U2E} 
	&	
	&	
	&	ейгулэткэлин [?], не умеет, не знает [74] % TODO: нужна транскрипция
	&	эjuletke [æjgulætkæ?] [ИЛИ:1.10] % TODO: нужен перевод, нужна транскрипция
		\tabularnewline \midrule 
\tenevilglyph[yes][4]{cD_2k} 
	&	
	&	
	&	ненкукин, там находящийся [78] % TODO: нужна транскрипция
	& 	\cite[364]{davydova2015a} \linebreak
		там [33.4, 34.1] \linebreak
		ŋenko [ŋænku, ӈэнку = там, далеко] [ИЛИ:2.21]
		\tabularnewline \midrule
\tenevilglyph[yes][4]{i_qY_vD} 
	&	тапак [табак] \cite[л. 68 об.]{spbfaran79}
	&	
	&	таак [тааӄ = табак], табак [110]
	& 	[4.1об]
		\tabularnewline \midrule
\tenevilglyph[yes][4]{c_q_cD_q} 
	&	
	&	
	&	имынан [амынан = один он], один (он) [19]
	& 	\cite[360,364]{davydova2015a} \linebreak
		осын [один] [37.2]
		\tabularnewline \midrule
\tenevilglyph[yes][4]{с_jY_cD_q} 
	&	
	&	
	&	
	& 	qunece [ӄунэче = однажны, как-то раз] [ИЛИ:1.14]
		\tabularnewline \midrule
\tenevilglyph[yes][2]{UD_uD} 
	&	
	&	
	&	вагыргын [= бытие, жизнь, история, общественный строй, событие], бытие, обычай, божество, образ жизни [61] % TODO: проверить, вагыргын = вааргын или нет
	& 	waьrgьn [vaьrgьn, вааргын = божество] [ИЛИ:1.8]
		\tabularnewline \midrule
\tenevilglyph[yes][4]{UD_uDE} 
	&	
	&	
	&	н'аргынэн [ӈаргынэн = вселенная, наружное пространство, климат, погода], природа, вселенная [62]
	& 	натваре [на дворе] [30.2об, 34.11об] \linebreak
		ŋargьnen [ӈаргынэн] [ИЛИ:1.5]
		\tabularnewline \midrule
\tenevilglyph[yes][4]{UD_uD_2q} 
	&	
	&	
	&	торвагыргын [= новая жизнь], новая жизнь [61]
	& 	torvaьrga [напечатано в книге, знак рядом] [12.20об] \linebreak % TODO: нужен перевод, нужна транскрипция
		torwaьrgьn [торвагыргын] [ИЛИ:1.16]
		\tabularnewline \midrule
\tenevilglyph[yes][4]{UD_uD_'} 
	&	
	&	
	&	кергынаргынан, ясная погода [62] % TODO: нужна транскрипция
	& 	qergьqer [ӄэргыӄэр = свет] [ИЛИ:1.15]
		\tabularnewline \midrule
\tenevilglyph[yes][4]{q_c_cD_q} 
	&	соровно [всё равно] \cite[л. 66]{spbfaran79} 
	&	
	&	%оптыма [о'птыма = подобно, словно, как], подобно, словно, как [54] % видимо, ошибка
	& 	сороно [всё равно] [34.11об] \linebreak
		tьmŋalgulaq [ИЛИ:1.9] % TODO: нужен перевод тымӈэԓгуԓеӄ ?
		\tabularnewline \midrule
\tenevilglyph[yes][4]{c_cD} 
	&	
	&	
	&	оптыма [о'птыма = подобно, словно, как], подобно, словно, как [54]
	& 	iomtьma [о'птыма] [ИЛИ:1.3] % TODO: нужен перевод
		\tabularnewline \midrule
\tenevilglyph[yes][1]{O_JX_b} 
	&	
	&	
	&
	& 	насоеас [?] [33.4]
		\tabularnewline \midrule
\tenevilglyph[yes][4]{3iX} 
	&	
	&	
	&
	& 	месе [вместе] [29.11, 34.21об, 39.5об] \linebreak
		cekj [cæәkæj, чеэкэй = вместе, совместно] [ИЛИ:1.5]
		ceke [чеэкэй] [ИЛИ:1.14] \linebreak
		ceььkj [чеэкэй] [ИЛИ:1.21]
		\tabularnewline \midrule
\tenevilglyph[yes][4]{k_j_jF} 
	&	
	&	
	&
	& 	нехосыт [не хочет] [37.2, 37.2об]
		\tabularnewline \midrule
\tenevilglyph[yes][4]{i_2q_l_q_i_L} 
	&	
	&	
	&
	& 	ролтырхын [руԓтыркын = сторониться] [34.10] % руԓтыркын
		\tabularnewline \midrule
\tenevilglyph[yes][4]{o_2q_l} 
	&	
	&	
	&	кейыкын [ӄээӄын = еще немного, еще], еще [46]
	& 	iесо [еще] [32.1] \linebreak
		iечо [еще] [32.15об] \linebreak
		iезо [еще] [39.1об] \linebreak
		qeqьn [ӄээӄын = еще немного, еще] [ИЛИ:2.28]
		\tabularnewline \midrule
\tenevilglyph[yes][4]{G-G} 
	&	nьnnь [нынны = имя] \cite[л. 65]{spbfaran79} % нынны
	&	
	&	нынны, название, имя [133]
	& 	[11.3] \linebreak
		nьnnь [напечатано в книге, знак рядом] [7.13] \linebreak
		nьnnь [ИЛИ:1.14]
		\tabularnewline \midrule
\tenevilglyph[yes][3]{O_oN} 
	&	
	&	озеро \cite{lavrov1969}
	&	гытгын [= озеро], озеро [101]
	& 	[1.71] \linebreak
		gьtgьk [гытгык = на озере] [12.17об] % TODO: уточнить перевод и транскрипция, нужна транскрипция латиницей
		\tabularnewline \midrule
\tenevilglyph[yes][3]{z_JX} 
	&	киска [кишка] \cite[л. 66 об]{spbfaran79}
	&	
	&
	& 	[4.3]
		\tabularnewline \midrule
\tenevilglyph[yes][4]{cF_2JY} 
	&	
	&	
	&
	& 	неветау [не ведаю] [30.7об] \linebreak
		gewwtь [gewetь, гэвэты = неизвестно, неведомо] [ИЛИ:1.7]
		\tabularnewline \midrule
\tenevilglyph[yes][4]{cD_2q_p} 
	&	
	&	
	&	ынкам [ынкъам = и, далее], и (союз) [57]
	& 	\cite[364]{davydova2015a} \linebreak
		потом [30.8] \linebreak
		ьnkiam [әnkam, ынкъам] [ИЛИ:1.2]
		\tabularnewline \midrule
\tenevilglyph[yes][4]{sM} 
	&	
	&	
	&
	& 	qьlьe [ӄэгԓын = правильно, верно, правду] [ИЛИ:2.2] % TODO: нужна транскрипция латиницей
		\tabularnewline \midrule
\tenevilglyph[yes][4]{sM_jF} 
	&	
	&	
	&
	& 	правелно [правильно] [30.7об] \linebreak
		праувлно [правильно] [29.12] \linebreak
		qeьlьnanget [qәglenanget, ӄэгԓынангэт = правильно, правда, действительно, верно] [ИЛИ:1.16]
		\tabularnewline \midrule
\tenevilglyph[yes][1]{jY} 
	&	
	&	
	&
	& 	веен [?] [34.7] \linebreak% vaәŋ? (155)
		ween [?] [ИЛИ:1.10]	% TODO: нужен перевод
		\tabularnewline \midrule
\tenevilglyph[yes][2]{iY_iX} 
	&	часть «обрез (кусок) шкуры лахтака» \cite[л. 48]{spbfaran79}
	&	
	&
	& 	\cite[364]{davydova2015a} \linebreak
		в «половена» [половина] [30.7]
		\tabularnewline \midrule
\tenevilglyph[yes][4]{2c_i} 
	&	
	&	
	&	ымын [ымы, ымыӈ = тоже], тоже [19]
	& 	\cite[360, 364]{davydova2015a} \linebreak
		ьmь [ымы] [ИЛИ:1.12]
		\tabularnewline \midrule
\tenevilglyph[yes][4]{iY_l} 
	&	
	&	
	&	
	& 	\cite[364]{davydova2015a} \linebreak
		gemo [гэмо = неизвестно, не знать, не заметить] [ИЛИ:1.12]
		\tabularnewline \midrule
\tenevilglyph[yes][4]{J_2lX} 
	&	
	&	
	&	
	& 	\cite[360]{davydova2015a} \linebreak
		penen [pænin, пэнин = прежний, старый] [ИЛИ:2.27]
		\tabularnewline \midrule
\tenevilglyph[yes][4]{J_2lX_j} 
	&	
	&	
	&	
	& 	panena [panena, panaa, панаа = всё еще] [ИЛИ:1.6]
		\tabularnewline \midrule
\tenevilglyph[yes][4]{uD_iXX} 
	&	
	&	
	&	
	& 	\cite[364]{davydova2015a} \linebreak
		qьrьm [ӄырым = не, нет, никогда, всё равно же] [ИЛИ:2.20]
		\tabularnewline \midrule
\tenevilglyph[yes][4]{uD_iXX_jF} 
	&	
	&	
	&	
	& 	qьrьmen [ӄырымэн = не, это не, ничей] [ИЛИ:1.3, ИЛИ:2.3]
		\tabularnewline \midrule
\tenevilglyph[yes][4]{iY_J} 
	&	
	&	
	&	вытку [wьtku = впервые, только], только что, теперь [69] % TODO: нужна транскрипция
	& 	\cite[361, 363]{davydova2015a} \linebreak
		wьtku  [ИЛИ:2.8]
		\tabularnewline \midrule
\tenevilglyph[yes][4]{u_lN} 
	&	
	&	
	&	чен'эт  [чеӈэт = ведь], если, так как [68]
	& 	\cite[364]{davydova2015a} \linebreak
		ceŋet [чеӈэт]  [ИЛИ:2.5]
		\tabularnewline \midrule
\tenevilglyph[yes][4]{cD_i_c} 
	&	
	&	
	&	янра [= янрэты = отдельно], отдельно [125]
	& 	\cite[364]{davydova2015a} \linebreak
		am-janra [амъянра = порознь, отдельно; напечатано в книге, знак рядом] [12.20об] \linebreak % TODO: проверить транскрипцию латиницей
	 	janra [янра] [ИЛИ:1.5] \linebreak
		jara [janra, янра] [ИЛИ:2.13]
		\tabularnewline \midrule
\tenevilglyph[yes][3]{cD_c} 
	&	
	&	
	&	
	& 	\cite[364]{davydova2015a} \linebreak
		iukeŋan [ИЛИ:1.6] \linebreak % TODO: нужна транскрипция, нужен перевод
	 	iuke [iwkә, ивкэ = хоть бы, вот бы, пожалуйста] [ИЛИ:1.18] 
		\tabularnewline \midrule
\tenevilglyph[yes][2]{LD_q_c} 
	&	
	&	
	&	кутти [qutti = quli = другой], другие [49] % кэтэм?
	& 	iepe [ипэ = поистине, на самом деле, именно] [ИЛИ:2.2] % TODO: уточнить перевод, проверить в контексте
		\tabularnewline \midrule
\tenevilglyph[yes][3]{LD_jX} 
	&	
	&	
	&	
	& 	mikin [микын = чей, чьи; напечатано в книге, знак рядом] [7.13, 12.13] \linebreak
		mekьne [микын] [ИЛИ:1.10]
		\tabularnewline \midrule
\tenevilglyph[yes][4]{L-l_q} 
	&	
	&	
	&	
	& 	weler [veler, вэԓер = хоть бы, хоть, довольно, хватит, достаточно] [ИЛИ:1.4] \linebreak
		wler [вэԓер] [ИЛИ:2.12]
		\tabularnewline \midrule
\tenevilglyph[yes][4]{o_2LE} 
	&	
	&	
	&	пын'ыл [пыӈыԓ = новость, известие], новость [132]
	& 	pьŋьl [пыӈыԓ; напечатано в книге, знак рядом] [12.20об] \linebreak
		pьŋьl [пыӈыԓ] [ИЛИ:2.25]
		\tabularnewline \midrule
\tenevilglyph[yes][3]{o_L_LE} 
	&	
	&	
	&	
	& 	qәtwuunat [= назови;  напечатано в книге, знак рядом] [12.24об] \linebreak % TODO: уточнить перевод, нужна транскрипция
		qәtwuun [= расскажите;  напечатано в книге, знак рядом] [12.25] % TODO: уточнить перевод, нужна транскрипция
		\tabularnewline \midrule
\tenevilglyph[yes][4]{c_c_p} 
	&	
	&	
	&	вечым [вэчьым = очевидно, пожалуй, должно быть, может быть], вероятно [69] 
	& 	weciьm [вэчьым] [ИЛИ:1.5] % TODO: нужна транскрипция латиницей
		\tabularnewline \midrule
\tenevilglyph[yes][4]{c_i_p_i} 
	&	
	&	
	&	
	& 	эqьlpэ [әqәlpæ, эӄыԓпэ = скорее, быстро] [ИЛИ:1.4]
		\tabularnewline \midrule
\tenevilglyph[yes][4]{i_sXY_jFY} 
	&	
	&	
	&	
	& 	\cite[360]{davydova2015a} \linebreak
		jacь [яачы = сзади, после, потом] [ИЛИ:1.17] \linebreak
		jacьken [яачыӈкэн = последний] [ИЛИ:1.17] 
		\tabularnewline \midrule
\tenevilglyph[yes][4]{4j} 
	&	
	&	
	&	ытри, они [89]
	& 	\cite[360, 361, 364]{davydova2015a} \linebreak
		ьree [ытри] [ИЛИ:1.5] % TODO: нужна транскрипция латиницей, странно, уточнить контекст
		\tabularnewline \midrule
\tenevilglyph[yes][4]{c_IY} 
	&	
	&	
	&	чымкык [чымӄык = часть, взятый частично, другой, иной], тэйвын [тэйвыӈ = часть, доля, пай], часть [68]
	& 	çьmqьk [чымӄык; напечатано в книге, знак рядом] [12.20об] \linebreak
		cьmqьk [чымӄык] [ИЛИ:1.5] 
		\tabularnewline \midrule
\tenevilglyph[yes][4]{2b} 
	&	
	&	
	&	
	& 	\cite[364]{davydova2015a} \linebreak
		ietьkeun [itьk eun, итыкэвын = хотя, и всё же, вообще-то] [ИЛИ:1.11]  
		ietьk [itьk, итык = а ... вот, -то, быть, являться, служить] [ИЛИ:1.5]
		\tabularnewline \midrule
\tenevilglyph[yes][4]{2b_2q} 
	&	
	&	
	&	вэты [= надо, необходимо, усилительная частица], поистине [113]
	& 	\cite[364]{davydova2015a} \linebreak
		vetь [вэты; напечатано в книге, знак рядом] [12.15] \linebreak % TODO: нужна транскрипция латиницей
		wtutь [?] [ИЛИ:1.4] \linebreak % TODO: нужен перевод
		wetь [вэты] [ИЛИ:2.27] % TODO: нужна транскрипция латиницей
		\tabularnewline \midrule
\tenevilglyph[yes][4]{uD-uD_2cD} 
	&	
	&	
	&	
	& 	raqьlqьl [раӄыԓӄыԓ = ненужная вещь, старье, хлам] [ИЛИ:1.4, ИЛИ:1.24] % TODO: нужна транскрипция латиницей
		\tabularnewline \midrule
\tenevilglyph[yes][4]{o_IY-_IY} 
	&	
	&	
	&	кооператык, кооперат, кооператив [52] % TODO: нужна транскрипция
	& 	kaparaceu [кооператив] [ИЛИ:1.5] % TODO: уточнить перевод
		\tabularnewline \midrule
\tenevilglyph[yes][4]{S} 
	&	
	&	
	&	etle [этԓы = нет, не так, никогда], не (отрицательная частица при причастии и деепричастии) [15] 
	& 	elte [этԓы; напечатано в книге, знак рядом] [12.15об] \linebreak
		elь [этԓы] [ИЛИ:1.11] % TODO: нужна транскрипция латиницей
		\tabularnewline \midrule
\tenevilglyph[yes][3]{k_jF_k_jFX} 
	&	
	&	
	&	
	& 	\cite[364]{davydova2015a} \linebreak
		puriu [пууръу = взамен, вместо, наоборот] [ИЛИ:2.1] % TODO: нужна транскрипция латиницей
		\tabularnewline \midrule
\tenevilglyph[yes][2]{jE_jFE_jF} 
	&	
	&	
	&	выентогыргын [= дыхание, выдыхание], дыхание [88]
	& 	\cite[364]{davydova2015a} \linebreak
		wejьrьrgьn [ИЛИ:1.7] \linebreak % TODO: здесь и ниже нужен перевод
		wejьrrьgьn [ИЛИ:1.8] \linebreak
		weьrrgьn [ИЛИ:1.13] \linebreak
		weьrgьn [ИЛИ:2.17]
		\tabularnewline \midrule
\tenevilglyph[yes][4]{c-cD_'} 
	&	
	&	
	&	
	& 	tenmьciьn [tenmьcьn, тэнмычьын = план, график, мера, образец] [ИЛИ:1.8, ИЛИ:2.11] % один из знаков зеркальный
		\tabularnewline \midrule
\tenevilglyph[yes][4]{UD_2j} 
	&	
	&	
	&	
	& 	amtьmŋe [амтымӈэ = просто, так себе] [ИЛИ:2.13] %TODO: нужна транскрипция латиницей
		\tabularnewline \midrule
\tenevilglyph[yes][1]{UD_2jD} 
	&	
	&	
	&	
	& 	naьnatьnat [ИЛИ:1.10] %TODO: нужен перевод
		\tabularnewline \midrule 
\tenevilglyph[yes][3]{2sX_j} 
	&	
	&	
	&	
	& 	kelь [kælь, кэԓы = злой дух, черт] [ИЛИ:2.23] \linebreak
		kelien* [ИЛИ:1.9] % TODO: нужен перевод, зеркально
		\tabularnewline \midrule 
\tenevilglyph[yes][4]{i_cX} 
	&	
	&	
	&	
	& 	\cite[364]{davydova2015a} \linebreak
		meceiu [mæcicu, мэчичьу = а всё же, всё-таки] [ИЛИ:1.14]
		\tabularnewline \midrule 
\tenevilglyph[yes][4]{rB_i_j} 
	&	
	&	
	&	рыпет [рыпэт = даже], даже [54]
	& 	\cite[364]{davydova2015a} \linebreak
		rьpet [рыпэт] [ИЛИ:1.14]
		\tabularnewline \midrule 
\tenevilglyph[yes][4]{SYE} 
	&	
	&	
	&	
	&	эerьm [ærьm, эрым = начальник, глава, староста] [ИЛИ:1.14]
		\tabularnewline \midrule
\tenevilglyph[yes][3]{SYE_2q} 
	&	
	&	
	&	
	&	tur-ermete [tur-ærmæt = советская власть; напечатано в книге, знак рядом] [12.15об] % TODO: уточнить перевод, нужна транскрипция
		\tabularnewline \midrule
\tenevilglyph[yes][2]{u-2j} 
	&	
	&	
	&	иам [ийъам = почему], почему [78]
	&	\cite[364]{davydova2015a} \linebreak
		ьnriam [?] [ИЛИ:1.8] % TODO: нужен перевод
		\tabularnewline \midrule 
\tenevilglyph[yes][4]{oF_j_q} 
	&	
	&	
	&	юрэк [юрэӄ = ԓюрэӄ], люрэк [ԓюрэӄ = возможно, может быть], может быть [77]
	&	joreq [юрэӄ] [ИЛИ:2.28] \linebreak % TODO: нужна транскрипция латиницей
		jurieq [юрэӄ] [ИЛИ:2.6]
		\tabularnewline \midrule 
\tenevilglyph[yes][1]{i_j_J_2j} 
	&	
	&	
	&	
	&	эmlььen [ИЛИ:1.20] \linebreak
		emlььэn [ИЛИ:2.4] \linebreak
		emlььn [ИЛИ:2.28]
		\tabularnewline \midrule 
\tenevilglyph[yes][4]{b-b} 
	&	
	&	
	&	алва [аԓва = иначе, не так], иной, чужой, иначе, не так, прочь [28]
	&	alwa [alva, аԓва] [ИЛИ:1.2]
		\tabularnewline \midrule 
\tenevilglyph[yes][1]{JF-jY} 
	&	
	&	
	&	
	&	ьŋe [ИЛИ:2.12] % TODO: нужен перевод, нужна транскрипция
		\tabularnewline \midrule 
\tenevilglyph[yes][1]{JFE-jY} 
	&	
	&	
	&	
	&	ьŋetal [ИЛИ:2.14] % TODO: нужен перевод, нужна транскрипция
		\tabularnewline \midrule 
\tenevilglyph[yes][3]{dDE} 
	&	
	&	
	&	ээк [= жирник], жирник, лампа [37]
	&	eek [aak, ээк; напечатано в книге, знак рядом] [12.11] 
		\tabularnewline \midrule 
\tenevilglyph[yes][3]{i_JY_j} 
	&	
	&	
	&	чаат [аркан, лассо], чавыт, аркан [58] % TODO: нужен перевод, нужна транскрипция
	&	çaat [caat, чаат; напечатано в книге, знак рядом] [12.22] 
		\tabularnewline \midrule
\tenevilglyph[yes][3]{lE-lE} 
	&	
	&	
	&	пароль [пароԓ = лишний, плюс, излишек], лишний, сверхкомплектный, плюс [132]
	&	\cite[361]{davydova2015a} \linebreak
		parol [пароԓ; напечатано в книге, знак рядом] [12.19] 
		\tabularnewline \midrule 
\tenevilglyph[yes][3]{cL_cR} 
	&	
	&	
	&	
	&	vilut [виԓют = уши;  напечатано в книге, знак рядом] [12.23об] 
		\tabularnewline \midrule 
\tenevilglyph[yes][4]{I_2q_2c} 
	&	
	&	
	&	пипикылгын [пипиӄыԓгын = мышь], мышь [23]
	&	pipәkьlgьn [пипиӄыԓгын;  напечатано в книге, знак рядом] [12.20, 14.20] \linebreak
		~[pядом с изображением мыши] [19.6] \linebreak
		pepeqьlgьn [pipәkьlgьn] [ИЛИ:2.18]
		\tabularnewline \midrule 
\tenevilglyph[yes][4]{3b} 
	&	
	&	
	&	
	&	kьmiьlgьn [kьmьlgьn, кымъыԓгын = червь] [ИЛИ:2.18]
		\tabularnewline \midrule 
\tenevilglyph[yes][4]{3b_k} 
	&	
	&	
	&	
	&	apapaьlŋn [apaapaglьŋьn, апаапагԓыӈын = паук] [19.1]
		\tabularnewline \midrule 
\tenevilglyph[yes][3]{l_lX} 
	&	
	&	
	&	
	&	geçejwutkulin [= пошли] [12.22] % TODO: уточнить перевод, нужна транскрипция
		\tabularnewline \midrule 
\bottomrule
\end{longtable}
\end{landscape}

\section{Как читать таблицу} 

\subsection{Первый столбец}
Содержит: 

\begin{itemize}
\item Порядковый номер знака, может меняться по мере добавления новых знаков; 
\item В скобках степень уверенности в интерпретации знака:
	\begin{itemize}
		\item 0:	нет вообще ничего
		\item 1: 	есть только фонетическая какая-то наводка, значение которой неясно
		\item 2:	есть интерпретация, но сомнительная, либо есть противоречия между интерпретациями
		\item 3:	есть однозначная интерпретация, но источник всего один (или источники зависимые), и это не Теневиль, либо интерпретация написанного Теневилем вызывает сомнения, либо это имя личное без понятной транскрипции
		\item 4:	есть интерпретация, совпадающая или близкая в нескольких источниках,  либо есть однозначный перевод приведенный Теневилем, а противоречия, если есть, объясняются ошибкой
		\item 5:	сильная уверенность, несколько независимых источников сходятся на одном значении, есть надежная интерпретация приведенная Теневилем для этого или для близкородственных знаков 
		\item 6:	однозначная интерпретация, проверенная в контексте % пока таких записей нет, появятся, когда буду смотреть в контексте
	\end{itemize}
\item Сам знак.
\end{itemize}

\subsection{Остальные столбцы}

Если указана только ссылка, то есть изображение знака без интерпретации.

Если ссылка сопровождается текстом, то вначале текст написан так, как в документе (с точностью до сложностей разбора почерков). 

Если текст в источнике требует расшифровки, она указана в квадратных скобках. Для чукотских слов, написанных латиницей, используется орфография из словаря Богораз-Тана\cite{bogoraz1937}.

Если интерпретация приводится для группы знаков или знак является частью «лигатуры», интерпретация написана в кавычках.

Звездочкой помечены источники, где знак отличается по начертанию от прочих.

\section{Описания источников} 

\subsection{СПбФ АРАН. Ф. 250. Оп. 1. Д. 79}

\subsubsection{Листы 40–51}

Сводная таблица с символами и переводами на русский и иногда чукотский (латиницей).

\subsubsection{Листы 39 и об., 52 и об., 54, 56}

Листы с символами и переводами на чукотский (латиницей).

\subsubsection{Листы 37, 52, 53, 54 об, 55}

Листы со словами на русском и отдельными символами.

\subsubsection{Листы 64, 65 с оборотами}

Листы с символами и переводами на чукотский (латиницей) и русский. На л. 64 приведены числительные от 1 до 10, плюс 19 и 20.

\subsubsection{Листы 66–71 с оборотами}

Листы из тетради с символами и переводами на русский и, местами, чукотский (кириллицей). Почерк неуверенный, плохая орфография. На обороте л. 71 написано другим почерком «Altol [или Avtol] — Андрей Краснино 28/XII 39 г.»

\subsubsection{Остальные листы}

Символов не содержат, кроме 38 об., где шесть символов без переводов, но с комментариями.

\subsection{Статья В. Г. Богораз-Тана «Луораветланский (чукотский) язык»}

Таблица с переводом ряда символов на русский. Изображение (перерисовка) одной из табличек Теневиля с переводом одной из трех строк на ней на русский язык.

\subsection{Статья А. М. Миндалевича Hieroglyphic Characters of the Chuckchees}

Переводы на английский ряда отдельных символов. Перевод на английский двух фрагментов записей Теневиля.

\subsection{Статья И. П. Лаврова «Чукотский феномен»}

Три таблицы с переводами ряда символов на русский. Изображения символов сильно стилизованы, поэтому их затруднительно использовать в отсутствие других источников. Фотографии «лунного календаря» и «родового дерева» , нарисованных Теневилем.

\subsection{Диссертация Е. А. Давыдовой «Властные отношения в семейно-родственных коллективах оленных чукчей»}

Пять фотографий «дневниковых записей» Теневиля из архива отдела этнографии Сибири МАЭ РАН.

\printbibliography

\end{document}