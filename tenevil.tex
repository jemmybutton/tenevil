\documentclass{article}
\usepackage[a4paper,margin=1.5cm]{geometry}
%\usepackage{lua-visual-debug}
\usepackage{luaotfload,luacode}
\usepackage[hidelinks]{hyperref}
\usepackage{pdflscape}
\usepackage{xcolor}
\usepackage{longtable}
\usepackage{booktabs}
\usepackage{array}
\usepackage{graphicx}
\usepackage{fontspec}
\usepackage[utf8]{inputenc}
\usepackage[russian]{babel}
\usepackage{tipa}
\usepackage{xparse}
\usepackage[style=russian]{csquotes}

%\usepackage[datamodel=archive,backend=biber,style=gost-numeric]{biblatex}
\usepackage[datamodel=archive,backend=biber]{biblatex}

\setmainfont[Scale=0.8]{Noto Serif}

\newfontfamily\tenevilfont[Renderer = HarfBuzz]{tenevil-font.otf}

\begin{luacode}
 documentdata = documentdata or { }

 local stringformat = string.format
 local texsprint = tex.sprint
 local slot_of_name = luaotfload.aux.slot_of_name

 documentdata.fontchar = function (chr)
 local chr = slot_of_name(font.current(), chr, false)
 if chr and type(chr) == "number" then
 texsprint
 (stringformat ([[\char"%X]], chr))
 end
 end
\end{luacode}

\def\fontchar#1{\directlua{documentdata.fontchar "#1"}}

\newcounter{glyph}
\setcounter{glyph}{1}

\newcounter{charvariantcounter}

\def\allchars{}
\def\allshortcuts{}

\newwrite \shortcuts
\openout \shortcuts = charactershortcuts.tex\relax

\ExplSyntaxOn

\DeclareDocumentCommand{\tenevilglyph}{o o o m}{%
\def\tmpyes{yes}
\def\tmpone{#1}
\theglyph~{\IfNoValueTF{#2}{}{(#2)}}\hfill~\linebreak{%
	\IfNoValueTF{#1}{}{%
		\ifx\tmpone\tmpyes%
			%
		\else%
			\color{gray}%
		\fi%
		}
		\setcounter{charvariantcounter}{0}
		\clist_map_inline:nn { #4 } {%
				\stepcounter{charvariantcounter}%
				\edef\curname{currentChar\thecharvariantcounter}%
				{\tenevilfont\fontsize{40pt}{40pt}\selectfont\fontchar{##1}\ %
				\global\expandafter\def\csname\curname\endcsname{{\tenevilfont\selectfont\fontchar{##1}}}%
				\global\expandafter\def\expandafter\allchars\expandafter{\allchars{ }\tenevilfont\fontsize{40pt}{40pt}\selectfont\fontchar{##1}\ }%
				\IfNoValueTF{#3}{}{%
					\global\def\currentGlyphName{#3}%
					%\global\expandafter\edef\csname TenevilGlyph\currentGlyphName\endcsname{#4}%
					\ifnum\value{charvariantcounter} = 1%
						\write\shortcuts{\string\def\expandafter\string\csname TenevilGlyph#3\endcsname{##1}}%
					\else
						\write\shortcuts{\string\def\expandafter\string\csname TenevilGlyph#3\endcsname\roman{charvariantcounter}{##1}}
					\fi
				}%
				}%
			}%
		}%
	\stepcounter{glyph}%
}


\DeclareDocumentCommand{\TenevilGlyphByShortcut}{m}{%
\ifcsname TenevilGlyph#1\endcsname%
	\tenevilfont\selectfont\fontchar{\csname TenevilGlyph#1\endcsname}%
\else%
?
\fi%
}

\DeclareDocumentCommand{\currentGlyphWithAffixes}{o m m}{%
\def\argone{#1}%
\def\argtwo{#2}%
\def\argthree{#3}%
	{%
	\fontsize{30pt}{\the\baselineskip}
	\ifx\argtwo\empty{}\else%
		\clist_map_inline:nn { #2 } {\TenevilGlyphByShortcut{##1}}%
	\fi%
	\IfNoValueTF{#1}{\csname currentChar1\endcsname}{\csname currentChar#1\endcsname}%
	%\ifx\argone\empty{}\csname currentChar1\endcsname\else\csname currentChar#1\endcsname\fi%
	\ifx\argthree\empty{}\else%
		\clist_map_inline:nn { #3 } {\TenevilGlyphByShortcut{##1}}%
	\fi%
	}
}
\ExplSyntaxOff

\IfFileExists{charactershortcuts.tex}{\input{charactershortcuts.tex}}{}

%
%	первый опциональный аргумент — найден ли этот знак, написанный рукой Теневиля, yes/no
%
%	второй — уверенность в интерпретации: см. пояснение под таблицей
%	
%	третий — ключ, по которому можно обращаться к знаку, ниже правила транскрипции
%
%	А/Я -> a
%	Я -> ja
%	В -> w
%	Г -> g	
%	И -> i
%	Й -> j
%	К -> k
%	Ӄ -> q
%	Ԓ -> l
%	М -> m
%	Н -> n
%	Ӈ -> v
%	О/Ё -> o
%	Ё -> jo
%	П -> p
%	Р -> r
%	Т -> t
%	У/Ю -> u
%	Ю -> ju
%	Ч/С -> c
%	Ъ/Ь/' -> x
%	Ы -> y
%	Э/Е -> e
%	Е -> je
%


\addbibresource{tenevil.bib}

\input bibliography-macros.tex

\begin{document}
\begin{landscape}
\begin{longtable}{p{1.25cm}>{\raggedright}p{2.5cm}>{\raggedright}p{3cm}>{\raggedright}p{3.5cm}>{\raggedright}p{6.5cm}>{\raggedright}p{7.5cm}}
\toprule
	&	Выбранная интерпретация
 	& 	Опубликованные интерпретации \cite{bogoraz1934,mindalevich1934,lavrov1969} 
 	&	Картотека И. П. Лаврова
 	& 	СПбФ АРАН \cite{spbfaran79} 
	& 	В записях самого Теневиля \cite{davydova2015a,lavrov1969,bogoraz1934} 
		\tabularnewline \midrule
\multicolumn{6}{c}{«Фонетические» знаки} \\ \midrule
\tenevilglyph[yes][4][A]{*cTR} 
	&	а/о/у
	&	
	&	а [2] \linebreak
		о [3]
	&	
	&	a [слово напечатано] [12.12об] \linebreak
		та \currentGlyphWithAffixes{T}{} [32.14] 
		\tabularnewline \midrule 
\tenevilglyph[yes][3][O]{*cTR_2jF} 
	&	о % вероятно также У
	&	
	&	
	&	
	&	карломаркз [Карл Маркс] \currentGlyphWithAffixes{K,A,L}{M,melgytanvyn} [44.62] \linebreak
		зехорзкй [возможно, Загоскин\cite{druri1989}, зам. по хозяйственной части] \currentGlyphWithAffixes{C,Q}{R,melgytanvyn} [42.11] % TODO: уточнить интерпретацию
		\tabularnewline \midrule 
\tenevilglyph[yes][4][W]{*cTD} 
	&	у/о/в
	&	
	&	в [2]
	&	
	&	ту \currentGlyphWithAffixes{T}{} [32.14] \linebreak
		а б в к д \currentGlyphWithAffixes{A,P}{K,T} [42.8] \linebreak
		eulьm \currentGlyphWithAffixes{E}{L,M} [ИЛИ:2.28об]  % ilispread:2.28
		\tabularnewline \midrule 
\tenevilglyph[yes][3][N]{*jF} 
	&	н
	&	
	&	
	&	
	&	n [слово напечатано] [12.16] \linebreak
		çmn \currentGlyphWithAffixes{C,M}{} [4.9]
		\tabularnewline \midrule 
\tenevilglyph[yes][3][E]{*JFF} 
	&	е/э/и
	&	
	&	е [2]
	&	
	&	e [слово напечатано] [12.11] \linebreak
		ә [слово напечатано] [12.16] \linebreak
		eulьm \currentGlyphWithAffixes{}{W,L,M} [ИЛИ:2.28об]  % ilispread:2.28
		\tabularnewline \midrule 
\tenevilglyph[yes][5][K]{*',*3'} 
	&	к, ӄ
	&	
	&	к [2] 
	&	
	&	piŋkutkulьn [пиӈкуткуԓьын = прыгающий; слово напечатано] \currentGlyphWithAffixes{P,E}{A,K,A,L,N} [19.11об] \linebreak
		k [слово напечатано] \currentGlyphWithAffixes[2]{}{} [12.11] \linebreak
		q [слово напечатано] \currentGlyphWithAffixes[2]{}{} [12.15об]
		\tabularnewline \midrule 
\tenevilglyph[yes][5][Q]{*bTF_jT} 
	&	ӄ
	&	
	&	
	&	
	&	plaq \currentGlyphWithAffixes{P,L,A}{} [ИЛИ:2.24] \linebreak  % TODO: нужен перевод % ilispread:2.23
		qьnur [ӄынур] \currentGlyphWithAffixes{oxptyma}{N,A,R} [ИЛИ:2.11об]  % ilispread:2.11
		\tabularnewline \midrule 
\tenevilglyph[yes][4][b]{*jT,*g}
	&	ь,ъ,' 
	&	
	&	
	&	
	&	kelien [кэԓьин] \currentGlyphWithAffixes{kely}{N} [ИЛИ:2.11об] \linebreak  % ilispread:2.11
		waliьt [ваԓьыт = пребывающие, живущие] \currentGlyphWithAffixes{warkyn}{T} [ИЛИ:2.26] \linebreak  % ilispread:2.25
		wiek [въик] \currentGlyphWithAffixes{wxik}{K} [ИЛИ:1.13] \linebreak  % ilispread:1.13
		ь [слово напечатано] \currentGlyphWithAffixes[2]{}{} [12.11об] \linebreak
		ъ', ь' \currentGlyphWithAffixes[2]{}{} [32.14]
		\tabularnewline \midrule 
\tenevilglyph[yes][5][L]{*jFEN} 
	&	ԓ
	&	
	&	л [2]
	&	
	&	l [слово напечатано] [12.11об, 12.12об] \linebreak
		kьlkakьt [кыԓкакыт = беззубки; слово напечатано]\currentGlyphWithAffixes{Ki}{Ki} [7.13]
		\tabularnewline \midrule 
\tenevilglyph[yes][5][M]{*o} 
	&	м
	&	
	&	м [3]
	&	
	&	m [слово напечатано] [12.11, 12.13] 
		\tabularnewline \midrule 
\tenevilglyph[yes][5][P]{*c_cD}
	&	п
	&	
	&	п [3]
	&	
	&	plaq \currentGlyphWithAffixes{}{L,A,Q} [ИЛИ:2.24]  % TODO: нужен перевод % ilispread:2.23
		\tabularnewline \midrule 
\tenevilglyph[yes][5][R]{*jFE} 
	&	р
	&	
	&	р [3]
	&	
	&	r [слово напечатано] [12.12] 
		\tabularnewline \midrule 
\tenevilglyph[yes][5][C]{*iY} 
	&	с, ч
	&	
	&	с, ч [3]
	&	
	&	kluç \currentGlyphWithAffixes{Ki,L}{} [7.2] \linebreak
		СССР [слово напечатано] \currentGlyphWithAffixes{}{C,C,R} [44.9]
		\tabularnewline \midrule 
\tenevilglyph[yes][5][T]{*cT} 
	&	т
	&	
	&	т* [3]
	&	
	&	t [слово напечатано] [12.9, 12.12] \linebreak
		та \currentGlyphWithAffixes{}{A}, ту \currentGlyphWithAffixes{}{W} [32.14] 
		\tabularnewline \midrule 
\multicolumn{6}{c}{«Основные» знаки} \\ \midrule
\tenevilglyph[yes][5]{i_2cU_2cD}
	&	ӄԓявыԓ
	&	мужчина \cite{lavrov1969}
	&	клявыл [ӄԓявыԓ], мужчина [126] \linebreak
		кэлиныгйнетылин [кэԓиныгйивэтыԓьын = учитель], учитель \currentGlyphWithAffixes{}{kelikel} [60]
	&	qlaul [ӄԓявыԓ = мужчина] \cite[л. 64]{spbfaran79} % ӄԓявыԓ
	&	[38.1] \linebreak
		qlaula [ӄԓявыԓя = мужчина; слово напечатано] [12.17об] \linebreak % TODO: уточнить перевод
		qlaulte [ӄԓявыԓтэ = мужчины; слово напечатано] \currentGlyphWithAffixes{}{T} [12.18] \linebreak
		qьlauleьm [ӄԓявыԓегым = я мужчина] \currentGlyphWithAffixes{}{M} [ИЛИ:1.9] \linebreak  % ilispread:1.9
		гылауло [ӄԓявыԓё] \currentGlyphWithAffixes{}{Q,L,O} [44.2об] \linebreak % TODO: нужен перевод
		kaletko-laul [= учитель; слово напечатано] \currentGlyphWithAffixes{}{kelikel} [12.19об] % TODO: нужна транскрипция
		\tabularnewline \midrule
\tenevilglyph[yes][5]{i_2cU_2cD_'}
	&	ытԓыгын, пэԓӄэтык
	& 	отец \cite{bogoraz1934}
	&	ынпыначгын, старик [126]
	&	отец \cite[л. 40, 55]{spbfaran79} \linebreak
		әtlьgьn [ытԓыгын = отец] \cite[л. 52]{spbfaran79}\linebreak % ытԓыгын
		әtlьgә \cite[л. 52]{spbfaran79}\linebreak
		etlьgьn [әtlьgьn] \cite[л. 52 об.]{spbfaran79}\linebreak
		ьnpьŋaçgьn [ынпыначгын = старик] \cite[л. 64]{spbfaran79} % ынпыначгын
	&	\cite[360, 364]{davydova2015a} \linebreak
		әtlьgьn [ытԓыгын; слово напечатано] [12.17] \linebreak
		старек [старик] [34.12об] \linebreak % черточка тут справа, но не очень понятно, значит ли это что-то
		старик [не рукой Т.] [57.22] \linebreak
		pelqэrkьt [пэԓӄэркыт = стареют, умирают] [ИЛИ:1.16об] \linebreak  % ilispread:1.17
		ситарик [старик] \currentGlyphWithAffixes{}{P} [34.9об] \linebreak
		tьpelqerkьn \currentGlyphWithAffixes{}{T,E,R} [ИЛИ:1.10] \linebreak  % TODO: нужен перевод % ilispread:1.10
		nьpelqetken \currentGlyphWithAffixes{N,P}{N} [ИЛИ:2.11] \linebreak  % TODO: нужен перевод % ilispread:2.10
		ьlььn [ытԓыгын] \currentGlyphWithAffixes{}{N} [ИЛИ:1.21] \linebreak  % ilispread:1.21
		әtlьgьk [ытԓыгык = к отцу; слово напечатано] \currentGlyphWithAffixes{}{K} [12.22] 
		\tabularnewline \midrule
\tenevilglyph[yes][3]{i_2cU_j_2cD}
	&	ы'вэӄуч
	&	
	&
	&	uwaqug [uwæquc, ы'вэӄуч = муж] \cite[л. 65 об.]{spbfaran79} % ы'вэӄуч
	&	\cite[364]{davydova2015a} \tabularnewline \midrule
\tenevilglyph[yes][4]{i_cUY_2cD}
	&	иръын
	&	
	&
	&	
	&	irьn [иръын = кухлянка, мужская меховая рубаха, пальто, верхняя одежда; слово напечатано] [7.13, 12.23] \linebreak
		rarrok [ратрок = класть за пазуху; слово напечатано] \currentGlyphWithAffixes{}{R,K} [12.19об] % NB: Обрати внимание, фонетический ключ используется для совсем другого слова
		\tabularnewline \midrule
\tenevilglyph[yes][4]{i_cUY_2cD_2q}
	&	туриръын
	&	
	&
	&	
	&	tur-irьn [туриръын = новая кухлянка; слово напечатано] [7.13] % TODO: уточнить перевод
		\tabularnewline \midrule
\tenevilglyph[yes][5]{i_2cU_2C}
	&	ӈэвысӄэт
	&	a woman \cite{mindalevich1934}
	&	н'эвыскэт [ӈэвысӄэт], женщина [26]
	&	ŋәucqan [ŋausqan, ӈэвысӄэт = женщина] \cite[л. 65 об.]{spbfaran79} % ӈэвысӄэт
	&	\cite[364]{davydova2015a} \linebreak
		щенщны [женщины] [34.16]
		\tabularnewline \midrule
\tenevilglyph[yes][3]{i_2cU_j_2C}
	&
	&	
	&
	&	жена \cite[л. 65 об.]{spbfaran79}
	&	\cite[364]{davydova2015a}
		\tabularnewline \midrule
\tenevilglyph[yes][5]{i_2cU_l_2C}
	&	ытԓя
	&	
	&
	&	мать \cite[л. 64]{spbfaran79}\linebreak
		әtla [ытԓя = мать] \cite[л. 52]{spbfaran79}\linebreak % ытԓя
		etla [әtla] \cite[л. 52 об., 56]{spbfaran79}
	&	\cite[360, 364]{davydova2015a} \linebreak
		ьla [әtla, ытԓя] [ИЛИ:1.21]  % ilispread:1.21
		\tabularnewline \midrule
\tenevilglyph[yes][3]{i_2cU_t_2C}
	&
	&	a woman awaiting the birth of her child \cite{mindalevich1934}
	&
	&	родившая мать \cite[л. 64]{spbfaran79}
	&	[65.6об]
		\tabularnewline \midrule
\tenevilglyph[yes][3]{i_2cU_2C_h}
	&	
	&	
	&
	&	ьnpьŋәu [ьnpь-ŋæw, ынпыӈэв = старуха] \cite[л. 65 об]{spbfaran79} % ынпыӈэв
	&	[25.6]
	 	\tabularnewline \midrule
\tenevilglyph[yes][5]{i_2CF}
	&	экык
	& 	сын \cite{bogoraz1934}\linebreak
		сын \cite{lavrov1969}
	&	экык [экык = сын], сын [18] \linebreak
		эккин, сына [18] \linebreak
		эккет [эккэт], cыновья \currentGlyphWithAffixes{}{T} [18]
	&	сын \cite[л. 52]{spbfaran79}\linebreak
		сыновья \cite[л. 52]{spbfaran79} \linebreak
		әkkot [ækkæt, эккэт = сыновья] \cite[л. 39]{spbfaran79} \linebreak % экык, эккэт
		син [сын] \cite[л. 67]{spbfaran79}
	&	\cite[364]{davydova2015a} \linebreak 
		\cite{bogoraz1934} \linebreak
		ekьk [экык] [12.11] \linebreak
		ekkin [эккин = детский; слово напечатано] \currentGlyphWithAffixes{}{N} [12.13] \linebreak
		ekket [эккэт; слово напечатано] \currentGlyphWithAffixes{}{T} [12.11] \linebreak
		ekkete [эккэтэ = сыновья; слово напечатано] \currentGlyphWithAffixes{}{T} [12.11] \linebreak % TODO: уточнить перевод
		ekket [эккэт] \currentGlyphWithAffixes{}{T} [ИЛИ:2.22]  % ilispread:2.21
		\tabularnewline \midrule
\tenevilglyph[yes][5]{i_2cU_CF}
	&	ӈээкык
	&	
	&
	&	доса [дочь] \cite[л. 67]{spbfaran79}
	&	ееуске [девушка] [29.2об] \linebreak
	 	~[25.8об] \linebreak
	 	ŋэkьk [ӈээкык = дочь] [ИЛИ:1.23]  % ilispread:1.24
	 	\tabularnewline \midrule
\tenevilglyph[no][3]{i_2cU_3CF}
	&
	&	
	&
	&	ситра [сестра] \cite[л. 67]{spbfaran79} 
	& 	\tabularnewline \midrule
\tenevilglyph[no][3]{i_2CF_v_q_'}
	&
	&	
	&
	&	прат [брат] \cite[л. 67]{spbfaran79} % йичьэмиттумгын?
	& 	\tabularnewline \midrule
\tenevilglyph[yes][5]{i_vd_q_i} 
	&	тумгытум
	&	друг \cite{lavrov1969}
	&	тумгытум [= товарищ], товарищ [50]
	&	
	& 	\cite[364]{davydova2015a} \linebreak
		tumgьt [тумгыт = друзья; слово напечатано] [12.20] \linebreak 
		таварес [товарищ] [34.8об] \linebreak
		j\=elgь-tomgьn [йъэԓгытомгын = двоюродный брат; слово напечатано] \currentGlyphWithAffixes{jxilgyn}{} [19.5об]
		\tabularnewline \midrule
\tenevilglyph[yes][5]{i_2CF_j}
	&	нэнэны, ӈинӄэй
	&	
	&
	&	qlaul nenene [qlaul nænænæ, ӄԓявыԓ нэнэны = мужчина младенец] \cite[л. 65 об]{spbfaran79} % ӄԓявыԓ нэнэны
	& 	\cite[364]{davydova2015a} \linebreak
		nenenь [нэнэны = дитя, ребенок; слово напечатано] [7.13] \linebreak
		ŋenqej [ŋenqәj, ӈинӄэй = мальчик] [ИЛИ:1.7] \linebreak  % ilispread:1.7
		ŋenqьte [ӈинӄэгти = дети] [ИЛИ:1.10]  % ilispread:1.10
		\tabularnewline \midrule
\tenevilglyph[yes][3]{i_2cU_CF_h}
	&
	&	
	&
	&	ŋәusqan neneneŋ [ŋausqan nænænæŋ, ӈэвысӄэт нэнэны = женщина мланедец] \cite[л. 65 об]{spbfaran79} % ӈэвысӄэт нэнэны
	& 	[34.9]
		\tabularnewline \midrule
\tenevilglyph[yes][5]{o-_p_j}
	&	ытԓён
	& 	он \cite{bogoraz1934}
	&	ытлён [ытԓён], он [92]
	&	он \cite[л. 40]{spbfaran79} \linebreak 
		әtlon [ытԓён = он] \cite[л. 39 об, 52, 65 об]{spbfaran79} % ытԓён
	& 	\cite[360]{davydova2015a} \linebreak
		ылтон [ытԓён] [32.16об] \linebreak
		ьlon [ытԓён] [ИЛИ:2.3об]  % ilispread:2.3
		\tabularnewline \midrule
\tenevilglyph[yes][5][muri]{o_2j}
	&	мури, мургин
	& 	наш \cite{bogoraz1934}\linebreak
		we \cite{mindalevich1934}
	&	мури, мы [92]
	&	наш \cite[л. 40]{spbfaran79} \linebreak
		murgin [мургин = наш] \cite[л. 52]{spbfaran79} \linebreak % мургин
		muri [мури = мы] \cite[л. 39 об, 65 об]{spbfaran79} \linebreak % мури
		мы \cite[л. 68]{spbfaran79} \linebreak
		наса [наша] \cite[л. 68]{spbfaran79}
	& 	\cite[364]{davydova2015a} \linebreak
		\cite[28]{lavrov1969} \linebreak
		murgin [мургин; слово напечатано] [12.17] \linebreak
		муре [мури] [42.3] \linebreak
		mure [мури] [ИЛИ:2.27об] \linebreak  % ilispread:2.27
		мурхин [мургин] \currentGlyphWithAffixes{}{N} [41.7] \linebreak
		murgin [мургин; слово напечатано] \currentGlyphWithAffixes{}{N} [12.22] \linebreak
		murьk [мурык = у нас; слово напечатано] \currentGlyphWithAffixes{K}{} [12.22] \linebreak
		murgen [мургин] \currentGlyphWithAffixes{}{N} [ИЛИ:2.27об] \linebreak  % ilispread:2.27
		murgьnan [моргынан = мы (форма подлежащего при переходном глаголе)] \currentGlyphWithAffixes{}{ynan} [ИЛИ:2.26об]  % ilispread:2.26
		\tabularnewline \midrule
\tenevilglyph[yes][4]{o_2j_l}
	&	моргынан
	& 	
	&	мургынан [моргынан = мы (форма подлежащего при переходном глаголе)], нами [94]
	&	
	& 	\cite[364]{davydova2015a} \linebreak
		murgьnan [morgьnan, моргынан] [ИЛИ:2.1об]  % ilispread:2.1
		\tabularnewline \midrule
\tenevilglyph[yes][5]{o_j}
	&	гымнин
	& 	мой \cite{bogoraz1934}
	&	гымнин, мой [93]
	&	мой \cite[л. 40, 55]{spbfaran79} \linebreak
		gьmnin [гымнин = мой] \cite[л. 56]{spbfaran79} \linebreak % гымнин
		gumnin [gьmnin] \cite[л. 52 об, 65]{spbfaran79}
	&	мена [меня] [37.2] \linebreak
		gьmnin [гымнин; слово напечатано] [12.19об] \linebreak
		gьmnenet [гымнинэт = мои] \currentGlyphWithAffixes{}{N,N,T} [ИЛИ:2.22]  % ilispread:2.21
		\tabularnewline \midrule
\tenevilglyph[yes][5]{o}
	&	гым
	& 	я \cite{bogoraz1934}
	&	гым, я [92]
	&	я \cite[л. 40, 53, 65 об]{spbfaran79} \linebreak
		gьm [гым = я]\cite[л. 52,56]{spbfaran79} \linebreak % гым
		gum [gьm] \cite[л. 52 об, 65 об]{spbfaran79}
	& 	\cite[364]{davydova2015a} \linebreak
		хым [гым] [34.8] \linebreak
		я [34.11] \linebreak
		gьm [гым] [ИЛИ:2.7]  % ilispread:2.6
		\tabularnewline \midrule
\tenevilglyph[yes][5][gymnan]{o_j_q}
	&	гымнан
	&	
	&	гымнан, мой [93]
	&	мне \cite[л. 66]{spbfaran79} \linebreak
		в \textit{«мне»}, \textit{«я} восму» \cite[л. 66]{spbfaran79} \linebreak
		в \textit{«я} ниснаю», \textit{«я} упрала» \cite[л. 79]{spbfaran79}
	&	\cite{bogoraz1934} \linebreak
		gьmnan [гымнан = я, форма подлежащего при переходных глаголах; слово напечатано] [12.28] \linebreak
		gьmnan [гымнан] [ИЛИ:1.9об]  % ilispread:1.10
		\tabularnewline \midrule
\tenevilglyph[yes][5]{o-_s}
	&	гыт
	&	
	&
	&	gьt [гыт = ты] \cite[л. 65 об]{spbfaran79} % гыт
	& 	ььt [гыт] \currentGlyphWithAffixes{}{T} [5.1об] % закорючка справа, вероятно, фонетическая 
		\tabularnewline \midrule
\tenevilglyph[yes][4]{o-_jY}
	&	тури
	&	
	&	тури, вы [92]
	&	turi [тури = вы] \cite[л. 65 об]{spbfaran79} % тури
	& 	[57.33об]
		\tabularnewline \midrule
\tenevilglyph[yes][4]{o_j_j}
	&	эвытԓым
	&	
	&
	&	или такои [или такой] \cite[л. 67]{spbfaran79} \linebreak
		эвлм [эвытԓым = или] \cite[л. 68]{spbfaran79} % ЗТ
	& 	[38.1] \linebreak
		eulьm [эвытԓым] [ИЛИ:1.10]  % ilispread:1.10
		\tabularnewline \midrule
\tenevilglyph[yes][5]{o-_j}
	&	ынин
	&	
	&	ынин [= его, её], его [93]
	&	
	& 	\cite[360, 361, 362, 364]{davydova2015a} \linebreak
		ьnen [ынин] [ИЛИ:1.4] \linebreak  % ilispread:1.4
		ьnnen [ынин] [ИЛИ:1.8об]  % ilispread:1.9
		\tabularnewline \midrule
\tenevilglyph[yes][4]{o-_j_l}
	&	гынан
	&	
	&	
	&	
	& 	gьnan [гынан = ты (употребляется с переходным глаголом); слово напечатано] [12.25]
		\tabularnewline \midrule
\tenevilglyph[yes][4]{o-_j_2cD}
	&	этын
	&	
	&	этынва, хозяином \currentGlyphWithAffixes{}{A} [95] % TODO: уточнить перевод
	&	хозяин* \cite[л. 51]{spbfaran79}
	& 	эetьn [etьn, этын = хозяин] [ИЛИ:1.8об, ИЛИ:2.8] \linebreak   % ilispread:1.9, ilispread:2.7
		etьnwьt [этынвыт = хозяева] [ИЛИ:1.16об]  % ilispread:1.17
		\tabularnewline \midrule
\tenevilglyph[yes][3]{o-_j_jY}
	&	ӄымэԓыргынан
	&	
	&	
	&	
	& 	qьmelьrgьnan [ӄымэԓыргынан = так что] [ИЛИ:1.18об]  % МП, Weinstein % ilispread:1.19
		\tabularnewline \midrule
\tenevilglyph[yes][5][ynan]{o_l}
	&	ынан
	&	
	&	
	&	он \cite[л. 40]{spbfaran79}
	& 	ьnan [ынан = он (употребляется с переходным глаголом), он сам] [ИЛИ:1.8об]  % ilispread:1.9
		\tabularnewline \midrule
\tenevilglyph[yes][5]{o_l_jY}
	&	ыргынан
	&	
	&	ыргынан, ими [94]
	&	ergьnen [ыргынан = они (употребляется с переходным глаголом)] \cite[л. 56]{spbfaran79}
	& 	\cite[364]{davydova2015a} \linebreak
		ьrgьnan [ыргынан] [ИЛИ:1.2об, ИЛИ:1.10об]   % ilispread:1.3 , ilispread:1.11
		\tabularnewline \midrule
\tenevilglyph[yes][4]{o_l_j2Y}
	&	ыргин
	&	
	&	
	&	
	& 	ьrgen [ыргин = их, принадлежащий им] [ИЛИ:1.2об, ИЛИ:2.1об]   % ilispread:1.3, % ilispread:2.1
		\tabularnewline \midrule
\tenevilglyph[yes][4]{o_l_j2Y_2c}
	&	эмыргин
	&	
	&	
	&	
	& 	эmьrgen [эмыргин = именно их] [ИЛИ:1.16об]  % МП % ilispread:1.17
		\tabularnewline \midrule
\tenevilglyph[yes][5]{R_2bN}
	&	купрэн
	& 	сеть \cite{bogoraz1934}\linebreak
		a net \cite{mindalevich1934}
	&	купрэн [= сеть, невод], гин'инги [гиӈынгиӈ], сеть [113]
	&	сеть \cite[л. 40]{spbfaran79} \linebreak
		giŋingi [giŋingь, гиӈынгиӈ = сеть] \cite[л. 39]{spbfaran79} \linebreak % гиӈынгиӈ
		сетка \cite[л. 68]{spbfaran79}
	& 	\cite[361]{davydova2015a} \linebreak
		\cite{bogoraz1934} \linebreak
		копрен [купрэн] [42.1] \linebreak
		kupret [купрэт = сети; слово напечатано] [12.25] \linebreak
		kuprete [купрэтэ = сетями; слово напечатано] \currentGlyphWithAffixes{}{T} [12.25]
		\tabularnewline \midrule 
\tenevilglyph[yes][3]{R_2bN_bXX}
	&	
	& 	
	&	
	&	свашот [свяжут] \cite[л. 68]{spbfaran79} % TODO: уточнить интерпретацию
	& 	~[4.4об]
		\tabularnewline \midrule 
\tenevilglyph[yes][3]{j-jXXX} 
	&	рырыткук
	&	
	&	
	&	
	&	ррткок [рырыткук = расстилать, распутывать, развязывать] [42.2]  % в контексте сети
		\tabularnewline \midrule 
\tenevilglyph[yes][2]{sME_2b}
	&
	&	как звать? \cite{lavrov1969}
	&	как тебя звать [16]
	&	Teŋiwil — автор записей \cite[л. 40, 52, 54]{spbfaran79} \linebreak
		вместе* \cite[л. 51]{spbfaran79} % похоже на ошибку, знак перевернут
	& 	\cite[360–364]{davydova2015a} \linebreak
		Tьŋewil [Теневиль] [4.7] \linebreak
		%Tьŋweil [7.29об] \linebreak
		атехае [атеӄай = папа] [33.5об] \linebreak
		атегаи [атеӄай] [30.3] \linebreak
		Тынеувел [Теневиль] [35.3] \linebreak
		Тыневил [Теневиль; не рукой Теневиля, подпись в ведомости приема-сдачи товаров] [82.1об] \linebreak
		папа \currentGlyphWithAffixes{}{Q,A} [30.1об]
		\tabularnewline \midrule
\tenevilglyph[yes][5]{sME}
	&	Теневиль
	&	Теневиль \cite{lavrov1969}
	&	имя Теневиля
	&
	& 	\cite[361]{davydova2015a} \linebreak
		\cite[28]{lavrov1969} \linebreak
		Тынеувел [Теневиль] [33.5об] \linebreak
		Тынувел [Теневиль] [45.2] \linebreak
		Tьŋewil [ИЛИ:1.20об] \linebreak  % ilispread:1.21
		Tьŋewelьn [ИЛИ:1.21об] \linebreak  % ilispread:1.22
		Tьŋeweilьn \currentGlyphWithAffixes{}{N} [ИЛИ:1.1] \linebreak  % ilispread:1.1
		Tьŋeweleьm \currentGlyphWithAffixes{}{M} [ИЛИ:2.22]  % ilispread:2.21
		\tabularnewline \midrule
\tenevilglyph[yes][4]{sM_2b}
	&	атэ
	&	
	&	атэ [= папа], папа [113] % на одну петельку меньше
	&
	& 	atena [атэна = папа (дательный падеж); слово напечатано] [7.13,12.10об,12.13,12.15,12.17об] % TODO: уточнить перевод
		\tabularnewline \midrule
\tenevilglyph[yes][2]{i_2lY}
	&
	&	Раглине [жена Теневиля] \cite{lavrov1969}
	&	Сергей (имя) [117]
	&
	& 	\cite[364]{davydova2015a} \linebreak
		\cite[28]{lavrov1969} 
		\tabularnewline \midrule
\tenevilglyph[yes][5]{i_l_q_lY}
	&	Рая, Раулина
	&	
	&	Раглынэ (Рая), имя жены Теневиля* [118]
	&
	& 	raьlьŋa [Раулина] [26.40об] \linebreak % чуть иначе нарисовано, в 39.4об так же
		Rая [Рая\cite{druri1989} = Раулина, жена Теневиля] [32.10, 39.4об]
		\tabularnewline \midrule
\tenevilglyph[yes][5]{i_2cY}
	&	Егор, Этувьи
	&	Этувьи \cite{lavrov1969}
	&	Этувьи (Егор), имя сына Теневиля [7]
	&
	& 	\cite[361, 363]{davydova2015a} \linebreak
		\cite[28]{lavrov1969} \linebreak
		Еекор [Егор, Этувьи\cite{lavrov1969} — сын Теневиля] [29.2, 33.5об] \linebreak
		Етуйэ [не рукой Т.] [57.9] \linebreak
		Eekor [Егор] [ИЛИ:1.9,ИЛИ:2.10]  % ilispread:2.9
		\tabularnewline \midrule
\tenevilglyph[yes][5]{UD_2b}
	& 	Эйчинкеу
	&	
	&	Эйчинкеу, имя сына Теневиля [115]
	&
	& 	\cite[362, 363]{davydova2015a} \linebreak
		\cite[28]{lavrov1969} \linebreak
		Эехынкеу [Игеннеу\cite{mindalevich1934a}, Эгенкау\cite{sergeev1956}, Эйчинкеу [46] — второй сын Теневиля] [29.13, 33.5об, 35.3] \linebreak
		Эхынкеу [29.2] \linebreak
		Эгинкеу [не рукой Т.] [57.9] \linebreak
		Эejnkeu [Эйчинкеу] [ИЛИ:1.9] \linebreak % ilispread:1.9
		Эegьnkeu [Эйчинкеу] [ИЛИ:2.9об] % ilispread:2.9
		\tabularnewline \midrule
\tenevilglyph[yes][5]{b-B}
	&	Mаусек
	&	
	&	Маусек, имя старшего сына Теневиля
	&
	& 	\cite[361, 362, 363]{davydova2015a} \linebreak
		Mаусек [[46] старший сын Теневиля] [33.5об] \linebreak
		Маузик [не рукой Т.] [57.9] \linebreak
		Maucek [Маусек] [ИЛИ:1.9]  % ilispread:1.9
		\tabularnewline \midrule
\tenevilglyph[yes][3]{U_2j}
	&	Вантоли
	&	
	&	Вантолин (имя)
	&
	& 	\cite[363]{davydova2015a} \linebreak
		Вантоли [не рукой Т.] [57.9]
		\tabularnewline \midrule
\tenevilglyph[yes][1]{i_2cU_CF_i_2l} % TODO: Вероятно имя собственное, но надо проверить
	&
	&	
	&
	&
	& 	Еетхеут [29.2, 35.3, 45.7]
		\tabularnewline \midrule
\tenevilglyph[yes][1]{f_i_2l} % TODO: Вероятно имя собственное, но надо проверить
	&
	&	
	&
	&
	& 	Еатхеынын [30.6об]
		\tabularnewline \midrule
\tenevilglyph[yes][4]{i_2cU_CF_i_2j}
	&
	&	
	&	Эвнэут, имя дочери Теневиля [125]
	&
	& 	iеунеут [Эвнэут] [29.13, 32.10, 33.5об, 35.3] \linebreak
		\.{и}еунеут [Эвнэут] [45.6об] \linebreak
		ieuŋeut [Эвнэут] [ИЛИ:1.9,ИЛИ:2.10об] % ilispread:1.9, ilispread:2.10
		\tabularnewline \midrule
\tenevilglyph[yes][1]{iY_2cDX_jF} % TODO: Вероятно имя собственное, но надо проверить
	&
	&	
	&
	&
	& 	нэiэн [29.10] \linebreak
		неiенхе [30.1об] \linebreak
		нейге [45.6об]
		\tabularnewline \midrule
\tenevilglyph[yes][2]{i_c_C_i_j}
	&
	& 	мать \cite{bogoraz1934}\linebreak
		мать \cite{lavrov1969}
	&	ытля [= мать], мать [69]
	&	мать \cite[л. 40]{spbfaran79} \linebreak
		Upenkew — враг автора \cite[л. 40]{spbfaran79} % Ближайшее по звучанию — шаман Упетеу \cite{mindalevich1934a}
	& 	[1.1] 
		\tabularnewline \midrule
\tenevilglyph[no][3]{i_c_C}
	&
	&	
	&
	&	Utenkew [Утенки, хозяин стойбища, люди которого отняли стадо Теневиля \cite[л. 9]{spbfaran79}] \cite[л. 52 об]{spbfaran79} \linebreak
		Utenkew [Утенки] \cite[л. 56]{spbfaran79}
	& 	\tabularnewline \midrule
\tenevilglyph[yes][5]{IY_j}
	&	чинит, чиниткин
	& 	сам \cite{bogoraz1934}
	&	чиниткин [= свой], свой, своя, свое, чинит [= сам], сам, сама, само [6]
	&	сам \cite[л. 40, 53]{spbfaran79} \linebreak
		cinit [cinit, чинит = сам] \cite[л. 52]{spbfaran79} \linebreak % чинит
		ꞓinit [cinit] \cite[л. 52 об]{spbfaran79}
	& 	\cite[364]{davydova2015a} \linebreak
		\cite{bogoraz1934} \linebreak
		чам [сам] [32.6об] \linebreak
		сам [34.8об] \linebreak
		сенеткен [чиниткин] [45.4об] \linebreak
		cenet [чинит] [ИЛИ:2.12] \linebreak  % ilispread:2.11
		cenetken [чиниткин] [ИЛИ:2.21]  % ilispread:2.20
		\tabularnewline \midrule
\tenevilglyph[yes][5]{iY}
	&	ынӄэн
	& 	тот \cite{bogoraz1934}
	&	ынкэн [ынӄэн], тот, та, то [52]
	&	тот \cite[л. 40]{spbfaran79} \linebreak
		әnqon [әnqan, ынӄэн = тот, этот] \cite[л. 52, 54]{spbfaran79} % ынӄэн
	& 	\cite[360, 361, 364]{davydova2015a} \linebreak
		\cite[28]{lavrov1969} \linebreak
		nqen [ынӄэн] [4.10об] \linebreak
		ето [это] [32.16] \linebreak
		ьnqen [ынӄэн] [ИЛИ:1.3об] \linebreak  % ilispread:1.4
		ынген [ынӄэн] [41.44] \linebreak
		ьnqenat [ынӄэнат = эти] \currentGlyphWithAffixes{}{T} [ИЛИ:1.8об] % ilispread:1.9
		\tabularnewline \midrule
\tenevilglyph[yes][5]{iY-2cD}
	&	ынӄоры
	& 	после того \cite{bogoraz1934}
	&	ынкоры [ынӄоры], после того, оттуда, потом [127]
	&	после того, әŋqorә [ынӄоры = потом, затем, оттуда] \cite[л. 51, 53]{spbfaran79} \linebreak
		әnqre \cite[л. 39]{spbfaran79} 
	& 	\cite[361, 362, 364]{davydova2015a} \linebreak
		\cite[28]{lavrov1969} \linebreak
		ьnqurь [ынӄоры] [ИЛИ:1.5] \linebreak  % ilispread:1.5
		әnqo [ынӄо = оттуда; слово напечатано] [12.25] % тут особенно похоже на модификацию ынӄэн
		\tabularnewline \midrule
\tenevilglyph[yes][5]{iY_q}
	&	ӈанӄэн
	& 	
	&	наанкэн [ӈанӄэн = вон тот (видимый), далекий], вон (там) [52]
	&	
	& 	\cite[364]{davydova2015a} \linebreak
		\cite[28]{lavrov1969} \linebreak
		ŋanqen [ŋanqәn, ӈанӄэн] [ИЛИ:1.2]  % ilispread:1.2
		\tabularnewline \midrule
\tenevilglyph[yes][5]{d_C}
	&	уйӈэ
	& 	нет \cite{bogoraz1934}\linebreak
		no \cite{mindalevich1934}
	&	уйн'э [уйӈэ], нет [51]
	&	нет \cite[л. 40]{spbfaran79} \linebreak
		ujŋә [уйӈэ = нет чего-нибудь] \cite[л. 39]{spbfaran79} \linebreak % уйӈэ
		нету \cite[л. 66 об]{spbfaran79} \linebreak
		в \textit{«не}било» \cite[л. 66]{spbfaran79}
	& 	\cite[360, 361, 364]{davydova2015a} \linebreak
		\cite[28]{lavrov1969} \linebreak
		нету [32.18] \linebreak
		weŋe [уйӈэ] [ИЛИ:1.22] \linebreak  % ilispread:1.23
		uwŋe [уйӈэ] [ИЛИ:2.2об]  % ilispread:2.2
		\tabularnewline \midrule
\tenevilglyph[yes][3]{d_C_b}
	&
	& 	
	&	кунинет, кончили, истребили [51] % TODO: нужен перевод
	&	
	& 	getkulin [= убили; слово напечатано] [12.20] \linebreak % TODO: уточнить перевод, нужна транскрипция
		kuninet [= истребили; слово напечатано] \currentGlyphWithAffixes{}{T} [12.13]
		\tabularnewline \midrule
\tenevilglyph[yes][5]{G}
	&	игыр, игыркин
	& 	теперь \cite{bogoraz1934}
	&	игыр, игыт, сегодня, сейчас, теперь [64] 
	&	теперь \cite[л. 40]{spbfaran79} \linebreak
		igьt [igьr, игыр = сегодня, теперь] \cite[л. 39, 52 об]{spbfaran79} \linebreak % игыр
		чиперче [теперь] \cite[л. 67 об]{spbfaran79} \linebreak
		свонтенеи [сегодня] \cite[л. 67 об]{spbfaran79} 
	& 	\cite[361, 364]{davydova2015a} \linebreak
		\cite[28]{lavrov1969} \linebreak
		сесеяс [сейчас] [37.2] \linebreak
		свонтенеи [сегодня] [37.2] \linebreak
		ieьr [igьr, игыр] [ИЛИ:1.6об] \linebreak  % ilispread:1.7
		ieьrken [игыркин = cегодняшний] [ИЛИ:2.6об] \linebreak  % ilispread:2.6
		ieьtken [игыткин = сегодняшний] \currentGlyphWithAffixes{}{K} [ИЛИ:1.20]  % ilispread:1.20
		\tabularnewline \midrule
\tenevilglyph[yes][4]{G_'}
	&	игыттэгнык
	& 	
	&	
	&	
	& 	ieьteьnk [игыттэгнык = до сего дня, до сих пор] [ИЛИ:1.4] \linebreak  % NB: обрати внимание ` тут и еще в паре мест обозначает -тэгнык % ilispread:1.4
		ieьteьnьk [игыттэгнык] \currentGlyphWithAffixes{}{T} [ИЛИ:2.3об]  % ilispread:2.3
		\tabularnewline \midrule
\tenevilglyph[yes][5]{o_q}
	&	ынкы
	& 	там \cite{bogoraz1934}
	&	ынкы, там
	&	там \cite[л. 50]{spbfaran79} \linebreak
		әnkь [ынкы = там] \cite[л. 39 об]{spbfaran79} \linebreak % ынкы
		тут \cite[л. 66]{spbfaran79} \linebreak
		дут [тут] \cite[л. 68]{spbfaran79}
	& 	\cite[360, 361, 364]{davydova2015a}\linebreak 
		\cite[28]{lavrov1969}\linebreak 
		әnkь [ынкы; слово напечатано] [12.17] \linebreak
		тот [тут] [32.13] \linebreak
		вотут [вот тут] \currentGlyphWithAffixes{votqen}{} [32.6] \linebreak
		тут [32.15об] \linebreak
		ынкы [32.16об, 42.31] \linebreak
		ьnkь [ынкы] [ИЛИ:1.3] \linebreak  % ilispread:1.3
		ьnkeken [ынкэкин = тамошний] \currentGlyphWithAffixes{}{N} [ИЛИ:2.12] \linebreak  % ilispread:2.11
		эneru \currentGlyphWithAffixes{}{N,E,R,A} [ИЛИ:2.2]  % TODO: нужен перевод % ilispread:2.16
		\tabularnewline \midrule
\tenevilglyph[yes][5]{o_q_'}
	&	ынкэтэгнык
	& 	с тех пор \cite{bogoraz1934}
	&	ынкэтэгнык, с тех пор [46]
	&	с тех пор \cite[л. 40]{spbfaran79} \linebreak
		әnkәtegnek [ынкэтэгнык = к этому времени, в тот момент] \cite[л. 39]{spbfaran79} \linebreak % Weinstein
		әnketegnek [ынкэтэгнык]\cite[л. 39 об]{spbfaran79} \linebreak
		әnkәtegnьik [ынкэтэгнык]\cite[л. 54]{spbfaran79} 
	& 	\cite[360, 364]{davydova2015a} \linebreak
		ьnketeьnk [ынкэтэгнык] [ИЛИ:1.13]  % ilispread:1.13
		\tabularnewline \midrule
\tenevilglyph[yes][4]{o_q_2c}
	&	эмынкы
	& 	
	&	
	&	
	& 	эmьnkь [эмынкы = именно там] [ИЛИ:1.6об]  % МП % ilispread:1.7
		\tabularnewline \midrule
\tenevilglyph[yes][4]{o_q_b}
	&	ԓымынкы
	& 	
	&	
	&	
	& 	lьmьnkь [ԓымынкы = повсюду, всюду, везде] [ИЛИ:2.2об]  % ilispread:2.2
		\tabularnewline \midrule
\tenevilglyph[yes][3]{o_q-q}
	&
	& 	
	&	
	&	
	& 	\cite[360, 364]{davydova2015a} \linebreak
		qetәm [= очень; слово напечатано] [12.22] % TODO: уточнить перевод, нужна транскрипция
		\tabularnewline \midrule
\tenevilglyph[yes][3]{o_q_jFY}
	&
	& 	
	&	
	&	
	& 	\cite[364]{davydova2015a} \linebreak
		morekqatken [= в нашей стране; слово напечатано] \currentGlyphWithAffixes{muri}{} [12.20] % TODO: уточнить перевод; нужна транскрипция
		\tabularnewline \midrule
\tenevilglyph[yes][5]{l-l}
	&	ӈутку
	&	
	&	нутку [ӈутку = здесь, тут, здесь [49]
	&	
	& 	нутку [ӈутку] [34.8] \linebreak
		ŋutku [ӈутку] [ИЛИ:2.5] \linebreak  % ilispread:2.4
		ŋukeken [ӈуткэкин = здешний, местный] \currentGlyphWithAffixes{}{K,E} [ИЛИ:1.2об] \linebreak  % ilispread:1.3
		ŋutkeken [ӈуткэкин] \currentGlyphWithAffixes{}{K,K} [ИЛИ:2.19об]  % ilispread:2.19
		\tabularnewline \midrule
\tenevilglyph[yes][4]{l-l_'}
	&
	& 	с этих пор \cite{bogoraz1934}
	&	с этих пор (Б. Т. 39)
	&	с этих пор \cite[л. 40]{spbfaran79} \linebreak
		wutkelegnek \cite[л. 54]{spbfaran79} % TODO: нужен перевод выткутэгнык ? Уточнить запись
	& 	[1.4об, 4.2об] \linebreak
		ŋuketeьnk [ИЛИ:1.4] \linebreak  % TODO: нужна транскрипция % ilispread:1.4
		ŋotketeьnьk \currentGlyphWithAffixes{}{T,K} [ИЛИ:2.6]  % ilispread:2.5
		\tabularnewline \midrule
\tenevilglyph[yes][3]{l-l_2c}
	&
	&	
	&	
	&	
	& 	эmŋutkuken [эмӈуткукин = здешний] [ИЛИ:1.4об]  % ЗТ % ilispread:1.5
		\tabularnewline \midrule
\tenevilglyph[yes][4]{l-l_'_2cD}
	&	ӈанӄо, ӈанӄоры
	& 	
	&	
	&	
	&	ŋanqo [ӈанӄо = оттуда сюда (к говорящему); слово напечатано] [12.20об] \linebreak
		ŋanqorь [ӈанӄоры = сюда, оттуда сюда (к говорящему)] [ИЛИ:1.20]  % ilispread:1.20
		\tabularnewline \midrule
\tenevilglyph[yes][3]{2i_P}
	&
	& 	на реке \cite{bogoraz1934}
	&
	&	на реке \cite[л. 41]{spbfaran79} \linebreak
		veemьk [vææmьk, вээмык = на реке] \cite[л. 39]{spbfaran79} % вээмык
	& 	\cite[361]{davydova2015a} 
		\tabularnewline \midrule
\tenevilglyph[yes][4]{2i_2q}
	&	вээм
	&	
	&	веем [вээм = река], река [41]
	&	vaamete [vææmьte, вээмэты = к реке] \cite[л. 56]{spbfaran79} \linebreak % вээмэты
		около рещки [около речки] \cite[л. 68 об]{spbfaran79}
	& 	\cite[361]{davydova2015a} \linebreak
		\cite[28]{lavrov1969} \linebreak
		weemьk [вээмык; слово напечатано] [12.19об] \linebreak
		waametь [вээмэты; слово напечатано] \currentGlyphWithAffixes{}{T} [12.24об]
		\tabularnewline \midrule
\tenevilglyph[yes][4]{i_g_b_jX}
	&	кычав
	& 	хариус \cite{bogoraz1934}
	&	кычав, хариус [22] % 
	&	хариус \cite[л. 41, 54 об]{spbfaran79} \linebreak
		qeꞓaw [kьcaw, кычав = хариус] \cite[л. 39]{spbfaran79} % кычав
	& 	\cite[361]{davydova2015a} \linebreak
		хартон [33.4] % TODO: нужна интерпретация
		\tabularnewline \midrule
\tenevilglyph[yes][5]{i_g_b}
	&	ыннээн, ӄэтаӄэт
	& 	кета \cite{bogoraz1934}\linebreak
		рыба кета \cite{lavrov1969}
	&	ыннээн [= рыба], рыба [22]
	&	кета \cite[л. 44, 45, 54 об]{spbfaran79} \linebreak
		рипа [рыба] \cite[л. 68 об]{spbfaran79}
	& 	\cite[361]{davydova2015a} \linebreak 
		\cite[26]{lavrov1969} \linebreak
		ьnnьt [ынныт = рыбы; слово напечатано] [12.24об] \linebreak 
		ьnnьt [ынныт; слово напечатано] \currentGlyphWithAffixes{}{T} [12.24об] \linebreak
		qetaqet [ӄэтаӄэт = кета; слово напечатано] [12.24об] \linebreak 
		кетат [ӄэтат = кеты] \currentGlyphWithAffixes{}{Q,T} [33.4] \linebreak
		эрепо [рыба] [33.4]
		\tabularnewline \midrule
\tenevilglyph[yes][3]{i_g_2b}
	&
	& 	налим \cite{bogoraz1934}\linebreak
		налим \cite{lavrov1969}
	&	гачгагыргын [= налим], налим [22]
	&	налим \cite[л. 45, 54 об]{spbfaran79} 
	& 	[1.30]
		\tabularnewline \midrule
\tenevilglyph[yes][3]{i_g_b_z}
	&
	&	
	&
	&	сиг \cite[л. 45]{spbfaran79} 
	& 	[15.11] 
		\tabularnewline \midrule
\tenevilglyph[yes][4]{i_g_b_hL}
	&
	& 	щука \cite{bogoraz1934} \linebreak
		щука \cite{lavrov1969}
	&	юуткуннээн [юукуннээн], ээун, щука [22] % TODO: нужна транскрипция
	&	щука \cite[л. 45]{spbfaran79} 
	& 	[2.101] \linebreak
		~[рядом с изображением щуки] [12.18]
		\tabularnewline \midrule %
\tenevilglyph[no][3]{i_g_2b_q_k}
	&
	&	
	&
	&	бычок \cite[л. 45]{spbfaran79} 
	& 	\tabularnewline \midrule
\tenevilglyph[yes][3]{i_g_b_2cD}
	&	вирувир
	&	нярка \cite{lavrov1969}
	&	вирувир [= нерка], нярка (род рыбы) [22]
	&	[без перевода] \cite[л. 54 об]{spbfaran79} 
	& 	\cite[361]{davydova2015a} \linebreak
		viruvir [вирувир; слово напечатано] [12.24об]
		\tabularnewline \midrule
\tenevilglyph[yes][3]{i_g_b_T}
	&	ԓыгиннээн
	&	
	&	
	&	
	& 	lgь-ьnnьt [ԓыгинныт = гольцы; слово напечатано] [12.24об]
		\tabularnewline \midrule
\tenevilglyph[yes][5]{u_j_jX-l_j}
	&	рооԓӄыԓ
	& 	еда, есть \cite{bogoraz1934} \linebreak
		food \cite{mindalevich1934}
	&
	&	еда, есть \cite[л. 41]{spbfaran79} \linebreak
		roьlqal [roolqәl, рооԓӄыԓ = пища, продукты] \cite[л. 39]{spbfaran79} % рооԓӄыԓ
	& 	\cite[364]{davydova2015a} \linebreak
		roolqal [roolqәl, рооԓӄыԓ; слово напечатано] [12.24об] \linebreak		roolqьl [roolqәl, рооԓӄыԓ] [ИЛИ:2.24об] \linebreak  % ilispread:2.24
		этэйнетке \currentGlyphWithAffixes{}{E,T,N,K} [42.3] % TODO: нужен перевод
		\tabularnewline \midrule
\tenevilglyph[yes][3][ruk]{u_j_jX-l} 
	&
	&	
	&	нэнунэт [нэнунэт = поели], съели, роолкыл [рооԓӄыԓ], еда [111] \linebreak % TODO: проверить перевод
		руркининэт, едят (руркинын (гл.) употребляемый в пищу) \currentGlyphWithAffixes{R}{T} [111] % TODO: нужен перевод
	&	
	& 	\cite[364]{davydova2015a} \linebreak
		nenunet [нэнунэт] \currentGlyphWithAffixes{}{T} [12.11об] \linebreak
		ruk [рук = есть] \currentGlyphWithAffixes{}{K} [ИЛИ:1.8об] \linebreak  % ilispread:1.9
		ruk [рук = есть] \currentGlyphWithAffixes{R}{} [ИЛИ:2.27об] \linebreak  % ilispread:2.27
		ruk [рук = есть] \currentGlyphWithAffixes{R}{K} [ИЛИ:2.24] \linebreak  % ilispread:2.23
		qametwak [ӄамэтвак = есть] \currentGlyphWithAffixes{}{K} [ИЛИ:1.14] \linebreak % ilispread:1.14
		nьqametwaŋuian \currentGlyphWithAffixes{}{qymek} [ИЛИ:1.21] % TODO: нужен перевод % ilispread:1.21
		\tabularnewline \midrule
\tenevilglyph[yes][5]{I_iX_2qY}
	&	айыԓгавык
	& 	бояться \cite{bogoraz1934}
	&	бояться (Б. Т. 26), апылгавык, экэлин'этык, пъэчикэтык [8] % TODO: нужна транскрипция
	&	бояться \cite[л. 41]{spbfaran79} \linebreak
		в «мы \textit{боимся»} \cite[л. 52]{spbfaran79} \linebreak
		в «я оцин \textit{боюс»} [«я очень боюсь»] \cite[л. 67 об]{spbfaran79}
	& 	поеысея [боялся] [37.2] \linebreak
		ajlgawk [айыԓгавык = бояться, испугаться] [ИЛИ:1.21об] \linebreak  % ilispread:1.22
		niajlaurkьn \currentGlyphWithAffixes{}{b,N} [ИЛИ:1.18] \linebreak  % ilispread:1.18
		ajlgaurkьn \currentGlyphWithAffixes{}{K} [ИЛИ:1.21об]  % ilispread:1.22
		\tabularnewline \midrule
\tenevilglyph[yes][4]{I_iX_u_2qY}
	&	айыԓга
	& 	
	&	
	&	
	& 	ajьla [айыԓга = бояться] [ИЛИ:1.3] \linebreak  % МП % ilispread:1.3
		ajьlga [айыԓга] [ИЛИ:1.3] \linebreak  % ilispread:1.3
		ajlga [айыԓга] [ИЛИ:1.4] \linebreak  % ilispread:1.4
		\tabularnewline \midrule
\tenevilglyph[yes][1]{I_2qY} 
	&	
	&	
	&	
	&	
	&	rьcьk \currentGlyphWithAffixes{}{K} [ИЛИ:1.5] \linebreak  % ilispread:1.5
		necьien \currentGlyphWithAffixes{}{N} [ИЛИ:1.6об] \linebreak  % ilispread:1.7
		rэсьnen \currentGlyphWithAffixes{}{N} [ИЛИ:1.7] \linebreak  % ilispread:1.7
		nьcьien \currentGlyphWithAffixes{N}{b,N} [ИЛИ:2.4]  % ilispread:2.3
		\tabularnewline \midrule 
\tenevilglyph[yes][5][tejkyk]{o_q_jF}
	&	тэйкык
	& 	сделал \cite{bogoraz1934}\linebreak
		made \cite{mindalevich1934}
	&	сделал, тэйкык [= делать, изготовлять], рытчик [рытчык = делать], делать* [67] \linebreak
		тайкыё, делаемое \currentGlyphWithAffixes{}{A} [80]
	&	сделал \cite[л. 41]{spbfaran79} \linebreak
		елат [делать] \cite[л. 68]{spbfaran79}
	& 	\cite[361, 364]{davydova2015a} \linebreak
		писат [писать] [32.6] \currentGlyphWithAffixes{}{kelikel} \linebreak
		tajkjo [тайкыё = дело, документ, делаемое] \currentGlyphWithAffixes{}{A} [ИЛИ:1.4]  % ilispread:1.4
		\tabularnewline \midrule
\tenevilglyph[yes][4]{o_l_jF}
	&	пыԓыткук
	& 	
	&	пылиткунин, закончил [79] % TODO: здесь и ниже нужна транскрипция, нужен перевод
	&	конщил* [кончил] \cite[л. 66 об]{spbfaran79} \linebreak % вверх ногами
	& 	зиокочайс [закончилось] [32.15] \linebreak
		эemlьkuke [ИЛИ:1.5] \linebreak  % ilispread:1.5
		gemlьtkulen [ИЛИ:1.10об] \linebreak  % ilispread:1.11
		mьtьplьtkunet [мытыпԓыткунэт = окончили; слово напечатано] [12.24] \linebreak % TODO: уточнить перевод и транскрипцию
		pьlьtkuk [пыԓыткук = заканчивать] [ИЛИ:1.16]  % ilispread:1.16
		\tabularnewline \midrule
\tenevilglyph[yes][5]{o_q_jF_b}
	&	ваӈэк
	&	делать, работать \cite{lavrov1969}
	&	нуван'экэн [нываӈэӄэн = шьет, работает], работает [79] 
	&	
	& 	\cite[364]{davydova2015a} \linebreak
		ырыпотали [работали] [34.8об] \linebreak
		waŋek [ваӈэк = шить, работать] \currentGlyphWithAffixes{}{K} [ИЛИ:1.13] \linebreak  % ilispread:1.13
		waŋerkn \currentGlyphWithAffixes{}{R} [ИЛИ:1.4об] \linebreak  % TODO: нужен перевод % ilispread:1.5
		nuwaŋeqen [нываӈэӄэн = работящий, шъёт] \currentGlyphWithAffixes{}{N} [ИЛИ:1.7об] \linebreak  % ilispread:1.8
		waŋejo [ваӈэё = дело] \currentGlyphWithAffixes{}{A} [ИЛИ:1.15об]  % ilispread:1.16
		\tabularnewline \midrule
\tenevilglyph[yes][5]{U_2Q}
	&	нивӄин
	& 	сказал* \cite{bogoraz1934}
	&	нивкин [нивӄин = сказал], говорит [69] 
	&	сказал* \cite[л. 41]{spbfaran79} \linebreak % тут везде знак перевернут, похоже
		giulin* \cite[л. 52]{spbfaran79} % TODO: НУЖЕН ПЕРЕВОД гэвъилин ?
	& 	каварйт [говорит] [32.16] \linebreak
		кавареу [говорю] [30.7] \linebreak
		кавареят [говорят] [30.7] \linebreak
		neuqen [нивӄин] [ИЛИ:1.4] \linebreak  % ilispread:1.4
		коварит [говорит, нивӄин] \currentGlyphWithAffixes{}{N,W,Q} [29.12] \linebreak
		teuk [тывык = говорить] \currentGlyphWithAffixes{T}{} [ИЛИ:1.5об]  % ilispread:1.6
		\tabularnewline \midrule
\tenevilglyph[yes][3]{U-k_2Q}
	&
	&	
	&
	&	
	& 	nenuqen [нинивӄин = сказал ему] [ИЛИ:1.4] \linebreak  % Weinstein TODO: уточнить перевод % ilispread:1.4
		niniwqinet [нинивӄинэт = сказал им; слово напечатано] \currentGlyphWithAffixes{}{T} [12.18об] 
		\tabularnewline \midrule
\tenevilglyph[yes][4]{U_Q_b}
	&	вэтгав
	&	
	&
	&	
	& 	wetgau [wetgaw, вэтгав = слово, речь; слово напечатано] [12.24] \linebreak
		kalewetgawkь [каԓевэтгавкы = научится читать; слово напечатано] \currentGlyphWithAffixes{}{kelikel} [12.18] \linebreak % TODO: уточнить перевод
		mьnьtwgьian \currentGlyphWithAffixes{}{M,N,A} [ИЛИ:1.2] % TODO: нужен перевод % ilispread:1.2
		\tabularnewline \midrule
\tenevilglyph[yes][4]{U_b}
	&
	&	
	&
	&	
	& 	еесик [язык] [32.6об] \linebreak
		еежг [язык] [34.16]
		\tabularnewline \midrule
\tenevilglyph[yes][5][warkyn]{c_CE}
	&	варкын, вак
	& 	пребывать, быть \cite{bogoraz1934}
	&	варкын [варкын = пребывает, живет], иметься, есть (что-нибудь), (пребывать, быть, Б. Т. 32) [71]
	&	пребывать, быть \cite[л. 41]{spbfaran79} \linebreak
		в \textit{«било} он», «не\textit{било»}, «какои» \cite[л. 66]{spbfaran79}
	& 	\cite[360, 361, 364]{davydova2015a} \linebreak
		\cite[28]{lavrov1969} \linebreak
		ееч [есть] [32.16] \linebreak
		еес [есть] [33.4] \linebreak
		warkьn [ѵarkьn, варкын] [ИЛИ:2.21об] \linebreak  % ilispread:2.21
		nьtwaqenat [нытваӄэнат = были] \currentGlyphWithAffixes{}{A,T} [26.40об] \linebreak
		mьtьtwa \currentGlyphWithAffixes{M,T}{A} [27.18об] \linebreak % TODO: нужен перевод
		warkьt [варкыт = имеются] \currentGlyphWithAffixes{}{T} [ИЛИ:1.2, ИЛИ:2.20] \linebreak   % ilispread:1.2, ilispread:2.19
		waliьt [ваԓьыт = пребывающие, живущие] \currentGlyphWithAffixes{}{T} [ИЛИ:1.17] \linebreak  % ilispread:1.17
		waliьt [ваԓьыт = пребывающие, живущие] \currentGlyphWithAffixes{}{b,T} [ИЛИ:2.26] \linebreak  % ilispread:2.25
		valьt [слово напечатано] \currentGlyphWithAffixes{}{T} [12.20] \linebreak
		vak [вак = быть; слово напечатано] \currentGlyphWithAffixes{}{K} [12.20] \linebreak
		вак \currentGlyphWithAffixes{}{W,K} [ИЛИ:41.7] \linebreak
		wak [вак] \currentGlyphWithAffixes{}{A,K} [ИЛИ:1.14] \linebreak  % ilispread:1.14
		vama [слово напечатано] \currentGlyphWithAffixes{}{A} [12.20] \linebreak
		valieьm [слово напечатано] \currentGlyphWithAffixes{}{M} [12.20] \linebreak
		valieьt [слово напечатано] \currentGlyphWithAffixes{}{b,T} [12.20] \linebreak
		waliьn \currentGlyphWithAffixes{}{A,N} [ИЛИ:1.2, ИЛИ:1.2об]   % ilispread:1.2, ilispread:1.3
		\tabularnewline \midrule
\tenevilglyph[yes][5]{UD_2B}
	&	егтэԓ
	& 	жить \cite{bogoraz1934}\linebreak
		lived \cite{mindalevich1934}\linebreak
		жить \cite{lavrov1969}
	&	нымытвак [= жить], жить (жизнь) [27]
	&	жить \cite[л. 41]{spbfaran79} \linebreak
		nyegtel \cite[л. 39]{spbfaran79} \linebreak % TODO: НУЖЕН ПЕРЕВОД ны-егтэԓ ?
		nьegtel \cite[л. 39 об]{spbfaran79} \linebreak
		сивоы [живой] \cite[л. 68]{spbfaran79}
	& 	\cite[360, 364]{davydova2015a} \linebreak
		jeьtel [егтэԓ = жизнь, средства к существованию] [ИЛИ:2.27об]  % ilispread:2.27
		\tabularnewline \midrule
\tenevilglyph[yes][5]{UE}
	&	нъэԓык
	& 	становиться \cite{bogoraz1934}
	&	нъэлык [нъэԓык = становиться], делаться, становиться \currentGlyphWithAffixes{}{K} [74] \linebreak 
		нынъэлыркын, пусть становится \currentGlyphWithAffixes{}{R,K} [74] \linebreak
		нынъэлкын, становится \currentGlyphWithAffixes{}{E,E} [74]
	&	становиться \cite[л. 41]{spbfaran79} \linebreak
		mьnneel \cite[л. 39]{spbfaran79} \linebreak % TODO: НУЖЕН ПЕРЕВОД мыннъэԓ ? мытнъэԓ?
		mьtьnnel \cite[л. 39 об]{spbfaran79} \linebreak
		mьn-neel \cite[л. 52]{spbfaran79}
	& 	\cite[360, 364]{davydova2015a} \linebreak
		nelgi [нъэԓгъи = стал; слово напечатано] [11.22] \linebreak 
		н\.{и}лх\.{и} [нъэԓгъи] [45.1об] \linebreak
		nilgei [нъэԓгъи] [ИЛИ:1.7] \linebreak  % ilispread:1.7
		nielige [нъэԓгъи] [ИЛИ:2.6об] \linebreak  % ilispread:2.6
		ninielьn [нынъэԓын = появился] \currentGlyphWithAffixes{}{N} [ИЛИ:1.5об] \linebreak  % TODO: уточнить перевод % ilispread:1.6
		nielьk [нъэԓык = становиться] \currentGlyphWithAffixes{}{K} [ИЛИ:1.13] \linebreak  % ilispread:1.13
		nьnielьn [нынъэԓын] \currentGlyphWithAffixes{}{N,N} [ИЛИ:1.22] \linebreak % ilispread:1.23
		ninialьŋunat \currentGlyphWithAffixes{}{qymek} [ИЛИ:1.20] % TODO: нужен перевод % ilispread:1.20
		\tabularnewline \midrule
\tenevilglyph[yes][5][kolo]{2OX} 
	&	коԓё мэй	
	&	
	&	нымэйынкин [нымэйыӈӄин], большой [123]
	&	mejŋь [mæjŋ, мэйыӈ = большой (основа)] \cite[л. 64 об]{spbfaran79} \linebreak % мэйыӈ
		оыё \cite[л. 66]{spbfaran79} \linebreak % TODO: нужна интерпретация
		коломеи [коԓё мэй = очень большой] \cite[л. 68 об]{spbfaran79} % коԓё мэй
	& 	\cite[361, 364]{davydova2015a} \linebreak
		\cite[28]{lavrov1969} \linebreak
		коло [коԓё = очень, слишком, ого] [41.7] \linebreak
		kolomj [kolo mej, коԓё мэй] [ИЛИ:2.27об]  % ilispread:2.27
		\tabularnewline \midrule
\tenevilglyph[yes][4]{2OX_j}
	&
	& 	расти, большой \cite{bogoraz1934}
	&	
	&	расти и большой \cite[л. 41]{spbfaran79} \linebreak
		nьmejŋqin* [nьmæjьŋqin, нымэйыӈӄин = большой] \cite[л. 54]{spbfaran79} \linebreak 
		mejn* [mæjŋ, мэйыӈ = большой (основа)] \cite[л. 39 об]{spbfaran79} 
	& 	\cite[360, 364]{davydova2015a} \linebreak
		полчои [большой]* \currentGlyphWithAffixes{}{A} [29.12] \linebreak
		necьqen [нитчыӄин = тяжелый] \currentGlyphWithAffixes{}{N,C,Q} [ИЛИ:2.26]  % ilispread:2.25
		\tabularnewline \midrule
\tenevilglyph[yes][4]{2OX_l} 
	&	ныкэтгуӄин
	&	
	&	ныкетгукин [ныкэтгуӄин = сильный, мощный могучий], сильный, мэйн'этык [мэйӈэтык = расти, развиваться, повышаться], увеличиваться* [123] % зеркально
	&	
	& 	\cite[364]{davydova2015a} \linebreak
		nьketguqenet [ныкэтгуӄинэт] [ИЛИ:1.2об] \linebreak  % ilispread:1.3
		nermeqen [нэрмэӄин = сильный] \currentGlyphWithAffixes{}{N,R,M} [ИЛИ:2.6об] \linebreak % ilispread:2.6
		katgutь\currentGlyphWithAffixes{K}{A,T} [ИЛИ:2.15об]  % TODO: нужен перевод, явно от кэтгут % ilispread:2.15
		\tabularnewline \midrule
\tenevilglyph[yes][3]{2OX_j_3q} 
	&	
	&	
	&	
	&	
	& 	maŋatьlama [майӈатыԓяма = с малых лет] \currentGlyphWithAffixes{}{M,A} [ИЛИ:1.15об] \linebreak  % ilispread:1.16
		maŋalьlama [майӈаттыԓяма] \currentGlyphWithAffixes{}{T,L} [ИЛИ:1.16]  % ilispread:1.16
		\tabularnewline \midrule
\tenevilglyph[yes][4]{o_4i}
	&
	&	
	&	убить, тымык [= убить зверя, добыть зверя] (жив.), тэгынн'этык [тэгинӈэтык = убить (человека)] (чел.) [131]
	&	nenanmuqen [нэнанмыӄэн = убил его] \cite[л. 54]{spbfaran79} \linebreak % МП
		в \textit{«убил} волка» \cite[л. 68 об]{spbfaran79} 
	& 	\cite[360, 361]{davydova2015a} \linebreak
		\cite{bogoraz1934} \linebreak
		tьnmьn [тымнэн = убил; слово напечатано] \currentGlyphWithAffixes{}{N} [12.9] \linebreak
		nenanmьqen [нэнанмыӄэн] \currentGlyphWithAffixes{}{N} [ИЛИ:1.8об] \linebreak  % ilispread:1.9
		tьmnenat [тымнэнат = убил; слово напечатано] \currentGlyphWithAffixes{}{T} [12.13об] \linebreak
		nenanmьqen [нэнанмыӄэн] \currentGlyphWithAffixes{}{N,N} [ИЛИ:2.8]  % ilispread:2.7
		\tabularnewline \midrule
\tenevilglyph[yes][4][wxik]{o_4i_k}
	&	въик
	&	
	&	въик [= умереть, скончаться], вайн'ык [вайӈык = угасать, умирать], умирать [132]
	&	умереть \cite[л. 41]{spbfaran79} \linebreak
		умирают \cite[л. 52]{spbfaran79} \linebreak
		wu [vi, въи = умереть (основа)] \cite[л. 52]{spbfaran79} \linebreak % въи
		vu [vi] \cite[л. 52]{spbfaran79} 
	& 	\cite[360]{davydova2015a} \linebreak
		омейр [умер] [32.16] \linebreak
		wiэrkьt \currentGlyphWithAffixes{}{K,T} [ИЛИ:2.21об] \linebreak  % TODO: нужен перевод % ilispread:2.21
		wiek [въик] \currentGlyphWithAffixes{}{b,K} [ИЛИ:1.13] \linebreak  % ilispread:1.13
		wieьrgьn [въэгыргын = смерть] \currentGlyphWithAffixes{}{R,N} [ИЛИ:2.7об] \linebreak  % ilispread:2.7
		nuieqen \currentGlyphWithAffixes{}{N,Q} [ИЛИ:2.8об]  % TODO: нужен перевод % ilispread:2.8
		\tabularnewline \midrule
\tenevilglyph[yes][4]{c_JY}
	&	пэԓяк
	&	
	&	ныппелакенат [ныпэԓяӄэнат = покидали], (их) покидают [8] 
	&	покинул \cite[л. 41]{spbfaran79} \linebreak
		enapelae [энапэԓя = оставлять (основа)] \cite[л. 52]{spbfaran79} \linebreak % энапэԓя 
		enapela \cite[л. 56]{spbfaran79} \linebreak
		оставил \cite[л. 68 об]{spbfaran79}
	& 	nьpelaqenat [ныпэԓяӄэнат, слово напечатано] [12.20об] \linebreak
		[25.3] 
		\tabularnewline \midrule
\tenevilglyph[yes][4]{c_sY} 
	&	ӄутти
	&	
	&	
	&	
	& 	qutti [ӄутти = другие; слово напечатано] [12.20] \linebreak % ср. ӄоԓ, совсем другой знак
		qute [ӄутти] [ИЛИ:1.9об]  % ilispread:1.10
		\tabularnewline \midrule
\tenevilglyph[yes][5]{b_2q_L}
	&	пэтԓе
	&	
	&	пэтле [пэтԓе = скоро, быстро, вскоре], скоро % очень стилизовано, не хватает одного хвостика
	&	покинул \cite[л. 41]{spbfaran79} \linebreak % Надо полагать, ошибка из-за схожести слов
		pela [пэԓя = покидать, оставлять (основа)] \cite[л. 52]{spbfaran79} % пэԓя, вероятно, неправильно 
	& 	\cite[364]{davydova2015a} \linebreak
		чкорйе [вскоре] [32.16] \linebreak
		ычкоре [вскоре] [30.6] \linebreak
		скоро [не рукой Т.] [57.9] \linebreak
		pele [petlә, пэтԓе] [ИЛИ:1.20]  % ilispread:1.20
		\tabularnewline \midrule
\tenevilglyph[yes][4]{b_2q_L_2c}
	&	эмпэтԓе
	&	
	&
	&	
	& 	эpele [эмпэтԓе = часто] [ИЛИ:1.13об]  % ilispread:1.14
		\tabularnewline \midrule
\tenevilglyph[yes][3]{4L}
	&
	&	
	&
	&	плакать \cite[л. 41]{spbfaran79} \linebreak
		nьtergьtьm [нытэргатӄэн = плачет] \cite[л. 52]{spbfaran79} % МП
	& 	\cite[360]{davydova2015a} 
		\tabularnewline \midrule
\tenevilglyph[yes][5][qorany]{a}
	&	ӄораӈы
	& 	стадо, олень \cite{bogoraz1934}\linebreak
		reindeer \cite{mindalevich1934}\linebreak
		олень \cite{lavrov1969}
	&	коран'ы [ӄораӈы = олень], олень [82]
	&	стадо, олень \cite[л. 42]{spbfaran79} \linebreak
		ŋәlvil [ŋælvьl, ӈэԓвыԓ = стадо (преимущественно оленей)] \cite[л. 56]{spbfaran79} % ӈэԓвыԓ
	& 	\cite[364]{davydova2015a} \linebreak
		\cite{bogoraz1934} \linebreak
		олене [олень] [33.4] \linebreak
		qoraŋь [ӄораӈы] [ИЛИ:2.18об] \linebreak  % ilispread:2.18
		qorat [ӄорат = олени, слово напечатано] \currentGlyphWithAffixes{}{R,T} [12.15об] \linebreak
		qaat [ӄаат = олени, слово напечатано] \currentGlyphWithAffixes{}{T} [12.15об] \linebreak
		qoren [ӄорэн = олений] \currentGlyphWithAffixes{}{N} [ИЛИ:1.2]  % ilispread:1.2
		\tabularnewline \midrule
\tenevilglyph[yes][5]{a_k}
	&	ӄэюу
	&	
	&
	&	силонок [теленок] \cite[л. 68 об]{spbfaran79} 
	& 	\cite[362]{davydova2015a} \linebreak
		[1.61] \linebreak
		qeju [qәju, ӄэюу = олений теленок] [ИЛИ:2.10об]  % ilispread:2.10
		\tabularnewline \midrule
\tenevilglyph[yes][5]{a_k_j}
	&	ӄԓикин
	&	теленок \cite{lavrov1969}
	&
	&
	& 	[1.61] \linebreak
		qьleken [ӄԓикин = теленок-бычок возрастом до года] [ИЛИ:2.10об]  % ilispread:2.10
		\tabularnewline \midrule
\tenevilglyph[yes][4]{a_k_j_'}
	&	ычвэк
	&	
	&
	&
	& 	ьcwek [æcwæk, ычвэк = олениха телка, неотелившаяся важенка] [ИЛИ:2.10об]  % ilispread:2.10
		\tabularnewline \midrule
\tenevilglyph[yes][5]{a_q}
	&	рэквыт
	&	важенка \cite{lavrov1969}
	&
	&	васонка [важенка] \cite[л. 68 об]{spbfaran79} 
	& 	[25.6об] \linebreak
		важенка [не рукой Т.] [57.26] \linebreak
		rekwьt [rækwut, рэквыт = важенка трех лет и старше] [ИЛИ:2.13]  % ilispread:2.12
		\tabularnewline \midrule
\tenevilglyph[yes][4]{a_q_l}
	&	ваӈӄасӄор
	&	
	&
	&	 
	& 	ванкаскор [ваӈӄасӄор = важенка в возрасте двух лет, яловая важенка] [25.6об] \linebreak % ваӈӄасӄор
		wakacqur [ваӈӄасӄор] [ИЛИ:2.10об]  % ilispread:2.10
		\tabularnewline \midrule
\tenevilglyph[yes][4]{a_t}
	&
	&	
	&	гыргольгын [гыръоԓьын = отелившийся, ощенившийся], отелившаяся самка* [84]
	&	силилас [телилась] \cite[л. 68 об]{spbfaran79} 
	& 	\cite[362]{davydova2015a} \linebreak
		\cite[26]{lavrov1969} \linebreak
		отел [не рукой Т.] [57.26]
		\tabularnewline \midrule
\tenevilglyph[yes][5][velwyl]{aB}
	&	ӈэԓвыԓ
	&	reindeer herd \cite{mindalevich1934}\linebreak
		стадо оленей \cite{lavrov1969}
	&	н'элвыл [ӈэԓвыԓ = стадо (преимущественно оленей)], стадо оленей [82]
	&	табун \cite[л. 55]{spbfaran79} 
	& 	\cite[361]{davydova2015a} \linebreak
		\cite[26, 28]{lavrov1969} \linebreak
		тапон [табун] [33.4] \linebreak
		нелвыл [ӈэԓвыԓ] [44.2об] \linebreak
		ŋelwьl [ŋælvьl, ӈэԓвыԓ] [ИЛИ:2.8] \linebreak  % ilispread:2.7
		ŋelwliьt [ӈэԓвыԓьыт = стада] \currentGlyphWithAffixes{}{T} [ИЛИ:2.28] \linebreak  % ilispread:2.27
		ŋelwьliьt [ӈэԓвыԓьыт = стада] \currentGlyphWithAffixes{}{b,T} [ИЛИ:2.27об]  % ilispread:2.27
		\tabularnewline \midrule
\tenevilglyph[yes][3]{a_o}
	&
	&	дикий олень \cite{lavrov1969}
	&
	&	
	& 	[18.1об?] 
		\tabularnewline \midrule
\tenevilglyph[yes][5]{a_jT}
	&	чымӈы
	&	
	&	чим'ны [чымӈы = старый бык], бык* [82]
	&	бик [бык] \cite[л. 68 об]{spbfaran79} 
	& 	[1.2] \linebreak
		cьmŋь [чымӈы] [ИЛИ:2.6]  % ilispread:2.5
		\tabularnewline \midrule
\tenevilglyph[yes][3]{a_2jX}
	&
	&	
	&
	&	хромой олень \cite[л. 43]{spbfaran79} 
	& 	[25.9] \tabularnewline \midrule
\tenevilglyph[yes][1]{b_a}
	&
	&	
	&
	&	
	& 	капетка [29.12] % TODO: проверить, может это тот же знак, что и предыдущий, что-то вроде atka-qor
		\tabularnewline \midrule
\tenevilglyph[yes][4]{a_b}
	&	пээчвак
	&	
	&	
	&	
	& 	пеcвак [пээчвак = молодой олень-самец] [29.11] \linebreak
		печвак [пээчвак] [30.5об] 		
		\tabularnewline \midrule
\tenevilglyph[yes][4]{a_bD}
	&	таачымынты
	&	
	&	пеечвак [пээчвак = молодой олень-самец], теленок оленя до одного года* [82] % вероятно, ошибка
	&	
	& 	tacьmьtь [taacьmentь, таачымынты = олень в возрасте четырех лет] [ИЛИ:2.10об]  % ilispread:2.10
		\tabularnewline \midrule
\tenevilglyph[yes][4][mooqor]{aE}
	&	мооӄор
	&	
	&	
	&	
	& 	moqor [mooqor, мооӄор = упряжной олень (грузовой)] [ИЛИ:1.3] \linebreak  % ilispread:1.3
		moqaat [мооӄаат = упряжные олени] [ИЛИ:1.4об] \linebreak  % ilispread:1.5
		moqoren \currentGlyphWithAffixes{}{N} [ИЛИ:1.6об]  % TODO: нужен перевод % ilispread:1.7
		\tabularnewline \midrule
\tenevilglyph[yes][4]{aE_'}
	&	эквэв
	&	
	&	
	&	
	& 	эkweu [ækwæw, эквэв = левый упряжной олень] [ИЛИ:2.11]  % ilispread:2.10
		\tabularnewline \midrule
\tenevilglyph[yes][5]{a_jX}
	&	ӄаанмэ
	&	
	&	каанми [ӄаанмэ = неразделанный убитый олень], каанмыйо [ӄаанмыё = тот, для которого забили, убили оленя], убитый олень, оленья туша [82]
	&	
	& 	qaanmj [qaanmi, ӄаанмэ] [ИЛИ:2.14]  % ilispread:2.13
		\tabularnewline \midrule
\tenevilglyph[yes][5]{a_lD}
	&	пэнвэԓ
	&	
	&	самец 2-х лет [84]
	&	
	& 	penwel [пэнвэԓ = олень-самец в возрасте двух лет, морж в возрасте двух лет] [ИЛИ:2.10об]  % ilispread:2.10
		\tabularnewline \midrule
\tenevilglyph[yes][4]{aY}
	&	гакаӈӄор
	&	
	&	гекенылын [гэкэӈыԓьын = ездок на оленьей упряжке], на оленях [86]
	&	
	& 	gakaqor [гакаӈӄор = ездовой олень, правый упряжной олень] [ИЛИ:2.11]  % ilispread:2.10
		\tabularnewline \midrule
\tenevilglyph[yes][1]{a_2q}
	&
	&	
	&	
	&	
	& 	kьrьmьt [ИЛИ:2.10об]  % TODO: нужен перевод % ilispread:2.10
		\tabularnewline \midrule
\tenevilglyph[yes][4]{a_q_2l}
	&	кырымӄор
	&	
	&	
	&	
	& 	kьrmqur [кырымӄор = трёхлетняя важенка] [ИЛИ:2.10об] % ilispread:2.10
		\tabularnewline \midrule
\tenevilglyph[yes][3]{a_iE-lE}
	&
	&	
	&	
	&	
	& 	tәl-qaat [= больные олени; слово напечатано] [12.20об] % TODO: уточнить перевод, нужна транскрипция
		\tabularnewline \midrule
\tenevilglyph[yes][1]{a_jT_B-}
	&
	&	
	&	
	&	
	& 	ŋerquliьn [ИЛИ:2.10об]  % TODO: нужен перевод % ilispread:2.10
		\tabularnewline \midrule
\tenevilglyph[yes][1]{a_jT_T_l}
	&
	&	
	&	
	&	
	& 	ŋьroquliьn [ИЛИ:2.10об]  % TODO: нужен перевод % ilispread:2.10
		\tabularnewline \midrule
\tenevilglyph[yes][1]{a_jT_i_c_T}
	&
	&	
	&	
	&	
	& 	ŋьraquliьn [ИЛИ:2.10об]  % TODO: нужен перевод % ilispread:2.10
		\tabularnewline \midrule
\tenevilglyph[yes][1]{a_2k-z-q}
	&
	&	
	&	
	&	
	& 	irelьt \currentGlyphWithAffixes{}{T} [19.5об] \linebreak % TODO: нужен перевод
		ierek \currentGlyphWithAffixes{}{K} [ИЛИ:1.15]  % TODO: нужен перевод % ilispread:1.15
		\tabularnewline \midrule
\tenevilglyph[yes][5]{s_b}
	&	нымкыӄин
	&	
	&
	&	много \cite[л. 42]{spbfaran79} \linebreak
		много \cite[л. 37]{spbfaran79} \linebreak
		numkәqin [nьmkәqin, нымкыӄин = многочисленный] \cite[л. 54]{spbfaran79} \linebreak % нымкыӄин
		nьemkәqin [nьmkәqin] \cite[л. 54]{spbfaran79} \linebreak
		nьmkәqin \cite[л. 52 об]{spbfaran79} \linebreak
		мноко [много] \cite[л. 66 об, 67]{spbfaran79}
	& 	\cite[360–364]{davydova2015a} \linebreak
		\cite[28]{lavrov1969} \linebreak
		\cite{bogoraz1934} \linebreak
		nьmkәqin [нымкыӄин; слово напечатано] [12.17об] \linebreak
		мноко [много] [34.11] \linebreak
		нымкыген [нымкыӄин] [42.31] \linebreak
		nьmkьqen [нымкыӄин] [ИЛИ:1.4об]  % ilispread:1.5
		\tabularnewline \midrule
\tenevilglyph[yes][4]{s_j_b}
	&	нымкъэв
	&	
	&
	&	
	& 	nьmkeu [нымкъэв = много; слово напечатано] [12.16] \linebreak
		nьmkieu [нымкъэв] [ИЛИ:1.6об,ИЛИ:2.22]   % ilispread:1.7, ilispread:2.21
		\tabularnewline \midrule
\tenevilglyph[yes][4]{s_b_jFY}
	&	мыкычьын
	&	
	&
	&	
	& 	mьkьciьn [мыкычьын = большинство] [ИЛИ:1.10об] \linebreak  % ilispread:1.11
		mьkьciu \currentGlyphWithAffixes{}{W} [ИЛИ:1.12об]  % TODO: нужен перевод % ilispread:1.13
		\tabularnewline \midrule
\tenevilglyph[yes][4]{s_b_jFE}
	&	мыкыԓьын
	&	
	&	мыкычин [мыкычьын = большинство], многочисленный [56]
	&	
	& 	mьkeliьn [мыкыԓьын = многочисленный] [ИЛИ:1.3об]  % ilispread:1.4
		\tabularnewline \midrule
\tenevilglyph[yes][5]{f}
	&	о'равэтԓьан
	& 	человек \cite{bogoraz1934}\linebreak
		человек \cite{lavrov1969}
	&	орявэтлян [о'равэтԓьан = человек], человек [39] % очень сильно стилизовано
	&	человек, народ \cite[л. 42]{spbfaran79} \linebreak
		человек \cite[л. 53]{spbfaran79} \linebreak
		соловк [человек] \cite[л. 68 об]{spbfaran79} 
	& 	\cite[360, 361, 364]{davydova2015a}\linebreak
		\cite{bogoraz1934} \linebreak
		соловек [человек] [37.7об] \linebreak
		iorawelian [orawetlan, о'равэтԓьан] [ИЛИ:1.3] \linebreak  % ilispread:1.3
		orawetlan [слово напечатано] \currentGlyphWithAffixes{}{N} [12.13об] \linebreak
		orawetlen [слово напечатано] \currentGlyphWithAffixes{}{N} [12.13об] \linebreak
		iorawetlweliatь \currentGlyphWithAffixes{}{T} [ИЛИ:1.12] \linebreak  % ilispread:1.12
		orawetletь [слово напечатано] \currentGlyphWithAffixes{}{T} [12.20об] \linebreak
		чукси [чукчи] \currentGlyphWithAffixes{}{L} [32.6об] \linebreak
		luorawetla [ԓыгъоравэтԓьа = чукчи; слово напечатано] \currentGlyphWithAffixes{}{L} [12.15] \linebreak 
		cukьcke \currentGlyphWithAffixes{}{L} [ИЛИ:2.22об] \linebreak  % Эти два с Ԓ надо пологать потому, что ԓыгъоравэтԓьа % ilispread:2.22
		cukьcke \currentGlyphWithAffixes{L}{} [ИЛИ:2.27] \linebreak  % ilispread:2.26
		сокодка [чукотка] \currentGlyphWithAffixes{}{L,A,L} [34.16] % TODO: найти эндоним?
		\tabularnewline \midrule
\tenevilglyph[yes][4]{f_4q}
	&	умэкэтык
	&	
	&	нумэкэтынэт [нумэкэтӄинэт = пусть собираются], пусть соединяются \currentGlyphWithAffixes{}{T} [40]
	&	
	& 	сапране [собрание] [36.1] \linebreak
		numeketьnet [нумэкэтӄинэт] \currentGlyphWithAffixes{}{T} [12.16об] 
		\tabularnewline \midrule
\tenevilglyph[yes][5]{f_c}
	&	варат
	&	
	&	варат [варат = народ], рэмкын, народ [39] 
	&	
	& 	\cite[364]{davydova2015a} \linebreak
		човет [совет] [30.3об] \linebreak
		в «советски» [советский] [34.8об] \linebreak
		варат [41.7] \linebreak
		warat [варат] [ИЛИ:2.18] \linebreak  % ilispread:2.17
		warat [варат] \currentGlyphWithAffixes{}{T} [ИЛИ:1.5об] \linebreak  % ilispread:1.6
		warata [варата = народом] \currentGlyphWithAffixes{}{T,A} [ИЛИ:2.5об]  % ilispread:2.5
		\tabularnewline \midrule
\tenevilglyph[yes][4]{f_jFE}
	&	а'ртеԓь
	&	
	&	
	&	
	& 	artel [а'ртеԓь = артель; слово напечатано] [12.12об] \linebreak
		artelo [слово напечатано] \currentGlyphWithAffixes{}{A} [12.12об] \linebreak % TODO: нужен перевод, нужна транскрипция		
		artela [слово напечатано] \currentGlyphWithAffixes{}{A} [12.17]
		\tabularnewline \midrule
\tenevilglyph[yes][4]{f-b}
	&	пионер
	&	
	&	пионертэ [= пионеры] \currentGlyphWithAffixes{}{T} [40]
	&	
	& 	pioneren [пионерэн = пионерский; слово напечатано] \currentGlyphWithAffixes{}{N} [12.18об] \linebreak
		pionerte [пионертэ; слово напечатано] \currentGlyphWithAffixes{}{T} [12.18об] \linebreak
		pioneretь [слово напечатано] \currentGlyphWithAffixes{}{T} [12.18об] \linebreak % TODO: нужен перевод, нужна транскрипция
		pionero [слово напечатано] \currentGlyphWithAffixes{}{A} [12.18об] \linebreak
		\tabularnewline \midrule
\tenevilglyph[yes][1]{f_lE}
	&
	&	
	&	
	&	
	& 	ierьl [ИЛИ:2.10]  % TODO: нужен перевод % ilispread:2.9
		\tabularnewline \midrule
\tenevilglyph[yes][1]{BD_f}
	&
	&	
	&	
	&	
	& 	iaqorawlian [ИЛИ:1.12об] \linebreak % TODO: нужен перевод а'къоравэтльан? % ilispread:1.13
		iaqiorawelian [ИЛИ:2.27]  % ilispread:2.26
		\tabularnewline \midrule
\tenevilglyph[yes][4]{i_G_f}
	&	таӈъоравэтԓьан
	&	
	&	
	&	
	& 	taŋiorawelian [таӈъоравэтԓьан = хороший человек] [ИЛИ:2.27]  % Weinstein % ilispread:2.26
		\tabularnewline \midrule
\tenevilglyph[yes][5]{i_l}
	&	ӄоԓ
	& 	другой \cite{bogoraz1934}
	&	кол [ӄоԓ = другой, один из], другой [49]
	&	другой \cite[л. 42]{spbfaran79} \linebreak
		другой \cite[л. 53]{spbfaran79} 
	& 	\cite[361–364]{davydova2015a} \linebreak
		\cite{bogoraz1934} \linebreak
		трхой [другой] [32.16] \linebreak
		qol [ӄоԓ] [ИЛИ:1.19об]  % ilispread:1.20
		\tabularnewline \midrule
\tenevilglyph[yes][4]{i_l_jFY}
	&	ӄоԓ нитӄин
	& 	
	&	колниткин [ӄоԓ нитӄин = иногда], другой раз [49]
	&	
	& 	qolnetqen [ӄоԓ нитӄин] [ИЛИ:2.7об]  % ilispread:2.7
		\tabularnewline \midrule
\tenevilglyph[yes][5]{i_jF_q}
	&	чит
	&	
	&	чит, раньше [54]
	&	раньше, ꞓit [cit, чит = раньше, прежде] \cite[л. 42]{spbfaran79} \linebreak % чит
		ꞓit [cit] \cite[л. 52 об, 56]{spbfaran79} 
	& 	\cite[364]{davydova2015a} \linebreak
		ceet [cit, чит] [ИЛИ:1.6об]  % ilispread:1.7
		\tabularnewline \midrule
\tenevilglyph[yes][5]{i_LX}
	&	ӄэйвэ
	& 	действительно \cite{bogoraz1934}
	&	кейве [ӄэйвэ], правда [50]
	&	действительно, qәjve [ӄэйвэ = право, и действительно] \cite[л. 42]{spbfaran79} \linebreak % ӄэйвэ
		qojve [qәjve] \cite[л. 56]{spbfaran79} \linebreak
		qoive [qәjve] \cite[л. 54, 52 об]{spbfaran79}
	& 	\cite[360–362, 364]{davydova2015a} \linebreak
		qejwe [qәjve, ӄэйвэ] [ИЛИ:1.10об]  % ilispread:1.11
		\tabularnewline \midrule
\tenevilglyph[yes][5]{b_2j}
	&	еп
	& 	еще не \cite{bogoraz1934}
	&	jep, еще, пока еще, еще в то время как, еще не [11]
	&	еще не, jep [еп = еще; пока еще; еще в то время, как] \cite[л. 42]{spbfaran79} \linebreak % еп
		jep \cite[л. 52, 52 об, 56]{spbfaran79}
	& 	\cite[360]{davydova2015a} \linebreak
		jep [слово напечатано] [12.20об] \linebreak
		ееб [еп] [45.2] \linebreak
		eep [jep, еп] [ИЛИ:1.9об]  % ilispread:1.10
		\tabularnewline \midrule
\tenevilglyph[yes][5]{b_2jF} 
	&	ваԓы
	&	
	&	валы [ваԓы = нож], рэскын [ръысӄын = охотничий нож], нож [114] 
	&	носик [ножик] \cite[л. 68об]{spbfaran79}
	&	\cite[361]{davydova2015a} \linebreak
		valь [ваԓы] [12.15] 
		\tabularnewline \midrule
\tenevilglyph[yes][4]{b_2jF_2q} 
	&	торвыԓы
	&	
	&	торвалы [торвыԓы = новый нож], новый нож [114] % TODO: уточнить транскрипцию
	&	
	&	tor-valь [торвыԓы; слово напечатано] [12.15] % TODO: уточнить транскрипцию
		\tabularnewline \midrule 
\tenevilglyph[yes][5]{2c}
	&	мэткиит
	&	somehow \cite{mindalevich1934}
	&
	&	кое как, насилу \cite[л. 42]{spbfaran79} \linebreak
		metkiit [mætkiit, мэткиит = насилу, еле-еле] \cite[л. 39, 52]{spbfaran79} % мэткиит
	&	\cite{bogoraz1934} \linebreak
		metkeet [mætkiit, мэткиит] [ИЛИ:1.4] \linebreak  % ilispread:1.4
		metket [мэткиит] [ИЛИ:2.23об]  % ilispread:2.23
		\tabularnewline \midrule
\tenevilglyph[yes][5]{I-2l}
	&	ӄонпы
	& 	совсем \cite{bogoraz1934}
	&	конпын' [ӄонпыӈ], совсем, вовсе
	&	совсем, qonpь [ӄонпы = совсем, совершенно] \cite[л. 42]{spbfaran79} \linebreak % ӄонпыӈ
		qonpь* \cite[л. 39]{spbfaran79} \linebreak
		совсем \cite[л. 67]{spbfaran79}
	& 	\cite[360, 361, 364]{davydova2015a} \linebreak
		\cite[28]{lavrov1969} \linebreak
		qonpь [ӄонпы] [ИЛИ:1.17об]  % ilispread:1.18
		\tabularnewline \midrule
\tenevilglyph[yes][5]{wD}
	&	ратанӈавӈэн
	&	
	&	раттанн'авн'эн [rattanŋawŋьn, ратанӈавӈэн = довольно, хватит, достаточно], довольно, хватит, худо, перестаньте [26]
	&	довлно [довольно] \cite[л. 68 об]{spbfaran79} 		
	& 	[33.14] \linebreak
		rataŋauŋen [rattanŋawŋьn, ратанӈавӈэн] [ИЛИ:1.13]  % ilispread:1.13
		\tabularnewline \midrule
\tenevilglyph[yes][5]{wD2E}
	&	ӄынвэр
	& 	притом \cite{bogoraz1934}
	&	кынвэр [ӄынвэр], поэтому [26]
	&	притом, qәnver [qanver, ӄынвэр = также] \cite[л. 42]{spbfaran79} \linebreak 
		qanver \cite[л. 39, 56]{spbfaran79} \linebreak
		qәnver \cite[л. 52, 56]{spbfaran79} 		
	& 	\cite[360, 361]{davydova2015a} \linebreak
		qьnwer [qanver, ӄынвэр] [ИЛИ:1.3об] \linebreak  % ilispread:1.4
		qьnwrur [ӄынвэрэвыр = в конце концов, мало помалу] \currentGlyphWithAffixes{}{ewyr} [ИЛИ:1.7]  % ilispread:1.7
		\tabularnewline \midrule
\tenevilglyph[yes][3]{wD_4q}
	&
	& 	
	&	
	&		
	& 	veçьrourgьn [vecьrourgьn, вэчыръогыргын = затруднение; слово напечатано] [12.20об] 
		\tabularnewline \midrule
\tenevilglyph[yes][5]{2o_2iY}
	&	тэԓенъеп
	&	
	&	айгоон [= давно], давно [6]
	&	давно \cite[л. 42]{spbfaran79} \linebreak
		telenjep [tælænjæp, тэԓенъеп = давно] \cite[л. 39 об, 52, 56]{spbfaran79} \linebreak % тэԓенъеп
		давно \cite[л. 66 об]{spbfaran79}
	& 	\cite[360]{davydova2015a} \linebreak
		telen-jep [тэԓенъеп; слово напечатано] [12.20о] \linebreak
		тавыно [давно] [30.7об] \linebreak
		teleeip [тэԓенъеп] [ИЛИ:2.2об]  % ilispread:2.2
		\tabularnewline \midrule
\tenevilglyph[yes][4]{2o_2iY_j}
	&	тэԓеӈкин
	&	
	&	
	&	
	& 	teleŋken [тэԓеӈкин = старинный, древний] [ИЛИ:1.15об] \linebreak  % ilispread:1.16
		telŋken [тэԓеӈкин] \currentGlyphWithAffixes{}{K,N} [ИЛИ:2.7об]  % ilispread:2.7
		\tabularnewline \midrule
\tenevilglyph[yes][5]{b_q}
	&	чама
	& 	так же, тоже \cite{bogoraz1934}
	&	чама, тоже [113]
	&	также, ꞓama [cama, чама = тоже] \cite[л. 42]{spbfaran79} \linebreak % чама
		тоже \cite[л. 37]{spbfaran79} \linebreak
		ꞓama [cama] \cite[л. 39 об, 54]{spbfaran79}
	& 	\cite[360, 361, 362, 364]{davydova2015a} \linebreak
		\cite[28]{lavrov1969} \linebreak 
		\cite{bogoraz1934} \linebreak
		çama [чама; слово напечатано] [12.25] \linebreak
		cama [чама] [ИЛИ:1.18об]  % ilispread:1.19
		\tabularnewline \midrule
\tenevilglyph[yes][5]{2i_2cD_2l}
	&	ымыԓьо
	& 	все \cite{bogoraz1934}\linebreak
		all \cite{mindalevich1934}
	&
	&	все \cite[л. 42]{spbfaran79} \linebreak	
		ьmьlo [ымыԓьо = весь] \cite[л. 52 об]{spbfaran79} % ымыԓьо
	& 	\cite[360, 361, 364]{davydova2015a} \linebreak
		сех [всех] [29.11об] \linebreak
		чех [всех] [29.11об] \linebreak
		учехы [всех] [30.3об] \linebreak
		ьmьlo [ымыԓьо; слово напечатано] [12.24] \linebreak
		ьmьlio [ымыԓьо] [ИЛИ:1.2об]  % ilispread:1.3
		\tabularnewline \midrule
\tenevilglyph[yes][4]{2i_cD_j_2l}
	&	ромакавкы
	& 	
	&	ромакавкы [= все вместе, в сумме, всего], всего [127]
	&	
	& 	умесе [вместе] [34.16об] \linebreak
		ромакавыкы [ромакавкы] [42.2] % TODO: убедиться, что относится к этом знаку
		\tabularnewline \midrule
\tenevilglyph[yes][5]{U_q}
	&	наӄам
	&	but \cite{mindalevich1934}
	&	накам [наӄам] (союз), но, к [36]
	&	но \cite[л. 42]{spbfaran79} \linebreak	
		naqam [наӄам = но, однако] \cite[л. 39, 52 об, 54, 56]{spbfaran79} % наӄам
	& 	\cite[360, 361, 364]{davydova2015a} \linebreak
		нахам [наӄам] [34.8] \linebreak
		naqam [наӄам; слово напечатано] [12.20об] \linebreak
		naqm [наӄам] [68.34об] \linebreak
		naqam [наӄам] [ИЛИ:1.4об]  % ilispread:1.5
		\tabularnewline \midrule
\tenevilglyph[yes][5]{o_2JY}
	&	мэчынкы
	&	well \cite{mindalevich1934}
	&	мечынкы [мэчынкы], достаточно [7]
	&	хорошо \cite[л. 43]{spbfaran79} \linebreak	
		meꞓәnkь [mæсәnkь, мэчынкы = довольно, достаточно, хорошо, в состоянии] \cite[л. 39, 52]{spbfaran79} \linebreak % мэчынкы
		латно [ладно] \cite[л. 67]{spbfaran79} \linebreak
		хвачит [хватит] \cite[л. 68 об]{spbfaran79}
	& 	\cite[360, 361, 364]{davydova2015a} \linebreak
		мэсынкы [мэчынкы] [45.1об] \linebreak
		mecьnk [мэчынкы] [ИЛИ:2.9об]  % ilispread:2.9
		\tabularnewline \midrule
\tenevilglyph[yes][3]{o_JY_JE}
	&	этъопэԓ
	&	
	&	
	&	
	& 	әtopel [этъопэԓ = лучше, лучше сделать так; слово напечатано] [12.25]
		\tabularnewline \midrule
\tenevilglyph[yes][5][milger]{o_2JE}
	&	миԓгэр
	&	
	&	милгэр [миԓгэр], ружье, милгэрыткук [миԓгэрыткук = стрелять], стрелять [133]
	&	milgьrә [milgьr, миԓгэр = ружье]* \cite[л. 54]{spbfaran79} \linebreak % миԓгэр
		русёь [ружье] \cite[л. 68 об]{spbfaran79}
	& 	\cite[360, 364]{davydova2015a} \linebreak
		\cite[28]{lavrov1969} \linebreak
		milgьrә [миԓгэр; слово напечатано] [12.25]
		\tabularnewline \midrule
\tenevilglyph[yes][3]{o_2J_2o}
	&
	&	
	&	увик [= варить мясо], варить [133]
	&	
	& 	uwii [= варит; слово напечатано] [12.13об] % TODO: уточнить перевод, нужна транскрипция
		\tabularnewline \midrule
\tenevilglyph[yes][3]{S_iX}
	&
	&	
	&
	&	не могли* \cite[л. 43]{spbfaran79} \linebreak % вверх ногами опять
		nanilrawcьm* (?) \cite[л. 39]{spbfaran79} % TODO: нужен перевод
	& 	[33.7]
		\tabularnewline \midrule
\tenevilglyph[yes][4]{SMY_iX}
	&	ныруԓӄин
	&	
	&
	&	
	& 	\cite[360]{davydova2015a} \linebreak
		nьrulqen [ныруԓӄин] [ИЛИ:1.3] \linebreak  % ilispread:1.3
		\tabularnewline \midrule
\tenevilglyph[yes][3]{SMYX_iX}
	&	ныруԓӄин
	&	
	&	юргымтэк [юргымтэӄ = безумный, слабоумный], безумный, глупый [35]
	&	слабый \cite[л. 43]{spbfaran79} \linebreak
		nьrulqin [ныруԓӄин = слабый] \cite[л. 52, 52 об]{spbfaran79} \linebreak % ныруԓӄин
		nьrulium \cite[л. 52 об, 56]{spbfaran79} \linebreak
		nьrulqinet \cite[л. 39 об]{spbfaran79}
	& 	[11.5] \linebreak
		jurgьmetq [юргымтэӄ; слово напечатано] \currentGlyphWithAffixes{}{T} [12.20об] 
		nьrulqen [ныруԓӄин] \currentGlyphWithAffixes{}{N,R} [ИЛИ:2.26]  % ilispread:2.25
		\tabularnewline \midrule
\tenevilglyph[yes][5]{i_4l}
	&	чимгъун
	&	
	&	чимгун [чимгъун = ум, разум, дума], гемгогыргын, разум [125] % TODO: нужна транскрипция
	&	
	& 	ом [ум] [37.2] \linebreak
		cemion [cimŋun, чимгъун] [ИЛИ:1.8об, ИЛИ:2.8об] \linebreak   % ilispread:1.9, ilispread:2.8
		cemiot [чимгъут = умы, думы] \currentGlyphWithAffixes{}{b,A,T} [ИЛИ:1.12об] \linebreak  % ilispread:1.13
		cemiot [чимгъут = умы, думы] \currentGlyphWithAffixes{}{T} [ИЛИ:2.5] \linebreak  % ilispread:2.4
		gecemiulen [гэчимгъуԓин = подумал] \currentGlyphWithAffixes{}{b,L} [ИЛИ:1.7]  % ilispread:1.7
		\tabularnewline \midrule
\tenevilglyph[yes][3]{i_4l_2l}
	&
	&	
	&
	&	тосковать* \cite[л. 43]{spbfaran79} 
	& 	[9.3] 
		\tabularnewline \midrule % тут вверх ногами
\tenevilglyph[yes][1]{i_4l_2zRX}
	&
	&	
	&	
	&	
	& 	gemlьcemiolen [ИЛИ:2.19]  % TODO: нужен перевод, нужна транскрипция % ilispread:2.18
		\tabularnewline \midrule
\tenevilglyph[yes][3]{i_4l_b}
	&
	&	
	&	
	&	
	& 	nьtemqen \currentGlyphWithAffixes{}{N} [ИЛИ:1.17] \linebreak  % TODO: нужен перевод, нужна транскрипция % ilispread:1.17
		gьtepьciьn [гыттапъёчгын = мудрость] \currentGlyphWithAffixes{}{b,N} [ИЛИ:2.10]  % ilispread:2.9
		\tabularnewline \midrule
\tenevilglyph[yes][4]{U2EN_JX}
	&
	&	
	&	элен'ит [эԓеӈит = лето], элеэл [эԓеэԓ = лето], лето [60]
	&	лето наступило \cite[л. 43]{spbfaran79} \linebreak	
		elerurkьn = лето начинается [ælærurkьn = лето начинается] \cite[л. 52 об]{spbfaran79} \linebreak % TODO: нужна транскрипция 
		лет [лето] \cite[л. 66]{spbfaran79}
	& 	\cite[362]{davydova2015a} \linebreak
		\cite[28]{lavrov1969} \linebreak
		летым [летом] [30.6] \linebreak
		elek [эԓек = летом] \currentGlyphWithAffixes{}{K} [ИЛИ:2.25об]  % ilispread:2.25
		\tabularnewline \midrule
\tenevilglyph[yes][5]{U_JX_3'}
	&	ԓьэԓеӈ, ԓьэԓеӈкы
	&	
	&	леэлен' [ԓьэԓеӈ = зима], зима
	&	симои [зимой] \cite[л. 66]{spbfaran79}
	& 	семои [зимой] [34.11] \linebreak
		lieleŋkь [lәlæŋqь, ԓьэԓеӈкы = зимой] [ИЛИ:1.6] \linebreak  % ilispread:1.6
		lileŋkь [ԓьэԓеӈкы = зимой] \currentGlyphWithAffixes{}{K} [ИЛИ:2.3]  % ilispread:2.2
		\tabularnewline \midrule
\tenevilglyph[yes][4]{U_JX_j}
	&	кыткыт, кыткытык
	&	
	&	кыткыт [= поздняя весна], анон [= раннее лето], весна [60] 
	&	
	& 	kьtkьtьk [кыткытык = весной; слово напечатано] [12.11, 12.20об] 
		\tabularnewline \midrule
\tenevilglyph[yes][5]{2O}
	&	опопы
	&	
	&	опопы, пусть [123]
	&	пусть, чтобы, opopo [опопы = пусть] \cite[л. 43]{spbfaran79} \linebreak % опопы
		пусть \cite[л. 53]{spbfaran79} \linebreak
		opopo \cite[л. 52 об]{spbfaran79} 
	& 	\cite[364]{davydova2015a} \linebreak
		опопы [34.8] \linebreak
		opopь [опопы] [ИЛИ:2.4]  % ilispread:2.3
		\tabularnewline \midrule
\tenevilglyph[yes][5]{o_3iS}
	&	мачынан
	&	
	&	мачынан, пускай, пусть [46]
	&	пускай, maꞓьnan [macьnan, мачынан = пусть] \cite[л. 43]{spbfaran79} \linebreak % мачынан
		maꞓьnan [мачынан] \cite[л. 52 об, 56]{spbfaran79} \linebreak
		мачнан [мачынан] \cite[л. 68]{spbfaran79} 
	& 	\cite[364]{davydova2015a} \linebreak
		\cite{bogoraz1934} \linebreak
		почкаи [пускай] [30.7] \linebreak
		macьnan [мачынан] [ИЛИ:1.4]  % ilispread:1.4
		\tabularnewline \midrule
\tenevilglyph[yes][5]{u-o_b}
	&	миӈкыри, миӈкыриԓы
	&	
	&	мин'кри [миӈкыри], куда [114]
	&	как же, miŋkri [миӈкыри = куда, как] \cite[л. 43]{spbfaran79} \linebreak 
		miŋkri-lu [миӈкыриԓы = куда]\cite[л. 56]{spbfaran79} \linebreak 
		кута [куда] \cite[л. 66]{spbfaran79} \linebreak
		как \cite[л. 66 об]{spbfaran79} \linebreak
		в «какои» \cite[л. 66]{spbfaran79} 
	& 	\cite[364]{davydova2015a} \linebreak
		miŋkri [миӈкыри; слово напечатано] [12.25] \linebreak
		mekьrelь [миӈкыриԓы] [ИЛИ:2.26]  % ilispread:2.25
		\tabularnewline \midrule
\tenevilglyph[yes][4]{u-o}
	&	миӈкы
	&	
	&
	&	аке \cite[л. 68]{spbfaran79} % TODO: нужна интерпретация
	& 	[25.9] \linebreak
		mekь [miŋkь, миӈкы = где] [ИЛИ:2.2об] \linebreak  % ilispread:2.2
		mekeken [миӈкэкин = откуда?] \currentGlyphWithAffixes{}{K,K,N} [ИЛИ:2.28об]  % ilispread:2.28
		\tabularnewline \midrule
\tenevilglyph[yes][1]{u-o_'}
	&	
	&	
	&
	&	
	& 	meketьnk \currentGlyphWithAffixes{}{T,K} [ИЛИ:1.7]  % TODO: нужен перевод; то же -тэгнык % ilispread:1.7
		\tabularnewline \midrule
\tenevilglyph[yes][1]{u-o_b_jX}
	&
	&	
	&
	&
	& 	riemekrelь [ИЛИ:1.7]  % TODO: нужен перевод % ilispread:1.7
		\tabularnewline \midrule
\tenevilglyph[yes][2]{U_iX_b}
	&
	&	divide the reindeer herd \cite{mindalevich1934}
	&
	&	завидовать \cite[л. 43]{spbfaran79}
	& 	[12.1об] \linebreak
		nьtemauqen [нытэнмавӄэн = готвится] \currentGlyphWithAffixes{N,T,M}{} [ИЛИ:1.3]  % TODO: уточнить перевод % ilispread:1.3
		\tabularnewline \midrule
\tenevilglyph[yes][1]{U_iX_2b}
	&
	&	
	&
	&	
	& 	meŋьqen [ИЛИ:1.4] \linebreak  % TODO: нужен перевод % ilispread:1.4
		teьmeŋьk \currentGlyphWithAffixes{}{K} [ИЛИ:1.4об] \linebreak  % ilispread:1.5
		tegemeŋьk \currentGlyphWithAffixes{}{K} [ИЛИ:1.13] \linebreak  % ilispread:1.13
		reteьmeŋьrkьn \currentGlyphWithAffixes{}{R} [ИЛИ:1.4об] \linebreak  % ilispread:1.5
		эlьэtgemeŋьke \currentGlyphWithAffixes{etly}{} [ИЛИ:1.6об]  % ilispread:1.7
		\tabularnewline \midrule
\tenevilglyph[yes][4]{i_2kU_2kD}
	&	выԓгынаԓгын
	&	
	&
	&	шкура \cite[л. 44]{spbfaran79} \linebreak
		nelgьn [nælgьn, нэԓгын = шкура]* \cite[л. 39 об]{spbfaran79} \linebreak % нэԓгын
		писик [пыжик] \cite[л. 68]{spbfaran79}
	& 	\cite[364]{davydova2015a} \linebreak
		wlgьnalgьn [выԓгынаԓгын = тонкошерстная летняя шкура оленя, неблюй] [ИЛИ:1.18об]  % ilispread:1.19
		\tabularnewline \midrule
\tenevilglyph[yes][4]{i_2kU_kD_2Q}
	&	нэԓгын
	&	
	&
	&	неторос [недоросль] \cite[л. 68]{spbfaran79} 
	& 	nlgьn [nælgьn, нэԓгын = шкура] [ИЛИ:1.6,ИЛИ:2.17]   % ilispread:1.6, ilispread:2.16
		\cite[364]{davydova2015a} 
		\tabularnewline \midrule
\tenevilglyph[yes][3]{i_2kU_kD_2Q_iX}
	&
	&	
	&
	&	посела [постель (шкура взрослого оленя)] \cite[л. 68]{spbfaran79} 
	& 	[32.17об]
		\tabularnewline \midrule
\tenevilglyph[yes][4]{i_kU_2kD_2CY}
	&	риӄугнэԓгын
	&	
	&
	&	
	& 	requnelgьn [риӄугнэԓгын = лисья шкура] [ИЛИ:1.18об]  % ilispread:1.19
		\tabularnewline \midrule
\tenevilglyph[yes][3]{i_kU_b_3Q_c}
	&
	&	
	&
	&	саски [чашки] \cite[л. 68]{spbfaran79} 
	& 	\cite[364]{davydova2015a} \linebreak
		чашка [не рукой Т.] [57.23]
		\tabularnewline \midrule
\tenevilglyph[yes][3]{k_o_oN}
	&
	&	
	&
	&	випороток [выпороток] \cite[л. 68]{spbfaran79} 
	& 	[1.1] \tabularnewline \midrule
\tenevilglyph[yes][4]{2k}
	&
	& 	мука \cite{bogoraz1934}
	&	пин'вытрын [пиӈвытрын = мука], пинмэкычгын, мука [14] % TODO: нужна транскрипция 
	&	мука \cite[л. 44]{spbfaran79} \linebreak
		мука \cite[л. 66 об]{spbfaran79}
	& 	[25.6]
		\tabularnewline \midrule
\tenevilglyph[yes][5][melgytanvyn]{vY_z}
	&	мэԓгытанӈын
	&	Russian holding fire-arms \cite{mindalevich1934} \linebreak 
		русский* \cite{lavrov1969}
	&	тангыт [танӈыт = коряки], руссилын [русиԓьын = русский (народ)], русский [119] 
	&	русский \cite[л. 44]{spbfaran79} 
	& 	\cite[364]{davydova2015a} \linebreak
		melgьtaŋь [melgьt-tanŋьtan, мэԓгытанӈын = русский] [ИЛИ:1.2] \linebreak  % ilispread:1.2
		melgьtaŋьt [мэԓгытанӈыт = русские] \currentGlyphWithAffixes{}{T} [ИЛИ:1.3об] \linebreak  % ilispread:1.4
		ьrucket \currentGlyphWithAffixes{R,K,T}{} [ИЛИ:2.20об] \linebreak  % ilispread:2.20
		ьruceliьn [русиԓьын = русский] \currentGlyphWithAffixes{R,N}{} [ИЛИ:2.27]  % ilispread:2.26
		\tabularnewline \midrule
\tenevilglyph[yes][4]{a_vY_z}
	&	совхоз
	&	
	&	каатангыт, совхоз (русские олени) [83] \linebreak % TODO: нужна транскрипция 
		совхоз (русский олень) [119]
	&	
	& 	совхос [совхоз] [32.13] \linebreak % \cite[148]{sergeev1956} «мельгитаньги чаучуа» — русские оленные люди
		оленсохоси [оленсовхоз] [34.11об] \linebreak
		coqocьk \currentGlyphWithAffixes{}{K} [ИЛИ:2.5]  % ilispread:2.4
		\tabularnewline \midrule
\tenevilglyph[yes][5]{bD_b_vY_z}
	&	Ленин
	&	Ленин \cite{lavrov1969}
	&	Ленин [118]
	&	
	& 	ленн [Ленин] [34.9] \linebreak % NB: часть «ԓыгэн» тут, видимо, в качестве фонетической компоненты 
		lenen [Ленин; рядом с портретом Ленина] [ИЛИ:1.1] \linebreak  % ilispread:1.1
		Ленин \currentGlyphWithAffixes{}{L,N,N} [43.31об]
		\tabularnewline \midrule
\tenevilglyph[yes][3]{vY_j}
	&	мэԓмэԓ
	&	
	&	
	&	
	& 	melmel [мэԓмэԓ = хорошая погода; слово напечатано] [14.10] \linebreak % TODO: интерпретация сомнительная
		мэлете \currentGlyphWithAffixes{}{T} [45.1об] % TODO: нужна интерпретация
		\tabularnewline \midrule
\tenevilglyph[yes][3]{i-z-i_c}
	&
	& 	следы \cite{bogoraz1934}
	&	вины [= след], след [117]
	&	следы \cite[л. 45]{spbfaran79} 
	& 	~[рядом с изображением следов] [19.9]
		\tabularnewline \midrule
\tenevilglyph[yes][3]{i-z-i_2q}
	&
	& 	
	&	
	&	
	& 	tur-reta [= новому пути; слово напечатано] [7.12об, 12.12об] % TODO: уточнить перевод
		\tabularnewline \midrule
\tenevilglyph[yes][4]{c_2cD_q}
	&
	& 	волк \cite{bogoraz1934}\linebreak
		волк \cite{lavrov1969}
	&	пинны, эгичгын [э'гычгын = волк (волчище)], волк [50] % TODO: нужна транскрипция
	&	волк \cite[л. 45, 53]{spbfaran79} \linebreak
		волк \cite[л. 68 об]{spbfaran79} \linebreak
		в «убил \textit{волка»} \cite[л. 68 об]{spbfaran79}
	& 	\cite[360]{davydova2015a} \linebreak
		воки [волки] [34.12] \linebreak
		ieьe [э'гъё = стая волков] [ИЛИ:2.7об]  % ilispread:2.7
		\tabularnewline \midrule
\tenevilglyph[yes][4]{J_b_i}
	&
	&	
	&	кейнын [кэйӈын = медведь (бурый)], медведь бурый [86]
	&	мтвет [медведь] \cite[л. 68 об]{spbfaran79}
	& 	мысвеси [медведи] [34.12]
		\tabularnewline \midrule
\tenevilglyph[yes][4]{J_b_2b_c}
	&	умӄы
	&	
	&	
	&	
	& 	umqь [umqæ, умӄы = белый медведь; слово напечатано] [12.16]
		\tabularnewline \midrule
\tenevilglyph[yes][4]{I-IE_'} 
	&
	&	
	&
	&	
	& 	рысомака [росомаха] [34.12]
		\tabularnewline \midrule
\tenevilglyph[yes][4]{2CY} % С лисами сплошная путаница
	&
	&	песец \cite{lavrov1969}
	&	рекокалгын [рэӄокаԓгын], элгар [эԓгар = песец], песец [97]
	&	reqokalgьn [рэӄокаԓгын = лисица, песец] \cite[л. 54]{spbfaran79} % рэӄокаԓгын
	& 	[11.3] \linebreak
		лесесея [лисица] [34.12]
		\tabularnewline \midrule
\tenevilglyph[yes][3]{2CY_c} 
	&
	& 	голубой песец \cite{bogoraz1934}
	&	кытгым [= соболь], голубой песец [97]
	&	голубой песец \cite[л. 46]{spbfaran79} 
	& 	[1.14об] %\linebreak
		%riqukete [?; слово напечатано] [12.20] % TODO: нужен перевод, уточнить транскрипцию; вероятно, форма рэӄокаԓгын
		\tabularnewline \midrule
\tenevilglyph[no][2]{2CY_2c} 
	&
	&	
	&
	&	песец \cite[л. 45]{spbfaran79} \linebreak
		сорнпур [чернобурая] \cite[л. 69 об]{spbfaran79} 
	& 	\tabularnewline \midrule
\tenevilglyph[yes][3]{2CY_cFD} 
	&
	&	
	&
	&	красная лисица* \cite[л. 45]{spbfaran79} \linebreak
		лисита [лисица] \cite[л. 69 об]{spbfaran79}
	& 	[11.3] 
		\tabularnewline \midrule
\tenevilglyph[yes][2]{2CY_o_I_3q} 
	&
	&	
	&
	&	огневка \cite[л. 45]{spbfaran79} \linebreak
		песеч [песец] \cite[л. 69 об]{spbfaran79}
	& 	[1.14об]
		\tabularnewline \midrule
\tenevilglyph[no][2]{2CY_o_I_3q_c} 
	&
	&	
	&
	&	чернобурая \cite[л. 45]{spbfaran79} \linebreak
		кулубои [голубой] \cite[л. 69 об]{spbfaran79}
	& 	\tabularnewline \midrule
\tenevilglyph[no][3]{2CY_o_I_3q_2jF} 
	&
	&	
	&
	&	сиводушка \cite[л. 45]{spbfaran79}
	& 	\tabularnewline \midrule
\tenevilglyph[yes][5]{2cF_k_2qY} 
	&	мэԓётаԓгын, миԓютэт
	& 	заяц \cite{bogoraz1934}\linebreak
		a hare \cite{mindalevich1934}
	&	мелёталгын [мэԓётаԓгын], заяц [23]
	&	заяц \cite[л. 46]{spbfaran79} \linebreak
		melitolgьn [melotalgьn, мэԓётаԓгын = заяц] \cite[л. 54]{spbfaran79} % мэԓётаԓгын
	& 	milutet [миԓютэт = зайцы; слово напечатано] [7.3] \linebreak
		~[рядом с изображением кроликов] [11.1] \linebreak
		milutet [миԓютэт = зайцы; слово напечатано] \currentGlyphWithAffixes{}{T} [12.13]
		\tabularnewline \midrule
\tenevilglyph[yes][4]{v-_jF}
	&
	&	
	&	кукэн'ы [кукэӈы = котел, кастрюля, ведро], котел [52]
	&	кострула [кастрюля] \cite[л. 68]{spbfaran79}
	& 	\cite[364]{davydova2015a} \linebreak
		в «7 ветра» [7 ведер] [34.19об]
		\tabularnewline \midrule
\tenevilglyph[yes][3]{O_v}
	&
	&	
	&
	&	тас [таз] \cite[л. 66]{spbfaran79}
	& 	~[12.21]
		\tabularnewline \midrule
\tenevilglyph[yes][3]{O_v_vD}
	&
	& 	кастрюля \cite{bogoraz1934}
	&	в «кастрюля» [133]
	&	в «кастрюлька» \cite[л. 46]{spbfaran79}
	& 	~[12.23]
		\tabularnewline \midrule
\tenevilglyph[no][3]{O_v_2jF}
	&
	& 	тарелка \cite{bogoraz1934}
	&	вычгоккам [вычгоӄӄам = тарелка], илгуккэм, тарелка [133] % TODO: нужна транскрипция 
	&	тарелка \cite[л. 46]{spbfaran79}
	& 	\tabularnewline \midrule
\tenevilglyph[yes][3]{i_c_c_2j}
	&
	& 	свечка \cite{bogoraz1934}
	&	пин'инэн [пиӈинэӈ = лучина, свеча], свеча [97]
	&	
	& 	\cite[364]{davydova2015a}
		\tabularnewline \midrule
\tenevilglyph[yes][3]{R_o-o}
	&
	& 	молоко \cite{bogoraz1934}
	&	молоко (Б. Т. 69, 70) [109]
	&	молоко \cite[л. 49]{spbfaran79} 
	& 	[4.4об]
		\tabularnewline \midrule
\tenevilglyph[yes][3]{R_o-o_2j}
	&
	& 	молоко \cite{bogoraz1934}
	&	молоко (Б. Т. 69, 70) [109]
	&	молоко \cite[л. 49]{spbfaran79} 
	& 	[2.101]
		\tabularnewline \midrule
\tenevilglyph[no][3]{R_o-o_2b}
	&
	& 	банка с керосином \cite{bogoraz1934}
	&	банка с керосином (Б. Т. 68) [109]
	&	банка с керосином \cite[л. 46]{spbfaran79} 
	& 	\tabularnewline \midrule
\tenevilglyph[no][3]{R-o-o_3iS_'}
	&
	& 	банка с жиром \cite{bogoraz1934}
	&	банка с жиром (Б. Т. 66) [109]
	&	банка с жиром \cite[л. 46]{spbfaran79} 
	& 	\tabularnewline \midrule
\tenevilglyph[yes][3]{R_o-o_c_zR}
	&
	& 	банка с салом \cite{bogoraz1934}
	&	банка с салом (Б. Т. 68) [109]
	&	банка с салом (с маслом) \cite[л. 46]{spbfaran79} 
	& 	[4.1]
		\tabularnewline \midrule
\tenevilglyph[yes][3]{R_o-o_2CE}
	&
	&	
	&	кэнтикэй [= конфета], конфеты [98]
	&	банка с сахаром \cite[л. 49]{spbfaran79} 
	& 	[4.7]
		\tabularnewline \midrule
\tenevilglyph[yes][4]{C_c_zR} 
	&	ыпаԓгын
	&	
	&
	&	в «олений \textit{жир} это» \cite[л. 46]{spbfaran79}
	& 	[4.5об] \linebreak
		ьpalgьn [ыпаԓгын = топленый жир] [ИЛИ:1.15]  % ilispread:1.15
		\tabularnewline \midrule
\tenevilglyph[yes][4][nilgyqin]{c_2b}
	&
	& 	белый \cite{bogoraz1934}
	&	нильгыкин [ниԓгыӄин = белый], белый [113]
	&	белый \cite[л. 46]{spbfaran79} \linebreak
		пелои [белый] \cite[л. 68]{spbfaran79}
	& 	\cite[360, 364]{davydova2015a} \linebreak
		\cite[28]{lavrov1969}
		\tabularnewline \midrule
\tenevilglyph[yes][5][nyppyluqin]{o-o_J}
	&	ныппыԓюӄин
	& 	маленький \cite{bogoraz1934}
	&	ныппылюкин [ныппыԓюӄин], маленький-ая-ое [34]
	&	маленький \cite[л. 46]{spbfaran79} \linebreak
		nьpluqin [ныппыԓюӄин = маленький] \cite[л. 46]{spbfaran79} % ныппыԓюӄин
	& 	\cite[360]{davydova2015a} \linebreak
		малйнкй [маленький] [37.7об] \linebreak
		nьpьluqen [ныппыԓюӄин] [ИЛИ:1.2об]  % ilispread:1.3
		\tabularnewline \midrule
\tenevilglyph[no][3]{o-o_J_2q}
	&
	& 	
	&	
	&	ысовая [дешевая] \cite[л. 69 об]{spbfaran79} \linebreak
	& 	
		\tabularnewline \midrule
\tenevilglyph[yes][3]{O_bN}
	&
	& 	крадет \cite{bogoraz1934}
	&
	&	
	&	\cite{bogoraz1934}
		\tabularnewline \midrule
\tenevilglyph[yes][4]{U_bN}
	&
	& 	вороватый \cite{bogoraz1934}
	&	нытулыкин, вороватый [10] % TODO: нужен перевод
	&	вороватый \cite[л. 47]{spbfaran79} 
	&	\cite{bogoraz1934} \linebreak
		tulerkьnin [= украл; слово напечатано] [12.13об] % TODO: нужна транскрипция
		\tabularnewline \midrule
\tenevilglyph[yes][5]{i_G}
	&	нытэӈӄин
	& 	хороший \cite{bogoraz1934}
	&	нытэн'кин [нытэӈӄин = славный, добрый, хороший], хороший [64]
	&	хороший \cite[л. 47]{spbfaran79} \linebreak
		хоросой [хороший]* \cite[л. 66, 68 об]{spbfaran79} 
	& 	\cite[360, 364]{davydova2015a} \linebreak
		\cite{bogoraz1934} \linebreak
		хоросо [хорошо] [33.4] \linebreak
		nьteqen [нытэӈӄин] [ИЛИ:2.24]  % ilispread:2.23
		\tabularnewline \midrule
\tenevilglyph[yes][5]{i_J}
	&	нымэԓьэв
	&	
	&	
	&	лучей [лучший] \cite[л. 66 об]{spbfaran79} \linebreak
		со споконо \cite[л. 67 об]{spbfaran79} \linebreak % TODO: нужна интерпретация
	& 	nьmeleu [нымэԓьэв = хорошо, разумно] [12.17] \linebreak 
		nьmelieu [нымэԓьэв] [ИЛИ:2.1об] \linebreak  % ilispread:2.1
		nьmliu [нымэԓьэв] [ИЛИ:2.12об] \linebreak  % ilispread:2.12
		nьmelqen [нымэԓӄин = хороший, доброкачественный] [ИЛИ:1.6]  % ilispread:1.6
		\tabularnewline \midrule
\tenevilglyph[yes][3]{i_o_G}
	&
	& 	мастероватый \cite{bogoraz1934}
	&	нытэминн'ыкин [нытэминӈыӄин = умелый, умелец, мастер], мастер [64]
	&	
	&	\cite{bogoraz1934} \linebreak
		[25.13об]
		\tabularnewline \midrule
\tenevilglyph[yes][3]{i_G_b}
	&
	&	
	&
	&	поправилас [поправилась] \cite[л. 66 об]{spbfaran79}
	& 	[25.13] \linebreak
		melwrmen \currentGlyphWithAffixes{erym}{} [ИЛИ:1.8]  % TODO: нужен перевод % ilispread:1.8
		\tabularnewline \midrule
\tenevilglyph[yes][4]{i_G_bX}
	&	йыӄӄайъым
	&	
	&
	&	блно \cite[л. 66]{spbfaran79} % TODO: нужна интерпретация
	& 	[4.8] \linebreak
		jьqajiьm [йыӄӄайъым = замечательно, прекрасно, чудесно] [ИЛИ:1.18]  % ilispread:1.18
		\tabularnewline \midrule
\tenevilglyph[yes][4]{i_G_cD}
	&	кавэты
	&	
	&	кавэты [= удобно, уютно], удобно [64]
	&	
	& 	kawetь [кавэты] [ИЛИ:1.5]  % ilispread:1.5
		\tabularnewline \midrule
\tenevilglyph[yes][4]{i_G_cFD} 
	&	таӈычьэты = весело, приятно, увлекательно % выведено: таӈычьэты = весело, приятно, увлекательно
	&	
	&	нытэн'ычеэркын, пусть веселятся \currentGlyphWithAffixes{}{N} [64] % TODO: нужен перевод
	&	
	& 	\cite[364]{davydova2015a} \linebreak
		nьteŋьcieerkьn \currentGlyphWithAffixes{N}{} [ИЛИ:1.15] \linebreak  % TODO: нужен перевод % ilispread:1.15
		nьteŋьcieerkьn \currentGlyphWithAffixes{}{N,T} [ИЛИ:1.13] \linebreak  % TODO: нужен перевод % ilispread:1.13
		эlьteŋьcike \currentGlyphWithAffixes{etly}{} [ИЛИ:1.14] \linebreak  % TODO: нужен перевод % ilispread:1.14
		taŋkьnmal [таӈкынмаԓ = дружно] \currentGlyphWithAffixes{kynmal}{} [ИЛИ:1.14об]  % ilispread:1.15
		\tabularnewline \midrule
\tenevilglyph[yes][4]{i_G_cFD_2c}
	&	амтаӈычьэты % МП
	&	
	&	
	&	
	& 	amtaŋьciet [амтаӈычьэты = очень весело] [ИЛИ:1.12об]  % ilispread:1.13
		\tabularnewline \midrule
\tenevilglyph[yes][5]{BD,B}
	&	э'тки
	& 	худой \cite{bogoraz1934}
	&	этки [э'тки = плохо, скверно], плохой [10]
	&	худой \cite[л. 47]{spbfaran79} \linebreak
		хутои [худой] \cite[л. 68 об]{spbfaran79} 
	& 	\cite[364]{davydova2015a} \linebreak 
		\cite{bogoraz1934} \linebreak
		etqi [э'тки; слово напечатано] [12.18] \linebreak
		iэтке [э'тки] [34.8] \linebreak % э'тки
		плохо [34.11] \linebreak
		эткин [э'ткиӈ = плохой] [41.7] \linebreak
		iэkeŋ [э'ткиӈ] \currentGlyphWithAffixes[2]{}{} [ИЛИ:2.25]  % ilispread:2.24
		\tabularnewline \midrule
\tenevilglyph[yes][4]{BD_cD}   % Возможно просто э'тки + А, см. ilispread:1.11 (ilispread:1.6)
	&	а'ткэвма
	&	
	&
	&	
	& 	плохои [плохой] [37.2, 37.2об] \linebreak
		iatkuma [а’ткэвма = дурно, плохо] [ИЛИ:2.7об]  % ilispread:2.7
		\tabularnewline \midrule
\tenevilglyph[yes][3]{BD_b}  
	&	э'ӄэрэк
	&	
	&
	&	
	& 	niqereqen [нъэӄэрэӄин = ухудшается] \currentGlyphWithAffixes{}{N} [ИЛИ:2.2об]  % ilispread:2.2
		\tabularnewline \midrule
\tenevilglyph[yes][3]{O_jN}
	&
	&	
	&	никитэ [ныкитэ = ночью], ночью [65]
	&	ночью \cite[л. 47]{spbfaran79} 
	& 	\cite[360, 362]{davydova2015a} 
		\tabularnewline \midrule
\tenevilglyph[yes][5][nymnym]{2o_2j}
	&	нымным
	&	
	&	нымным [= поселок, селение], селение [121]
	&	стойбище, олени (богатый оленевод) \currentGlyphWithAffixes{}{velwyl} \cite[л. 47]{spbfaran79} \linebreak
		на стойбище \cite[л. 53]{spbfaran79} \linebreak
		отлакир [әtlьq-, эԓԓыӄ- = тундра] \cite[л. 68]{spbfaran79} % эԓԓыӄ- , вероятно, ошибка
	& 	\cite[364]{davydova2015a} \linebreak
		nьmnьm [нымным] [ИЛИ:2.10об]  \linebreak % ilispread:2.10
		remkьn [рэмкын = народ] \currentGlyphWithAffixes{}{N} [12.15]
		\tabularnewline \midrule
\tenevilglyph[yes][3]{2o_2j_JFE}
	&	рэмкын
	&	
	&	рэмкын* [народ, толпа] [121] % *«гребенка» отдельно
	&	
	& 	remkьn [ræmkьn, рэмкын; слово напечатано] [12.12об]
		\tabularnewline \midrule
\tenevilglyph[yes][5]{2o_2j_a}
	&	чавчыв, чавчыват
	&	
	&	
	&	
	& 	çawçuwaçgьn [чавчывачгын = богатый оленями; слово напечатано] [12.12об] \linebreak
		cawcu [чавчыв = богатый оленями, оленевод, оленный] [ИЛИ:2.14об] \linebreak  % ilispread:2.14
		cawcawat [чавчыват = оленеводы] [ИЛИ:1.20]  % ilispread:1.20
		\tabularnewline \midrule
\tenevilglyph[no][3]{i_j_jF}
	&
	& 	ложка \cite{bogoraz1934}
	&
	&	ложка \cite[л. 48]{spbfaran79}
	& 	\tabularnewline \midrule
\tenevilglyph[yes][4]{u_p}
	&
	& 	чайник \cite{bogoraz1934}
	&	чайкок [= чайник], чайник [79]
	&	чайник \cite[л. 48]{spbfaran79} \linebreak
		саиник [чайник] \cite[л. 53]{spbfaran79}
	& 	\cite[364]{davydova2015a}
		\tabularnewline \midrule
\tenevilglyph[yes][3]{u_p_b}
	&
	& 	белый чайник \cite{bogoraz1934}
	&
	&	белый чайник \cite[л. 48]{spbfaran79} 
	& 	\cite[364]{davydova2015a}
		\tabularnewline \midrule
\tenevilglyph[no][3]{u_pD_bD}
	&
	& 	медный чайник \cite{bogoraz1934}
	&
	&	медный чайник \cite[л. 48]{spbfaran79} 
	& 	\tabularnewline \midrule
\tenevilglyph[yes][3]{u_p_2b}
	&
	&	
	&
	&	широкий чайник \cite[л. 48]{spbfaran79} 
	& 	\cite[364]{davydova2015a}
		\tabularnewline \midrule
\tenevilglyph[yes][2]{JF-JFN_jF}
	&
	&	
	&
	&	ремень* \cite[л. 48]{spbfaran79} \linebreak
		ремен [ремень] \cite[л. 66 об]{spbfaran79}
	& 	[32.2об] \linebreak
		ырымен [ремень] [29.11об] \linebreak
		ŋelgьn [нэԓгын = шкура] [ИЛИ:2.28]  % ilispread:2.27
		\tabularnewline \midrule
\tenevilglyph[yes][3]{JF-JFN_jFN}
	&
	&	
	&
	&	
	& 	cetewen [читэвын = к счастью] [ИЛИ:2.28]  % ilispread:2.27
		\tabularnewline \midrule
\tenevilglyph[yes][5]{O_jXX} % TODO: проверить размер
	&	тэнуйгын
	& 	тюленья шкура \cite{bogoraz1934}
	&	тюленья шкура (Б. Т. 6) [25]
	&	тюленья шкура \cite[л. 48]{spbfaran79} \linebreak
		нерпа \cite[л. 66 об]{spbfaran79}
	& 	[1.13] \linebreak
		tenugьn [тэнуйгын = нерпичья шкура]* [ИЛИ:2.28]  % зеркально % ilispread:2.27
		\tabularnewline \midrule
\tenevilglyph[yes][2]{O_2b}
	&
	&	
	&
	&	шкура лахтака* \cite[л. 48]{spbfaran79} \linebreak
		лахтак \cite[л. 66 об]{spbfaran79}
	& 	потоска [29.11об] % TODO: нужна интерпретация
		\tabularnewline \midrule
\tenevilglyph[no][3]{O_2b_c_zR}
	&
	&	
	&
	&	морча [морж] \cite[л. 66 об]{spbfaran79}
	& 	\tabularnewline \midrule
\tenevilglyph[yes][4]{O_jXX_2b}
	&	коԓтаԓгын
	&	
	&
	&	
	& 	kultalgьn [qoltalgьn, коԓтаԓгын = шкура лахтака] [ИЛИ:2.28]  % ilispread:2.27
		\tabularnewline \midrule
\tenevilglyph[yes][4]{O_jXXE}
	&	рыркы, рыркэн
	&	морж \cite{lavrov1969}
	&	рыркы [= морж], морж [24]
	&	
	& 	rьrkь [рыркы; слово напечатано] [12.12] \linebreak
		rьrken [рыркэн = моржовый; слово напечатано] [12.12об]
		\tabularnewline \midrule
\tenevilglyph[yes][3]{O_jXX_C_c}
	&	мэмыԓ
	&	нерпа \cite{lavrov1969}
	&	мемыл [мэмыԓ = тюлень, нерпа], нерпа (тюлень) [23]
	&	
	& 	[49.1] \linebreak
		memьl [мэмыԓ; слово напечатано] [14.10]
		\tabularnewline \midrule
\tenevilglyph[yes][3]{O_jXX_2zRX}
	&	унъэԓ
	&	
	&	унел [унъэԓ = лахтак], лахтак (морской заяц), Erignathus barbatus [23]
	&	
	& 	% unele [унъэԓ; слово напечатано] [12.11об] \linebreak
		unel [unæl, унъэԓ; слово напечатано] [12.12]
		\tabularnewline \midrule
\tenevilglyph[yes][3]{O_jXX-2b}
	&	кеԓиԓьын, кеԓиԓьыт
	&	
	&	келилин [кеԓиԓьын = нерпа], пестрая нерпа [24]
	&	
	& 	kelilьt [кеԓиԓьыт = нерпы; слово напечатано] [12.22] 
		\tabularnewline \midrule
\tenevilglyph[yes][3]{O_jXX-2b_С}
	&	эчьыԓьыт
	&	
	&	
	&	
	& 	eçelьt [æcьlьt, эчьыԓьыт = жирные; слово напечатано] [12.22] % TODO: уточнить перевод
		\tabularnewline \midrule
\tenevilglyph[yes][3]{O_jXX_c_2b}
	&	пуврэӄ, пуврэӄэт
	&	
	&	порэк [пуврэӄ = белуха], белуха [24]
	&	
	& 	purәqet [пуврэӄэт = белухи; слово напечатано] [12.20] 
		\tabularnewline \midrule
\tenevilglyph[yes][3]{O_jXX_pF}
	&	нэннэт
	&	
	&	нэннэт [= выдра], выдра [24]
	&	
	& 	nennet [нэннэт; слово напечатано] \currentGlyphWithAffixes{}{T} [12.11] 
		\tabularnewline \midrule
\tenevilglyph[yes][5]{2CE}
	&	чакар
	&	
	&	чакар [= сахар], сахар [97]
	&	сахар \cite[л. 44, 49]{spbfaran79}
	& 	[25.6] \linebreak
		caqar [чакар] [ИЛИ:1.15]  % ilispread:1.15
		\tabularnewline \midrule
\tenevilglyph[no][3]{I_q} 
	&
	& 	трубка \cite{bogoraz1934} \linebreak
		a pipe \cite{mindalevich1934}
	&	таакойнын [таа'койӈын = курительная трубка], трубка (курительная) [49]
	&	трубка \cite[л. 49]{spbfaran79} 
	& 	\tabularnewline \midrule
\tenevilglyph[no][3]{I_q_UE_JX}
	&
	& 	папироска \cite{bogoraz1934}
	&	сигрис [чиӄырич = папироса, сигарета], папироса [61]
	&	папироска \cite[л. 49]{spbfaran79} 
	& 	\tabularnewline \midrule
\tenevilglyph[no][3]{I_q_UE_JX_b_q}
	&
	&	
	&
	&	трубка с мундштуком \cite[л. 49]{spbfaran79} 
	& 	\tabularnewline \midrule
\tenevilglyph[yes][5][kelikel]{UE_JX} 
	&	кэԓикэԓ
	&	
	&	кэликэл [кэԓикэԓ = рисунок, картина], книга, письмо, рисунок [60]
	&	
	& 	\cite[364]{davydova2015a} \linebreak
		kelekel [kælikæl, кэԓикэԓ] [4.10об] \linebreak % кэԓикэԓ
		kalejolqal [калеёԓӄыԓ = домашний урок; слово напечатано] [12.24] \linebreak % TODO: уточнить перевод
		келекел [кэԓикэԓ] [45.4об] \linebreak
		kelekel [кэԓикэԓ] [ИЛИ:2.28об] \linebreak  % ilispread:2.28
		касет [газета] [34.8об] \currentGlyphWithAffixes{}{K,A} \linebreak
		kelit [кэԓит = книги (буквы?); слово напечатано] \currentGlyphWithAffixes{}{T} [12.24] \linebreak
		kalepь [каԓейпы = из книги] \currentGlyphWithAffixes{}{P} [ИЛИ:1.16] \linebreak  % TODO: уточнить перевод % ilispread:1.16
		kelek [кэԓик = в книге / рисовать] \currentGlyphWithAffixes{}{K} [ИЛИ:1.15об]  % ilispread:1.16
		\tabularnewline \midrule
\tenevilglyph[yes][4]{UE_JX_j_q} 
	&	кэԓитъуԓ
	&	
	&
	&	
	& 	еенки [деньги] [34.1] \linebreak
		keletiul [kælitul, кэԓитъуԓ = бумага, деньги] [ИЛИ:2.26]  % ilispread:2.25
		\tabularnewline \midrule
\tenevilglyph[yes][3]{UE_JX_CF-q} 
	&
	&	
	&	тетрадка [60]
	&	
	& 	cetrakan [тетрадь] [ИЛИ:1.23] \linebreak  % ЗТ ilispread:1.24
		kacet \currentGlyphWithAffixes{K,A}{} [ИЛИ:2.22]  % TODO: нужен перевод, газета? % ilispread:2.21
		\tabularnewline \midrule
\tenevilglyph[yes][1]{UE_JX_jY,UE_JX_jYD} 
	&
	&	
	&	
	&	
	& 	ковхоситы \currentGlyphWithAffixes[2]{}{K,A,C,T} [37.2об] \linebreak % TODO: нужна интерпретация
		ковхоз \currentGlyphWithAffixes[2]{}{K,O,Q} [44.2об] \linebreak
		koqocetь \currentGlyphWithAffixes{}{K,A,T} [ИЛИ:2.12об] \linebreak  % TODO: нужен перевод % ilispread:2.12
		koqocetь \currentGlyphWithAffixes{}{T} [ИЛИ:2.14] \linebreak  % ilispread:2.13
		koqocetь \currentGlyphWithAffixes{}{K,T} [ИЛИ:2.15] \linebreak  % ilispread:2.14
		koqocken \currentGlyphWithAffixes{}{K,A,Q} [ИЛИ:2.28об]  % ilispread:2.28
		\tabularnewline \midrule
\tenevilglyph[yes][4]{l_JXE} % знаки на л. 50 и л. 66 зеркальны
	&	ԓывавык
	&	
	&	тъытлен [тъытԓьэн = больной], больной [71] % похоже на ошибку
	&	не мог* \cite[л. 50]{spbfaran79} \linebreak
		нимнок \cite[л. 66 об]{spbfaran79} % TODO: нужна интерпретация
	& 	[9.1] \linebreak
		lwaurkьt [ԓывавыркыт = не могут] \currentGlyphWithAffixes{}{T} [ИЛИ:2.20об] \linebreak  % TODO: уточнить перевод % ilispread:2.20
		lwaurkьt [лывавыркыт] \currentGlyphWithAffixes{L}{K,T} [ИЛИ:2.23]  % ilispread:2.22 
		\tabularnewline \midrule
\tenevilglyph[yes][4]{lE_JXE} 
	&	эӈъэԓетык
	&	
	&	
	&	
	& 	эŋilerkьt [эӈъэԓетыркыт = устают] [ИЛИ:1.4об] \linebreak  %TODO: уточнить перевод % ilispread:1.5
		neŋiletqen [нэӈъэԓетӄин = с трудом] \currentGlyphWithAffixes{}{N} [ИЛИ:2.8] \linebreak  % ilispread:2.7
		aŋiacgrrgьn [аӈъачгыргын = трудность, трудно, тяжело] \currentGlyphWithAffixes{A}{R,R} [ИЛИ:2.27об] \linebreak  % ilispread:2.27
		neŋielenmure [нэӈъэԓенмури = с трудом мы] \currentGlyphWithAffixes{}{muri} [ИЛИ:2.25об]  % ilispread:2.25
		\tabularnewline \midrule
\tenevilglyph[yes][5]{cF_CF}
	&	ынӈин, ынӈинкин
	&	
	&	ынн'ын, ынн'от [ынӈин, ынӈот = так, такой], так, эдак [54]
	&	так \cite[л. 50]{spbfaran79} \linebreak
		әnmen [энмэн = также, итак] \cite[л. 39 об]{spbfaran79} \linebreak % энмэн, возможно, ошибка
		дак [так] \cite[л. 66 об]{spbfaran79}
	& 	\cite[360, 361, 364]{davydova2015a} \linebreak
		\cite[26, 28]{lavrov1969} \linebreak
		етак [так] [36.1] \linebreak
		ьŋen [ынӈин] [ИЛИ:1.3] \linebreak  % ilispread:1.3
		ьnŋen [ынӈин] [ИЛИ:1.5об] \linebreak  % ilispread:1.6
		ьŋenken [ынӈинкин = столько, таковой] [ИЛИ:1.3об]  \linebreak % зеркально % ilispread:1.4
		эмынхен [эмынӄэн = только это] \currentGlyphWithAffixes{ymy}{} [41.7] % NB: Обрати внимание, используется ымы вместо двух отдельных скобочек да ам-/эм-. Возможно, указание на происхождение последних от знака для ымы, но основной знак не тот
		\tabularnewline \midrule
\tenevilglyph[yes][3]{cF_CF_2c}
	&	эмынӈин
	&	
	&	
	&	
	& 	эmьŋen [эмынӈин = только так] [ИЛИ:1.6]  % ilispread:1.6
		\tabularnewline \midrule
\tenevilglyph[yes][5]{o_jX}
	&	а'тав 
	& 	так себе \cite{bogoraz1934}
	&	атав [а'тав], аттав, давай, ну (подбадривающее междометие) % TODO: нужна транскрипция 
	&	так себе, attaw [а'тав = ну, что же, напрасно, зря] \cite[л. 50]{spbfaran79} \linebreak % а'тав
		attaw [а'тав] \cite[л. 52 об]{spbfaran79} \linebreak
		всетакии (нука) (?) \cite[л. 53]{spbfaran79} 
	& 	\cite[361]{davydova2015a} \linebreak
		так [32.6] \linebreak
		attaw [а'тав; слово напечатано] [12.20об] \linebreak
		iatav [а'тав] [ИЛИ:1.2]  % МП: интересно, гортанная смычка ` звучит действительно перед «а» % ilispread:1.2
		\tabularnewline \midrule % [25.12]
\tenevilglyph[yes][1]{o_qX_f}
	&
	&	
	&
	&	
	& 	iатаро [34.10] \linebreak % атарго?
		iataru [ИЛИ:2.10]  % TODO: нужен перевод % ilispread:2.9
		\tabularnewline \midrule % [25.12]
\tenevilglyph[yes][3]{i_2j}
	&
	&	
	&
	&	долстои [толстый]* \cite[л. 69 об]{spbfaran79} % TODO: убедиться, что l и j тут одно и то же
	& 	\cite[364]{davydova2015a} \linebreak
		\cite[28]{lavrov1969} 
		\tabularnewline \midrule
\tenevilglyph[yes][5]{i_2j_iSY}
	&	э'квырга
	&	
	&	эквырга* [э'квырга = однако же, тем не менее], однако [90]
	&	несмотря на то, что \cite[л. 50]{spbfaran79}
	& 	\cite[360]{davydova2015a} \linebreak
		iekwьrga* [ekwurga, э'квырга] [ИЛИ:1.8об]  % ilispread:1.9
		\tabularnewline \midrule
\tenevilglyph[yes][3]{B_2BD}
	&
	&	
	&
	&	быть \cite[л. 50]{spbfaran79} 
	& 	\cite[364]{davydova2015a} \linebreak
		teerkьn \currentGlyphWithAffixes{}{T,R,K} [ИЛИ:2.7об]  % ilispread:2.7
		\tabularnewline \midrule
\tenevilglyph[yes][5]{o_L}
	&	киткит
	&	
	&	кит-кит [киткит], чуть чуть [46]
	&	мало \cite[л. 50]{spbfaran79} \linebreak
		kitkit [киткит = мало, немного] \cite[л. 39 об]{spbfaran79}
	& 	\cite[360, 361, 364]{davydova2015a} \linebreak
		немноско [немножко] [34.11] \linebreak
		ketket [киткит] [ИЛИ:1.17]  % ilispread:1.17
		\tabularnewline \midrule
\tenevilglyph[yes][5]{oI_vD}
	&	эты-ым
	&	
	&	этым [эты-ым = значит, наверное], наверно [95]
	&	вероятно \cite[л. 50]{spbfaran79} \linebreak
		наверно \cite[л. 67]{spbfaran79}
	& 	\cite[364]{davydova2015a} \linebreak
		анако [33.4] \linebreak % TODO: нужна интерпретация
		эtiьm [эты-ым] [ИЛИ:1.5]  % ilispread:1.5
		\tabularnewline \midrule
\tenevilglyph[yes][5]{bD_b}
	&	ԓыгэн
	& 	только \cite{bogoraz1934}
	&	лыгэн [ԓыгэн], инстинно, несомненно, только [113]
	&	только \cite[л. 50]{spbfaran79} \linebreak
		lien [liæn, ԓыгэн = как только] \cite[л. 52 об, 56]{spbfaran79} % ԓыгэн
	& 	\cite[361, 364]{davydova2015a} \linebreak
		\cite[28]{lavrov1969} \linebreak
		lien [liæn; слово напечатано] [12.20об] \linebreak
		lьen [liæn, ԓыгэн] [68.34об, ИЛИ:2.5об]  % ilispread:2.5
		\tabularnewline \midrule
\tenevilglyph[yes][3]{bD_2b}
	&	ԓыгэнитык
	& 	
	&	
	&	
	& 	\cite[364]{davydova2015a} \linebreak
		lьenetьk [ԓыгэнитык = напрасно, зря, просто так] \currentGlyphWithAffixes{}{T} [ИЛИ:2.14об] \linebreak  % ilispread:2.14
		lьenetьk [ԓыгэнитык] \currentGlyphWithAffixes{}{T,K} [ИЛИ:2.26] \linebreak  % ilispread:2.25
		\tabularnewline \midrule
\tenevilglyph[yes][4]{u_2k_uN_2k}
	&
	&	
	&
	&	спали (ушли?) \cite[л. 50]{spbfaran79} % судя по всему "спали"
	& 	[25.9] \linebreak
		ычпау [спал] [30.6] \linebreak
		спас [спать] [33.4] \linebreak
		nejlqetqen [ныйыԓӄэтӄин = спал] [ИЛИ:1.7об] \linebreak  % ilispread:1.8
		mьnjьlqənmьk [= пойдем спать; слово напечатано] \currentGlyphWithAffixes{M}{K} [12.21] % TODO: нужна транскрипция
		\tabularnewline \midrule
\tenevilglyph[yes][3]{cU_2q_cD_2q}
	&	яак
	&	
	&
	&	словно, как бы \cite[л. 50]{spbfaran79} \linebreak % видимо, ошибка
		qajaagьtkь \cite[л. 52 об]{spbfaran79} % TODO: нужен перевод
	& 	\cite[360–362, 364]{davydova2015a} \linebreak
		rajarkьnen [ИЛИ:1.4об] \linebreak  % TODO: нужен перевод % ilispread:1.5
		nenajaqen [нэнайъоӄэн] \currentGlyphWithAffixes{}{N} [ИЛИ:1.4] \linebreak  % TODO: нужен перевод % ilispread:1.4
		jaak [яак = использовать] \currentGlyphWithAffixes{}{K} [ИЛИ:2.11об,ИЛИ:1.8об] \linebreak   % ilispread:2.11, ilispread:1.9
		jajolqьl [яаёԓӄыԓ = нужный предмет, мне нужно] \currentGlyphWithAffixes{A}{L} [ИЛИ:2.16] \linebreak  % ilispread:2.15
		jajulqьl [яаёԓӄыԓ] \currentGlyphWithAffixes{}{A} [ИЛИ:1.6] \linebreak  % ilispread:1.6
		jaat \currentGlyphWithAffixes{}{T} [ИЛИ:1.18об] \linebreak  % ilispread:1.19
		nejaarkьnen \currentGlyphWithAffixes{N}{R,N,N} [ИЛИ:2.6об]  % ilispread:2.6
		\tabularnewline \midrule
\tenevilglyph[yes][3]{cU_q_j_cD_2q}
	&	яаёԓӄыԓ
	&	
	&
	&	
	& 	jajulqьl [яаёԓӄыԓ = нужный предмет, мне нужно] [ИЛИ:1.6]  % ilispread:1.6
		\tabularnewline \midrule
\tenevilglyph[yes][1]{cU_2q_cD_2q_o}
	&
	&	
	&
	&	
	& 	mьjarkьn [ИЛИ:1.12, ИЛИ:2.1об]   % TODO: нужен перевод % ilispread:1.12, ilispread:2.1
		\tabularnewline \midrule
\tenevilglyph[yes][4]{i_oB} % TODO: уточнить написание
	&	винвэ
	& 	тайный \cite{bogoraz1934}
	&	винвыкин [винвэткин = тайный], тайный* [68]
	&	тайно \cite[л. 50]{spbfaran79} \linebreak
		wьnvь [vinvә, винвэ = тайно, крадучись] \cite[л. 56]{spbfaran79} % винвэ
	& 	\cite[364]{davydova2015a} \linebreak
		\cite{bogoraz1934} \linebreak
		венве [винвэ] [41.7]
		\tabularnewline \midrule
\tenevilglyph[yes][5]{i_u} 
	&	ийъам
	&	
	&	
	&	посиму [почему] \cite[л. 66 об]{spbfaran79}
	& 	ejam [iam, ьjam, ийъам = почему] [5.1] \linebreak
		посиму [почему] [37.2об] \linebreak
		ejam [ийъам] [68.34об] \linebreak
		ijam [ийъам] [ИЛИ:1.4] \linebreak  % ilispread:1.4
		iejam [ийъам] [ИЛИ:1.6]  % ilispread:1.6
		\tabularnewline \midrule
\tenevilglyph[yes][5]{c_J}
	&	нутэнут
	& 	в поле \cite{bogoraz1934}\linebreak
		земля \cite{lavrov1969}
	&	нутэнут [= нутен = тундра, земля, страна], земля [67]
	&	в поле \cite[л. 50]{spbfaran79} \linebreak
		nutek [нутэк = на земле] \cite[л. 56]{spbfaran79} % нутэк
	& 	\cite[360]{davydova2015a} \linebreak
		\cite[28]{lavrov1969} \linebreak
		nutaьtь [нотагты = в тундру] \currentGlyphWithAffixes{}{T,T} [ИЛИ:1.2] \linebreak  % ilispread:1.2
		nutenut [нутэнут] [ИЛИ:2.2]  % ilispread:2.1
		\tabularnewline \midrule
\tenevilglyph[yes][5]{c_J_2j}
	&	нутэсӄын
	&	
	&	нутэкин [= земляной, тундровый], земной, полевой [67]
	&	nutesqan [nutæsqәn, нутэсӄын = земля, почва] \cite[л. 39]{spbfaran79} % нутэсӄын
	& 	\cite[362, 364]{davydova2015a} \linebreak
		\cite[28]{lavrov1969} \linebreak
		nutecqьn [nutæsqәn, нутэсӄын] [ИЛИ:1.12] \linebreak  % ilispread:1.12
		nutecqьken [нутэсӄыкин = земляной] \currentGlyphWithAffixes{}{K,N} [ИЛИ:2.3об] \linebreak  % ilispread:2.3
		nutecqepь [нотасӄэпы = от земли] \currentGlyphWithAffixes{}{P} [ИЛИ:2.4об]  % ilispread:2.4
		\tabularnewline \midrule
\tenevilglyph[yes][4]{O_cN_JN}
	&	камԓеԓыӈ
	&	
	&	камлелы нутэнут, вокруг земли [67] % TODO: нужна транскрипция, нужен перевод
	&	
	& 	\cite[364]{davydova2015a} \linebreak
		kamleleь [камԓеԓы = вокруг] [ИЛИ:1.12] \linebreak  % ilispread:1.12
		nutenut kamleleь [нутэнут камԓеԓы] [ИЛИ:1.5]  % ilispread:1.5
		\tabularnewline \midrule
\tenevilglyph[yes][3]{i_2bX}
	&	авэтываӄ
	& 	сразу \cite{bogoraz1934}
	&	авэтывак [авэтываӄ], сразу [126]
	&	сразу \cite[л. 51]{spbfaran79} \linebreak
		awetuwaq [авэтываӄ = быстро, проворно, сразу] \cite[л. 56]{spbfaran79} % авэтываӄ
	& 	[7.7]
		\tabularnewline \midrule
\tenevilglyph[yes][4]{o_m_j}
	&	ыммэмэӄэй
	&	
	&	ыммэмы [= мама (в речи близких людей)], ыммэй [= мама (при обращении)], мама [121]
	&	мама \cite[л. 51, 37]{spbfaran79} \linebreak
		мама \cite[л. 67]{spbfaran79} 
	& 	\cite[362]{davydova2015a} \linebreak
		\cite[28]{lavrov1969} \linebreak
		мама [33.5об] \linebreak
		ьmeqj [әmmәqәj, ыммэмэӄэй = мама, мать, кормилица] [ИЛИ:1.9] \linebreak  % ilispread:1.9
		ымеге [ыммэмэӄэй] [45.6об] \linebreak
		Reja [Рая?] [ИЛИ:2.10]  % ilispread:2.9
		\tabularnewline \midrule
\tenevilglyph[yes][4]{B_b_oX}
	&
	&	
	&
	&	остров Кулючин \cite[л. 51]{spbfaran79} \linebreak
		на Колючено \cite[л. 37]{spbfaran79} 
	& 	\cite[360]{davydova2015a} \linebreak
		kulucek \currentGlyphWithAffixes{}{K} [ИЛИ:2.20]  % TODO: нужен перевод, кулючэк = на Кулючине? % ilispread:2.19
		\tabularnewline \midrule
\tenevilglyph[yes][5]{UD_i_2l}
	&	яԓгытык
	&	кочевать* \cite{lavrov1969}
	&	ялгытык [яԓгытык = кочевать], кочевать [103]
	&	покочевали* \cite[л. 51]{spbfaran79} % вверх ногами
	& 	[25.8об] \linebreak
		ялхытык [яԓгытык] [34.8] \linebreak % яԓгытык
		покочоеоч [покочуешь] [30.5об] \linebreak
		jalgьtken [яԓгыткэн = к кочевке] \currentGlyphWithAffixes{}{K,N} [ИЛИ:2.16]  % Weinstein, TODO: уточнить перевод % ilispread:2.15
		\tabularnewline \midrule
\tenevilglyph[yes][2]{UD_i_2l_b}
	&
	&	
	&	кочуем [103]
	&	
	& 	moliьn \currentGlyphWithAffixes{}{N} [ИЛИ:2.11] \linebreak  % TODO: нужен перевод % ilispread:2.10
		mutьlerkьt \currentGlyphWithAffixes{}{T} [ИЛИ:1.6] \linebreak  % ilispread:1.6
		mutlek \currentGlyphWithAffixes{}{K} [ИЛИ:1.2об]  % ilispread:1.3
		\tabularnewline \midrule
\tenevilglyph[yes][1]{UD_i_2l_b_i_2q}
	&
	&	
	&	
	&	
	& 	nьmgutьleqenet [ИЛИ:1.5об]  % TODO: нужен перевод % ilispread:1.6
		\tabularnewline \midrule
\tenevilglyph[yes][5]{i_2iY}
	&	гыргоча
	& 	вверх \cite{bogoraz1934}
	&	гырголягты [гыргоԓягты = кверху, наверх], вверх [51] 
	&	вверх \cite[л. 51]{spbfaran79} 
	& 	\cite[361]{davydova2015a} \linebreak
		grguca [гыргоча = выше, над, поверх чего-л., вверх (по реке)] [ИЛИ:2.6об]  % ilispread:2.6
		\tabularnewline \midrule
\tenevilglyph[yes][4]{i_o_iY}
	&	эвыча
	& 	
	&	
	&	
	& 	эeuca [euca, эвыча = внизу] [ИЛИ:2.6об]  % ilispread:2.6
		\tabularnewline \midrule
\tenevilglyph[yes][5]{u_v_CD}
	&	аръаԓя
	&	
	&
	&	вполне \cite[л. 51]{spbfaran79} \linebreak
		arala [аръаԓя = совсем, вовсе] \cite[л. 52]{spbfaran79} % аръаԓя
	& 	\cite[361, 364]{davydova2015a} \linebreak
		\cite[28]{lavrov1969} \linebreak
		ariala [аръаԓя] [ИЛИ:1.9]  % ilispread:1.9
		\tabularnewline \midrule
\tenevilglyph[yes][5]{cF-cF}
	&
	& 	тоже, опять \cite{bogoraz1934}
	&	неме [нэмэ = опять, снова], опять [19]
	&	тоже, опять \cite[л. 51]{spbfaran79} \linebreak
		опять \cite[л. 53]{spbfaran79} 
	& 	\cite[361, 362]{davydova2015a} \linebreak
		апес [опять] [33.4] \linebreak
		neme [нэмэ] [ИЛИ:2.15]  % ilispread:2.14
		\tabularnewline \midrule
\tenevilglyph[yes][2]{c_cD_'} 
	&	чамъам
	&	
	&	
	&	кута \cite[л. 66 об]{spbfaran79} % TODO: нужна интерпретация
	& 	camiam [camam, чамъам = не мочь] [ИЛИ:2.2]  % ilispread:2.1
		\tabularnewline \midrule
\tenevilglyph[yes][5]{oF_2l_lG}
	&	чымче
	&	
	&
	&	близко \cite[л. 51, 53]{spbfaran79} \linebreak
		ꞓьmꞓә [cьmcь, чымче = близко] \cite[л. 54]{spbfaran79} \linebreak % чымче
		плиско [близко] \cite[л. 68 об]{spbfaran79}
	& 	\cite[364]{davydova2015a} \linebreak 
		\cite{bogoraz1934} \linebreak
		cьmce [cьmcь, чымче] [ИЛИ:1.20] \linebreak  % ilispread:1.20
		cьmceqej \currentGlyphWithAffixes{}{nyppyluqin} [27.18об] % TODO: нужен перевод
		\tabularnewline \midrule
\tenevilglyph[yes][3]{2cU_cD_jFY}
	&	ӄоныръым
	& 	
	&	
	&	
	& 	\cite[364]{davydova2015a} \linebreak
		qonьrьm [ӄоныръым = между прочим, притом, впрочем, да и] [12.20об]
		\tabularnewline \midrule
\tenevilglyph[yes][4]{o-o-o} 
	&	ыннаны
	&	
	&	иннаны [ыннаны = одинаково, одинаковые], одинаково [122]
	&	
	& 	осенако [одинаково] [34.12] \linebreak
		ьnanь [әnnanьŋ, ыннаны] [ИЛИ:1.5]  % ilispread:1.5
		\tabularnewline \midrule
\tenevilglyph[yes][5]{c_q_cD_q} 
	&	амынан
	&	
	&	имынан [амынан = один он], один (он) [19]
	&	
	& 	\cite[360,364]{davydova2015a} \linebreak
		осын [один] [37.2] \linebreak
		амынан [44.2об] \linebreak
		amьnan [амынан] [ИЛИ:2.25] \linebreak  % ilispread:2.24
		amьgьmnan [амгымнан = один я] \currentGlyphWithAffixes{}{gymnan} [ИЛИ:1.13об]  % ilispread:1.14
		\tabularnewline \midrule
\tenevilglyph[yes][1]{c_l_cD_q} 
	&	
	&	
	&	
	&	
	& 	macamьrgьnan \currentGlyphWithAffixes{M,A}{ynan} [ИЛИ:2.26об]  % TODO: нужен перевод % ilispread:2.26
		\tabularnewline \midrule
\tenevilglyph[yes][4]{с_jY_cD_q} 
	&	ӄунэче
	&	
	&	
	&	
	& 	qunece [ӄунэче = однажны, как-то раз] [ИЛИ:1.14]  % ilispread:1.14
		\tabularnewline \midrule
\tenevilglyph[yes][5][ynnen]{o_2q}
	&	1, ыннэн
	&	1 \cite{lavrov1969}
	&	ыннэн, 1, один, одна, одно [43]
	&	1 \cite[л. 64]{spbfaran79} \linebreak
		amunen [әnnæn?, ыннэн? = один] \cite[л. 39 об]{spbfaran79} % ыннэн
	& 	1 \cite[360, 362]{davydova2015a} \linebreak
		\cite[361, 364]{davydova2015a} \linebreak
		\cite[26]{lavrov1969} \linebreak
		ьnnen [ыннэн] [ИЛИ:1.21] \linebreak  % ilispread:1.21
		ьnencin [ыннэнчьэн = один из] [ИЛИ:1.9]  % ilispread:1.9
		\tabularnewline \midrule
\tenevilglyph[yes][5][qlikkin]{o_2q_j}
	&	20, ӄԓиккин
	&	20 \cite{lavrov1969}
	&	кликкин [ӄԓиккин = двадцать], 20, двадцать [44]
	&	20 \cite[л. 64]{spbfaran79} 
	& 	20 \cite[360, 362]{davydova2015a} \linebreak
		\cite[361, 363]{davydova2015a} \linebreak
		\cite[26]{lavrov1969} \linebreak
		qlekken [ӄԓиккин] [23.6]
		\tabularnewline \midrule
\tenevilglyph[yes][5]{i_b_s_j_o_2q,i_b_s_j}
	&	1000, тысяча
	&	1000* \cite{lavrov1969}
	&	мытлынча мынгытклеккэн [мытԓынча мынгытӄԓеккэн], тысяча [57] 
	&	
	& 	1000 [25.1об] \linebreak
		tecьce [тысяча] [ИЛИ:2.15] \linebreak  % ilispread:2.14
		tьsьç [тысяча; слово напечатано] \currentGlyphWithAffixes[2]{}{} [12.19] \linebreak
		1000 \currentGlyphWithAffixes[2]{}{} [7.11]
		\tabularnewline \midrule
\tenevilglyph[yes][3]{i_b_s_j_s_jX}
	&	
	&	1000000 \cite{lavrov1969}
	&	
	&	
	& 	[недалеко от степеней десятки от 10 до 1000000] [7.33]
		\tabularnewline \midrule
\tenevilglyph[yes][4]{i_b_s_j_o_q_j}
	&	20000
	&	
	&
	&	
	& 	20000 [36.2] \tabularnewline \midrule
\tenevilglyph[yes][5]{B-}
	&	2, ӈирэӄ
	&	2 \cite{lavrov1969}
	&	н'ирэк [ӈирэӄ = два], 2, два, две, двое [115]
	&	2 \cite[л. 64]{spbfaran79} \linebreak
		двоих \cite[л. 68]{spbfaran79}
	& 	2 \cite[360, 362]{davydova2015a} \linebreak
		\cite[361, 363, 364]{davydova2015a} \linebreak
		\cite[28]{lavrov1969} \linebreak
		ŋerq [ӈирэӄ] [ИЛИ:2.14об]  % ilispread:2.14
		\tabularnewline \midrule
\tenevilglyph[yes][3]{BY-}
	&	[ӈирэче]  % TODO: по аналогии с ӈыроча, плюс зачёркнуто рядом с ӈыроча в ilispread:1.9
	&	
	&	
	&	
	& 	[ӈирэче = дважды] [ИЛИ:1.9,23.5об]  % ilispread:1.9
		\tabularnewline \midrule
\tenevilglyph[yes][5]{B-_j}
	&	40, ӈирэӄӄԓиккин
	&	40 \cite{lavrov1969}
	&	н'ирэккликкин [ӈирэӄӄԓиккин = сорок], 40, сорок [114]
	&	
	& 	40 \cite[360]{davydova2015a} \linebreak
		40 — ŋirәq-qlikkin [ӈирэӄӄԓиккин; слово напечатано] [12.19]
		\tabularnewline \midrule
\tenevilglyph[yes][3]{oI_3j_B-}
	&	[200]
	&	
	&	
	&	
	& 	[200] [4.5об]
		\tabularnewline \midrule
\tenevilglyph[yes][4]{B-_2oI_jF_j}
	&	400, ӄԓиӄӄԓиккин
	&	400 \cite{lavrov1969}
	&	кликликкин [ӄԓиӄӄԓиккин = четыреста], 400, четыреста [122] 
	&	
	& 	[25.2] \linebreak
		400 — qlik-qlikkin [ӄԓиӄӄԓиккин; слово напечатано] [12.19]
		\tabularnewline \midrule
\tenevilglyph[yes][5]{i_b_s_j_B-}
	&	2000
	&	
	&
	&	
	& 	2000 [7.11,36.2] 
		\tabularnewline \midrule
\tenevilglyph[yes][5][vyroq]{o_2q_q_l,TD_l}
	&	3, ӈыроӄ
	&	3 \cite{lavrov1969}
	&	н'ырок [ӈыроӄ = три], 3, три [43] 
	&	три \cite[л. 41]{spbfaran79} \linebreak
		ŋьroq [ӈыроӄ = три] \cite[л. 39]{spbfaran79} \linebreak % ӈыроӄ
		3 \cite[л. 64]{spbfaran79}
	& 	3 \cite[360, 362]{davydova2015a} \linebreak
		3 \currentGlyphWithAffixes[2]{}{} [4.1об] \linebreak
		\cite[361, 363, 364]{davydova2015a} \linebreak
		ŋьrurgare [ӈыроргарэ = трое] [ИЛИ:1.9] \linebreak  % ilispread:1.9
		ŋьroq [ӈыроӄ; слово напечатано] [12.22] \linebreak
		ŋьroq [ӈыроӄ] \currentGlyphWithAffixes[2]{}{} [ИЛИ:2.14об]  % ilispread:2.14
		\tabularnewline \midrule
\tenevilglyph[yes][4]{o_2q_q_lY}
	&	ӈыроча
	&	
	&	
	&	
	& 	ŋьroca [ӈыроча = трижды] [ИЛИ:1.9]  % ilispread:1.9
		\tabularnewline \midrule
\tenevilglyph[yes][5]{o_2q_q_l_j,TD_l_J}
	&	60, ӈыроӄӄԓеккэн
	&	60 \cite{lavrov1969}
	&	н'ырокклеккэн [ӈыроӄӄԓеккэн = шестьдесят], 60, шестдесят [44]
	&	
	& 	60 \cite[360]{davydova2015a} \linebreak
		\cite[26]{lavrov1969} \linebreak
		60 — ŋьroq-qlekken [ӈыроӄӄԓеккэн; слово напечатано] [12.19] \linebreak
		~[60] \currentGlyphWithAffixes[2]{}{} [21.19об] \linebreak
		60 \currentGlyphWithAffixes[2]{}{} [20.1об]
		\tabularnewline \midrule
\tenevilglyph[yes][5]{oI_3j_TD_l}
	&	300
	&	
	&	
	&	
	& 	300 [12.2,12.21]
		\tabularnewline \midrule
\tenevilglyph[yes][4]{o_q_q_l_2oI_jF_j,TD_l_2oI_jF_j}
	&	600
	&	600 \cite{lavrov1969}
	&	кликкликан мынгытклеккэн пароль, 600, шестьсот [122] % TODO: нужна транскрипция
	&	
	& 	[11.4об] \linebreak
		[600] \currentGlyphWithAffixes[2]{}{} [29.3об]
		\tabularnewline \midrule
\tenevilglyph[yes][4]{i_b_s_j_o_q_q_l,i_b_s_j_o_TD_l}
	&	3000
	&	
	&
	&	
	& 	3000 \currentGlyphWithAffixes[2]{}{} [37.5] \linebreak
		[3000] [32.13об] 
		\tabularnewline \midrule
\tenevilglyph[yes][5]{o_q_c_T,q_c_t}
	&	4, ӈырaӄ
	&	4 \cite{lavrov1969}
	&	н'ырак [ӈырaӄ = четыре], 4, четыре [43] 
	&	4 \cite[л. 64]{spbfaran79}
	& 	\cite[361]{davydova2015a} \linebreak
		\cite[26]{lavrov1969} \linebreak
		4 \cite[360]{davydova2015a} \linebreak
		4 \currentGlyphWithAffixes[2]{}{} [4.1об] \linebreak
		ŋьraq [ӈырaӄ] \currentGlyphWithAffixes[2]{}{} [ИЛИ:2.14об]  % ilispread:2.14
		\tabularnewline \midrule
\tenevilglyph[yes][3]{o_q_c_T_qY}
	&	[ӈырача]
	&	
	&	
	&	
	& 	[ӈырача = четырежды] [21.9об,21.10]
		\tabularnewline \midrule
\tenevilglyph[yes][5]{o_q_c_T_j,q_c_T_j}
	&	80, ӈыраӄӄԓеккэн
	&	80 \cite{lavrov1969}
	&	н'ыракклеккен [ӈыраӄӄԓеккэн = восемьдесят], 80, восемдесят [44]
	&	
	& 	[25.4] \linebreak
		80 [57.50] \linebreak
		80 — ŋьraq-qlekken [ӈыраӄӄԓеккэн; слово напечатано] [12.19] \linebreak
		~[80] \currentGlyphWithAffixes[2]{}{} [21.18об] \linebreak
		80 \currentGlyphWithAffixes[2]{}{} [20.1об]
		\tabularnewline \midrule
\tenevilglyph[yes][4]{o_c_T_2oI_jF_j,c_T_2oI_jF_j}
	&	800
	&	800 \cite{lavrov1969}
	&	н'ыроча мынгытклеккэн [ӈырача мынгытӄԓеккэн], 800, восемьсот [122] % н'ыроча похоже на опечатку
	&	
	& 	[25.4] \linebreak
		800 [25.4об] \linebreak
		800 \currentGlyphWithAffixes[2]{}{} [32.5]
		\tabularnewline \midrule
\tenevilglyph[yes][4]{i_b_s_j_c_T} 
	&	4000
	&	
	&
	&	
	& 	[4000] [16.2]
		в «4506» [46a.40]
		\tabularnewline \midrule
\tenevilglyph[yes][5]{oI_2j,J_j}
	&	5, мытԓыӈэн
	&	5 \cite{lavrov1969}
	&	мытлын'эн [мытԓыӈэн = пять], 5, пять [43] 
	&	5 \cite[л. 64]{spbfaran79}
	& 	5 \cite[360]{davydova2015a} \linebreak
		\cite[361, 364]{davydova2015a} \linebreak
		mьlьŋen [мытԓыӈэн] [ИЛИ:2.14об] \linebreak  % ilispread:2.14
		mьtlьŋça [мытԓынча = пять раз; слово напечатано] [12.22] \linebreak
		305 \currentGlyphWithAffixes[2]{kylgynqlekken}{} [4.9об]
		\tabularnewline \midrule
\tenevilglyph[yes][5]{oI_3j,J_j_j}
	&	100, мытԓыӈӄԓеккэн
	&	100 \cite{lavrov1969}
	&	мытлын'клеккэн [мытԓыӈӄԓеккэн = сто], 100, сто [44] 
	&	
	& 	\cite[361]{davydova2015a} \linebreak
		100 [34.19] \linebreak
		100 — mьtlьŋ-qlekken [мытԓыӈӄԓеккэн; слово напечатано] [12.19] \linebreak
		100 \currentGlyphWithAffixes[2]{}{} [20.1об]
		\tabularnewline \midrule
\tenevilglyph[yes][3]{oI_3j_oI_2j}
	&	[500]
	&	
	&	
	&	
	& 	~[500] [17.30]
		\tabularnewline \midrule
\tenevilglyph[yes][4]{oI_2j_2oI_jF_j}
	&	1000
	&	
	&
	&	
	& 	10013 [1013] \currentGlyphWithAffixes{}{myngytken,vyroq} [7.7об]
		\tabularnewline \midrule
\tenevilglyph[yes][4]{i_b_s_j_oI_2j,i_b_s_j_2j}
	&	5000
	&	5000 \cite{lavrov1969}
	&
	&	
	& 	~[5000] [15.9] \linebreak
		~[5000] \currentGlyphWithAffixes[2]{}{} [15.8] \linebreak
		в «5066» [46а.60]
		\tabularnewline \midrule
\tenevilglyph[yes][5]{o-_q_jF_jT-q,i-jT-q_'}
	&	6, ыннанмытԓыӈэн
	&	6 \cite{lavrov1969}
	&	ыннанмытлын'эн [ыннанмытԓыӈэн = шесть], 6, шесть [43] 
	&	6 \cite[л. 64]{spbfaran79}
	& 	6 \cite[360]{davydova2015a}\linebreak
		ьnanmьlьŋen [ыннанмытԓыӈэн] [ИЛИ:2.15] \linebreak  % ilispread:2.14
		6 [11.2об] \linebreak
		26 \currentGlyphWithAffixes[2]{qlikkin}{} [20.34об] 
		\tabularnewline \midrule
\tenevilglyph[yes][5]{o-_q_jF_jT-q_j}
	&	120
	&	
	&	120, мытлын'клеккэн кликкин, сто двадцать [45] % TODO: нужна транскрипция
	&	
	& 	120 [34.20об,57.6]
		\tabularnewline \midrule
\tenevilglyph[yes][4]{i_b_s_j_o-_q_jF_jT-q} 
	&	6000
	&	
	&
	&	
	& 	в «6302» [46а.3об]
		\tabularnewline \midrule
\tenevilglyph[yes][5]{o_j_2q,j_2q}
	&	7, ӈэръамытԓыӈэн
	&	7 \cite{lavrov1969}
	&	н'эрэкмытлын'эн [ӈэръамытԓыӈэн = семь], 7, семь [43] 
	&	7 \cite[л. 64]{spbfaran79}
	& 	7 \cite[360]{davydova2015a} \linebreak
		\currentGlyphWithAffixes[2]{}{} \cite[361]{davydova2015a} \linebreak
		7 [напечатано] \currentGlyphWithAffixes[2]{}{} [12.14] \linebreak
		7 \currentGlyphWithAffixes[2]{}{} [4.8об] \linebreak
		\cite[361]{davydova2015a} \linebreak
		ŋeriamьlьŋen [ӈэръамытԓыӈэн] [ИЛИ:2.15]  % ilispread:2.14
		\tabularnewline \midrule
\tenevilglyph[yes][5]{o_j_2q_j}
	&	140
	&	
	&	140, мытлын'клеккэн н'ирэккликкин, сто сорок [45] % TODO: нужна транскрипция
	&	
	& 	140 [2.101,57.6] 
		\tabularnewline \midrule
\tenevilglyph[yes][3]{o_j_2q_2oI_jF_j} 
	&	[1400] % по аналогии
	&	
	&
	&	
	& 	[1400] [7.26]
		\tabularnewline \midrule
\tenevilglyph[yes][3]{i_b_s_j_o_j_2q} 
	&	[7000] % по аналогии
	&	
	&
	&	
	& 	[7000] [16.4об]
		\tabularnewline \midrule
\tenevilglyph[yes][5]{o-_2q_j,jF_2j}
	&	8, амӈырооткэн
	&	8 \cite{lavrov1969}
	&	амн'ырооткэн [амӈырооткэн = восемь], 8, восемь [44] % TODO: нужна транскрипция
	&	8 \cite[л. 64]{spbfaran79}
	& 	8 \cite[360]{davydova2015a} \linebreak
		8 [напечатано] \currentGlyphWithAffixes[2]{}{} [12.14] \linebreak
		8—am-ŋьrootken [амӈырооткэн; слово напечатано] [12.17об] \linebreak
		восмое [восьмое] \currentGlyphWithAffixes{}{Q,A}[34.19]
		\tabularnewline \midrule
\tenevilglyph[yes][5]{o-_2q_j_j}
	&	160
	&	
	&	160, мытлын'клеккэн н'ырокклеккэн [45] % TODO: нужна транскрипция
	&	
	& 	в «161» [160] [32.15] \linebreak
		160 [57.6,57.50]
		\tabularnewline \midrule
\tenevilglyph[yes][3]{o-_2q_j_2oI_jF_j,jF_2j_2oI_jF_j}
	&	[1600] % по аналогии
	&	
	&	
	&	
	& 	~[1600] [7.32] \linebreak
		~[1600] \currentGlyphWithAffixes[2]{}{} [7.34] \linebreak
		\tabularnewline \midrule
\tenevilglyph[yes][4]{i_b_s_j-jF_2j} 
	&	8000
	&	
	&
	&	
	& 	в «8074» [46а.4об]
		\tabularnewline \midrule
\tenevilglyph[yes][5]{o_2q_jN_jF_o,o-o_2'}
	&	9, ӄонъачгынкэн
	&	9 \cite{lavrov1969}
	&	коначгын'кэн [ӄонъачгынкэн = девять], 9, девять [44]
	&	9 \cite[л. 64]{spbfaran79}
	& 	9 \cite[360]{davydova2015a} \linebreak
		9 quniacьnken [ӄонъачгынкэн] [23.6] \linebreak
		9 [11.2об] \linebreak
		29 \currentGlyphWithAffixes[2]{qlikkin}{} [20.35] 
		\tabularnewline \midrule
\tenevilglyph[yes][5]{o_2q_jN_jF_o_j}
	&	180
	&	
	&	180, мытлын'клеккэн н'ыракклеккэн, сто восемдесят [45] % TODO: нужна транскрипция
	&	
	& 	[180] [25.3об] \linebreak
		180 [57.6]
		\tabularnewline \midrule
\tenevilglyph[yes][4]{i_b_s_j_o_2q_jN_jF_o} 
	&	9000
	&	
	&
	&	
	& 	[9000] [17.15об] \linebreak
		в «9190» [46а.39]
		\tabularnewline \midrule
\tenevilglyph[yes][5][myngytken]{2oI_2jF,SZ}
	&	10, мынгыткэн
	&	10 \cite{lavrov1969}
	&	мынгыткэн [= десять], 10, десять [122] % TODO: нужна транскрипция
	&	10 \cite[л. 64]{spbfaran79}
	& 	10 \cite[360]{davydova2015a} \linebreak
		\cite[361, 363]{davydova2015a} \linebreak
		\cite[26]{lavrov1969} \linebreak
		10—mьngьtket [мынгыткэн; словно напечатано] [12.17об] \linebreak
		10 \currentGlyphWithAffixes[2]{}{} [11.2об] \linebreak
		30 \currentGlyphWithAffixes[2]{qlikkin}{} [20.35] \linebreak
		30 — qlikkin mьngьtken parol [ӄԓиккин мынгыткэн пароԓ = тридцать; слово напечатано] \currentGlyphWithAffixes[2]{qlikkin}{} [12.9] 
		\tabularnewline \midrule
\tenevilglyph[yes][5]{2oI_2jF_j}
	&	200, мынгытӄԓеккэн
	&	200 \cite{lavrov1969}
	&	мынгытклеккен [мынгытӄԓеккэн = двести], 200, двести [122]
	&	
	& 	[25.3об] \linebreak
		200 [57.50] \linebreak
		200 — mьngьt-qlekken [мынгытӄԓеккэн; слово напечатано] [12.19]
		\tabularnewline \midrule
\tenevilglyph[yes][4]{i_b_s_j_2oI_2jF}
	&	10000
	&	10000 \cite{lavrov1969}
	&	10000, мынгыткэн мытлынга мынгытклеккен, десять тысяч [57]
	&	
	& 	10,0001 [10001] \currentGlyphWithAffixes{}{ynnen} [57.48]
		\tabularnewline \midrule
\tenevilglyph[yes][5]{o_T_2q_2o_jF}
	&	15, кыԓгынкэн
	&	15 \cite{lavrov1969}
	&	15, кылгынкэн [кыԓгынкэн = пятнадцать] [45] % TODO: нужна транскрипция
	&	
	& 	15 \cite[360]{davydova2015a} \linebreak 
		\cite[361]{davydova2015a} \linebreak
		15 [4.8об] \linebreak
		kәlgьnken [кыԓгынкэн; слово напечатано] [12.19] \linebreak
		питнасетои [пятнадцатый, кыԓгынӄавык?] \currentGlyphWithAffixes{}{Q,A,K} [35.1] % TODO: уточнить перевод
		\tabularnewline \midrule
\tenevilglyph[yes][5][kylgynqlekken]{o_T_2q_2o_jF_j} 
	&	300
	&	
	&
	&	
	& 	[300] \cite[26]{lavrov1969} \linebreak 
		в 301, 302, 303 и т. д. [300] [4.9об]
		\tabularnewline \midrule
\tenevilglyph[yes][3]{i_b_s_j_o_T_2q_2o_jF} 
	&	[15000]
	&	
	&
	&	
	& 	[15000] [17.9,17.15об]
		\tabularnewline \midrule
\tenevilglyph[yes][4]{U-UY}
	&	вэнԓыги
	&	
	&
	&	тем не менее, wenlьgь [vænligi, вэнԓыги = тем не менее] \cite[л. 42]{spbfaran79} \linebreak % вэнԓыги
		wenlьgь [vænligi] \cite[л. 52 об]{spbfaran79} \linebreak
		силно [сильно?] \cite[л. 66 об]{spbfaran79} 
	&	\cite{bogoraz1934} \linebreak
		wlьe [vænligi, вэнԓыги] [ИЛИ:1.3об]  % ilispread:1.4
		\tabularnewline \midrule
\tenevilglyph[yes][5][jxilgyn]{UD_2c}
	&	йъиԓгын
	&	
	&	йилгын [йъиԓгын = луна, месяц], чатам [= луна, месяц], луна, месяц [62] 
	&	мэсяч [месяц] \cite[л. 66]{spbfaran79} 
	& 	\cite[362]{davydova2015a} \linebreak
		\cite[26, 28]{lavrov1969} \linebreak
		мэсыс [месяц] [34.19] \linebreak
		jelgьn [jilgьn, йъиԓгын = луна, месяц] [ИЛИ:1.12об]  % ilispread:1.13
		\tabularnewline \midrule
\tenevilglyph[yes][3]{o_8q}
	&	йынйын
	&	the sun \cite{mindalevich1934}
	&
	&	
	& 	[25.8об] \linebreak
		jьnjьn [йынйын = огонь, пламя; слово напечатано] [12.21]
		\tabularnewline \midrule
\tenevilglyph[yes][4][tirkytir]{o_7q_Q}
	&	ы'ԓёӈэт
	&	солнце \cite{lavrov1969}
	&	элон'эт [ы'ԓёӈэт = день], течение дня, ыле [ы'ԓён = день] — день, тиркытир [= солнце] — солнце [131] \linebreak
		теркир \currentGlyphWithAffixes{}{T,R,K} [130]
	&	сонсо [солнце] \cite[л. 66]{spbfaran79} 
	& 	\cite[361, 364]{davydova2015a} \linebreak
		еен [день] [34.11об, 34.16об] \linebreak
		iloŋet [ы'ԓёӈэт] [23.5об] \linebreak
		tirkьtir [тиркытир = солнце; слово напечатано] \currentGlyphWithAffixes{}{R} [7.13] \linebreak
		terke \currentGlyphWithAffixes{}{R,K,E} [ИЛИ:1.14об] \linebreak  % TODO: нужен перевод % ilispread:1.15
		terker \currentGlyphWithAffixes{}{T,R} [ИЛИ:2.5об] \linebreak  % TODO: нужен перевод % ilispread:2.5
		terker \currentGlyphWithAffixes{}{T,R,K} [ИЛИ:1.12] \linebreak  % ilispread:1.12
		terker \currentGlyphWithAffixes{}{R,K} [ИЛИ:1.14об]  % ilispread:1.15
		\tabularnewline \midrule
\tenevilglyph[yes][4]{o_7q_L}
	&
	&	
	&
	&	
	& 	утырым [утром] [34.19об]
		\tabularnewline \midrule
\tenevilglyph[yes][4]{o_7q_LE}
	&
	&	
	&	омом [= тепло, жара], чиитэв, тепло [130] 
	&	
	& 	сопло [тепло, омом] \currentGlyphWithAffixes{}{A,M} [34.8]
		\tabularnewline \midrule
\tenevilglyph[yes][3]{o_O_8qX}
	&
	&	
	&
	&	
	& 	секраиче [30.2об] \linebreak % TODO: нужна интерпретация
		ucwetьk [уучвэтык = играть] \currentGlyphWithAffixes{}{T,K} [ИЛИ:1.13] 
		\tabularnewline \midrule
\tenevilglyph[yes][3]{rI_l_b}
	&
	&	
	&
	&	топор \cite[л. 68 об]{spbfaran79} 
	& 	\cite[364]{davydova2015a} 
		\tabularnewline \midrule
\tenevilglyph[yes][3]{c_2k}
	&
	&	
	&	валын [ваԓьын = живущий], сущий [103] % похоже на ошибку
	&	
	& 	eun [эвын = и, но] [ИЛИ:2.7об]  % ilispread:2.7
		\tabularnewline \midrule
\tenevilglyph[yes][4]{c_c_2k}
	&	ы'твъэт
	&	
	&	этвет [ы'твъэт = байдара, лодка], судно, лодка [103]
	&	лодка \cite[л. 68 об]{spbfaran79} 
	& 	\cite[361]{davydova2015a} \linebreak
		әttwet [әtwьt, ы'твъэт; слово напечатано] [12.16]
		\tabularnewline \midrule
\tenevilglyph[yes][4]{C_pF_c_2k}
	&	тэвыԓьын
	&	
	&	тевылын [тэвыԓьын = матрос], гребец [103]
	
	&	
	& 	teula [тэвыԓьа = гребцы; слово напечатано] [12.16] \linebreak 
		teulьt [тэвыԓьыт = гребцы; слово напечатано] \currentGlyphWithAffixes{}{T} [12.16]
		\tabularnewline \midrule
\tenevilglyph[yes][1]{c_c_2k_o_8q}
	&	
	&	
	&	
	&	
	& 	kaçr [5.1] \linebreak
		колачак \currentGlyphWithAffixes{}{K,A,L,C} [32.15об] \linebreak
		коло параготкен \currentGlyphWithAffixes{}{kolo,P,A} [42.31] % TODO: нужен перевод, коԓё+параӄоткэн = большой корабль? рядом с изображением парусника
		\tabularnewline \midrule
\tenevilglyph[yes][3]{c_2k_o_8q}
	&	
	&	
	&	
	&	
	& 	tьleciьn [тыԓечьын = мотор, двигатель] \currentGlyphWithAffixes{}{T,C} [ИЛИ:1.14об] % ilispread:1.15
		\tabularnewline \midrule
\tenevilglyph[yes][3]{i_2j_l}
	&
	&	
	&
	&	донкои [тонкий] \cite[л. 69 об]{spbfaran79} 
	& 	[25.4] 
		\tabularnewline \midrule
\tenevilglyph[yes][4]{i_2c}
	&
	&	
	&
	&	курба [крупа] \cite[л. 68 об]{spbfaran79} 
	& 	\cite[361, 364]{davydova2015a} \linebreak
		корпаи [крупой] [29.11об] \linebreak
		в «макаро» [макарон] [29.11об]
		\tabularnewline \midrule
\tenevilglyph[yes][4]{u_2l}
	&	ӈагчыӈын
	&	
	&
	&	ŋagcьnьn [ӈагчыӈын = гора] \cite[л. 64 об]{spbfaran79} \linebreak 
		в собки [в сопки] \cite[л. 68 об]{spbfaran79}
	& 	\cite[361]{davydova2015a} \linebreak
		~[30.8об] \linebreak
		ŋagcьŋьn [ӈагчыӈын; слово напечатано] [12.23]
		\tabularnewline \midrule
\tenevilglyph[yes][1]{u_2l_3p}
	&
	&	
	&
	&	
	& 	пеколнек [34.10] % TODO: нужна интерпретация
		\tabularnewline \midrule
\tenevilglyph[yes][3]{u_2l_c_z_oF_oN}
	&
	&	
	&	посёлок Снежное [110]
	&	
	& 	\cite[364]{davydova2015a} \linebreak
		qьeleutek \currentGlyphWithAffixes{}{T,K} [26.40об] \linebreak % TODO: нужен перевод, видимо от ӄээԓивтын, то есть типа ӄээԓивтык «в Снежном»
		пасык \currentGlyphWithAffixes{}{K} [34.8] \linebreak % TODO: нужна интерпретация
		пасык \currentGlyphWithAffixes{}{P,A} [34.11об] % TODO: нужна интерпретация
		\tabularnewline \midrule
\tenevilglyph[yes][5][qymek]{uF_2l} 
	&	ӄымэк
	&	
	&	кымэк [ӄымэк = чуть-чуть, почти, едва не, чуть не], почти [110]
	&	
	& 	чочо [чуть-чуть] [30.7об] \linebreak
		сосо [чуть-чуть] [34.11] \linebreak
		гымэк [ӄымэк] [45.1об] \linebreak
		qьmek [ӄымэк] [ИЛИ:1.5]  % ilispread:1.5
		\tabularnewline \midrule
\tenevilglyph[yes][5]{i_jX}
	&	ръэнут
	&	
	&
	&	
	& 	\cite[360, 364]{davydova2015a} \linebreak
		r\=әnut [ръэнут = что, что-нибудь; слово напечатано]] [12.19об] \linebreak
		rinut [ръэнут] [ИЛИ:1.14об] \linebreak  % ilispread:1.15
		riэnut [ръэнут] [ИЛИ:1.2]  % ilispread:1.2
		\tabularnewline \midrule
\tenevilglyph[yes][5][imyrxenut]{i_jX_z}
	&	имыръэнут
	&	
	&	имырэнут [имыръэнут], всякая всячина [117] 
	&	ime renut [imь-rәnut, имыръэнут = что угодно] \cite[л. 51]{spbfaran79} % имыръэнут
	& 	\cite[364]{davydova2015a} \linebreak
		ime-r\={ә}nut [имыръэнут; слово напечатано] [12.24] \linebreak
		iэmьrienut [имыръэнут] [ИЛИ:1.8]  % ilispread:1.8
		\tabularnewline \midrule
\tenevilglyph[yes][4]{i_jX_2z}
	&	имыкээрэчьын
	&	
	&	имыкееречин [имы- = всякий, каждый; кээрэчьын = муха, всякое живое существо (кроме человека), шероховатость], все живое [117] % TODO: уточнить перевод 
	&	
	& 	\cite[28]{lavrov1969} \linebreak
		iэmьkььrecien [имыкээрэчьын] [ИЛИ:1.8]  % TODO: уточнить транскрипцию % ilispread:1.8
		\tabularnewline \midrule
\tenevilglyph[yes][4]{i_jX_z_c-l}
	&	кимитъын
	&	
	&	кимитын [кимитъын = товар, продукция], товар, груз, кладь [118]
	&	
	& 	в «канпенат» [34.10] \linebreak
		в «канпинати» [37.5] \linebreak
		kemetiьk [кимитъык] [ИЛИ:1.6об] \linebreak  % TODO: уточнить перевод % ilispread:1.7
		kemetiьn [кимитъын] [ИЛИ:2.6]  % ilispread:2.5
		\tabularnewline \midrule
\tenevilglyph[yes][4]{i_JX}
	&	гитэк
	&	
	&	лылепыткук [ԓыԓепыткук = осматриваться, поглядеть, посмотреть], посмотреть [68] \linebreak
		гитэк [= смотреть на кого-либо, обсуждать, проверять], лылепэткук [ԓыԓепыткук], посмотреть [73]
	&	 ?...gite...* \cite[л. 39 об]{spbfaran79} % TODO: нужна расшифровка, нужен перевод
	& 	\cite[362]{davydova2015a} \linebreak
		почмотре [посмотри] [30.6об] \linebreak
		nьeteqen [ныгитэӄин = смотрят] [ИЛИ:1.3] \linebreak  % ilispread:1.3
		qьete [ӄыгите = смотри] \currentGlyphWithAffixes{}{Q,E,T} [ИЛИ:2.7]  % ilispread:2.6
		\tabularnewline \midrule
\tenevilglyph[yes][4]{i_JX_o}
	&
	&	
	&
	&	 mьgitegәn [мынгитэгъэн = давайте посмотрим] \cite[л. 64 об]{spbfaran79} % Weinstein, TODO: уточнить перевод
	& 	mьngeterkьn [мынгитэркын = давайте посмотрим] [ИЛИ:1.12об] \linebreak  % TODO: уточнить перевод % ilispread:1.13
		mьngeterkьn [мынгитэркын] \currentGlyphWithAffixes{}{T,R} [ИЛИ:2.5]  % ilispread:2.4
		\tabularnewline \midrule
\tenevilglyph[yes][3]{i_JX_o-o}
	&
	&	
	&	
	&	
	& 	nьlepьrkьn [ԓыԓепыркын = смотрит] [ИЛИ:1.14об]  % TODO: уточнить перевод и транскрипцию % ilispread:1.15
		\tabularnewline \midrule
\tenevilglyph[yes][4]{U_qD}
	&
	&	
	&	панралгын [панраԓгын = камус], камус, шкура, содранная с ног оленя [78]
	&	камусы \cite[л. 37]{spbfaran79} 
	& 	\cite[362, 364]{davydova2015a} \linebreak
		камозкй [камусы] [46а.42об]
		\tabularnewline \midrule
\tenevilglyph[yes][5]{U_qD_b}
	&	ԓеԓеԓгын, ԓиԓит
	&	
	&	лилит [ԓиԓит = варежки], рукавицы [79]
	&	рукавицы \cite[л. 37]{spbfaran79} 
	& 	\cite[362]{davydova2015a} \linebreak
		lelelgьn [ԓеԓеԓгын; слово напечатано] \currentGlyphWithAffixes{}{N} [12.23об] \linebreak
		lilit [ԓиԓит; слово напечатано] \currentGlyphWithAffixes{}{T} [12.23об] \linebreak
		лилит — рукавицы [не рукой Т.] [57.23]
		\tabularnewline \midrule
\tenevilglyph[yes][5]{sE}
	&	тэвъэԓ
	&	юкола сушеная рыба \cite{lavrov1969}
	&	тевел [тэвъэԓ = юкола], юкола (сушеная рыба) [88]
	&	юукула [юкола]* \cite[л. 68 об]{spbfaran79} 
	& 	\cite[361]{davydova2015a} \linebreak
		tewel [tawal, тэвъэԓ; слово напечатано] [12.13об]
		\tabularnewline \midrule
\tenevilglyph[yes][4]{sE_jFE}
	&	пирик
	&	
	&	нийгулеткин, учиться [74] \linebreak % похоже на ошибку, TODO: нужна транскрипция
		нэнаматакэн, брал, взял \currentGlyphWithAffixes{}{P} [72] % TODO: нужен перевод
	&	всала [взяла] \cite[л. 68 об]{spbfaran79} \linebreak
		в «я \textit{восму»} \cite[л. 66]{spbfaran79}
	& 	\cite[360]{davydova2015a} \linebreak
		pirirkьn [= берет; слово напечатано] [12.23об] \linebreak % TODO: уточнить перевод, нужна транскрипция
		perejo [= взятое; слово напечатано] \currentGlyphWithAffixes{}{A} [12.23об] \linebreak
		опере [берет] \currentGlyphWithAffixes{}{P,L} [30.7] \linebreak
		nьpereiэn \currentGlyphWithAffixes{}{b,N} [ИЛИ:2.17] \linebreak  % TODO: нужен перевод % ilispread:2.16
		mьrepereŋьn \currentGlyphWithAffixes{M}{P,R} [ИЛИ:2.28об]  % ilispread:2.28
		\tabularnewline \midrule
\tenevilglyph[yes][4]{sE_jFE_qY}
	&	мигчир, мигчирэтык
	&	
	&
	&	
	& 	ырыпота [работа] \currentGlyphWithAffixes{}{N,M} [35.1] \linebreak
		рыпосе [работе] \currentGlyphWithAffixes{}{L,N} [34.18об] \linebreak
		mьceretьk [мигчирэтык = работать, трудиться] \currentGlyphWithAffixes{}{K} [ИЛИ:1.4] \linebreak  % ilispread:1.4
		mьceretьk [мигчирэтык] \currentGlyphWithAffixes{}{T,K} [ИЛИ:1.3об] \linebreak  % ilispread:1.4
		mьcer [мигчир = работа, труд] \currentGlyphWithAffixes{}{R} [ИЛИ:2.21] \linebreak  % ilispread:2.20
		nьmeьcererkьn \currentGlyphWithAffixes{}{K,N} [ИЛИ:2.12об]  % ilispread:2.12
		\tabularnewline \midrule
\tenevilglyph[yes][3]{sE_jFE_qYE}
	&	мэгчэргыргын
	&	
	&
	&	
	& 	meьcergьrgьn [мэгчэргыргын = работа (нетипичное словообразование)] [ИЛИ:1.5об]  % МП TODO: уточнить перевод % ilispread:1.6
		\tabularnewline \midrule
\tenevilglyph[yes][3]{w_j}
	&
	&	
	&
	&	оцин [очень] \cite[л. 66]{spbfaran79} \linebreak
		в «я \textit{оцин} боюс», «я \textit{оцин} писпокоюс» \cite[л.66]{spbfaran79}
	& 	\cite[364]{davydova2015a} 
		\tabularnewline \midrule
\tenevilglyph[yes][4]{w_j_'}
	&	йъарат
	&	
	&
	&	
	& 	jarat [йъарат = очень, весьма, слишком, особенно; слово напечатано] [12.19об] \linebreak
		jarat [йъарат] [ИЛИ:1.6, ИЛИ:1.18] \linebreak   % ilispread:1.6, ilispread:1.18
		ijarat [йъарат] [ИЛИ:1.5об]  % NB: обрати внимание на i опять % ilispread:1.6
		\tabularnewline \midrule
\tenevilglyph[yes][5][kynmal]{i-q-JFX}
	&	кынмаԓ
	&	
	&	кынмал [кынмаԓ = быть вместе], вместе [36] % кынмаԓ
	&	вместе \cite[л. 55]{spbfaran79} 
	& 	\cite[360, 364]{davydova2015a} \linebreak
		kьnmal [кынмаԓ; слово напечатано] [12.22] \linebreak
		kьnmal [кынмаԓ] [ИЛИ:2.13] \linebreak  % ilispread:2.12
		am-kьnmal [амкынмаԓ = совместно; слово напечатано] \currentGlyphWithAffixes{A,M}{} [12.16]
		\tabularnewline \midrule
\tenevilglyph[yes][3]{i-q-JFX_2q}
	&	
	&	
	&	
	&	
	& 	reen [рээн = вместе с; слово напечатано] \currentGlyphWithAffixes{}{N} [12.12об,12.22]
		\tabularnewline \midrule
\tenevilglyph[yes][5]{B_2q} 
	&	
	&	
	&	винрэтык [= помогать], помогать [36] % зеркально
	&	
	&	помокаи [помогай] [30.8] \linebreak % зеркально
		nuwinretqin [нывинрэтӄин = помогает; слово напечатано] [12.17] \linebreak
		nuwinretqinet [нывинрэтӄинэт = помогают; слово напечатано] [12.18об] \linebreak
		winretьlьt [винрэтыԓьыт = помогающий; слово напечатано] \currentGlyphWithAffixes{}{T} [12.15об] \linebreak % зеркально
		mьnwrerkьn \currentGlyphWithAffixes{M}{} [ИЛИ:1.13]  % TODO: нужен перевод % ilispread:1.13
		\tabularnewline \midrule 
\tenevilglyph[yes][3]{SFE_jF}
	&	
	&	
	&	огрор [орвоор = нарта], нарта [89]
	&	нарта \cite[л. 68]{spbfaran79} 
	& 	\cite[364]{davydova2015a} \linebreak
		нарта [не рукой Т.] [57. 22]
		\tabularnewline \midrule
\tenevilglyph[yes][3]{SFE_jF_p}
	&	
	&	
	&	
	&	
	& 	jaacena [яаченаӈ = последняя грузовая нарта] \currentGlyphWithAffixes{}{mooqor} [ИЛИ:2. 10] 
		\tabularnewline \midrule
\tenevilglyph[yes][1]{SFE_jF_cF}
	&	
	&	
	&	
	&	
	& 	teneŋeŋkь \currentGlyphWithAffixes{}{T} [ИЛИ:1. 3] % TODO: нужен перевод
		\tabularnewline \midrule
\tenevilglyph[yes][4][ewyr]{CFE_q}
	&	эвыр
	&	
	&	эвыр [= если], если, и [88]
	&
	& 	\cite[360, 361, 364]{davydova2015a} \linebreak
		eur [эвыр; слово напечатано] [12.18об] \linebreak
		эеур [эвыр] [45.6об] \linebreak
		eur [эвыр] [ИЛИ:1.4об]  % ilispread:1.5
		\tabularnewline \midrule
\tenevilglyph[yes][5]{O_L_q}
	&	ёо
	&	
	&	ёо [= пурга], чеченёо , непогода, холодный ветер [47] % TODO: нужна транскрипция
	&	
	& 	cacajokь [чьэчеӈэюк = начало зимы] [ИЛИ:1.19] \linebreak  % ilispread:1.19
		joo [ёо] [ИЛИ:1.4об] \linebreak  % ilispread:1.5
		сiесенкы [чьэчеӈкы = во время мороза] \currentGlyphWithAffixes{}{C,C,K} [34.8] \linebreak
		холото [холодно] \currentGlyphWithAffixes{}{C,C,K} [34.8] \linebreak
		веесыр [ветер] \currentGlyphWithAffixes{}{K,T,A,R} [34.11] \linebreak % TODO: уточнить интерпретацию, найти чукотское слово
		тоеот [дует] \currentGlyphWithAffixes{}{K,T} [34.11] \linebreak 
		\u{е}о\.{и}а \currentGlyphWithAffixes{}{A} [45.1об] \linebreak % TODO: нужен перевод
		juien \currentGlyphWithAffixes{}{N} [ИЛИ:1.8] \linebreak  % ilispread:1.8
		jojien \currentGlyphWithAffixes{}{b,N} [ИЛИ:2.3] \linebreak  % ilispread:2.2
		ciэceŋ [чьэчеӈ = холод, мороз] \currentGlyphWithAffixes{}{C,C} [ИЛИ:2.21] \linebreak  % ilispread:2.20
		ciceŋkь [чьэчеӈкы] \currentGlyphWithAffixes{}{C,C,K} [ИЛИ:2.20об] \linebreak  % ilispread:2.20
		cicŋken [чьэчеӈкин = морозный] \currentGlyphWithAffixes{}{C,C,K} [ИЛИ:2.20об] \linebreak  % ilispread:2.20
		ciacajokь [чьэчеӈэюк] \currentGlyphWithAffixes{}{C,A,C,K} [ИЛИ:2.20об]  % ilispread:2.20
		\tabularnewline \midrule
\tenevilglyph[yes][4]{O_L_l}
	&	чьэчеӈкы
	&	
	&	
	&	
	& 	cieceŋkь [чьэчеӈкы = во время мороза] [ИЛИ:1.19]  % ilispread:1.19
		\tabularnewline \midrule
\tenevilglyph[yes][5]{O_L_qE}
	&	иԓииԓ
	&	
	&	элыёо, ветер со снегом [47] % TODO: нужна транскрипция
	&	доз [дождь] \cite[л. 68]{spbfaran79} 
	& 	\cite[361, 364]{davydova2015a} \linebreak
		ilel [иԓииԓ = дождь] [ИЛИ:2.9] \linebreak  % ilispread:2.8
		ŋlgqn [5.1об] \linebreak % TODO: нужен перевод
		эlepь [эԓепы = от дождя] \currentGlyphWithAffixes{}{P} [ИЛИ:2.4об]  % ilispread:2.4
		 % peŋepe [пиӈэпиӈ = падающий снег] \currentGlyphWithAffixes{}{P} [??] % TODO: найти источник, было ilispread:2.4
		\tabularnewline \midrule
\tenevilglyph[yes][3]{O_L_2q}
	&
	&	холодный ветер (в~тексте) \cite{lavrov1969}
	&
	&	холот [холод] \cite[л. 66]{spbfaran79} 
	& 	 \cite[26]{lavrov1969} 
		\tabularnewline \midrule
\tenevilglyph[no][3]{O_L}
	&
	&	
	&
	&	бурка [пурга] \cite[л. 68 об]{spbfaran79} 
	& 	 \tabularnewline \midrule
\tenevilglyph[yes][3]{O_L_q_C}
	&
	&	
	&
	&	
	& 	ajgьsqanqatken [= северяне; слово напечатано] [12.25] \linebreak % TODO: уточнить перевод, нужна транскрипция
		ajgьsqanqatken [слово напечатано] [14.1]
		\tabularnewline \midrule
\tenevilglyph[yes][3]{O_LE}
	&	пиӈэпиӈ
	&	
	&
	&	
	& 	peŋapь [пиӈэпи(ӈ) = падающий снег, вьюга] \currentGlyphWithAffixes{}{P} [ИЛИ:2.4об]  % TODO: уточнить перевод % ilispread:2.4
		\tabularnewline \midrule
\tenevilglyph[yes][5]{i_SX}
	&	аԓымы
	&	
	&	alьmь(ŋ) (союз), положим, что… [35] 
	&	alьmь [аԓымы = положим, что] \cite[л. 52 об]{spbfaran79} % аԓымы?
	& 	\cite[361, 364]{davydova2015a} \linebreak
		алымы [аԓымы] [41.7] \linebreak
		alьmь [аԓымы] [ИЛИ:1.4,ИЛИ:2.24об]   % ilispread:1.4, ilispread:2.24
		\tabularnewline \midrule
\tenevilglyph[yes][5]{2C_2c} 
	&	мимыԓ
	&	вода \cite{lavrov1969}
	&	мимыл [мимыԓ], вода [100] \linebreak 
		мимылкыин, водный \currentGlyphWithAffixes{}{E} [100]
	&	
	& 	\cite[364]{davydova2015a} \linebreak 
		\cite[26, 28]{lavrov1969} \linebreak
		вотой [водой] [32.15об] \linebreak
		вота [вода] [30.6] \linebreak
		memьl [mimьl, мимыԓ] [ИЛИ:2.9]  % ilispread:2.8
		\tabularnewline \midrule
\tenevilglyph[yes][3]{2C_2c_I} 
	&	гычормыеквэ
	&	
	&	
	&	
	& 	gьçormьjkwe [гычормыеквэ = вдоль берега; слово напечатано] [12.22] % TODO: уточнить транскрипцию
		\tabularnewline \midrule
\tenevilglyph[yes][4]{2C_2c_q} 
	&	аӈӄы, аӈӄэпы
	&	
	&	ан'кы [аӈӄы = море], море [101] \linebreak
		ан'какэн [аӈӄакэн = морской], морской \currentGlyphWithAffixes{}{E} [100]
	&	
	& 	aŋqajpu [аӈӄэпы = из моря; слово напечатано] [12.24 об] \linebreak % TODO: уточнить перевод, уточнить транскрипцию
		aŋqagtь [аӈӄагты = к морю; слово напечатано] \currentGlyphWithAffixes{}{T} [12.18] \linebreak
		aŋqaken [аӈӄакэн; слово напечатано] \currentGlyphWithAffixes{}{K,N} [19.11об]
		\tabularnewline \midrule
\tenevilglyph[yes][3]{2C_2c_q_z} 
	&	эвэнэгыргын
	&	
	&	
	&	
	& 	әweneьrgьn [эвэнэгыргын = охота на морского зверя; слово напечатано] [12.25] % TODO: уточнить перевод, уточнить транскрипцию
		\tabularnewline \midrule
\tenevilglyph[yes][5]{2kU_2QY} 
	&	ы'ԓьыԓ
	&	снег \cite{lavrov1969}
	&	ыльыль [ы'ԓьыԓ], лежащий снег [7] 
	&	
	& 	\cite[361, 364]{davydova2015a} \linebreak
		ычнек [снег] [30.6] \linebreak
		\.{и}л\.{и}л [ы'ԓьыԓ] [45.1об] \linebreak
		iliьl [ы'ԓьыԓ] [ИЛИ:2.7об] \linebreak  % ilispread:2.7
		iliьlgьpь [ы'ԓьыԓгыпы = из снега] \currentGlyphWithAffixes{}{P} [ИЛИ:2.4об]  % TODO: уточнить транскрипцию и перевод % ilispread:2.4
		\tabularnewline \midrule
\tenevilglyph[yes][3]{U_ux} 
	&
	&	
	&
	&	витал [видал] \cite[л. 67 об, 68 об]{spbfaran79}
	& 	\cite[360, 364]{davydova2015a} \linebreak
		lūәn [льугъэн = видел; слово напечатано][12.25] \linebreak % TODO: уточнить перевод, нужна транскрипция
		вечым [видим] [30.6об] \linebreak % TODO: уточнить интерпретацию
		neneliuqen \currentGlyphWithAffixes{}{N} [ИЛИ:1.7об] \linebreak  % TODO: нужен перевод % ilispread:1.8
		nьrelioŋьrkьnen \currentGlyphWithAffixes{}{N,R,L} [ИЛИ:2.4] \linebreak  % TODO: нужен перевод % ilispread:2.3
		nьrelioŋьrkьnen \currentGlyphWithAffixes{}{R,L,W} [ИЛИ:2.4об]  % TODO: нужен перевод % ilispread:2.4
		\tabularnewline \midrule
\tenevilglyph[no][3]{U_ux_j} % TODO: вероятно, просто отрицательная частица добавлена, можно грохнуть будет
	&
	&	
	&
	&	нивидал [не видал] \cite[л. 66 об]{spbfaran79}
	& 	\tabularnewline \midrule
\tenevilglyph[yes][3]{V_2l_i_2q} 
	&
	&	
	&	нъомрыкэн [нъомрыӄэн = крепкий], крепкий, прочный (о мехе) [30] 
	&	крепкои [крепкий] \cite[л. 69 об]{spbfaran79}
	& 	\cite[28]{lavrov1969} 
		\tabularnewline \midrule
\tenevilglyph[no][3]{V_l_lU_i_q_qU} 
	&
	&	
	&
	&	нирепкои [некрепкий] \cite[л. 69 об]{spbfaran79}
	& 	\tabularnewline \midrule
\tenevilglyph[yes][4]{v_i_2CX} 
	&
	&	приходить, приезжать \cite{lavrov1969}
	&	пикирык [пыкирык = по приходу], прихожу, приходит [30]
	&	
	& 	\cite[360]{davydova2015a} \linebreak
		\cite[26]{lavrov1969} \linebreak
		прлехалй [приехали] [32.13об] \linebreak
		преехали [приехали] [37.5] \linebreak
		pьkerie [пыкиргъи = пришел] [ИЛИ:1.17] \linebreak  % ilispread:1.17
		niьmkerьrkьn \currentGlyphWithAffixes{}{R} [ИЛИ:1.18об] \linebreak  % TODO: нужен перевод % ilispread:1.19
		pьkerьk [пыкирык] \currentGlyphWithAffixes{}{K} [ИЛИ:1.16об]  % ilispread:1.17
		\tabularnewline \midrule
\tenevilglyph[yes][1]{v_i_2CX_2q} 
	&
	&	
	&	
	&	
	& 	turpkre [ИЛИ:1.2]  % TODO: нужен перевод; видимо, что-то вроде "снова приходит" % ilispread:1.2
		\tabularnewline \midrule
\tenevilglyph[yes][5]{i_i_bX} 
	&	ӈъочьын
	& 	богатый \cite{bogoraz1934} % явная ошибка
	&	н'ъочын [ӈъочьын = бедняк], н'ъочыян, бедный [10]
	&	ŋocьm [ŋocьn, ӈъочьын = бедняк] \cite[л. 39 об]{spbfaran79} % ӈъочьын
	& 	петнаска [бедняжка] [34.8об] \linebreak
		петнак [бедняк] [30.3об] \linebreak
		ŋiociьn [ӈъочьын] [ИЛИ:2.14об] \linebreak  % ilispread:2.14
		ŋiocien [ӈъочьын] \currentGlyphWithAffixes{}{N} [ИЛИ:2.17об] \linebreak  % ilispread:2.17
		ŋiociьt [ӈъочьыт = бедняки] \currentGlyphWithAffixes{}{T} [ИЛИ:1.12] \linebreak  % ilispread:1.12
		ŋiocia [ӈъочьа = бедняк] \currentGlyphWithAffixes{}{A} [ИЛИ:2.17об] \linebreak  % ilispread:2.17
		aŋiociьka \currentGlyphWithAffixes{}{K,A} [ИЛИ:2.15об] \linebreak  % TODO: нужен перевод % ilispread:2.15
		\tabularnewline \midrule
\tenevilglyph[yes][5]{oEN_q} 
	&	гаймычьыԓьын
	& 	бедный \cite{bogoraz1934} % явная ошибка
	&	гаймычыллын [гаймычьыԓьын], богатый [95]
	&	goymьcьl(?) [gajmьcьjьn, гаймычьыԓьын = богач] \cite[л. 39 об]{spbfaran79} % гаймычьыԓьын
	& 	ьgamьciьliьn [гаймычьыԓьын] [ИЛИ:2.6об] \linebreak  % ilispread:2.6
		gamьciьlien [гаймычьыԓьын] \currentGlyphWithAffixes{}{b,N} [ИЛИ:2.17об] \linebreak  % ilispread:2.17
		\tabularnewline \midrule
\tenevilglyph[yes][4]{2i_2iX_4q} 
	&
	&	
	&	qeetlin [гэетԓин = приходил] % МП
	&	прсол [пришел] \cite[л. 68 об]{spbfaran79}
	& 	\cite[361]{davydova2015a} \linebreak
		geetlin [гэетԓин; слово напечатано] [12.19об] 
		еееот [30.6] \linebreak % TODO: нужна интерпретация
		пырыеехали [приехали] [30.6об] \linebreak
		eetie [ИЛИ:1.17об]  % ilispread:1.18
		\tabularnewline \midrule
\tenevilglyph[yes][3]{2i_iX_2q_cF_jF} 
	&
	&	
	&
	&	прныси [принеси] \cite[л. 68 об]{spbfaran79}
	& 	[4.3об] \linebreak
		qrtgьn [ӄырэтгын = принеси] [ИЛИ:1.3об]  % ilispread:1.4
		\tabularnewline \midrule
\tenevilglyph[yes][5]{i_CD} 
	&	тэԓпык
	&	
	&	тэльпыйе [тэԓпыгъи = закончился], кончился, тэльпык [тэԓпык = кончаться (завершаться), пройти (о месяце)], кончаться [54] 
	&	
	& 	telpьie [тэԓпыгъи] [ИЛИ:1.23] \linebreak  % ilispread:1.24
		tэlpьken \currentGlyphWithAffixes{}{K,N} [23.6] \linebreak % TODO: нужен перевод
		nьtelpьqen [нытэԓпыӄин = кончается] \currentGlyphWithAffixes{}{N,T,P} [ИЛИ:2.8] \linebreak  % ilispread:2.7
		telpьrkьt \currentGlyphWithAffixes{}{K,T} [ИЛИ:2.16об]  % TODO: нужен перевод % ilispread:2.16
		\tabularnewline \midrule
\tenevilglyph[yes][5]{i_CD_2jF} 
	&	ытръэч
	&	
	&
	&	длко [только] \cite[л. 68]{spbfaran79}
	& 	\cite[364]{davydova2015a} \linebreak
		толко [только] [34.11об] \linebreak
		ытрэч [ытръэч = только, всё (конец)] [42.3] \linebreak
		ьriэc [әrræc, ытръэч] [ИЛИ:1.6] \linebreak  % ilispread:1.6
		ьriecken [ытръэчкин = единственный] \currentGlyphWithAffixes{}{K,N} [ИЛИ:2.27об]  % ilispread:2.27
		\tabularnewline \midrule
\tenevilglyph[yes][5]{uD_jN} 
	&	гатԓе
	&	
	&	гатля [гатԓе = птица, утка, промысловая водяная птица], птица [89]
	&	кус [гусь] \cite[л. 66]{spbfaran79}
	& 	\cite[28]{lavrov1969} \linebreak
		gatlь [гатԓе; слово напечатано] [12.17об] \linebreak
		nerkuqьt [нэрӄуӄыт = лебеди; слово напечатано] \currentGlyphWithAffixes{}{nilgyqin} [12.17об]
		\tabularnewline \midrule
\tenevilglyph[yes][3]{i_u_uD} 
	&	% э'йӈэк
	&	
	&	
	&	
	& 	ejŋei [= крикнул; слово напечатано] [12.22об] % TODO: уточнить перевод, явно от ejŋæ = хрипеть, нужна транскрипция
		\tabularnewline \midrule
\tenevilglyph[yes][5]{i_u_uD_b} 
	&	грэп
	&	
	&	греп [грэп], песня [125]
	&	grep [græp, грэп = песня] \cite[л. 64 об]{spbfaran79} % грэп
	& 	grep [грэп] [12.23; слово напечатано] \linebreak
		поеот [поёт] \currentGlyphWithAffixes{}{N,T} [36.1]
		\tabularnewline \midrule
\tenevilglyph[yes][4]{i_u_uD_k_r} 
	&
	&	
	&
	&	
	& 	кармоска [гармошка] [36.1] \linebreak
		кармочка [гармошка] [31.1]
		\tabularnewline \midrule
\tenevilglyph[yes][3]{i_u_uD_k} 
	&
	&	
	&
	&	
	& 	[1.42об] \linebreak
		летара [гитара] [31.1] \currentGlyphWithAffixes{}{L,T,A} \linebreak
		маталина [мандалина] \currentGlyphWithAffixes{}{M,A,T} [31.1]
		\tabularnewline \midrule
\tenevilglyph[yes][5]{oF_oN_z} 
	&	ԓевыт
	&	
	&
	&	колова [голова] \cite[л. 68]{spbfaran79}
	& 	\cite[364]{davydova2015a} \linebreak
		leut [læwt, ԓевыт = голова; слово напечатано] [12.12об] \linebreak
		leut [ԓевыт] [ИЛИ:1.8об] \linebreak  % ilispread:1.9
		leute \currentGlyphWithAffixes{}{T} [ИЛИ:2.5]  % TODO: нужен перевод % ilispread:2.4
		\tabularnewline \midrule
\tenevilglyph[yes][4]{o_jN_m} 
	&
	&	
	&	яранга, чукотское жилище [106]
	&	
	& 	\cite[363,364]{davydova2015a} \linebreak
		~[рядом с изображением яранги] [12.9] \linebreak
		яранка [яранга] [29.2об] \linebreak
		ярнка [яранга] [29.2об] \linebreak
		еярнка [яранга] [29.2] 
		\tabularnewline \midrule
\tenevilglyph[yes][4]{o_lN_l} % значок почти наверняка другой, но в чем разница — непонятно
	&	яраӈы
	&	
	&	калеткоран [каԓеткоран = школа], школа \currentGlyphWithAffixes{}{kelikel} [107]
	&	
	& 	jaraŋь [яраӈы = яранга] [ИЛИ:1.12] \linebreak  % ilispread:1.12
		jarapь [ярайпы = из яранги] \currentGlyphWithAffixes{}{P} [ИЛИ:1.19] \linebreak  % ilispread:1.19
		kaletkoragtь [каԓеткорагты = в школу; слово напечатано] \currentGlyphWithAffixes{}{kelikel} [12.18] \linebreak 
		kaleran \currentGlyphWithAffixes{}{kelikel} [ИЛИ:2.21об] \linebreak  % TODO: нужен перевод % ilispread:2.21
		velьtkorak [вэԓы ткоран = магазин, лавка; слово напечатано] \currentGlyphWithAffixes{}{wilytkuk} [12.15] \linebreak
		столова [столовая] \currentGlyphWithAffixes{ruk}{Q,A,K,N} [34.16об]
		\tabularnewline \midrule
\tenevilglyph[yes][3]{o_jN_m_z} 
	&
	&	
	&
	&	домои [домой] \cite[л. 66 об]{spbfaran79}
	& 	\cite[363]{davydova2015a} 
		\tabularnewline \midrule
\tenevilglyph[yes][3]{o_lN_l_2jF}
	&	тытыԓ
	&	
	&	% яракача [?], около дома [107] % TODO: нужна транскрипция; больше похоже на другой знак со 106, но там крючок сверху другой
	&	
	& 	tьttьl [тытыԓ = дверь, вход; слово напечатано] [14.10] \linebreak
		товерес [37.7об] % TODO: нужна интерпретация
		\tabularnewline \midrule
\tenevilglyph[yes][3]{o_lN_l_c_2k}
	&	авынраԓьын
	&	
	&	аунралын [авынраԓьын = хозяин, дух, владеющий местностью], хозяин (главный домашний) [107] 
	&	
	& 	aunrala [авынраԓьо = хозяин; слово напечатано] [12.17] % TODO: уточнить перевод, уточнить транскрипцию
		\tabularnewline \midrule
\tenevilglyph[yes][5]{iE_b_i} 
	&	тъэркин
	&	
	&	мъэркин, киткит [= немного, чуть-чуть], мало [56] % TODO: нужна транскрипция
	&	мало \cite[л. 67]{spbfaran79}
	& 	\cite[361]{davydova2015a} \linebreak
		мало [29.11об, 30.6] \linebreak
		tierken [тъэркин = мало, немного, недостаточно] [ИЛИ:2.17] \linebreak  % ilispread:2.16
		tiэrkenet [тъэркинэт = мало (мн. ч.)] \currentGlyphWithAffixes{}{N,T} [ИЛИ:2.28]  % ilispread:2.27
		\tabularnewline \midrule
\tenevilglyph[yes][4]{iE_b_i_jL} 
	&
	&	
	&
	&	
	& 	немночко [немножко] [30.7об]
		\tabularnewline \midrule
\tenevilglyph[yes][3]{iE_b_i_jR} 
	&	тъэр
	&	
	&	тъэр [сколько], сколько [56]
	&	
	& 	ter [tær, тъэр; слово напечатано] [12.16]
		\tabularnewline \midrule
\tenevilglyph[yes][4]{iE-q_b_i} 
	&	тъэръэв
	&	
	&	
	&	
	& 	tereu [tæræw, тъэръэв = недостаточно, мало; слово напечатано] [12.16] \linebreak
		мало [34.12]
		\tabularnewline \midrule
\tenevilglyph[yes][4]{jF_b_q} 
	&
	&	
	&	кет'и [ӄытгъи = пошел], пошел, кетыркын — уходит [85]
	&	посла [пошла] \cite[л. 66]{spbfaran79}
	& 	\cite[360]{davydova2015a} \linebreak
		qәtji [ӄытгъи = пошел; слово напечатано] [12.22] \linebreak
		оеехау [уехал] [30.6об] \linebreak
		tьrelqьtь [трэԓӄыты = пойду] \currentGlyphWithAffixes{}{T,R} [ИЛИ:1.9об]  % Weinstein TODO: уточнить перевод % ilispread:1.10
		\tabularnewline \midrule
\tenevilglyph[yes][3]{jF_b_q_2q} 
	&
	&	
	&	тылама [тыԓяма = по дороге], по дороге, дорогой \currentGlyphWithAffixes{}{A}, \currentGlyphWithAffixes{}{T,K} [85]
	&	слы [шли] \cite[л. 68]{spbfaran79} \linebreak
		пёс [вёз] \cite[л. 66 об]{spbfaran79}
	& 	\cite[360]{davydova2015a} \linebreak
		tьlerkьt [тыԓеркыт = движутся; слово напечатано] [7.12об] \linebreak 
		tьlerkьt [тыԓеркыт; слово напечатано] \currentGlyphWithAffixes{}{T} [12.12об] \linebreak
		tьlama [тыԓяма; слово напечатано] \currentGlyphWithAffixes{}{A} [12.10об]
		\tabularnewline \midrule
\tenevilglyph[yes][3]{jF_b_q_L_uD} 
	&
	&	
	&	
	&	
	& 	iотихаи [отдыхай] \currentGlyphWithAffixes{}{P,A} [35.3об] \linebreak
		paŋieuŋьtoken [паӈъэвӈытокэн = отдохнуть] \currentGlyphWithAffixes{}{T,A,K} [ИЛИ:2.26] \linebreak  % Weinstein TODO: уточнить переводы % ilispread:2.25
		apaŋiuŋьtoka [апаӈъэвӈытока = без отдыха] \currentGlyphWithAffixes{}{K,A} [ИЛИ:2.25об] \linebreak  % ilispread:2.25
		paŋiuŋьtok [паӈъэвӈыток = отдых] \currentGlyphWithAffixes{}{T,K} [ИЛИ:2.27]  % ilispread:2.26
		\tabularnewline \midrule
\tenevilglyph[yes][3]{jF_b_2q} 
	&
	&	
	&	
	&	
	& 	genletetlin [гэнԓьэтэтԓин = его унесли; слово напечатано] [12.18] \linebreak 
		nerlieterkьn \currentGlyphWithAffixes{}{R} [ИЛИ:1.10]  % ilispread:1.10
		\tabularnewline \midrule
\tenevilglyph[yes][3]{jF_b_q-z-q} 
	&
	&	
	&	
	&	
	& 	penrьnen [пэнрынэн = бросился; слово напечатано] [12.20] \linebreak 
		penrьnen [слово напечатано] \currentGlyphWithAffixes{P}{} [14.20]
		\tabularnewline \midrule
\tenevilglyph[yes][3]{jF_b_q_o} 
	&
	&	
	&	тыйоан [тыйъогъан = я настиг его], я настиг, ёок, настигать [86] % TODO: нужен перевод
	&	
	& 	tьj\=oan [тыйъогъан; слово напечатано] [12.19об] 
		\tabularnewline \midrule
\tenevilglyph[yes][1]{jF_b_q_jF_iX} 
	&
	&	
	&	
	&	
	& 	qьrьmnilqььrkьnet [ИЛИ:1.18об]  % TODO: нужен перевод % ilispread:1.19
		\tabularnewline \midrule
\tenevilglyph[yes][4]{i_2j_2cY} 
	&	тиԓмытиԓ
	&	орел* \cite{lavrov1969}
	&	тилмытил [тиԓмытиԓ = орел, орлан], орел [98]
	&	
	& 	\cite[28]{lavrov1969} \linebreak
		tilme [тиԓмэ = орел; слово напечатано] [7.13, 12.13] \linebreak
		~[рядом с рисунком, изображающим орла] [56.1]
		\tabularnewline \midrule
\tenevilglyph[yes][3]{i_j_cY_s} 
	&	равыԓьэӈ
	&	
	&	
	&	
	& 	raulьŋ [rauleŋ, равыԓьэӈ = белка; слово напечатано] [12.19об] 
		\tabularnewline \midrule
\tenevilglyph[yes][4]{C-C_q_j} 
	&	вэтԓы
	&	ворон \cite{lavrov1969}
	&	вэтлы [вэтԓы = ваԓвыйӈын = ворон], ворон [23]
	&	
	& 	[25.13] \linebreak
		wetlь [vætlь, вэтԓы; слово напечатано] [12.13об] \linebreak
		walvьjŋьn [ваԓвыйӈын = ворон; слово напечатано] [12.22об]
		\tabularnewline \midrule
\tenevilglyph[yes][5]{CD-CDX} 
	&	айвэ
	&	
	&	айвэ [= вчера], вчера [34]
	&	недавно \cite[л. 50]{spbfaran79} \linebreak % значок только умеренно похож
		цирас [вчера] \cite[л. 67 об]{spbfaran79} \linebreak
		в «провсерас» [позавчера, рымагтайвэ] \cite[л. 67 об]{spbfaran79} % слово из двух знаков, и если один это «айвэ», а второй переводится как «дальше», то «рымагтайвэ» выглядит очень праводподобно
	& 	[25.4об] \linebreak
		ajwә [ajvә, айвэ; слово напечатано] [12.19] \linebreak
		черач [вчера] [30.6об]
		\tabularnewline \midrule
\tenevilglyph[yes][1]{CD-CDX_l} 
	&
	&	
	&
	&	ониметнис \cite[л. 66 об]{spbfaran79} % TODO: нужна интерпретация
	& 	\cite[364]{davydova2015a} 
		\tabularnewline \midrule
\tenevilglyph[yes][3]{CD-CDX_2q} 
	&
	&	
	&
	&	прослокот [прошлый год] \cite[л. 66 об]{spbfaran79}
	& 	[25.4] \linebreak
		kьtoor [кытоор = когда-то давно] \currentGlyphWithAffixes{}{T,A,R} [ИЛИ:1.15]  % ilispread:1.15
		\tabularnewline \midrule
\tenevilglyph[yes][3]{CD-CDX_q_2b_c} 
	&
	&	
	&
	&	
	& 	кота [когда] [37.2] 
		\tabularnewline \midrule
\tenevilglyph[yes][4]{i_b_qY} 
	&	тэгъеӈу
	&	
	&
	&	нусно [нужно] \cite[л. 66]{spbfaran79} \linebreak
		в «понравилас» \cite[л. 66]{spbfaran79} % TODO: нужна интерпретация, выписать полностью
	& 	[25.7] \linebreak
		teggeŋu [тэгъеӈу = хотеть, желать; слово напечатано] [12.19об] \linebreak 
		teьieŋu [тэгъеӈу] [68.34об] \linebreak 
		nьteьjŋьrkьn \currentGlyphWithAffixes{N}{T} [ИЛИ:1.12об] \linebreak  % TODO: нужен перевод % ilispread:1.13
		nьteьeŋьrkьn \currentGlyphWithAffixes{N}{T} [ИЛИ:1.12об]  % TODO: нужен перевод % ilispread:1.13
		\tabularnewline \midrule
\tenevilglyph[yes][1]{3k} 
	&
	&	
	&
	&	в «понравилас» \cite[л. 66]{spbfaran79} % TODO: нужна интерпретация, выписать полностью
	& 	\cite[364]{davydova2015a} \linebreak
		tlьrkьn [тыԓеркын] [68.34об] \linebreak % TODO: нужен перевод
		lьŋьtkь [слово напечатано] [12.19об] \linebreak % TODO: нужен перевод, нужна транскрипция
		lьŋьrkьnen [ԓыӈыркынин] \currentGlyphWithAffixes{}{R,N} [ИЛИ:1.8об] \linebreak  % TODO: нужен перевод % ilispread:1.9
		iьnьlgьrkьn [ы'ныԓгыркын] \currentGlyphWithAffixes{}{b,N} [ИЛИ:2.5об]  % TODO: нужен перевод % ilispread:2.5
		\tabularnewline \midrule
\tenevilglyph[yes][5]{i_j_3b} 
	&	мэӄым
	&	
	&	мэкым [мэӄым = стрела], рынныкон [= стрела с костяным наконечником], стрела [114]
	&	паска патро \cite[л. 68 об]{spbfaran79} % TODO: нужна интерпретация
	& 	\cite[364]{davydova2015a} \linebreak
		mәqьm [mæqьm, мэӄым; слово напечатано] [12.19об] \linebreak
		m\=әmit [мъэмит = стрелы; слово напечатано] \currentGlyphWithAffixes{}{T} [12.19об]
		\tabularnewline \midrule
\tenevilglyph[yes][1]{jY_3b} 
	&
	&	
	&	
	&	
	& 	\cite[364]{davydova2015a} \linebreak
		lьe [ИЛИ:2.12] \linebreak  % TODO: нужен перевод % ilispread:2.11
		lьeэ [ИЛИ:1.8об]  % ilispread:1.9
		\tabularnewline \midrule
\tenevilglyph[yes][4]{u_q_l} 
	&	ыяа
	&	
	&
	&	талок [далеко] \cite[л. 68 об]{spbfaran79}
	& 	\cite[360, 364]{davydova2015a} \linebreak
		\cite[28]{lavrov1969} \linebreak
		ejaa [jaa, ыяа = далеко, вдали] [ИЛИ:2.21об] \linebreak  % ilispread:2.21
		ejaaьt \currentGlyphWithAffixes{}{T} [ИЛИ:2.22]  % TODO: нужен перевод % ilispread:2.21
		\tabularnewline \midrule
\tenevilglyph[yes][3]{2cD_jY} 
	&
	&	вьюга (в~тексте) \cite{lavrov1969}
	&
	&	илплыл \cite[л. 68]{spbfaran79} % зеркально * [ы'ԓьыԓ = снег]?
	& 	\cite[361]{davydova2015a} \linebreak
		\cite[26]{lavrov1969} 
		\tabularnewline \midrule
\tenevilglyph[yes][4]{u_2j} 
	&
	&	
	&	галяк [гаԓяк = миновать, проходить], пройти, проехать [77]
	&	прошол [прошел] \cite[л. 66 об]{spbfaran79} % зеркально
	& 	[25.4] \linebreak
		ьalj [5.1] % TODO: нужен перевод, нужна транскрипция
		\tabularnewline \midrule
\tenevilglyph[yes][3]{U_4j} 
	&	
	&	
	&	этъылкылин [этъыԓкыԓьин = здоровый, не больной], здоровый [87]
	&	
	&	здоровый [не рукой Т.] [57.24об] 
		\tabularnewline \midrule 
\tenevilglyph[yes][5]{c_C_2j} 
	&	ы'ттъын
	&	
	&
	&	собака \cite[л. 68 об]{spbfaran79}
	& 	[25.3] \linebreak
		attьn [әttьn, ы'ттъын = собака; слово напечатано] [7.13, 12.10об] \linebreak
		attьn [ы'ттъын; слово напечатано] \currentGlyphWithAffixes{}{N} [12.16об] \linebreak
		attьt [ы'ттъыт = собаки; слово напечатано] \currentGlyphWithAffixes{}{T} [12.10об]
		\tabularnewline \midrule
\tenevilglyph[yes][4]{c_C_2j_f} 
	&	магԓяԓьын, магԓяԓьыт
	&	
	&
	&	
	& 	[в книге, рядом с изображением собачьей упряжки] [7.30] \linebreak
		maьlaliьt [магԓяԓьыт = ездоки на собаках] [ИЛИ:1.18об]  % ilispread:1.19
		\tabularnewline \midrule
\tenevilglyph[yes][4]{k_2j} 
	&
	&	
	&	чай [6]
	&	щаи [чай] \cite[л. 68 об]{spbfaran79}
	& 	[25.9] \linebreak
		чаеопат [«чай пить» или «чайпат» = вскипевший чай] \currentGlyphWithAffixes{ruk}{} [32.16об] \linebreak
		cajok [чаёк = пить чай] \currentGlyphWithAffixes{ruk}{K} [ИЛИ:1.15]  % ilispread:1.15
		\tabularnewline \midrule
\tenevilglyph[yes][5]{c_cD_b} 
	&	э'ӄимыԓ
	&	
	&	экымыл [э'ӄимыԓ = водка], водка, спирт [10]
	&	
	& 	eqemьl [э'ӄимыԓ] [ИЛИ:2.25] \linebreak  % ilispread:2.24
		пееано [пьяный] \currentGlyphWithAffixes{ruk}{} [30.3] \linebreak
		ieqemьluk [а'ӄэмԓётвак = быть выпившим] \currentGlyphWithAffixes{ruk}{A,K} [ИЛИ:2.27]  % TODO: уточнить перевод % ilispread:2.26 
		\tabularnewline \midrule
\tenevilglyph[yes][4]{c-c_cD_b} 
	&	чеԓгымэмыԓ
	&	
	&	
	&	
	& 	celgьmemьl [celgь-memьl, чеԓгымэмыԓ = красное вино] [ИЛИ:2.25] \linebreak  % TODO: нужна транскрипция, чеԓгымэмыԓ? % ilispread:2.24
		celgьmemlьqaj \currentGlyphWithAffixes{}{Q,A,E} [ИЛИ:2.26]  % ilispread:2.25
		\tabularnewline \midrule
\tenevilglyph[yes][4]{uD_z} 
	&	мынгытԓыӈын
	&	
	&
	&	рука \cite[л. 68]{spbfaran79}
	& 	[25.13] \linebreak
		mьngьtlьŋьn [мынгытԓыӈын = рука; слово напечатано] [12.17об] \linebreak % TODO: уточнить перевод
		mьngьt [мынгыт = руки; слово напечатано] \currentGlyphWithAffixes{}{T} [12.17об]
		\tabularnewline \midrule
\tenevilglyph[yes][3]{I_q_iSY} 
	&
	&	
	&
	&	нока [нога] \cite[л. 68]{spbfaran79} 
	& 	[19.6]
		\tabularnewline \midrule
\tenevilglyph[yes][3]{I_q_iSY_p} 
	&
	&	
	&
	&	сутав [сустав] \cite[л. 68]{spbfaran79} 
	& 	[32.7]
		\tabularnewline \midrule
\tenevilglyph[yes][5]{o-o_z} 
	&	ԓыԓет
	&	
	&	лилет, глаза [32] % TODO: нужна транскрипция
	&	клас [глаз] \cite[л. 68]{spbfaran79}
	& 	[6.1] \linebreak
		lelalgьn [ԓыԓяԓгын = глаз; слово напечатано] \currentGlyphWithAffixes{}{N} [12.23об] \linebreak
		lilet [ԓыԓет = глаза; слово напечатано] \currentGlyphWithAffixes{}{T} [12.23об]
		\tabularnewline \midrule
\tenevilglyph[yes][4]{o-o_z-q_I_q} 
	&	тинԓыԓет
	&	
	&	тинлылет [тинԓыԓет = очки], очки (ледяные глаза)* [32]
	&	
	& 	tinlilet [tin-lilæt, тинԓыԓет] [7.13] 
		\tabularnewline \midrule
\tenevilglyph[yes][4]{l_i} 
	&
	&	
	&	екаак [еӄааӄ = человеческий нос, перед обуви], ин'ин' [и'ӈъиӈ = нос животного, человека, клюв, носовая часть людки], нос [50]
	&	нос \cite[л. 68]{spbfaran79}
	& 	[25.15об] 
		\tabularnewline \midrule
\tenevilglyph[yes][3]{2c_2bX} 
	&	рымагты
	&	
	&
	&	в «провсерас» [позавчера, рымагтайвэ] \cite[л. 67 об]{spbfaran79}
	& 	талче [дальше, рымагты = дальше] [29.13] \linebreak
		в «талсе» [дальше, рымагты] [29.12] \linebreak
		[25.7]
		\tabularnewline \midrule
\tenevilglyph[yes][4]{o_2q_2j} 
	&
	&	
	&
	&	послы [пошли (от «слать»)] \cite[л. 68 об]{spbfaran79}
	& 	[25.10] \linebreak
		genŋiulinet [слово напечатано] [12.20об] \linebreak % TODO: нужен перевод, нужна транскрипция
		посевыке [посылке] [34.8об] \linebreak
		песмо [письмо] \currentGlyphWithAffixes{}{kelikel} [34.8об] \linebreak
		эlьэŋuke \currentGlyphWithAffixes{etly}{} [ИЛИ:1.2]  % TODO: нужен перевод % ilispread:1.2
		\tabularnewline \midrule
\tenevilglyph[yes][5]{vD_2qY} 
	&	а'ачек
	&	
	&
	&	aagek [aacek, а'ачек = молодой человек] \cite[л. 65 об]{spbfaran79} % а'ачек
	& 	[4.2?] \linebreak
		oraçekqaj [= юноша; слово напечатано] [12.22] \linebreak % TODO: уточнить перевод, нужна транскрипция
		молотои [молодой] [29.12] \linebreak
		молотои [молодой] \currentGlyphWithAffixes{}{K,T} [34.11об] \linebreak
		iaacek [aacek, а'ачек] [ИЛИ:2.11, ИЛИ:2.22об]   % ilispread:2.10, ilispread:2.22
		\tabularnewline \midrule
\tenevilglyph[yes][3]{2o_2jY} 
	&
	&	
	&
	&	в «я упрала» [«я убрала»] \cite[л. 67]{spbfaran79}
	& 	[4.2?] 
		\tabularnewline \midrule
\tenevilglyph[yes][5]{CD_jFN} 
	&	титэ
	&	
	&
	&	кова \cite[л. 66]{spbfaran79} \linebreak % TODO: нужна интерпретация
		ковта \cite[л. 66]{spbfaran79}
	& 	[4.2?] \linebreak
		какта [когда] [34.11] \linebreak
		tete [titә, титэ = когда] [ИЛИ:1.13] \linebreak  % ilispread:1.13
		когда [не рукой Т.] [57.9] \linebreak
		tete \currentGlyphWithAffixes{}{T,T} [ИЛИ:2.27]  % ilispread:2.26
		\tabularnewline \midrule
\tenevilglyph[yes][5][votqen]{i_b_jX} 
	&	ӈотӄэн
	&	
	&	н'откэн [ӈотӄэн = этот, эта, это], вот, этот [34]
	&	
	& 	\cite[363]{davydova2015a} \linebreak
		вот [32.6] \linebreak
		ŋotьqen [ŋotьnqәn, ӈотӄэн = вон тут] [4.10об] \linebreak
		ŋotqen [ӈотӄэн] [ИЛИ:1.5] \linebreak  % ilispread:1.5
		ŋutqen [ӈотӄэн] [ИЛИ:2.11] \linebreak  % ilispread:2.10
		nьmkьqen [нымкыӄин = много] [ИЛИ:2.15об,ИЛИ:2.19об]   % похоже на ошибку, но два раза одно и то же % ilispread:2.15, ilispread:2.19
		\tabularnewline \midrule
\tenevilglyph[yes][4]{i_b_jX_2cD} 
	&	ӈотӄоры
	&	
	&	н'откоры [ӈотӄоры = отсюда], н'отко [ӈотӄо = с этого места, отсюда], отсюда [34]
	&	
	& 	ŋutьqorь [ӈотӄоры] [ИЛИ:1.19]  % ilispread:1.19
		\tabularnewline \midrule
\tenevilglyph[yes][3]{2b_2l} 
	&	рэӄэ
	&	
	&
	&	
	& 	reqә [рэӄэ = что-то, как? чем?;слово напечатано] [12.19] \linebreak
		стотакои [что такое] [35.6, 37.2, 37.2об]
		\tabularnewline \midrule
\tenevilglyph[yes][4]{G_t} 
	&	тэгйиӈ
	&	
	&
	&	
	& 	теыеене [тэгйиӈ = кашель, грипп] [34.8] \linebreak % тэгйиӈ
		касли [кашель] [34.11]
		\tabularnewline \midrule
\tenevilglyph[yes][4]{r_t} 
	&	эӈэӈ
	&	
	&
	&	
	& 	пыраснек [праздник] [34.10об] \linebreak
		прачнек [праздник] [30.3] \linebreak
		в «захтре перхоймая» [завтра первое мая] [39.7об] \linebreak
		эeŋeŋkь [эӈэӈкы = на празднике] [ИЛИ:1.14] \linebreak  % TODO: уточнить перевод и транскрипцию % ilispread:1.14
		эeŋeŋkь [эӈэӈкы] \currentGlyphWithAffixes{}{K} [ИЛИ:2.27] \linebreak  % ilispread:2.26
		перхомаи [первомай] \currentGlyphWithAffixes{}{P,R} [30.3] \linebreak
		pergoemaj [первомай] \currentGlyphWithAffixes{}{P,R} [23.5об] 
		\tabularnewline \midrule
\tenevilglyph[yes][5]{i_b_JX} 
	&	эргатык
	&	
	&	эргатык [= завтра], завтра [34]
	&	
	& 	\cite[360]{davydova2015a} \linebreak
		сахтре [завтра] [34.18об] \linebreak
		в «захтре перхоймая» [завтра первое мая] [39.7об] \linebreak
		завтра [не рукой Т.] [57.26] \linebreak
		эrgatьk [әrgatьk, эргатык] [ИЛИ:1.15]  % ilispread:1.15
		\tabularnewline \midrule
\tenevilglyph[yes][3]{i_b_JX_b_c} 
	&	
	&	
	&	будущий год [35]
	&	
	& 	~[65.6] \linebreak
		будущий год [не рукой Т.] [57.24об]
		\tabularnewline \midrule
\tenevilglyph[yes][4]{i_b_JX_c} 
	&	эмӄынгивик
	&	
	&	
	&	
	& 	эmqьngewek [эмӄынгивик = каждый год] [ИЛИ:1.13об]  % ilispread:1.14
		\tabularnewline \midrule
\tenevilglyph[yes][4]{U2EN} 
	&
	&	учиться \cite{lavrov1969}
	&	рыгъюлевык [рыгъюԓевык = учить], нинэйвык, учить, учиться [74] % TODO: нужна транскрипция
	&	
	& 	усеся [учиться] [34.12] \linebreak
		усеся [учиться] \currentGlyphWithAffixes{}{kelikel} [34.11об] \linebreak
		mьneuolrekьnet \currentGlyphWithAffixes{M}{N,E} [ИЛИ:2.2] \linebreak  % TODO: нужен перевод % ilispread:2.1
		keleguletьk \currentGlyphWithAffixes{}{kelikel,T,K} [ИЛИ:2.20]  % ilispread:2.19
		\tabularnewline \midrule
\tenevilglyph[yes][5]{U2E} 
	&
	&	
	&	ейгулэткэлин [эгъюԓеткыԓьин = неспособный], не умеет, не знает [74] 
	&	
	&	эjuletke [эгъюԓеткэ = не умеют] [ИЛИ:1.9об] \linebreak  % TODO: уточнить перевод % ilispread:1.10
		эехулэткы\.{л}ин [эгъюԓеткыԓьин] \currentGlyphWithAffixes{}{K,L,N} [45.2] \linebreak
		эjoletke [эгъюԓеткэ] \currentGlyphWithAffixes{}{K,E} [ИЛИ:2.23об]  % ilispread:2.23
		\tabularnewline \midrule 
\tenevilglyph[yes][5]{cD_2k} 
	&	ӈэнку
	&	
	&	ненкукин [ӈэнкукин = тамошний], там находящийся [78]
	&	
	& 	\cite[364]{davydova2015a} \linebreak
		там [33.4, 34.1] \linebreak
		ŋenko [ŋænku, ӈэнку = там, далеко] [ИЛИ:2.21об] \linebreak  % ilispread:2.21
		ŋekuken [ӈэнкукин] \currentGlyphWithAffixes{}{K,K} [ИЛИ:1.2об]  % ilispread:1.3
		\tabularnewline \midrule
\tenevilglyph[yes][5]{i_qY_vD} 
	&
	&	
	&	таак [тааӄ = табак], табак [110]
	&	тапак [табак] \cite[л. 68 об.]{spbfaran79}
	& 	[4.1об] \linebreak
		табак [не рукой Т.] [57.22]
		\tabularnewline \midrule
\tenevilglyph[yes][5]{UD_uD} 
	&	вагыргын
	&	
	&	вагыргын [= бытие, жизнь, история, общественный строй, событие], бытие, обычай, божество, образ жизни [61] % TODO: проверить, вагыргын = вааргын или нет
	&	
	& 	waьrgьn [vaьrgьn, вагыргын = божество] [ИЛИ:1.8] \linebreak  % ilispread:1.8
		ваыгхын [вагыргын] [43.32] \linebreak
		waьrgьt [вагыргыт = божества] \currentGlyphWithAffixes{}{T} [ИЛИ:1.8об] \linebreak % Weinstein % ilispread:1.9
		waьrgьken [вагыргыкэн = жизненный, в жизни] \currentGlyphWithAffixes{}{K,N} [ИЛИ:1.15]  % ilispread:1.15
		\tabularnewline \midrule
\tenevilglyph[yes][5]{UD_uDE} 
	&	ӈаргынэн
	&	
	&	н'аргынэн [ӈаргынэн = вселенная, наружное пространство, климат, погода], природа, вселенная [62]
	&	
	& 	натваре [на дворе] [30.2об, 34.11об] \linebreak
		ŋargьnen [ӈаргынэн] [ИЛИ:1.4об] \linebreak  % ilispread:1.5
		ŋargьnen [ӈаргынэн] \currentGlyphWithAffixes{}{N} [ИЛИ:1.7об] \linebreak  % ilispread:1.8
		ŋargьnenak \currentGlyphWithAffixes{}{K} [ИЛИ:1.13] \linebreak  % TODO: нужен перевод % ilispread:1.13
		ŋargьnenen \currentGlyphWithAffixes{}{N,N} [ИЛИ:1.8] \linebreak  % ilispread:1.8
		ŋargьnenapь \currentGlyphWithAffixes{}{N,N,P} [ИЛИ:2.4об] \linebreak  % ilispread:2.4
		\tabularnewline \midrule
\tenevilglyph[yes][5]{UD_uD_2q} 
	&	торвагыргын
	&	
	&	торвагыргын [= новая жизнь], новая жизнь [61]
	&	
	& 	torvaьrga [торвагырга = новая жизнь; слово напечатано] [12.20об] \linebreak 
		torwaьrgьn [торвагыргын] [ИЛИ:1.15об]  % ilispread:1.16
		\tabularnewline \midrule
\tenevilglyph[yes][5]{UD_uD_'} 
	&	ӄэргыӄэр
	&	
	&	кергынаргынан, ясная погода [62] % TODO: нужна транскрипция
	&	
	& 	qergьqer [ӄэргыӄэр = свет] [ИЛИ:1.15]  % ilispread:1.15
		\tabularnewline \midrule
\tenevilglyph[yes][4]{q_c_cD_q} 
	&
	&	
	&	%оптыма [о'птыма = подобно, словно, как], подобно, словно, как [54] % видимо, ошибка
	&	соровно [всё равно] \cite[л. 66]{spbfaran79} 
	& 	сороно [всё равно] [34.11об] \linebreak
		tьmŋalgulaq [тымӈаԓгоԓяӄ = без разбору, всё равно] [ИЛИ:1.8об]  % ilispread:1.9
		\tabularnewline \midrule
\tenevilglyph[yes][5][oxptyma]{c_cD} 
	&	о'птыма
	&	
	&	оптыма [о'птыма = подобно, словно, как], подобно, словно, как [54]
	&	
	& 	iomtьma [о'птыма] [ИЛИ:1.3] \linebreak  % ilispread:1.3
		qьnur [ӄынур = как, словно] \currentGlyphWithAffixes{}{K,A,R} [ИЛИ:2.5] \linebreak  % ilispread:2.4
		qьnur [ӄынур] \currentGlyphWithAffixes{}{Q,N,R} [ИЛИ:2.9] \linebreak  % ilispread:2.8
		qьnur [ӄынур] \currentGlyphWithAffixes{}{Q,N,A,R} [ИЛИ:2.11об] \linebreak  % ilispread:2.11
		iemte [э'птэ = тоже] \currentGlyphWithAffixes{}{T} [ИЛИ:1.14об]  % TODO: нужен перевод % ilispread:1.15
		\tabularnewline \midrule
\tenevilglyph[yes][1]{O_JX_b} 
	&
	&	
	&
	&	
	& 	насоеас [33.4] % TODO: нужна интерпретация
		\tabularnewline \midrule
\tenevilglyph[yes][5]{3iX} 
	&	чеэкэй
	&	
	&	кынумекеквынэт, собирает [51] % TODO: уточнить перевод
	&	
	& 	месе [вместе] [29.11, 34.21об, 39.5об] \linebreak
		nьnumkeunet [= собирает; слово напечатано] \currentGlyphWithAffixes{}{T} [12.12об] \linebreak % TODO: уточнить перевод, нужна транскрипция
		cekj [cæәkæj, чеэкэй = вместе, совместно] [ИЛИ:1.5]  % ilispread:1.5
		ceke [чеэкэй] [ИЛИ:1.14] \linebreak  % ilispread:1.14
		ceььkj [чеэкэй] [ИЛИ:1.21]  % ilispread:1.21
		\tabularnewline \midrule
\tenevilglyph[yes][4]{k_j_jF} 
	&
	&	
	&
	&	
	& 	нехосыт [не хочет] [37.2, 37.2об]
		\tabularnewline \midrule
\tenevilglyph[yes][4]{i_2q_l_q_i_L} 
	&	руԓтыркын
	&	
	&
	&	
	& 	ролтырхын [руԓтыркын = сторониться] [34.10] % руԓтыркын
		\tabularnewline \midrule
\tenevilglyph[yes][5]{o_2q_l} 
	&	ӄээӄын
	&	
	&	кейыкын [ӄээӄын = еще немного, еще], еще [46]
	&	
	& 	iесо [еще] [32.1] \linebreak
		iечо [еще] [32.15об] \linebreak
		iезо [еще] [39.1об] \linebreak
		qeqьn [ӄээӄын = еще немного, еще] [ИЛИ:2.28об]  % ilispread:2.28
		\tabularnewline \midrule
\tenevilglyph[yes][5]{G-G} 
	&	нынны
	&	
	&	нынны, название, имя [133]
	&	nьnnь [нынны = имя] \cite[л. 65]{spbfaran79} % нынны
	& 	[11.3] \linebreak
		nьnnь [нынны; слово напечатано] [7.13] \linebreak
		нынны [45.6об] \linebreak
		nьnnь [нынны] [ИЛИ:1.13об]  % ilispread:1.14
		\tabularnewline \midrule
\tenevilglyph[yes][3]{O_oN} 
	&	гытгын, гытгык
	&	озеро \cite{lavrov1969}
	&	гытгын [= озеро], озеро [101]
	&	
	& 	[1.71] \linebreak
		gьtgьk [гытгык = на озере; слово напечатано] [12.17об] \linebreak % TODO: уточнить перевод и транскрипцю
		gьtgьt [гытгыт = озёра; слово напечатано] \currentGlyphWithAffixes{}{T} [12.22]
		\tabularnewline \midrule
\tenevilglyph[yes][3]{O_oN_cF} 
	&	
	&	
	&	гытгычормын [= берег озера], берег озера [100]
	&	
	& 	gьtgь-çormьk [гытгычормык = на берегу озера; слово напечатано] \currentGlyphWithAffixes{}{K} [12.22]
		\tabularnewline \midrule
\tenevilglyph[yes][3]{z_JX} 
	&
	&	
	&
	&	киска [кишка] \cite[л. 66 об]{spbfaran79}
	& 	[4.3]
		\tabularnewline \midrule
\tenevilglyph[yes][4]{cF_2JY} 
	&	гэвэты
	&	
	&
	&	
	& 	неветау [не ведаю] [30.7об] \linebreak
		geuwtь [gewetь, гэвэты = неизвестно, неведомо] [ИЛИ:1.7] \linebreak  % ilispread:1.7
		gewetь [гэвэты] \currentGlyphWithAffixes{}{T} [ИЛИ:2.12]  % ilispread:2.11
		\tabularnewline \midrule
\tenevilglyph[yes][5]{cD_2q_p} 
	&	ынкъам
	&	
	&	ынкам [ынкъам = и, далее], и (союз) [57]
	&	
	& 	\cite[364]{davydova2015a} \linebreak
		потом [30.8] \linebreak
		әnkam [ынкъам; слово напечатано] [12.24] \linebreak
		ьnkiam [әnkam, ынкъам] [ИЛИ:1.2]  % ilispread:1.2
		\tabularnewline \midrule
\tenevilglyph[yes][3]{sM} 
	&	ӄэгԓын
	&	
	&
	&	
	& 	qьlьe [ӄэгԓын = правильно, верно, правду] [ИЛИ:2.3]  % ilispread:2.2
		\tabularnewline \midrule
\tenevilglyph[yes][4]{sM_jF} 
	&	ӄэгԓынангэт
	&	
	&
	&	
	& 	правелно [правильно] [30.7об] \linebreak
		праувлно [правильно] [29.12] \linebreak
		qeьlьnanget [qәglenanget, ӄэгԓынангэт = правильно, правда, действительно, верно] [ИЛИ:1.15об]  % ilispread:1.16
		\tabularnewline \midrule
\tenevilglyph[yes][1]{sM_jFY} 
	&	
	&	
	&
	&	
	& 	waeeьqac [27.18об] % TODO: нужен перевод
		\tabularnewline \midrule
\tenevilglyph[yes][3]{jY} 
	&	виин
	&	
	&
	&	
	& 	веен [виин = временно, пока] [34.7] \linebreak % ЗТ
		ween [виин] [ИЛИ:1.9об]  % ilispread:1.10
		\tabularnewline \midrule
\tenevilglyph[yes][3]{iY_iX} 
	&
	&	
	&	гынунчвипыт [гынунчывипыт = половина], половина \currentGlyphWithAffixes{}{P,T} [51] \linebreak
		ритлит [ритԓит = груз на веревке, единица веса], фунт (единица веса) \currentGlyphWithAffixes{}{T} [51]
	&	в «обрез (кусок) шкуры лахтака» \cite[л. 48]{spbfaran79}
	& 	\cite[364]{davydova2015a} \linebreak
		половена [половина] \currentGlyphWithAffixes{}{P,T} [30.7]
		\tabularnewline \midrule
\tenevilglyph[yes][5][ymy]{2c_i} 
	&	ымы
	&	
	&	ымын [ымы, ымыӈ = тоже], тоже [19]
	&	
	& 	\cite[360, 364]{davydova2015a} \linebreak
		ьmь [ымы] [ИЛИ:1.12]  % ilispread:1.12
		\tabularnewline \midrule
\tenevilglyph[yes][5]{iY_l} 
	&	гэмо
	&	
	&	
	&	
	& 	\cite[364]{davydova2015a} \linebreak
		не знаю [не рукой Т.] [57.9] \linebreak
		gemo [гэмо = неизвестно, не знать, не заметить] [ИЛИ:1.12]  % ilispread:1.12
		\tabularnewline \midrule
\tenevilglyph[yes][4]{J_2lX} 
	&	пэнин
	&	
	&	
	&	
	& 	\cite[360]{davydova2015a} \linebreak
		penen [pænin, пэнин = прежний, старый] [ИЛИ:2.27об] \linebreak  % ilispread:2.27
		penenemure \currentGlyphWithAffixes{}{muri} [ИЛИ:1.15об]  % TODO: нужен перевод % ilispread:1.16
		\tabularnewline \midrule
\tenevilglyph[yes][4]{J_2lX_j} 
	&	панэна
	&	
	&	
	&	
	& 	panena [panena, панэна = всё еще] [ИЛИ:1.6]  % ilispread:1.6
		\tabularnewline \midrule
\tenevilglyph[yes][4]{uD_iXX} 
	&	ӄырым
	&	
	&	
	&	
	& 	\cite[364]{davydova2015a} \linebreak
		qьrьm [ӄырым = не, нет, никогда, всё равно же] [ИЛИ:2.21]  % ilispread:2.20
		\tabularnewline \midrule
\tenevilglyph[yes][4]{uD_iXX_jF} 
	&	ӄырымэн
	&	
	&	
	&	
	& 	qrьmen [ӄырымэн = не, это не, ничей] [ИЛИ:1.2об] \linebreak  % ilispread:1.3
		qьrьmen [ӄырымэн] [ИЛИ:2.3об]  % ilispread:2.3
		\tabularnewline \midrule
\tenevilglyph[yes][5]{iY_J} 
	&	вытку
	&	
	&	вытку [= впервые, только], только что, теперь [69]
	&	
	& 	\cite[361, 363]{davydova2015a} \linebreak
		вытко [вытку] [45.2] \linebreak
		wьtku [вытку] [ИЛИ:2.8об]  % ilispread:2.8
		\tabularnewline \midrule
\tenevilglyph[yes][5]{u_lN} 
	&	чеӈэт
	&	
	&	чен'эт [чеӈэт = ведь], если, так как [68]
	&	
	& 	\cite[364]{davydova2015a} \linebreak
		ceŋet [чеӈэт] [ИЛИ:2.5об]  % ilispread:2.5
		\tabularnewline \midrule
\tenevilglyph[yes][5][janra]{CD_i_C} 
	&	янра, амъянра
	&	
	&	янра [= янрэты = отдельно], отдельно [125]
	&	
	& 	\cite[364]{davydova2015a} \linebreak
		am-janra [амъянра = порознь, отдельно; слово напечатано] [12.20об] \linebreak 
	 	janra [янра] [ИЛИ:1.4об] \linebreak  % ilispread:1.5
		jara [janra, янра] [ИЛИ:2.13об]  % ilispread:2.13
		\tabularnewline \midrule
\tenevilglyph[yes][4]{CD_i_C_2c} 
	&	амъянра
	&	
	&	
	&	
	& 	amjara [амъянра = порознь, отдельно] [ИЛИ:1.5]  % ilispread:1.5
		\tabularnewline \midrule
\tenevilglyph[yes][3]{CD_C} 
	&	ивкэ
	&	
	&	
	&	
	& 	\cite[364]{davydova2015a} \linebreak
		iukeŋan [ИЛИ:1.5об] \linebreak  % TODO: нужна транскрипция, нужен перевод % ilispread:1.6
	 	iuke [iwkә, ивкэ = хоть бы, вот бы, пожалуйста] [ИЛИ:1.17об]  % ilispread:1.18
		\tabularnewline \midrule
\tenevilglyph[yes][3]{LD_q_c} 
	&	э'птэ
	&	
	&	кутти [qutti = quli = другой], другие [49] % кэтэм?
	&	
	& 	iepe [э'птэ = тоже] [ИЛИ:2.3]  % TODO: уточнить перевод, проверить в контексте % ilispread:2.2
		\tabularnewline \midrule
\tenevilglyph[yes][4]{LD_jX} 
	&	микын, микынэ, мэӈин
	&	
	&	
	&	
	& 	mikin [микын = чей, чьи; слово напечатано] [7.13, 12.13] \linebreak
		meŋen [мэӈин = кто, кто-то, кто же] [ИЛИ:1.17] % ilispread:1.17
		mekьne [микынэ = кто?] [ИЛИ:1.9об] % ilispread:1.10
		\tabularnewline \midrule
\tenevilglyph[yes][4]{L-l_q} 
	&	вэԓер
	&	
	&	
	&	
	& 	weler [veler, вэԓер = хоть бы, хоть, довольно, хватит, достаточно] [ИЛИ:1.4] \linebreak  % ilispread:1.4
		wler [вэԓер] [ИЛИ:2.12об]  % ilispread:2.12
		\tabularnewline \midrule
\tenevilglyph[yes][5]{o_2LE} 
	&	пыӈыԓ
	&	
	&	пын'ыл [пыӈыԓ = новость, известие], новость [132]
	&	
	& 	pьŋьl [пыӈыԓ; слово напечатано] [12.20об] \linebreak
		pьŋьl [пыӈыԓ] [ИЛИ:2.25об]  % ilispread:2.25
		\tabularnewline \midrule
\tenevilglyph[yes][3]{o_L_LE} 
	&
	&	
	&	
	&	
	& 	qәtwuunat [ӄытвыгынат = расскажи; слово напечатано] [12.24об] \linebreak 
		qәtwuun [ӄытвыгын = расскажите; слово напечатано] [12.25] 
		\tabularnewline \midrule
\tenevilglyph[yes][4]{o_2LE-q_jX} 
	&	ръапыӈыԓ?
	&	
	&	
	&	
	& 	r\=a-pьŋьl? [ръапыӈыԓ? = какие новости?; слово напечатано] [12.20] % Weinstein TODO: уточнить перевод
		\tabularnewline \midrule
\tenevilglyph[yes][4]{c_c_p} 
	&	вэчьым
	&	
	&	вечым [вэчьым = очевидно, пожалуй, должно быть, может быть], вероятно [69] 
	&	
	& 	везиым [вэчьым] [41.7] \linebreak
		weciьm [вэчьым] [ИЛИ:1.5]  % ilispread:1.5
		\tabularnewline \midrule
\tenevilglyph[yes][4]{c_i_p_i} 
	&	эӄыԓпэ
	&	
	&	экылпэ [эӄыԓпэ = скорее, быстро], скорее [100] % лишняя черта снизу ошибка
	&	
	& 	эqьlpэ [әqәlpæ, эӄыԓпэ] [ИЛИ:1.3об]  % ilispread:1.4
		\tabularnewline \midrule
\tenevilglyph[yes][3]{i_sXY} 
	&	
	&	
	&	
	&	
	& 	jaleьtь [яаԓегты = позднее] \currentGlyphWithAffixes{}{T} [ИЛИ:1.6, ИЛИ:1.15об] \linebreak   % ilispread:1.6, ilispread:1.16
		jalьtь [яаԓегты] \currentGlyphWithAffixes{}{T} [ИЛИ:1.6об]  % ilispread:1.7
		\tabularnewline \midrule
\tenevilglyph[yes][4]{i_sXY_jFY} 
	&	яачы, яачыӈкэн
	&	
	&	
	&	
	& 	\cite[360]{davydova2015a} \linebreak
		jacь [яачы = сзади, после, потом] [ИЛИ:1.17] \linebreak  % ilispread:1.17
		jacьken [яачыӈкэн = последний] [ИЛИ:1.16об]  % ilispread:1.17
		\tabularnewline \midrule
\tenevilglyph[yes][4]{4j} 
	&	ытри
	&	
	&	ытри [= они], они [89]
	&	
	& 	\cite[360, 361, 364]{davydova2015a} \linebreak
		ьree [ытри] [ИЛИ:1.4об]  % TODO: странно, уточнить контекст % ilispread:1.5
		\tabularnewline \midrule
\tenevilglyph[yes][5]{C_IY} 
	&	чымӄык
	&	
	&	чымкык [чымӄык = часть, взятый частично, другой, иной], тэйвын [тэйвыӈ = часть, доля, пай], часть [68]
	&	
	& 	çьmqьk [чымӄык; слово напечатано] [12.20об] \linebreak
		cьmqьk [чымӄык] [ИЛИ:1.4об]  % ilispread:1.5
		\tabularnewline \midrule
\tenevilglyph[yes][4]{C_IY_2c} 
	&	эмчымӄык
	&	
	&	
	&	
	& 	эemcmqьk [эмчымӄык = только часть] [ИЛИ:1.4об]  % Weinstein % ilispread:1.5
		\tabularnewline \midrule
\tenevilglyph[yes][4]{2b} 
	&	итык, итыкэвын
	&	
	&	
	&	
	& 	\cite[364]{davydova2015a} \linebreak
		иитык [итык = а ... вот, -то, быть, являться, служить] [41. 7] \linebreak
		ietьkeun [itьk eun, итыкэвын = хотя, и всё же, вообще-то] [ИЛИ:1.10об] \linebreak  % ilispread:1.11
		ietьk [itьk, итык] [ИЛИ:1.5]  % ilispread:1.5
		\tabularnewline \midrule
\tenevilglyph[yes][5]{2b_2q} 
	&	вэты
	&	
	&	вэты [= надо, необходимо, усилительная частица], поистине [113]
	&	
	& 	\cite[364]{davydova2015a} \linebreak
		vetь [вэты; слово напечатано] [12.15] \linebreak
		wtwtь [ИЛИ:1.3об] \linebreak  % TODO: нужен перевод % ilispread:1.4
		wetь [вэты] [ИЛИ:2.27об]  % ilispread:2.27
		\tabularnewline \midrule
\tenevilglyph[yes][4]{uD-uD_2cD} 
	&	раӄыԓӄыԓ
	&	
	&	ракалькал [раӄыԓӄыԓ = ненужная вещь, старье, хлам], ненужное, негодное [103] % одна из скобок выше
	&	
	& 	raqьlqьl [раӄыԓӄыԓ] [ИЛИ:1.4, ИЛИ:1.23] \linebreak  % ilispread:1.4 , ilispread:1.24
		худой [не рукой Т.] [57.22]
		\tabularnewline \midrule
\tenevilglyph[yes][4]{o_IY-_IY} 
	&
	&	
	&	кооператык, кооперат, кооператив [52] % TODO: нужна транскрипция
	&	
	& 	kaparaceu [кооператив] [ИЛИ:1.5]  % TODO: уточнить перевод % ilispread:1.5
		\tabularnewline \midrule
\tenevilglyph[yes][5][etly]{S} 
	&	этԓы
	&	
	&	etle [этԓы = нет, не так, никогда], не (отрицательная частица при причастии и деепричастии) [15] 
	&	
	& 	elte [этԓы; слово напечатано] [12.15об] \linebreak
		elь [этԓы] [ИЛИ:1.10об]  % ilispread:1.11
		\tabularnewline \midrule
\tenevilglyph[yes][3]{k_jF_k_jFX} 
	&	пууръу
	&	
	&	
	&	
	& 	\cite[364]{davydova2015a} \linebreak
		puriu [пууръу = взамен, вместо, наоборот] [ИЛИ:2.1об]  % ilispread:2.1
		\tabularnewline \midrule
\tenevilglyph[yes][4]{jE_jFE_jF} 
	&	выегыргын
	&	
	&	выентогыргын [= дыхание, выдыхание], дыхание [88]
	&	
	& 	\cite[364]{davydova2015a} \linebreak
		wejьrьrgьn [выегыргын = дыхание] [ИЛИ:1.7] \linebreak  % МП % ilispread:1.7
		wejьrrьgьn [выегыргын = дыхание] [ИЛИ:1.7об] \linebreak  % ilispread:1.8
		weьrrgьn [выегыргын = дыхание] [ИЛИ:1.12об] \linebreak  % ilispread:1.13
		weьrgьn [выегыргын = дыхание] [ИЛИ:2.17об]  % ilispread:2.17
		\tabularnewline \midrule
\tenevilglyph[yes][4]{c-cD_'} 
	&	тэнмычьын
	&	
	&	
	&	
	& 	tenmьciьn [tenmьcьn, тэнмычьын = план, график, мера, образец] [ИЛИ:1.8, ИЛИ:2.12] \linebreak   % один из знаков зеркальный % ilispread:1.8, ilispread:2.11
		temьciьn [тэнмычьын] [ИЛИ:2.5об] \linebreak  % ilispread:2.5
		tenmьcio [тэнмычьо = например] \currentGlyphWithAffixes{}{A} [ИЛИ:2.5об]  % ilispread:2.5
		\tabularnewline \midrule
\tenevilglyph[yes][4]{UD_2j} 
	&	амтымӈэ
	&	
	&	
	&	
	& 	amtьmŋe [амтымӈэ = просто, так себе] [ИЛИ:2.13об] \linebreak  % ilispread:2.13
		amtьmŋe [амтымӈэ] \currentGlyphWithAffixes{}{E} [ИЛИ:2.26, ИЛИ:2.25] \linebreak   % ilispread:2.24, ilispread:2.25
		amtьmŋe [амтымӈэ] \currentGlyphWithAffixes{}{T,M} [ИЛИ:2.6]  % ilispread:2.5
		\tabularnewline \midrule
\tenevilglyph[yes][1]{UD_2jD} 
	&
	&	
	&	
	&	
	& 	naьnatьnat [ИЛИ:1.9об] \linebreak  % нэнаа’тынат = беспокоят? TODO: нужен перевод % ilispread:1.10
		walak \currentGlyphWithAffixes{}{A,L,K} [ИЛИ:1.12]  % TODO: нужен перевод % ilispread:1.12
		\tabularnewline \midrule 
\tenevilglyph[yes][3][kely]{2sX_j} 
	&	кэԓы
	&	
	&	
	&	
	& 	kelь [kælь, кэԓы = злой дух, черт] [ИЛИ:2.24] \linebreak  % ilispread:2.23
		kelien [кэԓьин = злого духа] [ИЛИ:1.8об] \linebreak  % зеркально % ilispread:1.9
		kelien [кэԓьин] \currentGlyphWithAffixes{}{b,N} [ИЛИ:2.11об]  % ilispread:2.11
		\tabularnewline \midrule 
\tenevilglyph[yes][4]{i_cX} 
	&	мэчичьу
	&	
	&	
	&	
	& 	\cite[364]{davydova2015a} \linebreak
		mececiu [mæcicu, мэчичьу = а всё же, всё-таки] [ИЛИ:1.14]  % ilispread:1.14
		\tabularnewline \midrule 
\tenevilglyph[yes][4]{rB_i_j} 
	&	рыпэт
	&	
	&	рыпет [рыпэт = даже], даже [54]
	&	
	& 	\cite[364]{davydova2015a} \linebreak
		rьpet [рыпэт] [ИЛИ:1.14]  % ilispread:1.14
		\tabularnewline \midrule 
\tenevilglyph[yes][5][erym]{SYE} 
	&	эрым
	&	
	&	
	&	
	&	эerьm [ærьm, эрым = начальник, глава, староста] [ИЛИ:1.13об] \linebreak  % ilispread:1.14
		ermete [эрмэтэ = начальник] \currentGlyphWithAffixes{}{T} [ИЛИ:1.2] \linebreak  % ilispread:1.2
		насалик [начальник] [37.5] \linebreak
		armaьtь [армагты = начальству; сильно, мужественно] \currentGlyphWithAffixes{}{A,T} [ИЛИ:1.3,ИЛИ:1.12] \linebreak   % Weinstein TODO: уточнить перевод % ilispread:1.3, ilispread:1.12
		tirk-erьm [тиркэрым = царь; слово напечатано] \currentGlyphWithAffixes{tirkytir}{} [12.20об] \linebreak
		terkermen [тиркэрмин = царский] \currentGlyphWithAffixes{}{tirkytir,N} [ИЛИ:2.18]  % ilispread:2.17
		\tabularnewline \midrule
\tenevilglyph[yes][4]{SYE_2q} 
	&	турэрмэт
	&	
	&	
	&	
	&	tur-ermete [tur-ærmæt, турэрмэт = советская власть; слово напечатано] [12.15об] \linebreak
		tur-ermete [турэрмэт; слово напечатано] \currentGlyphWithAffixes{}{T} [12.15]
		\tabularnewline \midrule
\tenevilglyph[yes][4]{u-2j} 
	&	ынръам
	&	
	&	иам [ийъам = почему], почему [78] % видимо, ошибка
	&	
	&	\cite[364]{davydova2015a} \linebreak
		ьnriam [ынръам = видимо] [ИЛИ:1.7об]  % Weinstein % ilispread:1.8
		\tabularnewline \midrule 
\tenevilglyph[yes][5]{oF_j_q} 
	&	юрэӄ
	&	
	&	юрэк [юрэӄ = ԓюрэӄ], люрэк [ԓюрэӄ = возможно, может быть], может быть [77]
	&	
	&	joreq [юрэӄ] [ИЛИ:2.28об] \linebreak  % ilispread:2.28
		jurieq [юрэӄ] [ИЛИ:2.7]  % ilispread:2.6
		\tabularnewline \midrule 
\tenevilglyph[yes][1]{i_j_J_2j} 
	&
	&	
	&	
	&	
	&	эmlььen [ИЛИ:1.19об] \linebreak  % TODO: нужен перевод % ilispread:1.20
		emlььэn [ИЛИ:2.4об] \linebreak  % ilispread:2.4
		emlььn [ИЛИ:2.28об]  % ilispread:2.28
		\tabularnewline \midrule 
\tenevilglyph[yes][4]{b-b} 
	&	аԓва
	&	
	&	алва [аԓва = иначе, не так], иной, чужой, иначе, не так, прочь [28]
	&	
	&	alwa [alva, аԓва] [ИЛИ:1.2]  % ilispread:1.2
		\tabularnewline \midrule 
\tenevilglyph[yes][3]{b-b_2c} 
	&	амаԓваӈ
	&	
	&	
	&	
	&	amalwa [амаԓваӈ = по-разному, каждый по-своему, разнообразный] [ИЛИ:1.13об] \linebreak  % TODO: уточнить перевод % ilispread:1.14
		amaliam [ИЛИ:1.17] \linebreak  %TODO: нужен перевод % ilispread:1.17
		amalwaim [ИЛИ:1.13об]  %TODO: нужен перевод % ilispread:1.14
		\tabularnewline \midrule 
\tenevilglyph[yes][3]{JF-jY} 
	&	ынӈэ
	&	
	&	
	&	
	&	ьŋe [ынӈэ = хотя, нельзя] [ИЛИ:2.13]  % ЗТ % ilispread:2.12
		\tabularnewline \midrule 
\tenevilglyph[yes][3]{JFE-jY} 
	&	ынӈатаԓ
	&	
	&	
	&	
	&	 %ьŋetal [ынӈатаԓ = очень, нужно будет, с того времени] [??] \linebreak % ЗТ % TODO: найти источник, было ilispread:2.14
		ьŋatal [ынӈатаԓ = очень, нужно будет, с того времени] [ИЛИ:2.13, ИЛИ:2.15]   % ilispread:2.12, ilispread:2.14
		\tabularnewline \midrule 
\tenevilglyph[yes][4]{dDE} 
	&	ээк
	&	
	&	ээк [= жирник], жирник, лампа [37]
	&	
	&	eek [aak, ээк; слово напечатано] [12.11] \linebreak
		ээк [69.1]
		\tabularnewline \midrule 
\tenevilglyph[yes][4]{i_JY_j} 
	&	чаат
	&	
	&	чаат [аркан, лассо], чавыт, аркан [58] % TODO: нужен перевод, нужна транскрипция
	&	
	&	çaat [caat, чаат; слово напечатано] [12.22] \linebreak
		чаат [слово напечатано, рядом рисунок аркана] [42.22об]
		\tabularnewline \midrule
\tenevilglyph[yes][3]{lE-lE} 
	&	пароԓ
	&	
	&	пароль [пароԓ = лишний, плюс, излишек], лишний, сверхкомплектный, плюс [132]
	&	
	&	\cite[361]{davydova2015a} \linebreak
		parol [пароԓ; слово напечатано] [12.19] 
		\tabularnewline \midrule 
\tenevilglyph[yes][3]{cL_cR} 
	&	вэԓёԓгын, виԓют
	&	
	&	вилут [виԓют = уши], уши \currentGlyphWithAffixes{}{T} [127]
	&	
	&	vilut [виԓют; слово напечатано] [12.23об] \linebreak
		vilut [виԓют; слово напечатано] [12.13об] \currentGlyphWithAffixes{}{T} \linebreak
		velolgьn [вэԓёԓгын = ухо; слово напечатано] \currentGlyphWithAffixes{}{N} [12.23об]
		\tabularnewline \midrule 
\tenevilglyph[yes][5]{I_2q_2c} 
	&	пипиӄыԓгын
	&	
	&	пипикылгын [пипиӄыԓгын = мышь], мышь [23]
	&	
	&	peepeqьlьn [пипиӄыԓгын] [8.1] \linebreak
		pipәkьlgьn [пипиӄыԓгын; слово напечатано] [12.20, 14.20] \linebreak
		~[pядом с изображением мыши] [19.6] \linebreak
		pepeqьlgьn [pipәkьlgьn, пипиӄыԓгын] [ИЛИ:2.18об]  % ilispread:2.18
		\tabularnewline \midrule 
\tenevilglyph[yes][4]{3b} 
	&	кымъыԓгын
	&	
	&	
	&	
	&	kьmiьlgьn [kьmьlgьn, кымъыԓгын = червь] [ИЛИ:2.19] \linebreak  % ilispread:2.18
		kьmien [кымъын = червь] \currentGlyphWithAffixes{K}{N} [ИЛИ:2.19] \linebreak  % ilispread:2.18
		veqәn [вэӄын = навага; слово напечатано] \currentGlyphWithAffixes{}{N} [12.25] \linebreak
		gьtokalekal [гытокаԓе(каԓ) = гусеница] \currentGlyphWithAffixes{}{kelikel} [19.1]
		\tabularnewline \midrule 
\tenevilglyph[yes][4]{3b_k} 
	&	апаапагԓыӈын
	&	
	&	
	&	
	&	apapaьlŋn [apaapaglьŋьn, апаапагԓыӈын = паук] [19.1] \linebreak
		~[рядом с рисунком паука] [49.3]
		\tabularnewline \midrule 
\tenevilglyph[yes][4]{l_lX} 
	&	чейвыткук
	&	
	&	
	&	
	&	geçejwutkulin [гэчейвыткуԓин = ходил; слово напечатано] [12.22] \linebreak
		cewьtkuk [чейвыткук = ходить] \currentGlyphWithAffixes{}{K,K} [ИЛИ:2.27об] \linebreak  % ilispread:2.27
		cewьtkuk [чейвыткук] \currentGlyphWithAffixes{}{K,W,K} [ИЛИ:2.25об] \linebreak  % ilispread:2.25
		\tabularnewline \midrule 
\tenevilglyph[yes][1]{i_j_i_j_jE_iXX} 
	&
	&	
	&	
	&	
	&	nurman [ИЛИ:1.15]  % TODO: нужен перевод, нужна транскрипция % ilispread:1.15
		\tabularnewline \midrule 
\tenevilglyph[yes][4]{I_2q} 
	&	тинтин
	&	
	&	тинтин [= лёд], лёд (речной) [52]
	&	лот [лёд] \cite[л. 66, 66 об.]{spbfaran79}
	&	tintin [тинтин = лёд; слово напечатано] [12.22]
		\tabularnewline \midrule 
\tenevilglyph[yes][3]{L_JFT} 
	&	э'митԓён
	&	
	&	
	&	
	&	iemelon [æmi-tlon, э'митԓён = где же] [12.22] \linebreak 
		iemlun [э'митԓён] [12.22] \linebreak
		где [не рукой Т.] [57.9] 
		\tabularnewline \midrule 
\tenevilglyph[yes][2]{i_2j_ZRX} 
	&
	&	
	&	
	&	
	&	nьlwauŋuqen [ныԓвавыӈӈоӄэн = не могут] [ИЛИ:1.3об] \linebreak % ilispread:1.4
		ninetiŋu-muri [нинэтиӈумури = вытягиваем; слово напечатано] \currentGlyphWithAffixes{}{muri} [12.25] % TODO: нужен перевод
		\tabularnewline \midrule 
\tenevilglyph[yes][4]{SYY_jF_2q} 
	&	эӈэӈыԓьын
	&	
	&	эн'эн'ылын [эӈэӈыԓьын = шаман], шаман [88]
	&	
	&	эŋeŋьliьn [æŋæŋьlьn, эӈэӈыԓьын] [ИЛИ:1.8]  % ilispread:1.8
		\tabularnewline \midrule 
\tenevilglyph[yes][4]{J-jF} 
	&	уттуут
	&	
	&	уттуут [= дерево], дерево [13]
	&	
	&	uttuut [уттуут; слово напечатано] [12.11об]
		\tabularnewline \midrule 
\tenevilglyph[yes][4]{J-jFE} 
	&	умкуум
	&	
	&	умкуум [= леса, тайга], чаща (лесная) [13]
	&	
	&	umkuum [умкуум; слово напечатано] [12.11об] 
		\tabularnewline \midrule 
\tenevilglyph[yes][4]{J-jF_cF_q} 
	&
	&	
	&	оттысгепы, на вершине дерева [14] % TODO: уточнить перевод, нужна транскрипция
	&	
	&	ottьsqepu [= с вершины дерева; слово напечатано] [12.22об] % TODO: уточнить перевод, нужна транскрипция
		\tabularnewline \midrule 
\tenevilglyph[yes][3]{i_b_jF} 
	&
	&	
	&	поселок Анадырь [11]
	&	
	&	анатры \currentGlyphWithAffixes{}{T} [37.2] 
		\tabularnewline \midrule 
\tenevilglyph[yes][4]{2LE} 
	&
	&	
	&	тычгыргын [тъычгыргын = болезнь], аргыргын, болезнь \currentGlyphWithAffixes{R}{} [61]
	&	полноь [больной] \cite[л. 66 об]{spbfaran79}
	& 	[25.4об] \linebreak
		полнои [больной] [34.11] \linebreak
		tiьlьŋurkьt \currentGlyphWithAffixes{}{b,T,K} [ИЛИ:2.22] \linebreak  % TODO: нужен перевод; bT тут немного загадка % ilispread:2.21
		tiьcgьrgьn [тъычгыргын = болезнь] \currentGlyphWithAffixes{}{b,T} [ИЛИ:2.4об] \linebreak  % ilispread:2.4
		tәçьrgьn [tәçgьrgьn, тъычгыргын; слово напечатано] \currentGlyphWithAffixes{R}{} [12.20об] 
		\tabularnewline \midrule
\tenevilglyph[yes][3]{u-k} 
	&	ватап
	&	
	&	
	&	
	&	watap [ватап = ягель] [ИЛИ:1.6об]  % ilispread:1.7
		\tabularnewline \midrule
\tenevilglyph[yes][3]{C_4j} 
	&
	&	
	&	
	&	
	&	nresqiwqinet [нырэсӄивӄинэт = входят; слово напечатано] [12.24об]
		\tabularnewline \midrule 
\tenevilglyph[yes][3]{CD_4j} 
	&
	&	
	&	
	&	
	&	nьlqьrernen [ныԓӄырирӄин = искали] \currentGlyphWithAffixes{}{R,Q,L} [ИЛИ:2.4об]  % ilispread:2.4
		\tabularnewline \midrule 
\tenevilglyph[yes][3]{s_jF_jFY} 
	&	ынйив
	&	
	&	ынйив [= дядя], дядя [88]
	&	
	&	әnjiw [ынйив; слово напечатано] [12.21]
		\tabularnewline \midrule 
\tenevilglyph[yes][4]{2JFY_b} 
	&
	&	
	&	тымкытым, кочка [98]
	&	tьmkьtьm [тымкытым = кочка] \cite[л. 64 об.]{spbfaran79}
	&	[19.18]
		\tabularnewline \midrule 
\tenevilglyph[yes][3]{o-z-o} 
	&	пиԓгын
	&	
	&	
	&	
	&	pelgьn [pilgьn, пиԓгын = горло, морское устье рек, носик чайника] [ИЛИ:1.6]  % ilispread:1.6
		\tabularnewline \midrule 
\tenevilglyph[yes][4]{c-i_jFE} 
	&	тыӈачьыԓгын
	&	
	&	
	&	
	&	tьŋaciьlgьn [тыӈачьыԓгын = растение] [ИЛИ:1.14об]  % ilispread:1.15
		\tabularnewline \midrule
\tenevilglyph[yes][3]{c-i_q} 
	&	
	&	
	&	
	&	
	&	\cite[364]{davydova2015a}
		ininirkьn [ининиркын = всходит] [7.13] \linebreak % TODO: уточнить перевод
		ininirkьn [ининиркын] \currentGlyphWithAffixes{}{R} [12.13] 
		\tabularnewline \midrule 
\tenevilglyph[yes][3]{2b_J} 
	&
	&	
	&	
	&	
	&	ilen [ы'ԓин = снежный] [ИЛИ:1.16об]  % TODO: сомнительная интерпретация, проверить % ilispread:1.17
		\tabularnewline \midrule 
\tenevilglyph[yes][3]{2B_2jF} 
	&	йъэԓивэн
	&	
	&	
	&	
	&	jiэlewen [йъэԓивэн = как будто, разумеется, в порядке вещей] [ИЛИ:1.19] \linebreak  % ilispread:1.19
		jilewen [йъэԓивэн] [ИЛИ:1.19об]  % ilispread:1.20
		\tabularnewline \midrule 
\tenevilglyph[yes][1]{2B} 
	&
	&	
	&	
	&	
	&	netkenet [ИЛИ:1.10] \linebreak  % TODO: нужен перевод % ilispread:1.10
		ekenen \currentGlyphWithAffixes{K}{N,N} [ИЛИ:2.17об]  % ilispread:2.17
		\tabularnewline \midrule 
\tenevilglyph[yes][3]{i_4'} 
	&
	&	
	&	
	&	стрел [стрелять] \cite[л. 66 об.]{spbfaran79}
	&	qәjnewnin [ӄэгнэвнин = выстрелил; слово напечатано] \currentGlyphWithAffixes{milger}{} [12.22] % Weinstein TODO: уточнить перевод
		\tabularnewline \midrule 
\tenevilglyph[yes][1]{b_kY} 
	&
	&	
	&	
	&	
	&	rьrьt [ИЛИ:2.14]  % TODO: нужен перевод % ilispread:2.13
		\tabularnewline \midrule 
\tenevilglyph[yes][1]{j-q_j_q} 
	&
	&	
	&	
	&	
	&	четаи [30.3об] % TODO: нужна интерпретация
		\tabularnewline \midrule 
\tenevilglyph[yes][3]{B_jF_bT} 
	&
	&	
	&	
	&	
	&	nьretьlaqen [ИЛИ:1.7об] \linebreak  % TODO: нужен перевод % ilispread:1.8
		nьrtьqen [нырэтӄин = приносят] \currentGlyphWithAffixes{}{N} [ИЛИ:1.7об]  % ilispread:1.8
		\tabularnewline \midrule 
\tenevilglyph[yes][2]{b-b-u} 
	&
	&	
	&	ратьё, привезенный, рэтык, привезти [36] \linebreak
		генинтылин, брошенный \currentGlyphWithAffixes{}{T,L,N} [36] % TODO: нужен перевод
	&	
	&	nьneteqen \currentGlyphWithAffixes{}{N} [ИЛИ:1.3] \linebreak  % TODO: нужен перевод % ilispread:1.3
		neneteqen \currentGlyphWithAffixes{}{N} [ИЛИ:1.3об] \linebreak  % ilispread:1.4
		nenetьnet \currentGlyphWithAffixes{}{T} [ИЛИ:1.12]  % ilispread:1.12
		\tabularnewline \midrule 
\tenevilglyph[yes][1]{b-b-u-f} 
	&
	&	
	&	
	&	
	&	retwie [ИЛИ:2.10]  % TODO: нужен перевод, возможно, имя % ilispread:2.9
		\tabularnewline \midrule 
\tenevilglyph[yes][4]{uT_pF} 
	&
	&	
	&	
	&	
	&	ənjьlgьt [= пусть они дают тебе; слово напечатано] \currentGlyphWithAffixes{}{T} [12.21] \linebreak % TODO: нужна транскрипция
		nenejlqen [нинэйыԓӄин = отдает] \currentGlyphWithAffixes{}{N} [ИЛИ:1.4] \linebreak  % ilispread:1.4
		эjьlke [эйыԓкэ = не давать] \currentGlyphWithAffixes{etly}{} [ИЛИ:2.25об]  % TODO: уточнить перевод % ilispread:2.25
		\tabularnewline \midrule 		
\tenevilglyph[yes][3][wilytkuk]{uT_pF_b} 
	&
	&	покупать \cite{lavrov1969}
	&	кунин [куннин = купил], купил [11] \linebreak
		вэлыткук [виԓыткук = торговать], торговать \currentGlyphWithAffixes{}{K}, \currentGlyphWithAffixes{}{T} [11] 
	&	
	&	\cite[360]{davydova2015a} \linebreak
		в «макасен» [магазин] [32.14] \linebreak
		в «макачйн» [магазин] [32.15] \linebreak
		weleunwk [вэԓевванвык = на рынке] \currentGlyphWithAffixes{}{T} [ИЛИ:1.18об] \linebreak  % ilispread:1.19
		kunnin [куннин; слово напечатано] \currentGlyphWithAffixes{}{N} [12.15] \linebreak 
		nerkeukьn [= покупаются; слово напечатано] \currentGlyphWithAffixes{}{N} [12.15]
		\tabularnewline \midrule 
\tenevilglyph[yes][3]{g_oB} 
	&	оонъыԓгын
	&	
	&	оонылгын [оонъыԓгын = ягода], ягода [65] 
	&	
	&	uunet [уунъыт = ягоды; слово напечатано] \currentGlyphWithAffixes{}{T} [12.11об]
		\tabularnewline \midrule 
\tenevilglyph[yes][3]{UD_i_u} 
	&	
	&	
	&	
	&	
	&	потерял [не рукой Т.] [57.9] \linebreak
		nьtmŋuqen \currentGlyphWithAffixes{}{N} [ИЛИ:1.3об] \linebreak  % TODO: нужен перевод % ilispread:1.4 
		nьtьmŋuqen \currentGlyphWithAffixes{}{N,T} [ИЛИ:2.8]  % ilispread:2.7
		\tabularnewline \midrule 
\tenevilglyph[yes][3]{r-v} 
	&	
	&	
	&	
	&	
	&	\cite[362]{davydova2015a} \linebreak
		plekьt [пԓекыт = обувь; слово напечатано] \currentGlyphWithAffixes{}{L,T} [12.20] \linebreak
		tьplegutwaak [= я разулся; слово напечатано] \currentGlyphWithAffixes{}{janra} [12.20] % TODO: уточнить перевод
		\tabularnewline \midrule 
\tenevilglyph[no][3]{r-v_jY} 
	&	
	&	
	&	
	&	лапа \cite[л. 68]{spbfaran79}
	&	
		\tabularnewline \midrule 
\tenevilglyph[yes][3]{r-v_q_U} 
	&	
	&	
	&	панраплякыт [панрапԓякыт = обувь из камусов], плякылгын [пԓякыԓгын = обувь], камусная торбаса (чукотская обувь) [78]
	&	
	&	\cite[362]{davydova2015a} \linebreak
		плекыт [пԓекыт = обувь; не рукой Т.] [57.25]
		\tabularnewline \midrule 
\tenevilglyph[yes][3]{r-v_2l} 
	&	
	&	
	&	
	&	
	&	tьjpьnat [тыйпынат = надел; слово напечатано] \currentGlyphWithAffixes{}{T} [12.20]
		\tabularnewline \midrule 
\tenevilglyph[yes][3]{L_uD} 
	&	
	&	
	&	
	&	
	&	petьqetie [пинтыӄэтгъи = появился] [ИЛИ:2.18]  % Weinstein TODO: уточнить перевод % ilispread:2.17
		\tabularnewline \midrule 
\tenevilglyph[yes][3]{JE_q} 
	&	
	&	
	&	рыюлин [рыюԓьын = сторож, ночной пастух], пастух \currentGlyphWithAffixes{}{qorany} [83]
	&	
	&	\cite[361]{davydova2015a} \linebreak
		gьretьk [гынритык = пасти] \currentGlyphWithAffixes{}{T,K} [ИЛИ:2.23об] \linebreak  % ilispread:2.23
		mьgьrerkьnet \currentGlyphWithAffixes{M}{R,K} [ИЛИ:2.23об] \linebreak  % ilispread:2.23
		пачтох [пастух] \currentGlyphWithAffixes{}{qorany} [29.12] \linebreak
		оленевот [оленевод] \currentGlyphWithAffixes{}{qorany} [30.3об] \linebreak
		караули [караулить] \currentGlyphWithAffixes{}{qorany} [30.6] \linebreak
		qoraьretьlien [ӄорагынрэтыԓьэн = оленевода] \currentGlyphWithAffixes{}{qorany,b,N} [ИЛИ:2.16]  % ilispread:2.15
		\tabularnewline \midrule 
\tenevilglyph[yes][3]{U_2jF_i_uD} 
	&	
	&	
	&	
	&	
	&	понемаи [понимаю] \currentGlyphWithAffixes{T,A}{} [30.7] \linebreak % TODO: как на чукотском?
		nwalmure [наваԓёмморэ = слышим] \currentGlyphWithAffixes{N}{muri} [ИЛИ:2.26об]  % TODO: уточнить транскрипцию и перевод % ilispread:2.26
		\tabularnewline \midrule 
\tenevilglyph[yes][3]{i_2jT_2CE} 
	&	
	&	
	&	выквылгын [выквыԓгын = камень], камень [74]
	&	
	&	[рядом с изображением камня] [19.11] % TODO: получше описать контекст
		\tabularnewline \midrule 
\tenevilglyph[yes][3]{UT_U} 
	&	
	&	
	&	
	&	
	&	neniьqen [нинъыӄин = проворный] \currentGlyphWithAffixes{}{N,N,Q} [ИЛИ:2.22об] \linebreak  % ilispread:2.22
		neniьqen [нинъыӄин] \currentGlyphWithAffixes{}{N,N,b,K,Q} [ИЛИ:2.27об]  % ilispread:2.27
		\tabularnewline \midrule 
\tenevilglyph[yes][3]{I_r_2l_qY} 
	&	% ы’ттъыёл ?
	&	
	&	
	&	
	&	ьnaniьtool [ынанъыттъыёԓ = сначала] \currentGlyphWithAffixes{ynan}{} [ИЛИ:1.9] \linebreak  % ilispread:1.9
		ьnaniьtiejolken [ынанъыттъыёԓкэн = изначальный] \currentGlyphWithAffixes{ynan}{} [ИЛИ:1.16об]  % ilispread:1.17
		\tabularnewline \midrule 
\tenevilglyph[yes][4]{uD_uD} 
	&	
	&	
	&	
	&	пиристал [перестал] \cite[л. 67 об.]{spbfaran79}
	&	nьpaian [ныпаагъан = переставал] \currentGlyphWithAffixes{}{N} [ИЛИ:1.14об] \linebreak  % ilispread:1.15
		nьpaaian [ныпаагъан] \currentGlyphWithAffixes{N}{N} [ИЛИ:1.13] \linebreak  % ilispread:1.13
		nьpaaqen [ныпааӄэн = перестал] \currentGlyphWithAffixes{N}{N} [ИЛИ:2.11об]  % ilispread:2.11
		\tabularnewline \midrule 
\tenevilglyph[yes][3]{C_2q_CX_2q} 
	&	
	&	
	&	
	&	
	&	akaukeьt [акавкэгты = неудобно] \currentGlyphWithAffixes{}{T} [ИЛИ:1.6]  % ilispread:1.6
		\tabularnewline \midrule 
\tenevilglyph[yes][1]{I_o_q_CD} 
	&	
	&	
	&	
	&	
	&	tomko [4.8] % TODO: нужен перевод
		\tabularnewline \midrule 
\tenevilglyph[yes][1]{JEN_2j_j} 
	&	
	&	
	&	
	&	
	&	tьmŋtwarkьt [ИЛИ:1.4об]  % TODO: нужен перевод % ilispread:1.5
		\tabularnewline \midrule 
\tenevilglyph[yes][3]{cD_j} 
	&	
	&	
	&	
	&	
	&	tьjoan [тыёгъан = засунул; словно напечатано] \currentGlyphWithAffixes{T}{} [12.19об] % TODO: уточнить перевод
		\tabularnewline \midrule 
\tenevilglyph[yes][3]{I_JY} 
	&	
	&	
	&	мычунен [мычунэӈ = гребень, расческа], гребень для волос [90]
	&	
	&	гребень [не рукой Т.] [57.22] 
		\tabularnewline \midrule 
\tenevilglyph[yes][1]{i_UDYE} 
	&	
	&	
	&	
	&	
	&	mьraьnelet \currentGlyphWithAffixes{mooqor}{N,L,T} [ИЛИ:2.11]  % TODO: нужен перевод % ilispread:2.10
		\tabularnewline \midrule 
\tenevilglyph[yes][3]{IY_I-j} 
	&	
	&	
	&	
	&	
	&	relqeneŋ [рэԓӄиинэӈ = сани для перевозки жердей яранги] \currentGlyphWithAffixes{}{mooqor} [ИЛИ:2.11]  % TODO: уточнить перевод % ilispread:2.10
		\tabularnewline \midrule 
\tenevilglyph[yes][3]{BD_iX_jN} 
	&	
	&	
	&	пэнъёлгын [пэнъёԓгын = печь], печь, печка [130] % знак отличается, вероятно, сильно стилизован
	&	
	&	печка [не рукой Т.] [57.23] 
		\tabularnewline \midrule 
\tenevilglyph[yes][1]{r_2q} 
	&	
	&	
	&	
	&	
	&	поеыся \currentGlyphWithAffixes{}{Q,N} [29.12] % TODO: нужна интерпретация
		\tabularnewline \midrule 
\tenevilglyph[yes][3]{r_2q-z-q} 
	&	
	&	
	&	
	&	
	&	gьteurkьt [гыттэвыркыт = становятся хитрыми] \currentGlyphWithAffixes{}{T} [ИЛИ:1.19об]  % TODO: уточнить интерпретацию % ilispread:1.20
		\tabularnewline \midrule 
\tenevilglyph[yes][3]{I_2j_IX_2q} 
	&	
	&	
	&	
	&	
	&	iacata [а'тчата = ждать] \currentGlyphWithAffixes{}{T,A} [ИЛИ:1.19]  % ilispread:1.19
		\tabularnewline \midrule 
\tenevilglyph[yes][1]{JE_j} 
	&	
	&	
	&	
	&	
	&	qetrou [4.8об,4.9] % TODO: нужен перевод
		\tabularnewline \midrule 
\tenevilglyph[yes][3]{I_2l} 
	&	пыԓвынтын
	&	
	&	
	&	
	&	\cite[364]{davydova2015a} \linebreak
		пытвытын [пыԓвынтын = железо, металл] [42.31] \linebreak
		эmьriapьlugьtьn \currentGlyphWithAffixes{imyrxenut}{} [ИЛИ:1.8]  % TODO: нужен перевод % ilispread:1.8
		\tabularnewline \midrule 
\tenevilglyph[yes][3]{3cF_z} 
	&	
	&	
	&	
	&	туша \cite[л. 66 об]{spbfaran79} 
	&	[11.4,11.7] 
		\tabularnewline \midrule 
\tenevilglyph[yes][3]{J-I} 
	&	вай
	&	
	&	
	&	
	&	waj [вай = вот, на] [ИЛИ:2.5об] \linebreak  % ilispread:2.5
		wae [вай] [ИЛИ:2.12]  % ilispread:2.11
		\tabularnewline \midrule 
\tenevilglyph[yes][4]{i-jX_cD} 
	&	ӄоо
	&	
	&	
	&	я нипомлу [я не помню] \cite[л. 67]{spbfaran79}
	&	koo [ӄоо = неизвестно] [5.1об] \linebreak
		ieчнаеу [не знаю] [30.7об]
		\tabularnewline \midrule 
\tenevilglyph[yes][3]{u} 
	&	ритԓит
	&	
	&	
	&	
	&	relet [ритԓит = груз на веревке, единица веса, фунт] [ИЛИ:1.15]  % ilispread:1.15
		\tabularnewline \midrule 
\tenevilglyph[yes][3]{dD_j} 
	&	ванэван
	&	
	&	
	&	
	&	vanьvan [ванэван = нет, ничуть, нисколько; слово напечатано] [12.20об] 
		\tabularnewline \midrule 
\tenevilglyph[yes][1]{3c} 
	&	
	&	
	&	
	&	упала \cite[л. 67]{spbfaran79} %TODO: нужна интерпретация % немного другое начертание
	&	колика [37.1об] %TODO: нужна интерпретация
		\tabularnewline \midrule 
\bottomrule
\end{longtable}
\end{landscape}

\section{Как читать таблицу} 

\subsection{Первый столбец}
Содержит: 

\begin{itemize}
\item Порядковый номер знака, может меняться по мере добавления новых знаков; 
\item В скобках степень уверенности в интерпретации знака:
	\begin{itemize}
		\item 0:	нет вообще ничего
		\item 1: 	есть только фонетическая какая-то наводка, значение которой неясно
		\item 2:	есть интерпретация, но сомнительная, либо есть противоречия между интерпретациями
		\item 3:	есть однозначная интерпретация, но источник всего один (или источники зависимые), и это не Теневиль, либо интерпретация написанного Теневилем вызывает сомнения, либо это имя личное без понятной транскрипции
		\item 4:	есть интерпретация, совпадающая или близкая в нескольких источниках, либо есть однозначный перевод приведенный Теневилем, а противоречия, если есть, объясняются ошибкой
		\item 5:	сильная уверенность, несколько независимых источников сходятся на одном значении, есть надежная интерпретация приведенная Теневилем для этого или для близкородственных знаков 
		\item 6:	однозначная интерпретация, проверенная в контексте % пока таких записей нет, появятся, когда буду смотреть в контексте
	\end{itemize}
\item Сам знак.
\end{itemize}

\subsection{Второй столбец}
Наиболее вероятная интерпретация данного знака.

\subsection{Остальные столбцы}

Если указана только ссылка, то есть изображение знака без интерпретации.

Если ссылка сопровождается текстом, то вначале текст написан так, как в документе (с точностью до сложностей разбора почерков). 

Если текст в источнике требует расшифровки, она указана в квадратных скобках. %Для чукотских слов, написанных латиницей, используется орфография из словаря Богораз-Тана\cite{bogoraz1937}.

Если интерпретация приводится для группы знаков или знак является частью «лигатуры», интерпретация написана в кавычках.

Звездочкой помечены источники, где знак отличается по начертанию от прочих.

\section{Описания источников} 

\subsection{СПбФ АРАН. Ф. 250. Оп. 1. Д. 79}

Материалы о «новых письменах чукотских оленеводов» содержат 71 лист. Это разнообразные материалы по письменности Теневиля, написанные разными почерками, а также машинописные черновики статьи В. Г. Богораз–Тана. 

\subsubsection{Листы 40–51}

Сводная таблица с символами и переводами на русский и иногда чукотский (латиницей).

\subsubsection{Листы 39 и об., 52 и об., 54, 56}

Листы с символами и переводами на чукотский (латиницей).

\subsubsection{Листы 37, 52, 53, 54 об, 55}

Листы со словами на русском и отдельными символами.

\subsubsection{Листы 64, 65 с оборотами}

Листы с символами и переводами на чукотский (латиницей) и русский. На л. 64 приведены числительные от 1 до 10, плюс 19 и 20.

\subsubsection{Листы 66–71 с оборотами}

Листы из тетради с символами и переводами на русский и, местами, чукотский (кириллицей). Почерк неуверенный, плохая орфография. На обороте л. 71 написано другим почерком «Altol [или Avtol] — Андрей Краснино 28/XII 39 г.»

\subsubsection{Листы 1–13, 25–32, 58–61}
Машинописные черновики машинописные черновики статьи В. Г. Богораз–Тана, а также ключи к табличкам Теневиля.

\subsubsection{Остальные листы}

Символов не содержат, кроме 38 об., где шесть символов без переводов, но с комментариями.

\subsection{«Архив Теневиля»}

Большое количество записей сделанных Теневилем в разные периоды его творчества и материалы И. П. Лаврова.

\subsubsection{Дело 40 — «Картотека» И. П. Лаврова}

\subsubsection{Дело 46 — Черновик статьи И. П. Лаврова «Архив Теневиля» }

\subsection{Две тетради}

Содержат около 230 символов с переводами на чукотский. Как следует из приложенного к одной из тетрадей письма, датированного 1937 годом, тетрадь была написана Теневилем специально, чтобы познакомить с устройством письма. 

\subsection{Статья В. Г. Богораз-Тана «Луораветланский (чукотский) язык» (1934)}

Таблица с переводом ряда символов на русский. Изображение (перерисовка) одной из табличек Теневиля с переводом одной из трех строк на ней на русский язык.

\subsection{Статья А. М. Миндалевича Hieroglyphic Characters of the Chuckchees (1934) }

Переводы на английский ряда отдельных символов. Перевод на английский двух фрагментов записей Теневиля.

\subsection{Статья И. П. Лаврова «Чукотский феномен» (1969)}

Три таблицы с переводами ряда символов на русский. Изображения символов сильно стилизованы, поэтому их затруднительно использовать в отсутствие других источников. Фотографии «лунного календаря» и «родового дерева» , нарисованных Теневилем.

\subsection{Статья М. Филиппова «Письмо чукчи Теневиля» (1966)}

Несколько фотографий записей и рисунков из «Архива Теневиля». 

\subsection{Диссертация Е. А. Давыдовой «Властные отношения в семейно-родственных коллективах оленных чукчей» (2015)}

Пять фотографий «дневниковых записей» Теневиля из архива отдела этнографии Сибири МАЭ РАН. Оригиналы всех пяти можно найти в «Архиве Теневиля». 

360 = 62.52об

361 = 48.12об

362 = 57.2

363 = 62.63об

364 = 62.85об 

\printbibliography

%\allchars

\end{document}